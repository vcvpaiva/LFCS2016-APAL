\documentclass[preprint,12pt]{elsarticle}

\usepackage{version} 

\includeversion{paper}
\excludeversion{report}

\usepackage{amssymb,amsmath}
\usepackage{cmll}
\usepackage{stmaryrd}
\usepackage{todonotes}
\usepackage{mathpartir}
\usepackage{hyperref}
\usepackage[barr]{xy}

% Commands that are useful for writing about type theory and programming language design.
%% \newcommand{\case}[4]{\text{case}\ #1\ \text{of}\ #2\text{.}#3\text{,}#2\text{.}#4}
\newcommand{\interp}[1]{\llbracket #1 \rrbracket}
\newcommand{\normto}[0]{\rightsquigarrow^{!}}
\newcommand{\join}[0]{\downarrow}
\newcommand{\redto}[0]{\rightsquigarrow}
\newcommand{\nat}[0]{\mathbb{N}}
\newcommand{\fun}[2]{\lambda #1.#2}
\newcommand{\CRI}[0]{\text{CR-Norm}}
\newcommand{\CRII}[0]{\text{CR-Pres}}
\newcommand{\CRIII}[0]{\text{CR-Prog}}
\newcommand{\subexp}[0]{\sqsubseteq}
%% Must include \usepackage{mathrsfs} for this to work.
\newcommand{\powerset}[0]{\mathscr{P}}

\date{}

% Ott includes.
\usepackage{color}

% generated by Ott 0.24 from: FILL.ott
\newcommand{\FILLdrule}[4][]{{\displaystyle\frac{\begin{array}{l}#2\end{array}}{#3}\quad\FILLdrulename{#4}}}
\newcommand{\FILLusedrule}[1]{\[#1\]}
\newcommand{\FILLpremise}[1]{ #1 \\}
\newenvironment{FILLdefnblock}[3][]{ \framebox{\mbox{#2}} \quad #3 \\[0pt]}{}
\newenvironment{FILLfundefnblock}[3][]{ \framebox{\mbox{#2}} \quad #3 \\[0pt]\begin{displaymath}\begin{array}{l}}{\end{array}\end{displaymath}}
\newcommand{\FILLfunclause}[2]{ #1 \equiv #2 \\}
\newcommand{\FILLnt}[1]{\mathit{#1}}
\newcommand{\FILLmv}[1]{\mathit{#1}}
\newcommand{\FILLkw}[1]{\mathbf{#1}}
\newcommand{\FILLsym}[1]{#1}
\newcommand{\FILLcom}[1]{\text{#1}}
\newcommand{\FILLdrulename}[1]{\textsc{#1}}
\newcommand{\FILLcomplu}[5]{\overline{#1}^{\,#2\in #3 #4 #5}}
\newcommand{\FILLcompu}[3]{\overline{#1}^{\,#2<#3}}
\newcommand{\FILLcomp}[2]{\overline{#1}^{\,#2}}
\newcommand{\FILLgrammartabular}[1]{\begin{supertabular}{llcllllll}#1\end{supertabular}}
\newcommand{\FILLmetavartabular}[1]{\begin{supertabular}{ll}#1\end{supertabular}}
\newcommand{\FILLrulehead}[3]{$#1$ & & $#2$ & & & \multicolumn{2}{l}{#3}}
\newcommand{\FILLprodline}[6]{& & $#1$ & $#2$ & $#3 #4$ & $#5$ & $#6$}
\newcommand{\FILLfirstprodline}[6]{\FILLprodline{#1}{#2}{#3}{#4}{#5}{#6}}
\newcommand{\FILLlongprodline}[2]{& & $#1$ & \multicolumn{4}{l}{$#2$}}
\newcommand{\FILLfirstlongprodline}[2]{\FILLlongprodline{#1}{#2}}
\newcommand{\FILLbindspecprodline}[6]{\FILLprodline{#1}{#2}{#3}{#4}{#5}{#6}}
\newcommand{\FILLprodnewline}{\\}
\newcommand{\FILLinterrule}{\\[5.0mm]}
\newcommand{\FILLafterlastrule}{\\}
\newcommand{\FILLmetavars}{
\FILLmetavartabular{
 $ \FILLmv{term\_var} ,\, \FILLmv{w} ,\, \FILLmv{x} ,\, \FILLmv{y} ,\, \FILLmv{z} ,\, \FILLmv{v} $ &  \\
 $ \FILLmv{index\_var} ,\, \FILLmv{i} ,\, \FILLmv{j} ,\, \FILLmv{k} $ &  \\
}}

\newcommand{\FILLform}{
\FILLrulehead{\FILLnt{form}  ,\ \FILLnt{A}  ,\ \FILLnt{B}  ,\ \FILLnt{C}  ,\ \FILLnt{D}  ,\ \FILLnt{E}}{::=}{}\FILLprodnewline
\FILLfirstprodline{|}{ \top }{}{}{}{}\FILLprodnewline
\FILLprodline{|}{ \perp }{}{}{}{}\FILLprodnewline
\FILLprodline{|}{ \FILLnt{A}  \multimap   \FILLnt{B} }{}{}{}{}\FILLprodnewline
\FILLprodline{|}{ \FILLnt{A}  \otimes  \FILLnt{B} }{}{}{}{}\FILLprodnewline
\FILLprodline{|}{ \FILLnt{A}  \parr  \FILLnt{B} }{}{}{}{}\FILLprodnewline
\FILLprodline{|}{\FILLsym{(}  \FILLnt{A}  \FILLsym{)}} {\textsf{S}}{}{}{}\FILLprodnewline
\FILLprodline{|}{ \FILLnt{A} } {\textsf{M}}{}{}{}\FILLprodnewline
\FILLprodline{|}{ ( \FILLnt{A} )^N } {\textsf{M}}{}{}{}\FILLprodnewline
\FILLprodline{|}{\FILLsym{1}} {\textsf{M}}{}{}{}\FILLprodnewline
\FILLprodline{|}{ \lnot  \FILLnt{A} } {\textsf{M}}{}{}{}\FILLprodnewline
\FILLprodline{|}{ \FILLnt{A} ^{\perp} } {\textsf{M}}{}{}{}\FILLprodnewline
\FILLprodline{|}{ \bigotimes ( \Gamma ) }{}{}{}{}\FILLprodnewline
\FILLprodline{|}{ \bigparr ( \Delta ) }{}{}{}{}\FILLprodnewline
\FILLprodline{|}{ \bigotimes ( \Delta ) }{}{}{}{}\FILLprodnewline
\FILLprodline{|}{ \bigparr ( \Gamma ) }{}{}{}{}}

\newcommand{\FILLpatterns}{
\FILLrulehead{\FILLnt{patterns}  ,\ \FILLnt{p}}{::=}{}\FILLprodnewline
\FILLfirstprodline{|}{ * }{}{}{}{}\FILLprodnewline
\FILLprodline{|}{\FILLmv{x}}{}{}{}{}\FILLprodnewline
\FILLprodline{|}{ \FILLnt{p_{{\mathrm{1}}}}  \otimes  \FILLnt{p_{{\mathrm{2}}}} }{}{}{}{}\FILLprodnewline
\FILLprodline{|}{ \FILLnt{p_{{\mathrm{1}}}}  \parr  \FILLnt{p_{{\mathrm{2}}}} }{}{}{}{}\FILLprodnewline
\FILLprodline{|}{ - }{}{}{}{}\FILLprodnewline
\FILLprodline{|}{\FILLsym{(}  \FILLnt{p}  \FILLsym{)}} {\textsf{S}}{}{}{}}

\newcommand{\FILLterm}{
\FILLrulehead{\FILLnt{term}  ,\ \FILLnt{t}  ,\ \FILLnt{e}  ,\ \FILLnt{d}  ,\ \FILLnt{f}  ,\ \FILLnt{g}  ,\ \FILLnt{u}}{::=}{}\FILLprodnewline
\FILLfirstprodline{|}{\FILLmv{x}}{}{}{}{}\FILLprodnewline
\FILLprodline{|}{\FILLsym{*}}{}{}{}{}\FILLprodnewline
\FILLprodline{|}{ \circ }{}{}{}{}\FILLprodnewline
\FILLprodline{|}{ \FILLnt{e_{{\mathrm{1}}}}  \otimes  \FILLnt{e_{{\mathrm{2}}}} }{}{}{}{}\FILLprodnewline
\FILLprodline{|}{ \FILLnt{e_{{\mathrm{1}}}}  \parr  \FILLnt{e_{{\mathrm{2}}}} }{}{}{}{}\FILLprodnewline
\FILLprodline{|}{ \lambda  \FILLmv{x}  .  \FILLnt{t} }{}{}{}{}\FILLprodnewline
\FILLprodline{|}{ \mathsf{let}\, \FILLnt{t} \,\mathsf{be}\, \FILLnt{p} \,\mathsf{in}\, \FILLnt{e} }{}{}{}{}\FILLprodnewline
\FILLprodline{|}{\FILLnt{f} \, \FILLnt{e}}{}{}{}{}\FILLprodnewline
\FILLprodline{|}{ \mathsf{\operatorname{let-pat} }\, \FILLnt{t} \, \FILLnt{p} \, \FILLnt{e} } {\textsf{M}}{}{}{}\FILLprodnewline
\FILLprodline{|}{\FILLsym{[}  \FILLnt{t}  \FILLsym{/}  \FILLmv{x}  \FILLsym{]}  \FILLnt{t'}} {\textsf{M}}{}{}{}\FILLprodnewline
\FILLprodline{|}{\FILLsym{[}  \FILLnt{t}  \FILLsym{/}  \FILLmv{x}  \FILLsym{,}  \FILLnt{e}  \FILLsym{/}  \FILLmv{y}  \FILLsym{]}  \FILLnt{t'}} {\textsf{M}}{}{}{}\FILLprodnewline
\FILLprodline{|}{ ( \FILLnt{t} ) } {\textsf{S}}{}{}{}\FILLprodnewline
\FILLprodline{|}{ \FILLnt{t} } {\textsf{M}}{}{}{}\FILLprodnewline
\FILLprodline{|}{ {\color{blue}{ \FILLnt{t} } } } {\textsf{M}}{}{}{}\FILLprodnewline
\FILLprodline{|}{\FILLsym{\mbox{?}}} {\textsf{M}}{}{}{}}

\newcommand{\FILLG}{
\FILLrulehead{\Gamma}{::=}{}\FILLprodnewline
\FILLfirstprodline{|}{\FILLmv{x}  \FILLsym{:}  \FILLnt{A}}{}{}{}{}\FILLprodnewline
\FILLprodline{|}{ \cdot }{}{}{}{}\FILLprodnewline
\FILLprodline{|}{\Gamma  \FILLsym{,}  \Gamma'}{}{}{}{}\FILLprodnewline
\FILLprodline{|}{ {\color{blue}{ \FILLmv{x} } } :  \FILLnt{A} }{}{}{}{}\FILLprodnewline
\FILLprodline{|}{\FILLmv{x_{{\mathrm{1}}}}  \FILLsym{:}  \FILLnt{A_{{\mathrm{1}}}}  \FILLsym{,} \, ... \, \FILLsym{,}  \FILLmv{x_{\FILLmv{i}}}  \FILLsym{:}  \FILLnt{A_{\FILLmv{i}}}} {\textsf{M}}{}{}{}\FILLprodnewline
\FILLprodline{|}{ ( \Gamma )^N } {\textsf{M}}{}{}{}\FILLprodnewline
\FILLprodline{|}{ ( \Delta )^N } {\textsf{M}}{}{}{}\FILLprodnewline
\FILLprodline{|}{ \lnot ( \Delta ) } {\textsf{M}}{}{}{}\FILLprodnewline
\FILLprodline{|}{ \lnot ( \Gamma ) } {\textsf{M}}{}{}{}}

\newcommand{\FILLGL}{
\FILLrulehead{\Gamma}{::=}{}\FILLprodnewline
\FILLfirstprodline{|}{\FILLnt{A}}{}{}{}{}\FILLprodnewline
\FILLprodline{|}{ \cdot }{}{}{}{}\FILLprodnewline
\FILLprodline{|}{\Gamma  \FILLsym{,}  \Gamma'}{}{}{}{}\FILLprodnewline
\FILLprodline{|}{\FILLnt{A_{{\mathrm{1}}}}  \FILLsym{,} \, ... \, \FILLsym{,}  \FILLnt{A_{\FILLmv{i}}}}{}{}{}{}}

\newcommand{\FILLL}{
\FILLrulehead{\Delta}{::=}{}\FILLprodnewline
\FILLfirstprodline{|}{\FILLnt{t}  \FILLsym{:}  \FILLnt{A}}{}{}{}{}\FILLprodnewline
\FILLprodline{|}{ \cdot }{}{}{}{}\FILLprodnewline
\FILLprodline{|}{ \Delta  \mid  \Delta' }{}{}{}{}\FILLprodnewline
\FILLprodline{|}{ \Delta }{}{}{}{}\FILLprodnewline
\FILLprodline{|}{\Delta  \FILLsym{,}  \Delta'}{}{}{}{}\FILLprodnewline
\FILLprodline{|}{\FILLsym{[}  \FILLnt{t}  \FILLsym{/}  \FILLmv{x}  \FILLsym{]}  \Delta}{}{}{}{}\FILLprodnewline
\FILLprodline{|}{ \mathsf{let}\, \FILLnt{t} \,\mathsf{be}\, \FILLnt{p} \,\mathsf{in}\, \Delta }{}{}{}{}\FILLprodnewline
\FILLprodline{|}{\FILLsym{(}  \Delta  \FILLsym{)}}{}{}{}{}\FILLprodnewline
\FILLprodline{|}{ \mathsf{\operatorname{let-pat} }\, \FILLnt{t} \, \FILLnt{p} \, \Delta } {\textsf{M}}{}{}{}\FILLprodnewline
\FILLprodline{|}{\FILLnt{t_{{\mathrm{1}}}}  \FILLsym{:}  \FILLnt{A_{{\mathrm{1}}}}  \FILLsym{,} \, ... \, \FILLsym{,}  \FILLnt{t_{\FILLmv{i}}}  \FILLsym{:}  \FILLnt{A_{\FILLmv{i}}}} {\textsf{M}}{}{}{}}

\newcommand{\FILLLL}{
\FILLrulehead{\Delta}{::=}{}\FILLprodnewline
\FILLfirstprodline{|}{\FILLnt{A}}{}{}{}{}\FILLprodnewline
\FILLprodline{|}{ \cdot }{}{}{}{}\FILLprodnewline
\FILLprodline{|}{\Delta  \FILLsym{,}  \Delta'}{}{}{}{}\FILLprodnewline
\FILLprodline{|}{ \Delta }{}{}{}{}\FILLprodnewline
\FILLprodline{|}{\FILLnt{A_{{\mathrm{1}}}}  \FILLsym{,} \, ... \, \FILLsym{,}  \FILLnt{A_{\FILLmv{i}}}}{}{}{}{}}

\newcommand{\FILLterminals}{
\FILLrulehead{\FILLnt{terminals}}{::=}{}\FILLprodnewline
\FILLfirstprodline{|}{ \multimap }{}{}{}{}\FILLprodnewline
\FILLprodline{|}{ \parr }{}{}{}{}\FILLprodnewline
\FILLprodline{|}{ \otimes }{}{}{}{}}

\newcommand{\FILLformula}{
\FILLrulehead{\FILLnt{formula}}{::=}{}\FILLprodnewline
\FILLfirstprodline{|}{\FILLnt{judgement}}{}{}{}{}\FILLprodnewline
\FILLprodline{|}{ \FILLnt{formula_{{\mathrm{1}}}}  \quad  \FILLnt{formula_{{\mathrm{2}}}} }{}{}{}{}\FILLprodnewline
\FILLprodline{|}{\FILLsym{(}  \FILLnt{formula}  \FILLsym{)}}{}{}{}{}\FILLprodnewline
\FILLprodline{|}{ \FILLmv{x}  \not\in \mathsf{FV}(  \Delta  ) }{}{}{}{}\FILLprodnewline
\FILLprodline{|}{ \FILLmv{x}  \in \mathsf{FV}(  \FILLnt{t}  ) }{}{}{}{}\FILLprodnewline
\FILLprodline{|}{ \FILLmv{x} , \FILLmv{y}  \not\in \mathsf{FV}(  \Delta  ) }{}{}{}{}\FILLprodnewline
\FILLprodline{|}{ \FILLmv{x}  \not\in \mathsf{FV}(  \FILLnt{t}  ) }{}{}{}{}\FILLprodnewline
\FILLprodline{|}{ \FILLmv{x} , \FILLmv{y}  \not\in \mathsf{FV}(  \FILLnt{t}  ) }{}{}{}{}\FILLprodnewline
\FILLprodline{|}{\Delta_{{\mathrm{1}}}  \FILLsym{=}  \Delta_{{\mathrm{2}}}}{}{}{}{}\FILLprodnewline
\FILLprodline{|}{ \mathsf{FV}( \FILLnt{t} ) }{}{}{}{}\FILLprodnewline
\FILLprodline{|}{ \mathsf{FV}( \Delta ) }{}{}{}{}}

\newcommand{\FILLInferRules}{
\FILLrulehead{\FILLnt{InferRules}}{::=}{}\FILLprodnewline
\FILLfirstprodline{|}{ \Gamma  \vdash  \Delta }{}{}{}{}\FILLprodnewline
\FILLprodline{|}{ \Gamma  \vdash  \Delta }{}{}{}{}\FILLprodnewline
\FILLprodline{|}{\FILLnt{f}  \FILLsym{=}  \FILLnt{e}}{}{}{}{}}

\newcommand{\FILLjudgement}{
\FILLrulehead{\FILLnt{judgement}}{::=}{}\FILLprodnewline
\FILLfirstprodline{|}{\FILLnt{InferRules}}{}{}{}{}}

\newcommand{\FILLuserXXsyntax}{
\FILLrulehead{\FILLnt{user\_syntax}}{::=}{}\FILLprodnewline
\FILLfirstprodline{|}{\FILLmv{term\_var}}{}{}{}{}\FILLprodnewline
\FILLprodline{|}{\FILLmv{index\_var}}{}{}{}{}\FILLprodnewline
\FILLprodline{|}{\FILLnt{form}}{}{}{}{}\FILLprodnewline
\FILLprodline{|}{\FILLnt{patterns}}{}{}{}{}\FILLprodnewline
\FILLprodline{|}{\FILLnt{term}}{}{}{}{}\FILLprodnewline
\FILLprodline{|}{\Gamma}{}{}{}{}\FILLprodnewline
\FILLprodline{|}{\Gamma}{}{}{}{}\FILLprodnewline
\FILLprodline{|}{\Delta}{}{}{}{}\FILLprodnewline
\FILLprodline{|}{\Delta}{}{}{}{}\FILLprodnewline
\FILLprodline{|}{\FILLnt{terminals}}{}{}{}{}\FILLprodnewline
\FILLprodline{|}{\FILLnt{formula}}{}{}{}{}}

\newcommand{\FILLgrammar}{\FILLgrammartabular{
\FILLform\FILLinterrule
\FILLpatterns\FILLinterrule
\FILLterm\FILLinterrule
\FILLG\FILLinterrule
\FILLGL\FILLinterrule
\FILLL\FILLinterrule
\FILLLL\FILLinterrule
\FILLterminals\FILLinterrule
\FILLformula\FILLinterrule
\FILLInferRules\FILLinterrule
\FILLjudgement\FILLinterrule
\FILLuserXXsyntax\FILLafterlastrule
}}

% defnss
% defns InferRules
%% defn Ltyping
\newcommand{\FILLdruleLLXXAx}[1]{\FILLdrule[#1]{%
}{
 \FILLnt{A}  \vdash  \FILLnt{A} }{%
{\FILLdrulename{LL\_Ax}}{}%
}}


\newcommand{\FILLdruleLLXXCut}[1]{\FILLdrule[#1]{%
\FILLpremise{  \Gamma  \vdash  \FILLnt{A}  \FILLsym{,}  \Delta   \quad   \Gamma'  \FILLsym{,}  \FILLnt{A}  \vdash  \Delta'  }%
}{
 \Gamma  \FILLsym{,}  \Gamma'  \vdash  \Delta  \FILLsym{,}  \Delta' }{%
{\FILLdrulename{LL\_Cut}}{}%
}}


\newcommand{\FILLdruleLLXXNegl}[1]{\FILLdrule[#1]{%
\FILLpremise{ \Gamma  \FILLsym{,}  \FILLnt{A}  \vdash  \Delta }%
}{
 \Gamma  \vdash   \FILLnt{A} ^{\perp}   \FILLsym{,}  \Delta }{%
{\FILLdrulename{LL\_Negl}}{}%
}}


\newcommand{\FILLdruleLLXXNegr}[1]{\FILLdrule[#1]{%
\FILLpremise{ \Gamma  \vdash  \FILLnt{A}  \FILLsym{,}  \Delta }%
}{
 \Gamma  \FILLsym{,}   \FILLnt{A} ^{\perp}   \vdash  \Delta }{%
{\FILLdrulename{LL\_Negr}}{}%
}}


\newcommand{\FILLdruleLLXXTl}[1]{\FILLdrule[#1]{%
\FILLpremise{ \Gamma  \vdash  \Delta }%
}{
 \Gamma  \FILLsym{,}   \top   \vdash  \Delta }{%
{\FILLdrulename{LL\_Tl}}{}%
}}


\newcommand{\FILLdruleLLXXTr}[1]{\FILLdrule[#1]{%
}{
  \cdot   \vdash   \top  }{%
{\FILLdrulename{LL\_Tr}}{}%
}}


\newcommand{\FILLdruleLLXXTenl}[1]{\FILLdrule[#1]{%
\FILLpremise{ \Gamma  \FILLsym{,}  \FILLnt{A}  \FILLsym{,}  \FILLnt{B}  \vdash  \Delta }%
}{
 \Gamma  \FILLsym{,}   \FILLnt{A}  \otimes  \FILLnt{B}   \vdash  \Delta }{%
{\FILLdrulename{LL\_Tenl}}{}%
}}


\newcommand{\FILLdruleLLXXTenr}[1]{\FILLdrule[#1]{%
\FILLpremise{  \Gamma  \vdash  \FILLnt{A}  \FILLsym{,}  \Delta   \quad   \Gamma'  \vdash  \FILLnt{B}  \FILLsym{,}  \Delta'  }%
}{
 \Gamma  \FILLsym{,}  \Gamma'  \vdash     \FILLnt{A}  \otimes  \FILLnt{B}    \FILLsym{,}  \Delta   \FILLsym{,}  \Delta' }{%
{\FILLdrulename{LL\_Tenr}}{}%
}}


\newcommand{\FILLdruleLLXXPl}[1]{\FILLdrule[#1]{%
}{
  \perp   \vdash   \cdot  }{%
{\FILLdrulename{LL\_Pl}}{}%
}}


\newcommand{\FILLdruleLLXXPr}[1]{\FILLdrule[#1]{%
\FILLpremise{ \Gamma  \vdash  \Delta }%
}{
 \Gamma  \vdash   \perp   \FILLsym{,}  \Delta }{%
{\FILLdrulename{LL\_Pr}}{}%
}}


\newcommand{\FILLdruleLLXXParl}[1]{\FILLdrule[#1]{%
\FILLpremise{  \Gamma  \FILLsym{,}  \FILLnt{A}  \vdash  \Delta   \quad   \Gamma'  \FILLsym{,}  \FILLnt{B}  \vdash  \Delta'  }%
}{
 \Gamma  \FILLsym{,}  \Gamma'  \FILLsym{,}   \FILLnt{A}  \parr  \FILLnt{B}   \vdash  \Delta  \FILLsym{,}  \Delta' }{%
{\FILLdrulename{LL\_Parl}}{}%
}}


\newcommand{\FILLdruleLLXXParr}[1]{\FILLdrule[#1]{%
\FILLpremise{ \Gamma  \vdash    \Delta  \FILLsym{,}  \FILLnt{A}   \FILLsym{,}  \FILLnt{B}   \FILLsym{,}  \Delta' }%
}{
 \Gamma  \vdash   \Delta  \FILLsym{,}   \FILLnt{A}  \parr  \FILLnt{B}    \FILLsym{,}  \Delta' }{%
{\FILLdrulename{LL\_Parr}}{}%
}}


\newcommand{\FILLdruleLLXXImpl}[1]{\FILLdrule[#1]{%
\FILLpremise{  \Gamma  \vdash  \FILLnt{A}  \FILLsym{,}  \Delta   \quad   \Gamma'  \FILLsym{,}  \FILLnt{B}  \vdash  \Delta'  }%
}{
 \Gamma  \FILLsym{,}   \FILLnt{A}  \multimap   \FILLnt{B}   \FILLsym{,}  \Gamma'  \vdash  \Delta  \FILLsym{,}  \Delta' }{%
{\FILLdrulename{LL\_Impl}}{}%
}}


\newcommand{\FILLdruleLLXXImpr}[1]{\FILLdrule[#1]{%
\FILLpremise{ \Gamma  \FILLsym{,}  \FILLnt{A}  \vdash  \FILLnt{B}  \FILLsym{,}  \Delta }%
}{
 \Gamma  \vdash   \FILLnt{A}  \multimap   \FILLnt{B}   \FILLsym{,}  \Delta }{%
{\FILLdrulename{LL\_Impr}}{}%
}}


\newcommand{\FILLdruleLLXXExl}[1]{\FILLdrule[#1]{%
\FILLpremise{ \Gamma  \FILLsym{,}  \FILLnt{A}  \FILLsym{,}  \FILLnt{B}  \vdash  \Delta }%
}{
 \Gamma  \FILLsym{,}  \FILLnt{B}  \FILLsym{,}  \FILLnt{A}  \vdash  \Delta }{%
{\FILLdrulename{LL\_Exl}}{}%
}}


\newcommand{\FILLdruleLLXXExr}[1]{\FILLdrule[#1]{%
\FILLpremise{ \Gamma  \vdash    \Delta_{{\mathrm{1}}}  \FILLsym{,}  \FILLnt{A}   \FILLsym{,}  \FILLnt{B}   \FILLsym{,}  \Delta_{{\mathrm{2}}} }%
}{
 \Gamma  \vdash    \Delta_{{\mathrm{1}}}  \FILLsym{,}  \FILLnt{B}   \FILLsym{,}  \FILLnt{A}   \FILLsym{,}  \Delta_{{\mathrm{2}}} }{%
{\FILLdrulename{LL\_Exr}}{}%
}}

\newcommand{\FILLdefnLtyping}[1]{\begin{FILLdefnblock}[#1]{$ \Gamma  \vdash  \Delta $}{}
\FILLusedrule{\FILLdruleLLXXAx{}}
\FILLusedrule{\FILLdruleLLXXCut{}}
\FILLusedrule{\FILLdruleLLXXNegl{}}
\FILLusedrule{\FILLdruleLLXXNegr{}}
\FILLusedrule{\FILLdruleLLXXTl{}}
\FILLusedrule{\FILLdruleLLXXTr{}}
\FILLusedrule{\FILLdruleLLXXTenl{}}
\FILLusedrule{\FILLdruleLLXXTenr{}}
\FILLusedrule{\FILLdruleLLXXPl{}}
\FILLusedrule{\FILLdruleLLXXPr{}}
\FILLusedrule{\FILLdruleLLXXParl{}}
\FILLusedrule{\FILLdruleLLXXParr{}}
\FILLusedrule{\FILLdruleLLXXImpl{}}
\FILLusedrule{\FILLdruleLLXXImpr{}}
\FILLusedrule{\FILLdruleLLXXExl{}}
\FILLusedrule{\FILLdruleLLXXExr{}}
\end{FILLdefnblock}}

%% defn typing
\newcommand{\FILLdruleAx}[1]{\FILLdrule[#1]{%
}{
 \FILLmv{x}  \FILLsym{:}  \FILLnt{A}  \vdash  \FILLmv{x}  \FILLsym{:}  \FILLnt{A} }{%
{\FILLdrulename{Ax}}{}%
}}


\newcommand{\FILLdruleCut}[1]{\FILLdrule[#1]{%
\FILLpremise{  \Gamma  \vdash   \FILLnt{t}  \FILLsym{:}  \FILLnt{A}  \mid  \Delta    \quad   \Gamma'  \FILLsym{,}  \FILLmv{y}  \FILLsym{:}  \FILLnt{A}  \vdash  \Delta'  }%
}{
 \Gamma  \FILLsym{,}  \Gamma'  \vdash   \Delta  \mid  \FILLsym{[}  \FILLnt{t}  \FILLsym{/}  \FILLmv{y}  \FILLsym{]}  \Delta'  }{%
{\FILLdrulename{Cut}}{}%
}}


\newcommand{\FILLdruleTl}[1]{\FILLdrule[#1]{%
\FILLpremise{ \Gamma  \vdash  \Delta }%
}{
 \Gamma  \FILLsym{,}  \FILLmv{x}  \FILLsym{:}   \top   \vdash   \mathsf{let}\, \FILLmv{x} \,\mathsf{be}\,  *  \,\mathsf{in}\, \Delta  }{%
{\FILLdrulename{Tl}}{}%
}}


\newcommand{\FILLdruleTr}[1]{\FILLdrule[#1]{%
}{
  \cdot   \vdash  \FILLsym{*}  \FILLsym{:}   \top  }{%
{\FILLdrulename{Tr}}{}%
}}


\newcommand{\FILLdruleTenl}[1]{\FILLdrule[#1]{%
\FILLpremise{ \Gamma  \FILLsym{,}  \FILLmv{x}  \FILLsym{:}  \FILLnt{A}  \FILLsym{,}  \FILLmv{y}  \FILLsym{:}  \FILLnt{B}  \vdash  \Delta }%
}{
 \Gamma  \FILLsym{,}  \FILLmv{z}  \FILLsym{:}   \FILLnt{A}  \otimes  \FILLnt{B}   \vdash   \mathsf{let}\, \FILLmv{z} \,\mathsf{be}\,  \FILLmv{x}  \otimes  \FILLmv{y}  \,\mathsf{in}\, \Delta  }{%
{\FILLdrulename{Tenl}}{}%
}}


\newcommand{\FILLdruleUnTenl}[1]{\FILLdrule[#1]{%
\FILLpremise{ \Gamma  \FILLsym{,}  \FILLmv{z}  \FILLsym{:}   \FILLnt{A}  \otimes  \FILLnt{B}   \vdash   \mathsf{let}\, \FILLmv{z} \,\mathsf{be}\,  \FILLmv{x}  \otimes  \FILLmv{y}  \,\mathsf{in}\, \Delta  }%
}{
 \Gamma  \FILLsym{,}  \FILLmv{x}  \FILLsym{:}  \FILLnt{A}  \FILLsym{,}  \FILLmv{y}  \FILLsym{:}  \FILLnt{B}  \vdash  \Delta }{%
{\FILLdrulename{UnTenl}}{}%
}}


\newcommand{\FILLdruleTenr}[1]{\FILLdrule[#1]{%
\FILLpremise{  \Gamma  \vdash   \FILLnt{e}  \FILLsym{:}  \FILLnt{A}  \mid  \Delta    \quad   \Gamma'  \vdash   \FILLnt{f}  \FILLsym{:}  \FILLnt{B}  \mid  \Delta'   }%
}{
 \Gamma  \FILLsym{,}  \Gamma'  \vdash      \FILLnt{e}  \otimes  \FILLnt{f}   \FILLsym{:}   \FILLnt{A}  \otimes  \FILLnt{B}   \mid  \Delta    \mid  \Delta'  }{%
{\FILLdrulename{Tenr}}{}%
}}


\newcommand{\FILLdrulePl}[1]{\FILLdrule[#1]{%
}{
 \FILLmv{x}  \FILLsym{:}   \perp   \vdash   \cdot  }{%
{\FILLdrulename{Pl}}{}%
}}


\newcommand{\FILLdrulePr}[1]{\FILLdrule[#1]{%
\FILLpremise{ \Gamma  \vdash  \Delta }%
}{
 \Gamma  \vdash    \circ   \FILLsym{:}   \perp   \mid  \Delta  }{%
{\FILLdrulename{Pr}}{}%
}}


\newcommand{\FILLdruleParl}[1]{\FILLdrule[#1]{%
\FILLpremise{  \Gamma  \FILLsym{,}  \FILLmv{x}  \FILLsym{:}  \FILLnt{A}  \vdash  \Delta   \quad   \Gamma'  \FILLsym{,}  \FILLmv{y}  \FILLsym{:}  \FILLnt{B}  \vdash  \Delta'  }%
}{
 \Gamma  \FILLsym{,}  \Gamma'  \FILLsym{,}  \FILLmv{z}  \FILLsym{:}   \FILLnt{A}  \parr  \FILLnt{B}   \vdash     \mathsf{\operatorname{let-pat} }\, \FILLmv{z} \, \FILLsym{(}   \FILLmv{x}  \parr   -    \FILLsym{)} \, \Delta    \mid    \mathsf{\operatorname{let-pat} }\, \FILLmv{z} \, \FILLsym{(}    -   \parr  \FILLmv{y}   \FILLsym{)} \, \Delta'    }{%
{\FILLdrulename{Parl}}{}%
}}


\newcommand{\FILLdruleParr}[1]{\FILLdrule[#1]{%
\FILLpremise{ \Gamma  \vdash       \Delta  \mid  \FILLnt{e}  \FILLsym{:}  \FILLnt{A}    \mid  \FILLnt{f}  \FILLsym{:}  \FILLnt{B}    \mid  \Delta'  }%
}{
 \Gamma  \vdash     \Delta  \mid   \FILLnt{e}  \parr  \FILLnt{f}   \FILLsym{:}   \FILLnt{A}  \parr  \FILLnt{B}     \mid  \Delta'  }{%
{\FILLdrulename{Parr}}{}%
}}


\newcommand{\FILLdruleUnparr}[1]{\FILLdrule[#1]{%
\FILLpremise{ \Gamma  \vdash     \Delta  \mid   \FILLnt{e}  \parr  \FILLnt{f}   \FILLsym{:}   \FILLnt{A}  \parr  \FILLnt{B}     \mid  \Delta'  }%
}{
 \Gamma  \vdash       \Delta  \mid  \FILLnt{e}  \FILLsym{:}  \FILLnt{A}    \mid  \FILLnt{f}  \FILLsym{:}  \FILLnt{B}    \mid  \Delta'  }{%
{\FILLdrulename{Unparr}}{}%
}}


\newcommand{\FILLdruleImpl}[1]{\FILLdrule[#1]{%
\FILLpremise{  \Gamma  \vdash   \FILLnt{e}  \FILLsym{:}  \FILLnt{A}  \mid  \Delta    \quad   \Gamma'  \FILLsym{,}  \FILLmv{x}  \FILLsym{:}  \FILLnt{B}  \vdash  \Delta'  }%
}{
 \Gamma  \FILLsym{,}  \FILLmv{y}  \FILLsym{:}   \FILLnt{A}  \multimap   \FILLnt{B}   \FILLsym{,}  \Gamma'  \vdash   \Delta  \mid  \FILLsym{[}  \FILLmv{y} \, \FILLnt{e}  \FILLsym{/}  \FILLmv{x}  \FILLsym{]}  \Delta'  }{%
{\FILLdrulename{Impl}}{}%
}}


\newcommand{\FILLdruleImpr}[1]{\FILLdrule[#1]{%
\FILLpremise{  \Gamma  \FILLsym{,}  \FILLmv{x}  \FILLsym{:}  \FILLnt{A}  \vdash   \FILLnt{e}  \FILLsym{:}  \FILLnt{B}  \mid  \Delta    \quad   \FILLmv{x}  \not\in \mathsf{FV}(  \Delta  )  }%
}{
 \Gamma  \vdash    \lambda  \FILLmv{x}  .  \FILLnt{e}   \FILLsym{:}   \FILLnt{A}  \multimap   \FILLnt{B}   \mid  \Delta  }{%
{\FILLdrulename{Impr}}{}%
}}


\newcommand{\FILLdruleUncur}[1]{\FILLdrule[#1]{%
\FILLpremise{ \Gamma  \vdash    \lambda  \FILLmv{x}  .  \FILLnt{e}   \FILLsym{:}   \FILLnt{A}  \multimap   \FILLnt{B}   \mid  \Delta  }%
}{
 \Gamma  \FILLsym{,}  \FILLmv{x}  \FILLsym{:}  \FILLnt{A}  \vdash   \FILLnt{e}  \FILLsym{:}  \FILLnt{B}  \mid  \Delta  }{%
{\FILLdrulename{Uncur}}{}%
}}


\newcommand{\FILLdruleNegL}[1]{\FILLdrule[#1]{%
\FILLpremise{  \Gamma  \FILLsym{,}  \FILLmv{x}  \FILLsym{:}  \FILLnt{A}  \vdash  \Delta   \quad   \FILLmv{x}  \not\in \mathsf{FV}(  \Delta  )  }%
}{
 \Gamma  \vdash   \lambda  \FILLmv{x}  .   \circ    \FILLsym{:}   \lnot  \FILLnt{A}   \FILLsym{,}  \Delta }{%
{\FILLdrulename{NegL}}{}%
}}


\newcommand{\FILLdruleNegR}[1]{\FILLdrule[#1]{%
\FILLpremise{ \Gamma  \vdash  \FILLnt{t}  \FILLsym{:}  \FILLnt{A}  \FILLsym{,}  \Delta }%
}{
 \Gamma  \FILLsym{,}  \FILLmv{z}  \FILLsym{:}   \lnot  \FILLnt{A}   \vdash  \Delta }{%
{\FILLdrulename{NegR}}{}%
}}


\newcommand{\FILLdruleExl}[1]{\FILLdrule[#1]{%
\FILLpremise{ \Gamma  \FILLsym{,}  \FILLmv{x}  \FILLsym{:}  \FILLnt{A}  \FILLsym{,}  \FILLmv{y}  \FILLsym{:}  \FILLnt{B}  \vdash  \Delta }%
}{
 \Gamma  \FILLsym{,}  \FILLmv{y}  \FILLsym{:}  \FILLnt{B}  \FILLsym{,}  \FILLmv{x}  \FILLsym{:}  \FILLnt{A}  \vdash  \Delta }{%
{\FILLdrulename{Exl}}{}%
}}


\newcommand{\FILLdruleExr}[1]{\FILLdrule[#1]{%
\FILLpremise{ \Gamma  \vdash       \Delta_{{\mathrm{1}}}  \mid  \FILLnt{t_{{\mathrm{1}}}}  \FILLsym{:}  \FILLnt{A}    \mid  \FILLnt{t_{{\mathrm{2}}}}  \FILLsym{:}  \FILLnt{B}    \mid  \Delta_{{\mathrm{2}}}  }%
}{
 \Gamma  \vdash       \Delta_{{\mathrm{1}}}  \mid  \FILLnt{t_{{\mathrm{2}}}}  \FILLsym{:}  \FILLnt{B}    \mid  \FILLnt{t_{{\mathrm{1}}}}  \FILLsym{:}  \FILLnt{A}    \mid  \Delta_{{\mathrm{2}}}  }{%
{\FILLdrulename{Exr}}{}%
}}

\newcommand{\FILLdefntyping}[1]{\begin{FILLdefnblock}[#1]{$ \Gamma  \vdash  \Delta $}{}
\FILLusedrule{\FILLdruleAx{}}
\FILLusedrule{\FILLdruleCut{}}
\FILLusedrule{\FILLdruleTl{}}
\FILLusedrule{\FILLdruleTr{}}
\FILLusedrule{\FILLdruleTenl{}}
\FILLusedrule{\FILLdruleUnTenl{}}
\FILLusedrule{\FILLdruleTenr{}}
\FILLusedrule{\FILLdrulePl{}}
\FILLusedrule{\FILLdrulePr{}}
\FILLusedrule{\FILLdruleParl{}}
\FILLusedrule{\FILLdruleParr{}}
\FILLusedrule{\FILLdruleUnparr{}}
\FILLusedrule{\FILLdruleImpl{}}
\FILLusedrule{\FILLdruleImpr{}}
\FILLusedrule{\FILLdruleUncur{}}
\FILLusedrule{\FILLdruleNegL{}}
\FILLusedrule{\FILLdruleNegR{}}
\FILLusedrule{\FILLdruleExl{}}
\FILLusedrule{\FILLdruleExr{}}
\end{FILLdefnblock}}

%% defn catEq
\newcommand{\FILLdruleAlpha}[1]{\FILLdrule[#1]{%
\FILLpremise{ \FILLmv{y}  \not\in \mathsf{FV}(  \FILLnt{t}  ) }%
}{
\FILLnt{t}  \FILLsym{=}  \FILLsym{[}  \FILLmv{y}  \FILLsym{/}  \FILLmv{x}  \FILLsym{]}  \FILLnt{t}}{%
{\FILLdrulename{Alpha}}{}%
}}


\newcommand{\FILLdruleEtaFun}[1]{\FILLdrule[#1]{%
\FILLpremise{ \FILLmv{x}  \not\in \mathsf{FV}(  \FILLnt{f}  ) }%
}{
 (  \lambda  \FILLmv{x}  .   \FILLnt{f} \, \FILLmv{x}   )   \FILLsym{=}  \FILLnt{f}}{%
{\FILLdrulename{EtaFun}}{}%
}}


\newcommand{\FILLdruleBetaFun}[1]{\FILLdrule[#1]{%
}{
 (  \lambda  \FILLmv{x}  .  \FILLnt{e}  )  \, \FILLnt{e'}  \FILLsym{=}  \FILLsym{[}  \FILLnt{e'}  \FILLsym{/}  \FILLmv{x}  \FILLsym{]}  \FILLnt{e}}{%
{\FILLdrulename{BetaFun}}{}%
}}


\newcommand{\FILLdruleEtaOneI}[1]{\FILLdrule[#1]{%
}{
 \mathsf{let}\, \FILLsym{*} \,\mathsf{be}\,  *  \,\mathsf{in}\, \FILLnt{e}   \FILLsym{=}  \FILLnt{e}}{%
{\FILLdrulename{Eta1I}}{}%
}}


\newcommand{\FILLdruleEtaTwoI}[1]{\FILLdrule[#1]{%
\FILLpremise{ \FILLmv{y}  \not\in \mathsf{FV}(  \FILLnt{f}  ) }%
}{
 \mathsf{let}\, \FILLmv{y} \,\mathsf{be}\,  *  \,\mathsf{in}\, \FILLnt{f}   \FILLsym{=}  \FILLnt{f}}{%
{\FILLdrulename{Eta2I}}{}%
}}


\newcommand{\FILLdruleBetaI}[1]{\FILLdrule[#1]{%
}{
 \mathsf{let}\, \FILLnt{u} \,\mathsf{be}\,  *  \,\mathsf{in}\, \FILLsym{[}  \FILLsym{*}  \FILLsym{/}  \FILLmv{z}  \FILLsym{]}  \FILLnt{f}   \FILLsym{=}  \FILLsym{[}  \FILLnt{u}  \FILLsym{/}  \FILLmv{z}  \FILLsym{]}  \FILLnt{f}}{%
{\FILLdrulename{BetaI}}{}%
}}


\newcommand{\FILLdruleNatI}[1]{\FILLdrule[#1]{%
}{
\FILLsym{[}   \mathsf{let}\, \FILLnt{u} \,\mathsf{be}\,  *  \,\mathsf{in}\, \FILLnt{e}   \FILLsym{/}  \FILLmv{y}  \FILLsym{]}  \FILLnt{f}  \FILLsym{=}   \mathsf{let}\, \FILLnt{u} \,\mathsf{be}\,  *  \,\mathsf{in}\, \FILLsym{[}  \FILLnt{e}  \FILLsym{/}  \FILLmv{y}  \FILLsym{]}  \FILLnt{f} }{%
{\FILLdrulename{NatI}}{}%
}}


\newcommand{\FILLdruleEtaTen}[1]{\FILLdrule[#1]{%
\FILLpremise{ \FILLmv{x} , \FILLmv{y}  \not\in \mathsf{FV}(  \FILLnt{t}  ) }%
}{
 \mathsf{let}\, \FILLnt{t'} \,\mathsf{be}\,  \FILLmv{x}  \otimes  \FILLmv{y}  \,\mathsf{in}\, \FILLnt{t}   \FILLsym{=}  \FILLnt{t}}{%
{\FILLdrulename{EtaTen}}{}%
}}


\newcommand{\FILLdruleBetaOneTen}[1]{\FILLdrule[#1]{%
}{
 \mathsf{let}\,  \FILLnt{e}  \otimes  \FILLnt{t}  \,\mathsf{be}\,  \FILLmv{x}  \otimes  \FILLmv{y}  \,\mathsf{in}\, \FILLnt{u}   \FILLsym{=}  \FILLsym{[}  \FILLnt{e}  \FILLsym{/}  \FILLmv{x}  \FILLsym{,}  \FILLnt{t}  \FILLsym{/}  \FILLmv{y}  \FILLsym{]}  \FILLnt{u}}{%
{\FILLdrulename{Beta1Ten}}{}%
}}


\newcommand{\FILLdruleBetaTwoTen}[1]{\FILLdrule[#1]{%
}{
 \mathsf{let}\, \FILLnt{u} \,\mathsf{be}\,  \FILLmv{x}  \otimes  \FILLmv{y}  \,\mathsf{in}\, \FILLsym{[}   \FILLmv{x}  \otimes  \FILLmv{y}   \FILLsym{/}  \FILLmv{z}  \FILLsym{]}  \FILLnt{f}   \FILLsym{=}  \FILLsym{[}  \FILLnt{u}  \FILLsym{/}  \FILLmv{z}  \FILLsym{]}  \FILLnt{f}}{%
{\FILLdrulename{Beta2Ten}}{}%
}}


\newcommand{\FILLdruleNatTen}[1]{\FILLdrule[#1]{%
}{
\FILLsym{[}   \mathsf{let}\, \FILLnt{u} \,\mathsf{be}\,  \FILLmv{x}  \otimes  \FILLmv{y}  \,\mathsf{in}\, \FILLnt{g}   \FILLsym{/}  \FILLmv{w}  \FILLsym{]}  \FILLnt{f}  \FILLsym{=}   \mathsf{let}\, \FILLnt{u} \,\mathsf{be}\,  \FILLmv{x}  \otimes  \FILLmv{y}  \,\mathsf{in}\, \FILLsym{[}  \FILLnt{g}  \FILLsym{/}  \FILLmv{w}  \FILLsym{]}  \FILLnt{f} }{%
{\FILLdrulename{NatTen}}{}%
}}


\newcommand{\FILLdruleEtaParU}[1]{\FILLdrule[#1]{%
}{
\FILLnt{u}  \FILLsym{=}   \circ }{%
{\FILLdrulename{EtaParU}}{}%
}}


\newcommand{\FILLdruleEtaPar}[1]{\FILLdrule[#1]{%
}{
  (  \mathsf{let}\, \FILLnt{u} \,\mathsf{be}\,  \FILLmv{x}  \parr   -   \,\mathsf{in}\, \FILLmv{x}  )   \parr   (  \mathsf{let}\, \FILLnt{u} \,\mathsf{be}\,   -   \parr  \FILLmv{y}  \,\mathsf{in}\, \FILLmv{y}  )    \FILLsym{=}  \FILLnt{u}}{%
{\FILLdrulename{EtaPar}}{}%
}}


\newcommand{\FILLdruleBetaOnePar}[1]{\FILLdrule[#1]{%
}{
 \mathsf{let}\,  \FILLnt{u}  \parr  \FILLnt{t}  \,\mathsf{be}\,  \FILLmv{x}  \parr   -   \,\mathsf{in}\, \FILLnt{e}   \FILLsym{=}  \FILLsym{[}  \FILLnt{u}  \FILLsym{/}  \FILLmv{x}  \FILLsym{]}  \FILLnt{e}}{%
{\FILLdrulename{Beta1Par}}{}%
}}


\newcommand{\FILLdruleBetaTwoPar}[1]{\FILLdrule[#1]{%
}{
 \mathsf{let}\,  \FILLnt{u}  \parr  \FILLnt{t}  \,\mathsf{be}\,   -   \parr  \FILLmv{y}  \,\mathsf{in}\, \FILLnt{e}   \FILLsym{=}  \FILLsym{[}  \FILLnt{t}  \FILLsym{/}  \FILLmv{y}  \FILLsym{]}  \FILLnt{e}}{%
{\FILLdrulename{Beta2Par}}{}%
}}


\newcommand{\FILLdruleNatOnePar}[1]{\FILLdrule[#1]{%
}{
 \mathsf{let}\, \FILLnt{t} \,\mathsf{be}\,  \FILLmv{x}  \parr   -   \,\mathsf{in}\, \FILLsym{[}  \FILLnt{u}  \FILLsym{/}  \FILLmv{x}  \FILLsym{]}  \FILLnt{f}   \FILLsym{=}  \FILLsym{[}   \mathsf{let}\, \FILLnt{t} \,\mathsf{be}\,  \FILLmv{x}  \parr   -   \,\mathsf{in}\, \FILLnt{u}   \FILLsym{/}  \FILLmv{x}  \FILLsym{]}  \FILLnt{f}}{%
{\FILLdrulename{Nat1Par}}{}%
}}


\newcommand{\FILLdruleNatTwoPar}[1]{\FILLdrule[#1]{%
}{
 \mathsf{let}\, \FILLnt{t} \,\mathsf{be}\,   -   \parr  \FILLmv{y}  \,\mathsf{in}\, \FILLsym{[}  \FILLmv{v}  \FILLsym{/}  \FILLmv{y}  \FILLsym{]}  \FILLnt{f}   \FILLsym{=}  \FILLsym{[}   \mathsf{let}\, \FILLnt{t} \,\mathsf{be}\,   -   \parr  \FILLmv{y}  \,\mathsf{in}\, \FILLmv{v}   \FILLsym{/}  \FILLmv{y}  \FILLsym{]}  \FILLnt{f}}{%
{\FILLdrulename{Nat2Par}}{}%
}}


\newcommand{\FILLdruleLam}[1]{\FILLdrule[#1]{%
\FILLpremise{\FILLnt{t}  \FILLsym{=}  \FILLnt{t'}}%
}{
 \lambda  \FILLmv{x}  .  \FILLnt{t}   \FILLsym{=}   \lambda  \FILLmv{x}  .  \FILLnt{t''} }{%
{\FILLdrulename{Lam}}{}%
}}


\newcommand{\FILLdruleAppOne}[1]{\FILLdrule[#1]{%
\FILLpremise{\FILLnt{t_{{\mathrm{1}}}}  \FILLsym{=}  \FILLnt{t'_{{\mathrm{1}}}}}%
}{
\FILLnt{t_{{\mathrm{1}}}} \, \FILLnt{t_{{\mathrm{2}}}}  \FILLsym{=}  \FILLnt{t'_{{\mathrm{1}}}} \, \FILLnt{t_{{\mathrm{2}}}}}{%
{\FILLdrulename{App1}}{}%
}}


\newcommand{\FILLdruleAppTwo}[1]{\FILLdrule[#1]{%
\FILLpremise{\FILLnt{t_{{\mathrm{2}}}}  \FILLsym{=}  \FILLnt{t'_{{\mathrm{2}}}}}%
}{
\FILLnt{t_{{\mathrm{1}}}} \, \FILLnt{t_{{\mathrm{2}}}}  \FILLsym{=}  \FILLnt{t_{{\mathrm{1}}}} \, \FILLnt{t'_{{\mathrm{2}}}}}{%
{\FILLdrulename{App2}}{}%
}}


\newcommand{\FILLdruleTenOne}[1]{\FILLdrule[#1]{%
\FILLpremise{\FILLnt{t_{{\mathrm{1}}}}  \FILLsym{=}  \FILLnt{t'_{{\mathrm{1}}}}}%
}{
 \FILLnt{t_{{\mathrm{1}}}}  \otimes  \FILLnt{t_{{\mathrm{2}}}}   \FILLsym{=}   \FILLnt{t'_{{\mathrm{1}}}}  \otimes  \FILLnt{t_{{\mathrm{2}}}} }{%
{\FILLdrulename{Ten1}}{}%
}}


\newcommand{\FILLdruleTenTwo}[1]{\FILLdrule[#1]{%
\FILLpremise{\FILLnt{t_{{\mathrm{2}}}}  \FILLsym{=}  \FILLnt{t'_{{\mathrm{2}}}}}%
}{
 \FILLnt{t_{{\mathrm{1}}}}  \otimes  \FILLnt{t_{{\mathrm{2}}}}   \FILLsym{=}   \FILLnt{t_{{\mathrm{1}}}}  \otimes  \FILLnt{t'_{{\mathrm{2}}}} }{%
{\FILLdrulename{Ten2}}{}%
}}


\newcommand{\FILLdruleParOne}[1]{\FILLdrule[#1]{%
\FILLpremise{\FILLnt{t_{{\mathrm{1}}}}  \FILLsym{=}  \FILLnt{t'_{{\mathrm{1}}}}}%
}{
 \FILLnt{t_{{\mathrm{1}}}}  \parr  \FILLnt{t_{{\mathrm{2}}}}   \FILLsym{=}   \FILLnt{t'_{{\mathrm{1}}}}  \parr  \FILLnt{t_{{\mathrm{2}}}} }{%
{\FILLdrulename{Par1}}{}%
}}


\newcommand{\FILLdruleParTwo}[1]{\FILLdrule[#1]{%
\FILLpremise{\FILLnt{t_{{\mathrm{2}}}}  \FILLsym{=}  \FILLnt{t'_{{\mathrm{2}}}}}%
}{
 \FILLnt{t_{{\mathrm{1}}}}  \parr  \FILLnt{t_{{\mathrm{2}}}}   \FILLsym{=}   \FILLnt{t_{{\mathrm{1}}}}  \parr  \FILLnt{t'_{{\mathrm{2}}}} }{%
{\FILLdrulename{Par2}}{}%
}}


\newcommand{\FILLdruleLetOne}[1]{\FILLdrule[#1]{%
\FILLpremise{\FILLnt{t}  \FILLsym{=}  \FILLnt{t'}}%
}{
 \mathsf{let}\, \FILLnt{t} \,\mathsf{be}\, \FILLnt{p} \,\mathsf{in}\, \FILLnt{e}   \FILLsym{=}   \mathsf{let}\, \FILLnt{t'} \,\mathsf{be}\, \FILLnt{p} \,\mathsf{in}\, \FILLnt{e} }{%
{\FILLdrulename{Let1}}{}%
}}


\newcommand{\FILLdruleLetTwo}[1]{\FILLdrule[#1]{%
\FILLpremise{\FILLnt{e}  \FILLsym{=}  \FILLnt{e'}}%
}{
 \mathsf{let}\, \FILLnt{t} \,\mathsf{be}\, \FILLnt{p} \,\mathsf{in}\, \FILLnt{e}   \FILLsym{=}   \mathsf{let}\, \FILLnt{t} \,\mathsf{be}\, \FILLnt{p} \,\mathsf{in}\, \FILLnt{e'} }{%
{\FILLdrulename{Let2}}{}%
}}


\newcommand{\FILLdruleRefl}[1]{\FILLdrule[#1]{%
}{
\FILLnt{t}  \FILLsym{=}  \FILLnt{t}}{%
{\FILLdrulename{Refl}}{}%
}}


\newcommand{\FILLdruleSym}[1]{\FILLdrule[#1]{%
\FILLpremise{\FILLnt{t}  \FILLsym{=}  \FILLnt{t'}}%
}{
\FILLnt{t'}  \FILLsym{=}  \FILLnt{t}}{%
{\FILLdrulename{Sym}}{}%
}}


\newcommand{\FILLdruleTrans}[1]{\FILLdrule[#1]{%
\FILLpremise{ \FILLnt{t_{{\mathrm{1}}}}  \FILLsym{=}  \FILLnt{t_{{\mathrm{2}}}}  \quad  \FILLnt{t_{{\mathrm{2}}}}  \FILLsym{=}  \FILLnt{t_{{\mathrm{3}}}} }%
}{
\FILLnt{t_{{\mathrm{1}}}}  \FILLsym{=}  \FILLnt{t_{{\mathrm{3}}}}}{%
{\FILLdrulename{Trans}}{}%
}}

\newcommand{\FILLdefncatEq}[1]{\begin{FILLdefnblock}[#1]{$\FILLnt{f}  \FILLsym{=}  \FILLnt{e}$}{}
\FILLusedrule{\FILLdruleAlpha{}}
\FILLusedrule{\FILLdruleEtaFun{}}
\FILLusedrule{\FILLdruleBetaFun{}}
\FILLusedrule{\FILLdruleEtaOneI{}}
\FILLusedrule{\FILLdruleEtaTwoI{}}
\FILLusedrule{\FILLdruleBetaI{}}
\FILLusedrule{\FILLdruleNatI{}}
\FILLusedrule{\FILLdruleEtaTen{}}
\FILLusedrule{\FILLdruleBetaOneTen{}}
\FILLusedrule{\FILLdruleBetaTwoTen{}}
\FILLusedrule{\FILLdruleNatTen{}}
\FILLusedrule{\FILLdruleEtaParU{}}
\FILLusedrule{\FILLdruleEtaPar{}}
\FILLusedrule{\FILLdruleBetaOnePar{}}
\FILLusedrule{\FILLdruleBetaTwoPar{}}
\FILLusedrule{\FILLdruleNatOnePar{}}
\FILLusedrule{\FILLdruleNatTwoPar{}}
\FILLusedrule{\FILLdruleLam{}}
\FILLusedrule{\FILLdruleAppOne{}}
\FILLusedrule{\FILLdruleAppTwo{}}
\FILLusedrule{\FILLdruleTenOne{}}
\FILLusedrule{\FILLdruleTenTwo{}}
\FILLusedrule{\FILLdruleParOne{}}
\FILLusedrule{\FILLdruleParTwo{}}
\FILLusedrule{\FILLdruleLetOne{}}
\FILLusedrule{\FILLdruleLetTwo{}}
\FILLusedrule{\FILLdruleRefl{}}
\FILLusedrule{\FILLdruleSym{}}
\FILLusedrule{\FILLdruleTrans{}}
\end{FILLdefnblock}}


\newcommand{\FILLdefnsInferRules}{
\FILLdefnLtyping{}\FILLdefntyping{}\FILLdefncatEq{}}

\newcommand{\FILLdefnss}{
\FILLdefnsInferRules
}

\newcommand{\FILLall}{\FILLmetavars\\[0pt]
\FILLgrammar\\[5.0mm]
\FILLdefnss}





% Renewing some Ott commands to shrink some of the labels.
\renewcommand{\FILLdrulename}[1]{\scriptsize \textsc{#1}}
\newcommand{\ifrName}[1]{\scriptsize \textsc{#1}}

% Cat commands.
\newcommand{\cat}[1]{\mathcal{#1}}
\newcommand{\catop}[1]{\cat{#1}^{\mathsf{op}}}
\newcommand{\homs}[3]{\mathsf{Hom}_{\cat{#1}}(#2,#3)}
\newcommand{\limp}[0]{\multimap}
\newcommand{\dial}[0]{\mathsf{Dial_2}(\mathsf{Sets})}
\newcommand{\dialSets}[1]{\mathsf{Dial_{#1}}(\mathsf{Sets})}
\newcommand{\dcSets}[1]{\mathsf{DC_{#1}}(\mathsf{Sets})}
\newcommand{\sets}[0]{\mathsf{Sets}}
\newcommand{\obj}[1]{\mathsf{Obj}(#1)}
\newcommand{\mor}[1]{\mathsf{Mor(#1)}}
\newcommand{\id}[0]{\mathsf{id}}
\newcommand{\lett}[0]{\mathsf{let}\,}
\newcommand{\inn}[0]{\,\mathsf{in}\,}
\newcommand{\cur}[1]{\mathsf{cur}(#1)}
\newcommand{\curi}[1]{\mathsf{cur}^{-1}(#1)}

\newtheorem{theorem}{Theorem} 
\newtheorem{lemma}[theorem]{Lemma} 
\newtheorem{corollary}[theorem]{Corollary} 
\newdefinition{definition}{Definition} 
\newproof{proof}{Proof} 

\begin{document}

\newcounter{condCounter}

\newcommand{\rcond}[0]{\refstepcounter{condCounter}\arabic{condCounter}}

\begin{frontmatter}
\title{Multiple Conclusion Linear Logic: \\ Cut Elimination and more}

\author[au]{Harley Eades III\corref{cor1}}
\ead{heades@augusta.edu}

\author[nuance]{Valeria de Paiva}
\ead{valeria.depaiva@gmail.com}

\cortext[cor1]{Corresponding author}
\address[au]{Computer and Information Sciences, Augusta University, Augusta, GA}
\address[nuance]{Nuance Communications, Sunnyvale, CA}

\begin{abstract}
  Full Intuitionistic Linear Logic (FILL) was first introduced by
  Hyland and de Paiva, and went against current beliefs that it was
  not possible to incorporate all of the linear connectives,
  e.g. tensor, par, and implication, into an intuitionistic linear
  logic. Bierman showed that their formalization of FILL did not enjoy
  cut-elimination as such, but Bellin proposed a small change to the
  definition of FILL regaining cut-elimination and using proofnets.
  In this note we adopt Bellin's proposed change and give a direct
  proof of cut-elimination for the sequent calculus.  Then we show
  that a categorical model of FILL in the basic dialectica category is
  also a LinearNonLinear (LNL) model of Benton and a full tensor model
  of Melli\`es' and Tabareau's tensorial logic.  We give a
  double-negation translation of linear logic into FILL that
  explicitly uses par in addition to tensor.  Lastly, we introduce a
  new library to be used in the proof assistant Agda for proving
  properties of dialectica categories.
\end{abstract}

\begin{keyword}
full intuitionistic linear logic \sep classical linear logic \sep
dialectica category \sep cut-elimination \sep tensorial logic \sep
%linear/non-linear models \sep categorical model \sep proof theory \sep
par \sep formal proof \sep proof assistants
\end{keyword}

%% \maketitle
\end{frontmatter}
\section{Introduction}
\label{sec:introduction}

A commonly held belief during the early history of linear logic was
that the linear-connective par could not be incorporated into an
intuitionistic linear logic.  This belief was challenged when de Paiva
gave a categorical understanding of G\"odel's Dialectica
interpretation in terms of dialectica categories
\cite{dePaiva:1987,dePaiva:1988}.  

Dialectica categories were originally conceived  as models of
intuitionistic logic, but they are actually models of intuitionistic
linear logic, containing the linear connectives: tensor, implication,
the additives, and the exponentials.  Further work improved de Paiva's
models to capture both intuitionistic and classical linear logic.
Armed with this semantic insight de Paiva gave the first formalization
of Full Intuitionistic Linear Logic (FILL) \cite{dePaiva:1988}. The logical system FILL
is a sequent calculus with multiple conclusions in addition to
multiple hypotheses.  Logics of this type go back to Gentzen's work on
the sequent calculus for classical logic LK and for intuitionistic
logic LJ, and Maehara's work on LJ' \cite{Maehara:1954,Takeuti:1975}.
The sequents in these types of logics usually have the form $\Gamma\vdash \Delta$ where $\Gamma$ and $\Delta$ are multisets of formulas.
Sequents such as these are read as ``the conjunction of the formulas
in $\Gamma$ imply the disjunction of the formulas in $\Delta$''.  For
a brief, but more complete history of logics with multiple conclusions
see the introduction to \cite{dePaiva:2005}.

Gentzen showed that to obtain intuitionistic logic one could start
with the classical logic LK and  place a cardinality restriction on the
right-hand side of sequents, saying that only one or no formula was allowed in every inference rule. However, this is not the only means of
enforcing intuitionism.  Maehara showed that in the propositional case one could simply place the cardinality restriction on the premise of
the implication right rule, and leave all of the other rules of LK
unrestricted.  This restriction is sometimes called the Dragalin
restriction, as it appeared in his AMS textbook \cite{Dragalin:1988}.
The classical implication right rule has the form:
\begin{center}
  \begin{math}
    $$\mprset{flushleft}
    \inferrule* [right=impR] {
      \Gamma, \FILLnt{A} \vdash \FILLnt{B}, \Delta
    }{\Gamma \vdash  \FILLnt{A}  \multimap   \FILLnt{B} , \Delta}
  \end{math}
\end{center}
By placing the Dragalin restriction on the previous rule we obtain:
\begin{center}
  \begin{math}
    $$\mprset{flushleft}
    \inferrule* [right=impR] {
      \Gamma, \FILLnt{A} \vdash \FILLnt{B}
    }{\Gamma \vdash  \FILLnt{A}  \multimap   \FILLnt{B} }
  \end{math}
\end{center}
The first formalization of FILL used the Dragalin restriction,
see \cite{dePaiva:1988} p. 58, but Schellinx showed that this restriction has
the unfortunate consequence of breaking cut-elimination
\cite{Schellinx:1991}.

 Hyland and de Paiva gave an alternative formalization of FILL
with the intention of regaining cut-elimination \cite{Hyland:1993}.  This
new formalization lifted the Dragalin restriction by decorating
sequents with a term assignment.  Hypotheses were assigned variables,
and the conclusions were assigned terms.  Then using these terms one
can track the use of hypotheses throughout a derivation.  They
proposed a new implication right rule:
\begin{center}
  \begin{math}
    $$\mprset{flushleft}
    \inferrule* [right=impR] {
       \Gamma  \FILLsym{,}  \FILLmv{x}  \FILLsym{:}  \FILLnt{A}  \vdash  \FILLnt{t}  \FILLsym{:}  \FILLnt{B}  \FILLsym{,}  \Delta 
      \\
       \FILLmv{x}  \not\in \mathsf{FV}(  \Delta  ) 
    }{ \Gamma  \vdash   \lambda  \FILLmv{x}  .  \FILLnt{t}   \FILLsym{:}   \FILLnt{A}  \multimap   \FILLnt{B}   \FILLsym{,}  \Delta }
  \end{math}
\end{center}
Intuitionism is enforced in this rule by requiring that the variable
being discharged, $x$, is not free in terms annotating other conclusions.
This formalization did not enjoy cut-elimination either.

Bierman was able to give a counterexample to cut-elimination
\cite{Bierman:1996}.  As Bierman explains the problem then was with the
left rule for the multiplicative disjunction par.  The original rule
was as follows:
\begin{center}
  \begin{math}
    $$\mprset{flushleft}
    \inferrule* [right=parL] {
       \Gamma  \FILLsym{,}  \FILLmv{x}  \FILLsym{:}  \FILLnt{A}  \vdash  \Delta  
      \\
       \Gamma'  \FILLsym{,}  \FILLmv{y}  \FILLsym{:}  \FILLnt{B}  \vdash  \Delta' 
    }{ \Gamma  \FILLsym{,}  \Gamma'  \FILLsym{,}  \FILLmv{z}  \FILLsym{:}   \FILLnt{A}  \parr  \FILLnt{B}   \vdash     \mathsf{let}\, \FILLmv{z} \,\mathsf{be}\, \FILLsym{(}   \FILLmv{x}  \parr   -    \FILLsym{)} \,\mathsf{in}\, \Delta    \mid    \mathsf{let}\, \FILLmv{z} \,\mathsf{be}\, \FILLsym{(}    -   \parr  \FILLmv{y}   \FILLsym{)} \,\mathsf{in}\, \Delta'    }
  \end{math}
\end{center}
In this rule the pattern variables $x$ and $y$ are bound in each term
of $\Delta$ and $\Delta'$ respectively. Notice that the variable $z$
becomes free in every term in $\Delta$ and $\Delta'$. Bierman showed
that this rule mixed with the restriction on implication right
prevents the usual cut-elimination step that commutes cut with the
left rule for par.  The main idea behind the counterexample is that in
the derivation before commuting the cut it is possible to discharge
$z$ using implication right, but after the cut is commuted past the
left rule for par, the variable $z$ becomes free in more than one
conclusion, and thus, can no longer be discharged.

In the conclusion of Bierman's note he describes Bellin's  alternate left rule
for par.  This new left-rule is as follows:
\begin{center}
  \begin{math}
    \FILLdruleParl{}
  \end{math}
\end{center}
In this rule $ \mathsf{\operatorname{let-pat} }\, \FILLmv{z} \, \FILLsym{(}   \FILLmv{x}  \parr   -    \FILLsym{)} \, \FILLnt{t} $ and $ \mathsf{\operatorname{let-pat} }\, \FILLmv{z} \, \FILLsym{(}    -   \parr  \FILLmv{y}   \FILLsym{)} \, \FILLnt{t'} $ only let-bind $z$ in $t$ or $t'$ if $x \in FV(t)$ or $y \in
FV(t')$.  Otherwise the terms are left unaltered.  Bellin showed that
by adopting this rule cut-elimination can be proven by reduction to
the cut-elimination procedure for proof nets for multiplicative linear
logic with the mix rule \cite{Bellin:1997}.  However, this is an
indirect proof that requires the adoption of proof nets.

\textbf{Contributions.} In this paper our main contribution is to give
a direct proof of cut-elimination for FILL with Bellin's proposed
par-left rule (Section~\ref{sec:cut-elimination}).  A direct proof
accomplishes two goals: the first is to complete the picture of FILL
Hyland and de Paiva started, and the second is to view a direct proof
of cut-elimination as a means of checking the correctness of the
formulation of FILL given here.  The latter point is important for
future work.  Following the proof of cut-elimination we show that the
categorical model of FILL called $\dial$, the basic dialectica
category, is also a linear/non-linear model of Benton
(Section~\ref{sec:full_lnl_models}) and a full tensor model of
Melli\`es' and Tabareau's tensor logic
(Section~\ref{sec:tensor_logic}). Then we give a double-negation
translation of multi-conclusion classical linear logic into FILL
(Section~\ref{subsec:a_negative_translation_of_linear_logic_into_fill_using_tensorial_negation}).
Due to the complexities of working in $\dial$ we have formalized all
of the constructions and proofs used in
Section~\ref{sec:full_lnl_models} and Section~\ref{sec:tensor_logic}
-- although our formal verification does not include the
double-negation translation in
Section~\ref{subsec:a_negative_translation_of_linear_logic_into_fill_using_tensorial_negation}
-- in the Agda proof assistant\footnote{The Agda development can be
  found at \url{https://github.com/heades/cut-fill-agda}.}.  We
%extracted 
synthetized
a general library for proving properties of dialectica
categories and introduce it in
Section~\ref{sec:the_agda_formalization:_a_library_for_studying_dialectica_categories}
along with generalizations of dialectica categories using the notion
of a lineale.

\textbf{Related Work.} The first formalization of FILL with
cut-elimination was due to Bra\"uner and de Paiva \cite{Brauner:1998}.
Their formalization can be seen as a linear version of LK with a
sophisticated meta-level dependency tracking system.  A proof of a
FILL sequent in their formalization amounts to a classical derivation,
$\pi$, invariant in what they call the FILL property:
\begin{center}
  \begin{itemize}
  \item The hypothesis discharged by an application of the implication
    right rule in $\pi$ is a dependency of the conclusion of the
    implication being introduced.
  \end{itemize}
\end{center}
They were able to show that their formalization is sound, complete,
and enjoys cut-elimination.  In favor of the term assignment
formalization given here over Bra\"uner and de Paiva's formalization
we can say that the dependency tracking system complicates both the
definition of the logic and its use.  However, one might conjecture
that their system is more fundamental and hence more generalizable.
It might be possible to prove cut-elimination of the term assignment
formalization of FILL relative to Bra\"uner and de Paiva's dependency
tracking system by erasing the terms on conclusions and then tracking
which variable is free in which conclusion.  However, as we stated
above a direct proof is more desirable than a relative one.

The work of de Paiva and Pereira used annotations on the sequents of LK to arrive
at full intuitionistic logic (FIL) with multiple conclusion that
enjoys cut-elimination \cite{dePaiva:2005}. They annotate hypothesis
with natural number indices, and conclusions with finite sets of
indices.  The sets of indices on conclusions correspond to the
collection of the hypotheses that the conclusion depends on.  Then
they have a similar property to that of Bra\"uner and de Paiva's
formalization.  In fact, the dependency tracking system is very
similar to this formalization, but the dependency tracking has been
collapsed into the object language instead of being at the meta-level.

Clouston et al. give both a deep inference calculus and a display
calculus for FILL that admits cut-elimination \cite{Clouston:2013}.
Both of these systems are refinements of a larger one called
bi-intuitionistic linear logic (BiLL).  This logic contains every
logical connective of FILL with the addition of the exclusion (or
subtraction) connective.  This connective can be defined categorically
as the left-adjoint to par.  Thus, exclusion is the dual to
implication.  A positive aspect to this work is that the resulting
systems are annotation free, but at a price of complexity.  Deep
inference and display calculi are harder to understand, and their
system requires FILL to be defined as a refinement of a system with
additional connectives.  We show in this paper that such a refinement
is unnecessary.  In addition, a term assignment system is closer to
traditional logic than deep inference and display calculi, and it is
closer, through the lens of the Curry-Howard-Lambek correspondence, to
a type theoretic understanding of FILL.
% section introduction (end)

\section{Full Intuitionistic Linear Logic (FILL)}
\label{sec:full_intuitionistic_linear_logic_(fill)}

In this section we give a brief description of FILL.  We first give
the syntax of formulas, patterns, terms, and contexts.  Following the
syntax we define several meta-functions that will be used when
defining the inference rules of the logic.

\begin{definition}
  \label{def:syntax}
  The syntax for FILL is as follows:
  \begin{center}
    \begin{math}
      \begin{array}{cll}
        \text{(Formulas)}       & \FILLnt{A}, \FILLnt{B}, \FILLnt{C}, \FILLnt{D}, \FILLnt{E} ::=  \top  \mid  \perp 
        \mid  \FILLnt{A}  \multimap   \FILLnt{B}  \mid  \FILLnt{A}  \otimes  \FILLnt{B}  \mid  \FILLnt{A}  \parr  \FILLnt{B}  \\
        \text{(Patterns)} & \FILLnt{p} ::=  *  \mid  -  \mid \FILLmv{x} \mid  \FILLnt{p_{{\mathrm{1}}}}  \otimes  \FILLnt{p_{{\mathrm{2}}}}  \mid  \FILLnt{p_{{\mathrm{1}}}}  \parr  \FILLnt{p_{{\mathrm{2}}}} \\
        \text{(Terms)}          & \FILLnt{t}, \FILLnt{e} ::= \FILLmv{x} \mid \FILLsym{*} \mid  \circ  \mid
         \FILLnt{t_{{\mathrm{1}}}}  \otimes  \FILLnt{t_{{\mathrm{2}}}}  \mid  \FILLnt{t_{{\mathrm{1}}}}  \parr  \FILLnt{t_{{\mathrm{2}}}}  \mid  \lambda  \FILLmv{x}  .  \FILLnt{t}  \mid  \mathsf{let}\, \FILLnt{t} \,\mathsf{be}\, \FILLnt{p} \,\mathsf{in}\, \FILLnt{e}  \mid \FILLnt{t_{{\mathrm{1}}}} \, \FILLnt{t_{{\mathrm{2}}}}\\
        \text{(Left Contexts)}  & \Gamma ::=  \cdot  \mid \FILLmv{x}  \FILLsym{:}  \FILLnt{A} \mid \Gamma_{{\mathrm{1}}}  \FILLsym{,}  \Gamma_{{\mathrm{2}}}\\
        \text{(Right Contexts)} & \Delta ::=  \cdot  \mid \FILLnt{t}  \FILLsym{:}  \FILLnt{A} \mid \Delta_{{\mathrm{1}}}  \FILLsym{,}  \Delta_{{\mathrm{2}}}\\
      \end{array}
    \end{math}
  \end{center}
\end{definition}

The formulas of FILL are standard, but we denote the unit of tensor as
$\top$ and the unit of par as $\perp$. Patterns are used to
distinguish between the various let-expressions for tensor, par, and
their units.  There are three different let-expressions:
\begin{center}
  \begin{math}
    \begin{array}{lll}
      \begin{array}{lll}
        \text{Tensor:}\\
        \,\,\,\, \mathsf{let}\, \FILLnt{t} \,\mathsf{be}\,  \FILLnt{p_{{\mathrm{1}}}}  \otimes  \FILLnt{p_{{\mathrm{2}}}}  \,\mathsf{in}\, \FILLnt{e} 
      \end{array}
      &
      \begin{array}{lll}
        \text{Par:}\\
        \,\,\,\, \mathsf{let}\, \FILLnt{t} \,\mathsf{be}\,  \FILLnt{p_{{\mathrm{1}}}}  \parr  \FILLnt{p_{{\mathrm{2}}}}  \,\mathsf{in}\, \FILLnt{e} 
      \end{array}
      &
      \begin{array}{lll}
        \text{Tensor Unit:}\\
        \,\,\,\, \mathsf{let}\, \FILLnt{t} \,\mathsf{be}\,  *  \,\mathsf{in}\, \FILLnt{e} 
      \end{array}
    \end{array}
    %% \begin{array}{rll}
    %%   \text{Tensor:} &  \mathsf{let}\, \FILLnt{t} \,\mathsf{be}\,  \FILLnt{p_{{\mathrm{1}}}}  \otimes  \FILLnt{p_{{\mathrm{2}}}}  \,\mathsf{in}\, \FILLnt{e} \\
    %%   \text{Par:} &  \mathsf{let}\, \FILLnt{t} \,\mathsf{be}\,  \FILLnt{p_{{\mathrm{1}}}}  \parr  \FILLnt{p_{{\mathrm{2}}}}  \,\mathsf{in}\, \FILLnt{e} \\
    %%   \text{Tensor Unit:} &  \mathsf{let}\, \FILLnt{t} \,\mathsf{be}\,  *  \,\mathsf{in}\, \FILLnt{e} \\
    %% \end{array}
  \end{math}
\end{center}
In addition, each of these will have their own equational rules, see
Figure~\ref{def:FILL-eq}.  The role each term plays in the overall
logic will become clear after we introduce the inference rules.

At this point we introduce some syntax and meta-level functions that
will be used in the definition of the inference rules for FILL. Left
contexts are multisets of formulas each labeled with a variable, and
right contexts are multisets of formulas each labeled with a term.  We
will often write $ \Delta_{{\mathrm{1}}}  \mid  \Delta_{{\mathrm{2}}} $ as syntactic sugar for $\Delta_{{\mathrm{1}}}  \FILLsym{,}  \Delta_{{\mathrm{2}}}$.
The former should be read as ``$\Delta_{{\mathrm{1}}}$ or $\Delta_{{\mathrm{2}}}$.''  We denote the
usual capture-avoiding substitution by $\FILLsym{[}  \FILLnt{t}  \FILLsym{/}  \FILLmv{x}  \FILLsym{]}  \FILLnt{t'}$, and its
straightforward extension to right contexts as $\FILLsym{[}  \FILLnt{t}  \FILLsym{/}  \FILLmv{x}  \FILLsym{]}  \Delta$.
% \begin{definition}
%   \label{def:delta-sub}
%   We extend the capture-avoiding substitution function to right
%   contexts as follows:
%   \begin{center}
%     \begin{math}
%       \begin{array}{lll}
%         \FILLsym{[}  \FILLnt{t}  \FILLsym{/}  \FILLmv{x}  \FILLsym{]}   \cdot  =  \cdot \\
%         \FILLsym{[}  \FILLnt{t}  \FILLsym{/}  \FILLmv{x}  \FILLsym{]}  \FILLsym{(}  \FILLnt{t'}  \FILLsym{:}  \FILLnt{A}  \FILLsym{)} =  ( \FILLsym{[}  \FILLnt{t}  \FILLsym{/}  \FILLmv{x}  \FILLsym{]}  \FILLnt{t'} )   \FILLsym{:}  \FILLnt{A}\\
%         \FILLsym{[}  \FILLnt{t}  \FILLsym{/}  \FILLmv{x}  \FILLsym{]}  \FILLsym{(}   \Delta_{{\mathrm{1}}}  \mid  \Delta_{{\mathrm{2}}}   \FILLsym{)} =  \FILLsym{(}  \FILLsym{[}  \FILLnt{t}  \FILLsym{/}  \FILLmv{x}  \FILLsym{]}  \Delta_{{\mathrm{1}}}  \FILLsym{)}  \mid  \FILLsym{(}  \FILLsym{[}  \FILLnt{t}  \FILLsym{/}  \FILLmv{x}  \FILLsym{]}  \Delta_{{\mathrm{2}}}  \FILLsym{)} \\
%       \end{array}
%     \end{math}
%   \end{center}
% \end{definition}
Similarly, we find it convenient to be able to do this style of
extension for the let-binding as well.
\begin{definition}
  \label{def:delta-let}
  We extend let-binding terms to right contexts as follows:
  \begin{center}
    \begin{math}
      \begin{array}{lll}
         \mathsf{let}\, \FILLnt{t} \,\mathsf{be}\, \FILLnt{p} \,\mathsf{in}\,  \cdot   =  \cdot \\
         \mathsf{let}\, \FILLnt{t} \,\mathsf{be}\, \FILLnt{p} \,\mathsf{in}\, \FILLsym{(}  \FILLnt{t'}  \FILLsym{:}  \FILLnt{A}  \FILLsym{)}  =  (  \mathsf{let}\, \FILLnt{t} \,\mathsf{be}\, \FILLnt{p} \,\mathsf{in}\, \FILLnt{t'}  )   \FILLsym{:}  \FILLnt{A}\\
         \mathsf{let}\, \FILLnt{t} \,\mathsf{be}\, \FILLnt{p} \,\mathsf{in}\, \FILLsym{(}   \Delta_{{\mathrm{1}}}  \mid  \Delta_{{\mathrm{2}}}   \FILLsym{)}  =  \FILLsym{(}   \mathsf{let}\, \FILLnt{t} \,\mathsf{be}\, \FILLnt{p} \,\mathsf{in}\, \Delta_{{\mathrm{1}}}   \FILLsym{)}  \mid  \FILLsym{(}   \mathsf{let}\, \FILLnt{t} \,\mathsf{be}\, \FILLnt{p} \,\mathsf{in}\, \Delta_{{\mathrm{2}}}   \FILLsym{)} \\
      \end{array}
    \end{math}
  \end{center}
\end{definition}
\noindent
Lastly, we denote the usual function that computes the set of free
variables in a term by $\mathsf{FV}(\FILLnt{t})$, and its straightforward
extension to right contexts as $\mathsf{FV}(\Delta)$.
% \begin{definition}
%   \label{def:delta-FV}
%   We extend the free-variable function on terms to right contexts as
%   follows:
%   \begin{center}
%     \begin{math}
%       \begin{array}{lll}
%          \mathsf{FV}(  \cdot  )  = \emptyset\\
%          \mathsf{FV}( \FILLnt{t}  \FILLsym{:}  \FILLnt{A} )  =  \mathsf{FV}( \FILLnt{t} ) \\
%          \mathsf{FV}(  \Delta_{{\mathrm{1}}}  \mid  \Delta_{{\mathrm{2}}}  )  =  \mathsf{FV}( \Delta_{{\mathrm{1}}} )  \cup  \mathsf{FV}( \Delta_{{\mathrm{2}}} ) \\
%       \end{array}
%     \end{math}
%   \end{center}
% \end{definition}

The inference rules for FILL are defined in Figure~\ref{def:infr}.
\begin{figure}
    \begin{center}
    \scriptsize
      \begin{mathpar}
        \FILLdruleAx{}    \and 
        \FILLdruleCut{}     \and 
        \FILLdruleTl{}            \and 
        \FILLdruleTr{}    \and 
        \FILLdruleTenl{}    \and 
        \FILLdruleTenr{}    \and 
        \FILLdrulePl{}    \and 
        \FILLdrulePr{}    \and 
        \FILLdruleParl{}    \and 
        \FILLdruleParr{}    \and 
        \FILLdruleImpl{}    \and 
        \FILLdruleImpr{}    \and 
        \FILLdruleExl{}    \and 
        \FILLdruleExr{}    
    \end{mathpar}
  \end{center}
  \caption{Inference rules for FILL}
  \label{def:infr}
\end{figure}
The par left $\FILLdrulename{Parl}$ rule depends on the function $ \mathsf{\operatorname{let-pat} }\, \FILLmv{z} \, \FILLnt{p} \, \Delta $ which we define next.
\begin{definition}
  \label{def:let-pat-term}
  The function $ \mathsf{\operatorname{let-pat} }\, \FILLmv{z} \, \FILLnt{p} \, \FILLnt{t} $ is defined as follows:
  \begin{center}
  \footnotesize
    \begin{math}
      \begin{array}{lllllllll}      
        \begin{array}{lll}
           \mathsf{\operatorname{let-pat} }\, \FILLmv{z} \, \FILLsym{(}   \FILLmv{x}  \parr   -    \FILLsym{)} \, \FILLnt{t}  = \FILLnt{t}\\
          \,\,\,\,\,\,\text{where } \FILLmv{x} \not\in \mathsf{FV}(\FILLnt{t})\\
        \end{array}
        & \quad &
          \begin{array}{lll}
             \mathsf{\operatorname{let-pat} }\, \FILLmv{z} \, \FILLsym{(}    -   \parr  \FILLmv{y}   \FILLsym{)} \, \FILLnt{t}  = \FILLnt{t}\\
        \,\,\,\,\,\,\text{where } \FILLmv{y} \not\in \mathsf{FV}(\FILLnt{t})\\
          \end{array}
        & \quad &
          \begin{array}{lll}
             \mathsf{\operatorname{let-pat} }\, \FILLmv{z} \, \FILLnt{p} \, \FILLnt{t}  =  \mathsf{let}\, \FILLmv{z} \,\mathsf{be}\, \FILLnt{p} \,\mathsf{in}\, \FILLnt{t} \\
            & \\
          \end{array}
      \end{array}
    \end{math}
  \end{center}
  It is straightforward to extend the previous definition to
  right-contexts, and we denote this extension by $ \mathsf{\operatorname{let-pat} }\, \FILLmv{z} \, \FILLnt{p} \, \Delta $.
  % \begin{center}
  %   \begin{math}
  %     \begin{array}{lll}      
  %        \mathsf{\operatorname{let-pat} }\, \FILLmv{z} \, \FILLnt{p} \,  \cdot   =  \cdot \\                
  %        \mathsf{\operatorname{let-pat} }\, \FILLmv{z} \, \FILLnt{p} \, \FILLsym{(}  \FILLnt{t}  \FILLsym{:}  \FILLnt{A}  \FILLsym{)}  =  (  \mathsf{\operatorname{let-pat} }\, \FILLmv{z} \, \FILLnt{p} \, \FILLnt{t}  )   \FILLsym{:}  \FILLnt{A}\\
  %        \mathsf{\operatorname{let-pat} }\, \FILLmv{z} \, \FILLnt{p} \, \FILLsym{(}   \Delta_{{\mathrm{1}}}  \mid  \Delta_{{\mathrm{2}}}   \FILLsym{)}  =  \FILLsym{(}   \mathsf{\operatorname{let-pat} }\, \FILLmv{z} \, \FILLnt{p} \, \Delta_{{\mathrm{1}}}   \FILLsym{)}  \mid  \FILLsym{(}   \mathsf{\operatorname{let-pat} }\, \FILLmv{z} \, \FILLnt{p} \, \Delta_{{\mathrm{2}}}   \FILLsym{)} \\
  %     \end{array}
  %   \end{math}
  % \end{center}
\end{definition}
The motivation behind this function is that it only binds the pattern
variables in $ \FILLmv{x}  \parr   -  $ and $  -   \parr  \FILLmv{y} $ if and only if those
pattern variables are free in the body of the let.  This overcomes the
counterexample given by Bierman in \cite{Bierman:1996}.  

The terms of FILL are equipped with an equivalence relation defined in
Figure~\ref{def:FILL-eq}.
\begin{figure}[t]
  \begin{center}
    \footnotesize
    \begin{mathpar}
      \FILLdruleAlpha{} \and
      \FILLdruleEtaFun{} \and
      \FILLdruleBetaFun{} \and
      \FILLdruleEtaOneI{} \and
\begin{report}
  \FILLdruleEtaTwoI{} \and
\end{report}
      \FILLdruleBetaI{} \and
      \FILLdruleNatI{} \and
      \begin{report}
        \FILLdruleEtaTen{} \and
      \end{report}
      \FILLdruleBetaOneTen{} \and
      \FILLdruleBetaTwoTen{} \and
      \FILLdruleNatTen{} \and
      \FILLdruleEtaParU{} \and
      \FILLdruleEtaPar{} \and
      \FILLdruleBetaOnePar{} \and
      \FILLdruleBetaTwoPar{} \and
      \FILLdruleNatOnePar{} \and
      \FILLdruleNatTwoPar{} 
      \begin{report}
        \and
      \FILLdruleLam{} \and
      \FILLdruleAppOne{} \and
      \FILLdruleAppTwo{} \and
      \FILLdruleTenOne{} \and
      \FILLdruleTenTwo{} \and
      \FILLdruleParOne{} \and
      \FILLdruleParTwo{} \and
      \FILLdruleLetOne{} \and
      \FILLdruleLetTwo{} \and
      \FILLdruleRefl{} \and
      \FILLdruleSym{} \and
      \FILLdruleTrans{}
      \end{report}
    \end{mathpar}
  \end{center}
  \caption{Equivalence on terms}
  \label{def:FILL-eq}
\end{figure}
There are a number of $\alpha$, $\beta$, and $\eta$ like rules as well
as several rules we call naturality rules.  These rules are similar to
the rules presented in \cite{Hyland:1993}.
% section full_intuitionistic_linear_logic_(fill) (end)

\section{Cut-elimination}
\label{sec:cut-elimination}

The system FILL can be viewed from two different angles: i. as an intuitionistic
linear logic with par, or ii. as a restricted form of classical linear
logic.  Thus, to prove cut-elimination of FILL one only needs to start
with the cut-elimination procedure for intuitionistic linear logic,
and then dualize all of the steps in the procedure for tensor and its
unit to obtain the steps for par and its unit.  Similarly, one could
just as easily start with the cut-elimination procedure for classical
linear logic, and then apply the restriction on the implication right
rule producing a cut-elimination procedure for FILL.

The major difference between proving cut-elimination of FILL from
classical or intuitionistic linear logic is that we must prove an
invariant across each step in the procedure.  The invariant is that if
a derivation $\pi$ is transformed into a derivation $\pi'$, then the
terms in the conclusion of the final rule applied in $\pi$ must be be
transformable, when the equivalences defined in
Figure~\ref{def:FILL-eq} are taken as left-to-right rewrite rules,
into the terms in the conclusion of the final rule applied in $\pi'$.

\begin{report}
  The cut elimination procedure requires the following two basic
results:
\begin{lemma}[Substitution Distribution]
  \label{lemma:substitution_distribution}
  For any terms $\FILLnt{t}$, $\FILLnt{t_{{\mathrm{1}}}}$, and $\FILLnt{t_{{\mathrm{2}}}}$, $\FILLsym{[}  \FILLnt{t_{{\mathrm{1}}}}  \FILLsym{/}  \FILLmv{x}  \FILLsym{]}  \FILLsym{[}  \FILLnt{t_{{\mathrm{2}}}}  \FILLsym{/}  \FILLmv{y}  \FILLsym{]}  \FILLnt{t} = \FILLsym{[}  \FILLsym{[}  \FILLnt{t_{{\mathrm{1}}}}  \FILLsym{/}  \FILLmv{x}  \FILLsym{]}  \FILLnt{t_{{\mathrm{2}}}}  \FILLsym{/}  \FILLmv{y}  \FILLsym{]}  \FILLsym{[}  \FILLnt{t_{{\mathrm{2}}}}  \FILLsym{/}  \FILLmv{x}  \FILLsym{]}  \FILLnt{t}$.
\end{lemma}
\begin{proof}
  This proof holds by straightforward induction on the form of $t$.
\end{proof}

\begin{lemma}[Let-pat Distribution]
  \label{lemma:let-pat_distribution}
  For any terms $\FILLnt{t}$, $\FILLnt{t_{{\mathrm{1}}}}$, and $\FILLnt{t_{{\mathrm{2}}}}$, and pattern p, \\
  $ \mathsf{\operatorname{let-pat} }\, \FILLnt{t} \, \FILLnt{p} \, \FILLsym{[}  \FILLnt{t_{{\mathrm{1}}}}  \FILLsym{/}  \FILLmv{y}  \FILLsym{]}  \FILLnt{t_{{\mathrm{2}}}}  = \FILLsym{[}   \mathsf{\operatorname{let-pat} }\, \FILLnt{t} \, \FILLnt{p} \, \FILLnt{t_{{\mathrm{1}}}}   \FILLsym{/}  \FILLmv{y}  \FILLsym{]}  \FILLnt{t_{{\mathrm{2}}}}$.
\end{lemma}
\begin{proof}
  This proof holds by case splitting over $p$, and then using the
  naturality equations for the respective pattern.
\end{proof}
\end{report}

We finally arrive at cut-elimination.
\begin{theorem}
  \label{thm:cut-pro}
  If $ \Gamma  \vdash  \FILLnt{t_{{\mathrm{1}}}}  \FILLsym{:}  \FILLnt{A_{{\mathrm{1}}}}  \FILLsym{,} \, ... \, \FILLsym{,}  \FILLnt{t_{\FILLmv{i}}}  \FILLsym{:}  \FILLnt{A_{\FILLmv{i}}} $ steps to $ \Gamma  \vdash  \FILLnt{t'_{{\mathrm{1}}}}  \FILLsym{:}  \FILLnt{A_{{\mathrm{1}}}}  \FILLsym{,} \, ... \, \FILLsym{,}  \FILLnt{t'_{\FILLmv{i}}}  \FILLsym{:}  \FILLnt{A_{\FILLmv{i}}} $ using the cut-elimination procedure, then $\FILLnt{t_{\FILLmv{j}}}  \FILLsym{=}  \FILLnt{t'_{\FILLmv{j}}}$
  for $1 \leq j \leq i$.
\end{theorem}
\begin{proof}
  The cut-elimination procedure given here is the standard\\
  %are you sure you want this \\ here? looks odd
  cut-elimination
  procedure for classical linear logic except that the
  cases involving the implication right rule have the FILL
  restriction. The structure of our procedure follows the structure of
  the procedure found in \cite{Mellies:2009}. Throughout this proof we
  treat the equivalences defined in Figure~\ref{def:FILL-eq} as
  left-to-right rewrite rules. \begin{paper} For the entire proof see
    the companion report \cite{Eades:2015}.\end{paper}
  \begin{report}        
  \begin{itemize}
  \item[Case:] commuting conversion cut vs cut (first case).
    The following proof
\begin{center}
  \begin{math}
    $$\mprset{flushleft}
    \inferrule* [right=Cut] {
      $$\mprset{flushleft}
      \inferrule* [right=] {
        \inferrule* [right=,vdots=1.5em,fraction=\,] {
          \,
        }{\pi_1}
      }{ \Gamma  \vdash   \FILLnt{t}  \FILLsym{:}  \FILLnt{A}  \mid  \Delta  }
      \\
      $$\mprset{flushleft}
      \inferrule* [right=Cut] {
        $$\mprset{flushleft}
        \inferrule* [right=] {
          \inferrule* [right=,vdots=1.5em,fraction=\,] {
            \,
          }{\pi_2}
        }{ \Gamma_{{\mathrm{2}}}  \FILLsym{,}  \FILLmv{x}  \FILLsym{:}  \FILLnt{A}  \FILLsym{,}  \Gamma_{{\mathrm{3}}}  \vdash   \FILLnt{t_{{\mathrm{1}}}}  \FILLsym{:}  \FILLnt{B}  \mid  \Delta_{{\mathrm{1}}}  }
        \\
        $$\mprset{flushleft}
        \inferrule* [right=] {
          \inferrule* [right=,vdots=1.5em,fraction=\,] {
            \,
          }{\pi_3}
        }{ \Gamma_{{\mathrm{1}}}  \FILLsym{,}  \FILLmv{y}  \FILLsym{:}  \FILLnt{B}  \FILLsym{,}  \Gamma_{{\mathrm{4}}}  \vdash  \Delta_{{\mathrm{2}}} }
      }{ \Gamma_{{\mathrm{1}}}  \FILLsym{,}  \Gamma_{{\mathrm{2}}}  \FILLsym{,}  \FILLmv{x}  \FILLsym{:}  \FILLnt{A}  \FILLsym{,}  \Gamma_{{\mathrm{3}}}  \FILLsym{,}  \Gamma_{{\mathrm{4}}}  \vdash    \Delta_{{\mathrm{1}}}  \mid  \FILLsym{[}  \FILLnt{t_{{\mathrm{1}}}}  \FILLsym{/}  \FILLmv{y}  \FILLsym{]}  \Delta_{{\mathrm{2}}}   }
    }{ \Gamma_{{\mathrm{1}}}  \FILLsym{,}  \Gamma_{{\mathrm{2}}}  \FILLsym{,}  \Gamma  \FILLsym{,}  \Gamma_{{\mathrm{3}}}  \FILLsym{,}  \Gamma_{{\mathrm{4}}}  \vdash     \Delta  \mid  \FILLsym{[}  \FILLnt{t}  \FILLsym{/}  \FILLmv{x}  \FILLsym{]}  \Delta_{{\mathrm{1}}}    \mid  \FILLsym{[}  \FILLnt{t}  \FILLsym{/}  \FILLmv{x}  \FILLsym{]}   \FILLsym{[}  \FILLnt{t_{{\mathrm{1}}}}  \FILLsym{/}  \FILLmv{y}  \FILLsym{]}  \Delta_{{\mathrm{2}}}   }
  \end{math}
\end{center}
is transformed into the proof
\begin{center}
  \begin{math}
    $$\mprset{flushleft}
    \inferrule* [right=Cut] {
      $$\mprset{flushleft}
      \inferrule* [right=] {
        $$\mprset{flushleft}
      \inferrule* [right=] {
        \inferrule* [right=,vdots=1.5em,fraction=\,] {
          \,
        }{\pi_1}               
      }{ \Gamma  \vdash   \FILLnt{t}  \FILLsym{:}  \FILLnt{A}  \mid  \Delta  }
      \\
      $$\mprset{flushleft}
        \inferrule* [right=] {
          \inferrule* [right=,vdots=1.5em,fraction=\,] {
            \,
          }{\pi_2}
        }{ \Gamma_{{\mathrm{2}}}  \FILLsym{,}  \FILLmv{x}  \FILLsym{:}  \FILLnt{A}  \FILLsym{,}  \Gamma_{{\mathrm{3}}}  \vdash   \FILLnt{t_{{\mathrm{1}}}}  \FILLsym{:}  \FILLnt{B}  \mid  \Delta_{{\mathrm{1}}}  }
      }{ \Gamma_{{\mathrm{2}}}  \FILLsym{,}  \Gamma  \FILLsym{,}  \Gamma_{{\mathrm{3}}}  \vdash    \FILLsym{[}  \FILLnt{t}  \FILLsym{/}  \FILLmv{x}  \FILLsym{]}  \FILLnt{t_{{\mathrm{1}}}}   \FILLsym{:}  \FILLnt{B}  \mid  \FILLsym{[}  \FILLnt{t}  \FILLsym{/}  \FILLmv{x}  \FILLsym{]}  \Delta_{{\mathrm{1}}}  }
      \\
      $$\mprset{flushleft}
        \inferrule* [right=] {
          \inferrule* [right=,vdots=1.5em,fraction=\,] {
            \,
          }{\pi_3}
        }{ \Gamma_{{\mathrm{1}}}  \FILLsym{,}  \FILLmv{y}  \FILLsym{:}  \FILLnt{B}  \FILLsym{,}  \Gamma_{{\mathrm{4}}}  \vdash  \Delta_{{\mathrm{2}}} }
    }{ \Gamma_{{\mathrm{1}}}  \FILLsym{,}  \Gamma_{{\mathrm{2}}}  \FILLsym{,}  \Gamma  \FILLsym{,}  \Gamma_{{\mathrm{3}}}  \FILLsym{,}  \Gamma_{{\mathrm{4}}}  \vdash     \Delta  \mid  \FILLsym{[}  \FILLnt{t}  \FILLsym{/}  \FILLmv{x}  \FILLsym{]}  \Delta_{{\mathrm{1}}}    \mid  \FILLsym{[}  \FILLsym{[}  \FILLnt{t}  \FILLsym{/}  \FILLmv{x}  \FILLsym{]}  \FILLnt{t_{{\mathrm{1}}}}  \FILLsym{/}  \FILLmv{y}  \FILLsym{]}  \Delta_{{\mathrm{2}}}  }
  \end{math}
\end{center}
First, if $\Delta_{{\mathrm{2}}}$ is empty, then all the terms in the conclusion of
the previous two derivations are equivalent.  
So suppose $\Delta_{{\mathrm{2}}} =  \FILLnt{t_{{\mathrm{2}}}}  \FILLsym{:}  \FILLnt{C}  \mid  \Delta'_{{\mathrm{2}}} $.  Then we know that the term
$\FILLsym{[}  \FILLnt{t}  \FILLsym{/}  \FILLmv{x}  \FILLsym{]}  \FILLsym{[}  \FILLnt{t_{{\mathrm{1}}}}  \FILLsym{/}  \FILLmv{y}  \FILLsym{]}  \FILLnt{t_{{\mathrm{2}}}}$ in the first derivation above is equivalent to
$\FILLsym{[}  \FILLsym{[}  \FILLnt{t}  \FILLsym{/}  \FILLmv{x}  \FILLsym{]}  \FILLnt{t_{{\mathrm{1}}}}  \FILLsym{/}  \FILLmv{y}  \FILLsym{]}  \FILLsym{[}  \FILLnt{t}  \FILLsym{/}  \FILLmv{x}  \FILLsym{]}  \FILLnt{t_{{\mathrm{2}}}}$ by
Lemma~\ref{lemma:substitution_distribution}.  Furthermore, by
inspecting the first derivation we can see that $ \FILLmv{x}  \not\in \mathsf{FV}(  \FILLnt{t_{{\mathrm{2}}}}  ) $,
and thus, $\FILLsym{[}  \FILLsym{[}  \FILLnt{t}  \FILLsym{/}  \FILLmv{x}  \FILLsym{]}  \FILLnt{t_{{\mathrm{1}}}}  \FILLsym{/}  \FILLmv{y}  \FILLsym{]}  \FILLsym{[}  \FILLnt{t}  \FILLsym{/}  \FILLmv{x}  \FILLsym{]}  \FILLnt{t_{{\mathrm{2}}}}  \FILLsym{=}  \FILLsym{[}  \FILLsym{[}  \FILLnt{t}  \FILLsym{/}  \FILLmv{x}  \FILLsym{]}  \FILLnt{t_{{\mathrm{1}}}}  \FILLsym{/}  \FILLmv{y}  \FILLsym{]}  \FILLnt{t_{{\mathrm{2}}}}$.  This
argument may be repeated for any term in $\Delta'_{{\mathrm{2}}}$, and thus, we know
$\FILLsym{[}  \FILLnt{t}  \FILLsym{/}  \FILLmv{x}  \FILLsym{]}  \FILLsym{[}  \FILLnt{t_{{\mathrm{1}}}}  \FILLsym{/}  \FILLmv{y}  \FILLsym{]}  \Delta_{{\mathrm{2}}}  \FILLsym{=}  \FILLsym{[}  \FILLsym{[}  \FILLnt{t}  \FILLsym{/}  \FILLmv{x}  \FILLsym{]}  \FILLnt{t_{{\mathrm{1}}}}  \FILLsym{/}  \FILLmv{y}  \FILLsym{]}  \Delta_{{\mathrm{2}}}$.

\item[Case:] commuting conversion cut vs. cut (second case).  The second commuting conversion on cut begins with the proof
\begin{center}
  \begin{math}
    $$\mprset{flushleft}
    \inferrule* [right=Cut] {
      $$\mprset{flushleft}
      \inferrule* [right=] {
        \inferrule* [right=,vdots=1.5em,fraction=\,] {
          \,
        }{\pi_1}
      }{ \Gamma  \vdash   \FILLnt{t}  \FILLsym{:}  \FILLnt{A}  \mid  \Delta  }
      \\
      $$\mprset{flushleft}
      \inferrule* [right=Cut] {
        $$\mprset{flushleft}
        \inferrule* [right=] {
          \inferrule* [right=,vdots=1.5em,fraction=\,] {
            \,
          }{\pi_2}
        }{ \Gamma'  \vdash   \FILLnt{t'}  \FILLsym{:}  \FILLnt{B}  \mid  \Delta'  }
        \\
        $$\mprset{flushleft}
        \inferrule* [right=] {
          \inferrule* [right=,vdots=1.5em,fraction=\,] {
            \,
          }{\pi_3}
        }{ \Gamma_{{\mathrm{1}}}  \FILLsym{,}  \FILLmv{x}  \FILLsym{:}  \FILLnt{A}  \FILLsym{,}  \Gamma_{{\mathrm{2}}}  \FILLsym{,}  \FILLmv{y}  \FILLsym{:}  \FILLnt{B}  \FILLsym{,}  \Gamma_{{\mathrm{3}}}  \vdash  \Delta_{{\mathrm{1}}} }
      }{ \Gamma_{{\mathrm{1}}}  \FILLsym{,}  \FILLmv{x}  \FILLsym{:}  \FILLnt{A}  \FILLsym{,}  \Gamma_{{\mathrm{2}}}  \FILLsym{,}  \Gamma'  \FILLsym{,}  \Gamma_{{\mathrm{3}}}  \vdash   \Delta'  \mid  \FILLsym{[}  \FILLnt{t'}  \FILLsym{/}  \FILLmv{y}  \FILLsym{]}  \Delta_{{\mathrm{1}}}  }
    }{ \Gamma_{{\mathrm{1}}}  \FILLsym{,}  \Gamma  \FILLsym{,}  \Gamma_{{\mathrm{2}}}  \FILLsym{,}  \Gamma'  \FILLsym{,}  \Gamma_{{\mathrm{3}}}  \vdash     \Delta  \mid  \FILLsym{[}  \FILLnt{t}  \FILLsym{/}  \FILLmv{x}  \FILLsym{]}  \Delta'    \mid  \FILLsym{[}  \FILLnt{t}  \FILLsym{/}  \FILLmv{x}  \FILLsym{]}   \FILLsym{[}  \FILLnt{t'}  \FILLsym{/}  \FILLmv{y}  \FILLsym{]}  \Delta_{{\mathrm{1}}}   }
  \end{math}
\end{center}
is transformed into the following proof:
\begin{center}
  \begin{math}
    $$\mprset{flushleft}
    \inferrule* [right=\text{\scriptsize Series of Exchanges}] {
      $$\mprset{flushleft}
    \inferrule* [right=Cut] {
      $$\mprset{flushleft}
      \inferrule* [right=] {
        \inferrule* [right=,vdots=1.5em,fraction=\,] {
          \,
        }{\pi_2}
      }{ \Gamma'  \vdash   \FILLnt{t'}  \FILLsym{:}  \FILLnt{B}  \mid  \Delta'  }
      \\
      $$\mprset{flushleft}
      \inferrule* [right=Cut] {
        $$\mprset{flushleft}
        \inferrule* [right=] {
          \inferrule* [right=,vdots=1.5em,fraction=\,] {
            \,
          }{\pi_1}
        }{ \Gamma  \vdash   \FILLnt{t}  \FILLsym{:}  \FILLnt{A}  \mid  \Delta  }      
        \\
        $$\mprset{flushleft}
        \inferrule* [right=] {
          \inferrule* [right=,vdots=1.5em,fraction=\,] {
            \,
          }{\pi_3}
        }{ \Gamma_{{\mathrm{1}}}  \FILLsym{,}  \FILLmv{x}  \FILLsym{:}  \FILLnt{A}  \FILLsym{,}  \Gamma_{{\mathrm{2}}}  \FILLsym{,}  \FILLmv{y}  \FILLsym{:}  \FILLnt{B}  \FILLsym{,}  \Gamma_{{\mathrm{3}}}  \vdash  \Delta_{{\mathrm{1}}} }
      }{ \Gamma_{{\mathrm{1}}}  \FILLsym{,}  \Gamma  \FILLsym{,}  \Gamma_{{\mathrm{2}}}  \FILLsym{,}  \FILLmv{y}  \FILLsym{:}  \FILLnt{B}  \FILLsym{,}  \Gamma_{{\mathrm{3}}}  \vdash   \Delta  \mid  \FILLsym{[}  \FILLnt{t}  \FILLsym{/}  \FILLmv{x}  \FILLsym{]}  \Delta_{{\mathrm{1}}}  }
    }{ \Gamma_{{\mathrm{1}}}  \FILLsym{,}  \Gamma  \FILLsym{,}  \Gamma_{{\mathrm{2}}}  \FILLsym{,}  \Gamma'  \FILLsym{,}  \Gamma_{{\mathrm{3}}}  \vdash     \Delta'  \mid  \FILLsym{[}  \FILLnt{t'}  \FILLsym{/}  \FILLmv{y}  \FILLsym{]}  \Delta    \mid  \FILLsym{[}  \FILLnt{t'}  \FILLsym{/}  \FILLmv{y}  \FILLsym{]}   \FILLsym{[}  \FILLnt{t}  \FILLsym{/}  \FILLmv{x}  \FILLsym{]}  \Delta_{{\mathrm{1}}}   }
    }{ \Gamma_{{\mathrm{1}}}  \FILLsym{,}  \Gamma  \FILLsym{,}  \Gamma_{{\mathrm{2}}}  \FILLsym{,}  \Gamma'  \FILLsym{,}  \Gamma_{{\mathrm{3}}}  \vdash      \FILLsym{[}  \FILLnt{t'}  \FILLsym{/}  \FILLmv{y}  \FILLsym{]}  \Delta   \mid  \Delta'    \mid  \FILLsym{[}  \FILLnt{t'}  \FILLsym{/}  \FILLmv{y}  \FILLsym{]}   \FILLsym{[}  \FILLnt{t}  \FILLsym{/}  \FILLmv{x}  \FILLsym{]}  \Delta_{{\mathrm{1}}}   }
  \end{math}
\end{center}
We know $ \FILLmv{x} , \FILLmv{y}  \not\in \mathsf{FV}(  \Delta  ) $ by inspection of the first derivation, and so we know that
$\Delta  \FILLsym{=}  \FILLsym{[}  \FILLnt{t'}  \FILLsym{/}  \FILLmv{y}  \FILLsym{]}  \Delta$ and $\Delta'  \FILLsym{=}  \FILLsym{[}  \FILLnt{t}  \FILLsym{/}  \FILLmv{x}  \FILLsym{]}  \Delta'$.  Without loss of generality suppose $\Delta_{{\mathrm{1}}} =  \FILLnt{t_{{\mathrm{1}}}}  \FILLsym{:}  \FILLnt{C}  \mid  \Delta'_{{\mathrm{1}}} $. Then we know that $ \FILLmv{x} , \FILLmv{y}  \not\in \mathsf{FV}(  \FILLnt{t}  ) $ and $ \FILLmv{x} , \FILLmv{y}  \not\in \mathsf{FV}(  \FILLnt{t'}  ) $.  Thus, by this fact and
Lemma~\ref{lemma:substitution_distribution}, we know that $\FILLsym{[}  \FILLnt{t}  \FILLsym{/}  \FILLmv{x}  \FILLsym{]}  \FILLsym{[}  \FILLnt{t'}  \FILLsym{/}  \FILLmv{y}  \FILLsym{]}  \FILLnt{t_{{\mathrm{1}}}}  \FILLsym{=}  \FILLsym{[}  \FILLsym{[}  \FILLnt{t}  \FILLsym{/}  \FILLmv{x}  \FILLsym{]}  \FILLnt{t'}  \FILLsym{/}  \FILLmv{y}  \FILLsym{]}  \FILLsym{[}  \FILLnt{t}  \FILLsym{/}  \FILLmv{x}  \FILLsym{]}  \FILLnt{t_{{\mathrm{1}}}} = \FILLsym{[}  \FILLnt{t'}  \FILLsym{/}  \FILLmv{y}  \FILLsym{]}  \FILLsym{[}  \FILLnt{t}  \FILLsym{/}  \FILLmv{x}  \FILLsym{]}  \FILLnt{t_{{\mathrm{1}}}}$.  This argument can be repeated for any term in
$\Delta'_{{\mathrm{1}}}$, hence, $\FILLsym{[}  \FILLnt{t}  \FILLsym{/}  \FILLmv{x}  \FILLsym{]}  \FILLsym{[}  \FILLnt{t'}  \FILLsym{/}  \FILLmv{y}  \FILLsym{]}  \Delta_{{\mathrm{1}}}  \FILLsym{=}  \FILLsym{[}  \FILLnt{t'}  \FILLsym{/}  \FILLmv{y}  \FILLsym{]}  \FILLsym{[}  \FILLnt{t}  \FILLsym{/}  \FILLmv{x}  \FILLsym{]}  \Delta_{{\mathrm{1}}}$.

\item[Case:] the $\eta$-expansion cases: tensor.  
The proof
\begin{center}
  \begin{math}
    $$\mprset{flushleft}
    \inferrule* [right=Ax] {
      \,
    }{ \FILLmv{x}  \FILLsym{:}   \FILLnt{A}  \otimes  \FILLnt{B}   \vdash  \FILLmv{x}  \FILLsym{:}   \FILLnt{A}  \otimes  \FILLnt{B}  }
  \end{math}
\end{center}
is transformed into the proof
\begin{center}
  \begin{math}
    $$\mprset{flushleft}
    \inferrule* [right=Tl] {
      $$\mprset{flushleft}
      \inferrule* [right=Tr] {
        $$\mprset{flushleft}
        \inferrule* [right=Ax] {
          \,
        }{ \FILLmv{y}  \FILLsym{:}  \FILLnt{A}  \vdash  \FILLmv{y}  \FILLsym{:}  \FILLnt{A} }
        \\
        $$\mprset{flushleft}
        \inferrule* [right=Ax] {
          \,
        }{ \FILLmv{z}  \FILLsym{:}  \FILLnt{B}  \vdash  \FILLmv{z}  \FILLsym{:}  \FILLnt{B} }
      }{ \FILLmv{y}  \FILLsym{:}  \FILLnt{A}  \FILLsym{,}  \FILLmv{z}  \FILLsym{:}  \FILLnt{B}  \vdash   \FILLmv{y}  \otimes  \FILLmv{z}   \FILLsym{:}   \FILLnt{A}  \otimes  \FILLnt{B}  }
    }{ \FILLmv{x}  \FILLsym{:}   \FILLnt{A}  \otimes  \FILLnt{B}   \vdash    \mathsf{let}\, \FILLmv{x} \,\mathsf{be}\,  \FILLmv{y}  \otimes  \FILLmv{z}  \,\mathsf{in}\,  (  \FILLmv{y}  \otimes  \FILLmv{z}  )     \FILLsym{:}   \FILLnt{A}  \otimes  \FILLnt{B}  }
  \end{math}
\end{center}
By the rule $\FILLdrulename{EQ\_EtaTensor}$
we know $ \mathsf{let}\, \FILLmv{x} \,\mathsf{be}\,  \FILLmv{y}  \otimes  \FILLmv{z}  \,\mathsf{in}\,  (  \FILLmv{y}  \otimes  \FILLmv{z}  )    \FILLsym{=}  \FILLmv{x}$.

\item[Case:] the $\eta$-expansion cases: par.  
The proof
\begin{center}
  \begin{math}
    $$\mprset{flushleft}
    \inferrule* [right=Ax] {
      \,
    }{ \FILLmv{x}  \FILLsym{:}   \FILLnt{A}  \parr  \FILLnt{B}   \vdash  \FILLmv{x}  \FILLsym{:}   \FILLnt{A}  \parr  \FILLnt{B}  }
  \end{math}
\end{center}
is transformed into the proof
\begin{center}
  \begin{math}
    $$\mprset{flushleft}
    \inferrule* [right=Parr] {
      $$\mprset{flushleft}
      \inferrule* [right=Parl] {
        $$\mprset{flushleft}
        \inferrule* [right=Ax] {
          \,
        }{ \FILLmv{y}  \FILLsym{:}  \FILLnt{A}  \vdash  \FILLmv{y}  \FILLsym{:}  \FILLnt{A} }
        \\
        $$\mprset{flushleft}
        \inferrule* [right=Ax] {
          \,
        }{ \FILLmv{z}  \FILLsym{:}  \FILLnt{B}  \vdash  \FILLmv{z}  \FILLsym{:}  \FILLnt{B} }
      }{ \FILLmv{x}  \FILLsym{:}   \FILLnt{A}  \parr  \FILLnt{B}   \vdash     \mathsf{let}\, \FILLmv{x} \,\mathsf{be}\, \FILLsym{(}   \FILLmv{y}  \parr   -    \FILLsym{)} \,\mathsf{in}\, \FILLmv{y}    \FILLsym{:}  \FILLnt{A}  \mid    \mathsf{let}\, \FILLmv{x} \,\mathsf{be}\, \FILLsym{(}    -   \parr  \FILLmv{z}   \FILLsym{)} \,\mathsf{in}\, \FILLmv{z}    \FILLsym{:}  \FILLnt{B}  }
    }{ \FILLmv{x}  \FILLsym{:}   \FILLnt{A}  \parr  \FILLnt{B}   \vdash    (  \mathsf{let}\, \FILLmv{x} \,\mathsf{be}\, \FILLsym{(}   \FILLmv{y}  \parr   -    \FILLsym{)} \,\mathsf{in}\, \FILLmv{y}  )   \parr   (  \mathsf{let}\, \FILLmv{x} \,\mathsf{be}\, \FILLsym{(}    -   \parr  \FILLmv{z}   \FILLsym{)} \,\mathsf{in}\, \FILLmv{z}  )    \FILLsym{:}   \FILLnt{A}  \parr  \FILLnt{B}  }
  \end{math}
\end{center}
By rule $\FILLdrulename{Eq\_EtaPar}$ we know
$ (   (  \mathsf{let}\, \FILLmv{x} \,\mathsf{be}\, \FILLsym{(}   \FILLmv{y}  \parr   -    \FILLsym{)} \,\mathsf{in}\, \FILLmv{y}  )   \parr   (  \mathsf{let}\, \FILLmv{x} \,\mathsf{be}\, \FILLsym{(}    -   \parr  \FILLmv{z}   \FILLsym{)} \,\mathsf{in}\, \FILLmv{z}  )   )   \FILLsym{=}  \FILLmv{x}$.

\item[Case:] the $\eta$-expansion cases: implication.
The proof
\begin{center}
  \begin{math}
    $$\mprset{flushleft}
    \inferrule* [right=Ax] {
      \,
    }{ \FILLmv{x}  \FILLsym{:}   \FILLnt{A}  \multimap   \FILLnt{B}   \vdash  \FILLmv{x}  \FILLsym{:}   \FILLnt{A}  \multimap   \FILLnt{B}  }
  \end{math}
\end{center}
transforms into the proof 
\begin{center}
  \begin{math}
    $$\mprset{flushleft}
    \inferrule* [right=ImpR] {
      $$\mprset{flushleft}
      \inferrule* [right=ImpL] {
        $$\mprset{flushleft}
        \inferrule* [right=Ax] {
          \,
        }{ \FILLmv{y}  \FILLsym{:}  \FILLnt{A}  \vdash  \FILLmv{y}  \FILLsym{:}  \FILLnt{A} }
        \\
        $$\mprset{flushleft}
        \inferrule* [right=Ax] {
          \,
        }{ \FILLmv{z}  \FILLsym{:}  \FILLnt{B}  \vdash  \FILLmv{z}  \FILLsym{:}  \FILLnt{B} }
      }{ \FILLmv{y}  \FILLsym{:}  \FILLnt{A}  \FILLsym{,}  \FILLmv{x}  \FILLsym{:}   \FILLnt{A}  \multimap   \FILLnt{B}   \vdash  \FILLmv{x} \, \FILLmv{y}  \FILLsym{:}  \FILLnt{B} }
    }{ \FILLmv{x}  \FILLsym{:}   \FILLnt{A}  \multimap   \FILLnt{B}   \vdash   \lambda  \FILLmv{y}  .   \FILLmv{x} \, \FILLmv{y}    \FILLsym{:}   \FILLnt{A}  \multimap   \FILLnt{B}  }
  \end{math}  
\end{center}
All terms in the two derivations are equivalent, because
$ (  \lambda  \FILLmv{y}  .   \FILLmv{x} \, \FILLmv{y}   )   \FILLsym{=}  \FILLmv{x}$ by the $\FILLdrulename{Eq\_EtaFun}$ rule.

\item[Case:] the $\eta$-expansion cases: tensor unit.
The proof
\begin{center}
  \begin{math}
    $$\mprset{flushleft}
    \inferrule* [right=Ax] {
      \,
    }{ \FILLmv{x}  \FILLsym{:}   \top   \vdash  \FILLmv{x}  \FILLsym{:}   \top  }
  \end{math}
\end{center}
transforms into the proof
\begin{center}
  \begin{math}
    $$\mprset{flushleft}
    \inferrule* [right=Il] {
      $$\mprset{flushleft}
      \inferrule* [right=Ir] {
        \,
      }{  \cdot   \vdash  \FILLsym{*}  \FILLsym{:}   \top  }
    }{ \FILLmv{x}  \FILLsym{:}   \top   \vdash    \mathsf{let}\, \FILLmv{x} \,\mathsf{be}\,  *  \,\mathsf{in}\, \FILLsym{*}    \FILLsym{:}   \top  }
  \end{math}
\end{center}
We know $\FILLmv{x}  \FILLsym{=}   \mathsf{let}\, \FILLmv{x} \,\mathsf{be}\,  *  \,\mathsf{in}\, \FILLsym{*} $ by
$\FILLdrulename{Eq\_EtaI}$.  

\item[Case:] the $\eta$-expansion cases: par unit.
  The proof
\begin{center}
  \begin{math}
    $$\mprset{flushleft}
    \inferrule* [right=Ax] {
      \,
    }{ \FILLmv{x}  \FILLsym{:}   \perp   \vdash  \FILLmv{x}  \FILLsym{:}   \perp  }
  \end{math}
\end{center}
transforms into the proof
\begin{center}
  \begin{math}
    $$\mprset{flushleft}
    \inferrule* [right=Pr] {
      $$\mprset{flushleft}
      \inferrule* [right=Pl] {
        \,
      }{ \FILLmv{x}  \FILLsym{:}   \perp   \vdash   \cdot  }
    }{ \FILLmv{x}  \FILLsym{:}   \perp   \vdash   \circ   \FILLsym{:}   \perp  }
  \end{math}
\end{center}
We know $\FILLmv{x}  \FILLsym{=}   \circ $ by $\FILLdrulename{Eq\_EtaParU}$.

\item[Case:] the axiom steps: the axiom step.
The proof 
\begin{center}
  \begin{math}
    $$\mprset{flushleft}
    \inferrule* [right=Cut] {
      $$\mprset{flushleft}
      \inferrule* [right=Ax] {
        \,
      }{ \FILLmv{x}  \FILLsym{:}  \FILLnt{A}  \vdash  \FILLmv{x}  \FILLsym{:}  \FILLnt{A} }
      \\
      $$\mprset{flushleft}
      \inferrule* [right=] {
        \inferrule* [right=,vdots=1.5em,fraction=\,] {
            \,
          }{\pi}          
      }{ \Gamma_{{\mathrm{1}}}  \FILLsym{,}  \FILLmv{y}  \FILLsym{:}  \FILLnt{A}  \FILLsym{,}  \Gamma_{{\mathrm{2}}}  \vdash  \Delta }
    }{ \Gamma_{{\mathrm{1}}}  \FILLsym{,}  \FILLmv{x}  \FILLsym{:}  \FILLnt{A}  \FILLsym{,}  \Gamma_{{\mathrm{2}}}  \vdash  \FILLsym{[}  \FILLmv{x}  \FILLsym{/}  \FILLmv{y}  \FILLsym{]}  \Delta }
  \end{math}
\end{center}
transforms into the proof
\begin{center}
  \begin{math}
    $$\mprset{flushleft}
      \inferrule* [right=] {
        \inferrule* [right=,vdots=1.5em,fraction=\,] {
            \,
          }{\pi}          
      }{ \Gamma_{{\mathrm{1}}}  \FILLsym{,}  \FILLmv{y}  \FILLsym{:}  \FILLnt{A}  \FILLsym{,}  \Gamma_{{\mathrm{2}}}  \vdash  \Delta }
  \end{math}
\end{center}
By $\FILLdrulename{Eq\_Alpha}$, we know, for any $\FILLnt{t}$ in
$\Delta$, $\FILLnt{t}  \FILLsym{=}  \FILLsym{[}  \FILLmv{x}  \FILLsym{/}  \FILLmv{y}  \FILLsym{]}  \FILLnt{t}$, and hence $\Delta  \FILLsym{=}  \FILLsym{[}  \FILLmv{x}  \FILLsym{/}  \FILLmv{y}  \FILLsym{]}  \Delta$.

\item[Case:] the axiom steps: conclusion vs. axiom.
The proof 
\begin{center}
  \begin{math}
    $$\mprset{flushleft}
    \inferrule* [right=Cut] {
      $$\mprset{flushleft}
      \inferrule* [right=] {
        \inferrule* [right=,vdots=1.5em,fraction=\,] {
            \,
          }{\pi}          
      }{ \Gamma  \vdash   \FILLnt{t}  \FILLsym{:}  \FILLnt{A}  \mid  \Delta  }
      \,
      $$\mprset{flushleft}
      \inferrule* [right=Ax] {
        \,
      }{ \FILLmv{x}  \FILLsym{:}  \FILLnt{A}  \vdash  \FILLmv{x}  \FILLsym{:}  \FILLnt{A} }
    }{ \Gamma  \vdash   \Delta  \mid   \FILLsym{[}  \FILLnt{t}  \FILLsym{/}  \FILLmv{x}  \FILLsym{]}  \FILLmv{x}   \FILLsym{:}  \FILLnt{A}  }
  \end{math}
\end{center}
transforms into 
\begin{center}
  \begin{math}
    $$\mprset{flushleft}
    \inferrule* [right=\text{\tiny Series of Exchanges}] {
      \inferrule* [right=] {
        \inferrule* [right=,vdots=1.5em,fraction=\,] {
            \,
          }{\pi}          
      }{ \Gamma  \vdash   \FILLnt{t}  \FILLsym{:}  \FILLnt{A}  \mid  \Delta  }
    }{ \Gamma  \vdash   \Delta  \mid  \FILLnt{t}  \FILLsym{:}  \FILLnt{A}  }
  \end{math}
\end{center}
By the definition of the substitution function we know $\FILLnt{t}  \FILLsym{=}  \FILLsym{[}  \FILLnt{t}  \FILLsym{/}  \FILLmv{x}  \FILLsym{]}  \FILLmv{x}$.

\item[Case:] the exchange steps: conclusion vs. left-exchange (the
  first case).
The proof
\begin{center}
  \begin{math}
    $$\mprset{flushleft}
    \inferrule* [right=Cut] {
      \inferrule* [right=] {
        \inferrule* [right=,vdots=1.5em,fraction=\,] {
            \,
          }{\pi_1}          
      }{ \Gamma  \vdash   \FILLnt{t}  \FILLsym{:}  \FILLnt{A}  \mid  \Delta  }
      \\
      $$\mprset{flushleft}
      \inferrule* [right=Exl] {        
        $$\mprset{flushleft}
        \inferrule* [right=] {
          \inferrule* [right=,vdots=1.5em,fraction=\,] {
            \,
          }{\pi_2}          
        }{ \Gamma_{{\mathrm{1}}}  \FILLsym{,}  \FILLmv{x}  \FILLsym{:}  \FILLnt{A}  \FILLsym{,}  \FILLmv{y}  \FILLsym{:}  \FILLnt{B}  \FILLsym{,}  \Gamma_{{\mathrm{2}}}  \vdash  \Delta' }        
      }{ \Gamma_{{\mathrm{1}}}  \FILLsym{,}  \FILLmv{y}  \FILLsym{:}  \FILLnt{B}  \FILLsym{,}  \FILLmv{x}  \FILLsym{:}  \FILLnt{A}  \FILLsym{,}  \Gamma_{{\mathrm{2}}}  \vdash  \Delta' }
    }{ \Gamma_{{\mathrm{1}}}  \FILLsym{,}  \FILLmv{y}  \FILLsym{:}  \FILLnt{B}  \FILLsym{,}  \Gamma  \FILLsym{,}  \Gamma_{{\mathrm{2}}}  \vdash   \Delta  \mid  \FILLsym{[}  \FILLnt{t}  \FILLsym{/}  \FILLmv{x}  \FILLsym{]}  \Delta'  }
  \end{math}
\end{center}
transforms into the proof
\begin{center}
  \begin{math}
    $$\mprset{flushleft}
    \inferrule* [right=\text{Series of Exchanges}] {
      $$\mprset{flushleft}
      \inferrule* [right=Cut] {
        \inferrule* [right=] {
        \inferrule* [right=,vdots=1.5em,fraction=\,] {
            \,
          }{\pi_1}          
      }{ \Gamma  \vdash   \FILLnt{t}  \FILLsym{:}  \FILLnt{A}  \mid  \Delta  }
      \\
        $$\mprset{flushleft}
        \inferrule* [right=] {
          \inferrule* [right=,vdots=1.5em,fraction=\,] {
            \,
          }{\pi_2}          
        }{ \Gamma_{{\mathrm{1}}}  \FILLsym{,}  \FILLmv{x}  \FILLsym{:}  \FILLnt{A}  \FILLsym{,}  \FILLmv{y}  \FILLsym{:}  \FILLnt{B}  \FILLsym{,}  \Gamma_{{\mathrm{2}}}  \vdash  \Delta' }        
      }{ \Gamma_{{\mathrm{1}}}  \FILLsym{,}  \Gamma  \FILLsym{,}  \FILLmv{y}  \FILLsym{:}  \FILLnt{B}  \FILLsym{,}  \Gamma_{{\mathrm{2}}}  \vdash   \Delta  \mid  \FILLsym{[}  \FILLnt{t}  \FILLsym{/}  \FILLmv{x}  \FILLsym{]}  \Delta'  }
    }{ \Gamma_{{\mathrm{1}}}  \FILLsym{,}  \FILLmv{y}  \FILLsym{:}  \FILLnt{B}  \FILLsym{,}  \Gamma  \FILLsym{,}  \Gamma_{{\mathrm{2}}}  \vdash   \Delta  \mid  \FILLsym{[}  \FILLnt{t}  \FILLsym{/}  \FILLmv{x}  \FILLsym{]}  \Delta'  }
  \end{math}
\end{center}
Clearly, all terms are equivalent.  

\item[Case:] the exchange steps: conclusion vs. left-exchange (the
  second case). The proof
\begin{center}
  \begin{math}
    $$\mprset{flushleft}
    \inferrule* [right=Cut] {
      \inferrule* [right=] {
        \inferrule* [right=,vdots=1.5em,fraction=\,] {
            \,
          }{\pi_1}          
      }{ \Gamma  \vdash   \FILLnt{t}  \FILLsym{:}  \FILLnt{B}  \mid  \Delta  }
      \\
      $$\mprset{flushleft}
      \inferrule* [right=Exl] {        
        $$\mprset{flushleft}
        \inferrule* [right=] {
          \inferrule* [right=,vdots=1.5em,fraction=\,] {
            \,
          }{\pi_2}          
        }{ \Gamma_{{\mathrm{1}}}  \FILLsym{,}  \FILLmv{x}  \FILLsym{:}  \FILLnt{A}  \FILLsym{,}  \FILLmv{y}  \FILLsym{:}  \FILLnt{B}  \FILLsym{,}  \Gamma_{{\mathrm{2}}}  \vdash  \Delta' }        
      }{ \Gamma_{{\mathrm{1}}}  \FILLsym{,}  \FILLmv{y}  \FILLsym{:}  \FILLnt{B}  \FILLsym{,}  \FILLmv{x}  \FILLsym{:}  \FILLnt{A}  \FILLsym{,}  \Gamma_{{\mathrm{2}}}  \vdash  \Delta' }
    }{ \Gamma_{{\mathrm{1}}}  \FILLsym{,}  \Gamma  \FILLsym{,}  \FILLmv{x}  \FILLsym{:}  \FILLnt{A}  \FILLsym{,}  \Gamma_{{\mathrm{2}}}  \vdash   \Delta  \mid  \FILLsym{[}  \FILLnt{t}  \FILLsym{/}  \FILLmv{y}  \FILLsym{]}  \Delta'  }
  \end{math}
\end{center}
transforms into the proof
\begin{center}
  \begin{math}
    $$\mprset{flushleft}
    \inferrule* [right=\text{Series of Exchanges}] {
      $$\mprset{flushleft}
      \inferrule* [right=Cut] {
        \inferrule* [right=] {
        \inferrule* [right=,vdots=1.5em,fraction=\,] {
            \,
          }{\pi_1}          
      }{ \Gamma  \vdash   \FILLnt{t}  \FILLsym{:}  \FILLnt{B}  \mid  \Delta  }
      \\
        $$\mprset{flushleft}
        \inferrule* [right=] {
          \inferrule* [right=,vdots=1.5em,fraction=\,] {
            \,
          }{\pi_2}          
        }{ \Gamma_{{\mathrm{1}}}  \FILLsym{,}  \FILLmv{x}  \FILLsym{:}  \FILLnt{A}  \FILLsym{,}  \FILLmv{y}  \FILLsym{:}  \FILLnt{B}  \FILLsym{,}  \Gamma_{{\mathrm{2}}}  \vdash  \Delta' }        
      }{ \Gamma_{{\mathrm{1}}}  \FILLsym{,}  \FILLmv{x}  \FILLsym{:}  \FILLnt{A}  \FILLsym{,}  \Gamma  \FILLsym{,}  \Gamma_{{\mathrm{2}}}  \vdash   \Delta  \mid  \FILLsym{[}  \FILLnt{t}  \FILLsym{/}  \FILLmv{y}  \FILLsym{]}  \Delta'  }
    }{ \Gamma_{{\mathrm{1}}}  \FILLsym{,}  \Gamma  \FILLsym{,}  \FILLmv{x}  \FILLsym{:}  \FILLnt{A}  \FILLsym{,}  \Gamma_{{\mathrm{2}}}  \vdash   \Delta  \mid  \FILLsym{[}  \FILLnt{t}  \FILLsym{/}  \FILLmv{y}  \FILLsym{]}  \Delta'  }
  \end{math}
\end{center}
Clearly, all terms are equivalent.

\item[Case:] the exchange steps: conclusion vs. right-exchange
The proof
\begin{center}
  \begin{math}
    $$\mprset{flushleft}
    \inferrule* [right=Cut] {
        \inferrule* [right=] {
        \inferrule* [right=,vdots=1.5em,fraction=\,] {
            \,
          }{\pi_1}          
      }{ \Gamma  \vdash   \FILLnt{t}  \FILLsym{:}  \FILLnt{A}  \mid  \Delta  }
      \\
      $$\mprset{flushleft}
      \inferrule* [right=Exr] {
        $$\mprset{flushleft}
        \inferrule* [right=] {
          \inferrule* [right=,vdots=1.5em,fraction=\,] {
            \,
          }{\pi_2}          
        }{ \Gamma_{{\mathrm{1}}}  \FILLsym{,}  \FILLmv{x}  \FILLsym{:}  \FILLnt{A}  \FILLsym{,}  \Gamma_{{\mathrm{2}}}  \vdash       \Delta_{{\mathrm{1}}}  \mid  \FILLnt{t_{{\mathrm{1}}}}  \FILLsym{:}  \FILLnt{B}    \mid  \FILLnt{t_{{\mathrm{2}}}}  \FILLsym{:}  \FILLnt{C}    \mid  \Delta'  }        
      }{ \Gamma_{{\mathrm{1}}}  \FILLsym{,}  \FILLmv{x}  \FILLsym{:}  \FILLnt{A}  \FILLsym{,}  \Gamma_{{\mathrm{2}}}  \vdash       \Delta_{{\mathrm{1}}}  \mid  \FILLnt{t_{{\mathrm{2}}}}  \FILLsym{:}  \FILLnt{C}    \mid  \FILLnt{t_{{\mathrm{1}}}}  \FILLsym{:}  \FILLnt{B}    \mid  \Delta'  }
    }{ \Gamma_{{\mathrm{1}}}  \FILLsym{,}  \Gamma  \FILLsym{,}  \Gamma_{{\mathrm{2}}}  \vdash   \Delta  \mid         \FILLsym{[}  \FILLnt{t}  \FILLsym{/}  \FILLmv{x}  \FILLsym{]}  \Delta_{{\mathrm{1}}}   \mid   \FILLsym{[}  \FILLnt{t}  \FILLsym{/}  \FILLmv{x}  \FILLsym{]}  \FILLnt{t_{{\mathrm{2}}}}   \FILLsym{:}  \FILLnt{C}    \mid   \FILLsym{[}  \FILLnt{t}  \FILLsym{/}  \FILLmv{x}  \FILLsym{]}  \FILLnt{t_{{\mathrm{1}}}}   \FILLsym{:}  \FILLnt{B}    \mid  \FILLsym{[}  \FILLnt{t}  \FILLsym{/}  \FILLmv{x}  \FILLsym{]}  \Delta'    }
  \end{math}
\end{center}
transforms into this proof
\begin{center}
  \begin{math}
    $$\mprset{flushleft}
    \inferrule* [right=Exr] {
      $$\mprset{flushleft}
      \inferrule* [right=Cut] {
        \inferrule* [right=] {
        \inferrule* [right=,vdots=1.5em,fraction=\,] {
            \,
          }{\pi_1}          
      }{ \Gamma  \vdash   \FILLnt{t}  \FILLsym{:}  \FILLnt{A}  \mid  \Delta  }
      \\
      $$\mprset{flushleft}
        \inferrule* [right=] {
          \inferrule* [right=,vdots=1.5em,fraction=\,] {
            \,
          }{\pi_2}          
        }{ \Gamma_{{\mathrm{1}}}  \FILLsym{,}  \FILLmv{x}  \FILLsym{:}  \FILLnt{A}  \FILLsym{,}  \Gamma_{{\mathrm{2}}}  \vdash       \Delta_{{\mathrm{1}}}  \mid  \FILLnt{t_{{\mathrm{1}}}}  \FILLsym{:}  \FILLnt{B}    \mid  \FILLnt{t_{{\mathrm{2}}}}  \FILLsym{:}  \FILLnt{C}    \mid  \Delta'  }        
      }{ \Gamma_{{\mathrm{1}}}  \FILLsym{,}  \Gamma  \FILLsym{,}  \Gamma_{{\mathrm{2}}}  \vdash   \Delta  \mid         \FILLsym{[}  \FILLnt{t}  \FILLsym{/}  \FILLmv{x}  \FILLsym{]}  \Delta_{{\mathrm{1}}}   \mid   \FILLsym{[}  \FILLnt{t}  \FILLsym{/}  \FILLmv{x}  \FILLsym{]}  \FILLnt{t_{{\mathrm{1}}}}   \FILLsym{:}  \FILLnt{B}    \mid   \FILLsym{[}  \FILLnt{t}  \FILLsym{/}  \FILLmv{x}  \FILLsym{]}  \FILLnt{t_{{\mathrm{2}}}}   \FILLsym{:}  \FILLnt{C}    \mid  \FILLsym{[}  \FILLnt{t}  \FILLsym{/}  \FILLmv{x}  \FILLsym{]}  \Delta'    }
    }{ \Gamma_{{\mathrm{1}}}  \FILLsym{,}  \Gamma  \FILLsym{,}  \Gamma_{{\mathrm{2}}}  \vdash        \FILLsym{[}  \FILLnt{t}  \FILLsym{/}  \FILLmv{x}  \FILLsym{]}  \Delta_{{\mathrm{1}}}   \mid   \FILLsym{[}  \FILLnt{t}  \FILLsym{/}  \FILLmv{x}  \FILLsym{]}  \FILLnt{t_{{\mathrm{2}}}}   \FILLsym{:}  \FILLnt{C}    \mid   \FILLsym{[}  \FILLnt{t}  \FILLsym{/}  \FILLmv{x}  \FILLsym{]}  \FILLnt{t_{{\mathrm{1}}}}   \FILLsym{:}  \FILLnt{B}    \mid  \FILLsym{[}  \FILLnt{t}  \FILLsym{/}  \FILLmv{x}  \FILLsym{]}  \Delta'  }
  \end{math}
\end{center}
Clearly, all terms are equivalent.  
  
\item[Case:] principal formula vs. principal formula: tensor.
The proof 
\begin{center} 
    \begin{math}
    $$\mprset{flushleft}
    \inferrule* [right=\footnotesize Cut] {
      $$\mprset{flushleft}
      \inferrule* [right=\footnotesize Tr] {
        \inferrule* [right=] {
        \inferrule* [right=,vdots=1.5em,fraction=\,] {
            \,
          }{\pi_1}          
      }{ \Gamma_{{\mathrm{1}}}  \vdash   \FILLnt{t_{{\mathrm{1}}}}  \FILLsym{:}  \FILLnt{A}  \mid  \Delta_{{\mathrm{1}}}  }
      \\
      \inferrule* [right=] {
        \inferrule* [right=,vdots=1.5em,fraction=\,] {
            \,
          }{\pi_2}          
      }{ \Gamma_{{\mathrm{2}}}  \vdash   \FILLnt{t_{{\mathrm{2}}}}  \FILLsym{:}  \FILLnt{B}  \mid  \Delta_{{\mathrm{2}}}  }
      }{ \Gamma_{{\mathrm{1}}}  \FILLsym{,}  \Gamma_{{\mathrm{2}}}  \vdash    \FILLnt{t_{{\mathrm{1}}}}  \otimes  \FILLnt{t_{{\mathrm{2}}}}   \FILLsym{:}   \FILLnt{A}  \otimes  \FILLnt{B}   \mid    \Delta_{{\mathrm{1}}}  \mid  \Delta_{{\mathrm{2}}}    }
      \\
      $$\mprset{flushleft}
      \inferrule* [right=\footnotesize Tl] {
        \inferrule* [right=] {
          \inferrule* [right=,vdots=1.5em,fraction=\,] {
            \,
          }{\pi_3}          
        }{ \Gamma_{{\mathrm{3}}}  \FILLsym{,}  \FILLmv{x}  \FILLsym{:}  \FILLnt{A}  \FILLsym{,}  \FILLmv{y}  \FILLsym{:}  \FILLnt{B}  \FILLsym{,}  \Gamma_{{\mathrm{4}}}  \vdash  \Delta_{{\mathrm{3}}} }
      }{ \Gamma_{{\mathrm{3}}}  \FILLsym{,}  \FILLmv{z}  \FILLsym{:}   \FILLnt{A}  \otimes  \FILLnt{B}   \FILLsym{,}  \Gamma_{{\mathrm{4}}}  \vdash    \mathsf{let}\, \FILLmv{z} \,\mathsf{be}\,  \FILLmv{x}  \otimes  \FILLmv{y}  \,\mathsf{in}\, \Delta_{{\mathrm{3}}}   }
    }{ \Gamma_{{\mathrm{3}}}  \FILLsym{,}  \Gamma_{{\mathrm{1}}}  \FILLsym{,}  \Gamma_{{\mathrm{2}}}  \FILLsym{,}  \Gamma_{{\mathrm{4}}}  \vdash     \Delta_{{\mathrm{1}}}  \mid  \Delta_{{\mathrm{2}}}    \mid  \FILLsym{[}   \FILLnt{t_{{\mathrm{1}}}}  \otimes  \FILLnt{t_{{\mathrm{2}}}}   \FILLsym{/}  \FILLmv{z}  \FILLsym{]}  \FILLsym{(}    \mathsf{let}\, \FILLmv{z} \,\mathsf{be}\,  \FILLmv{x}  \otimes  \FILLmv{y}  \,\mathsf{in}\, \Delta_{{\mathrm{3}}}    \FILLsym{)}  }
  \end{math}
\end{center}
is transformed into the proof
\begin{center}
  \begin{math}
    $$\mprset{flushleft}
    \inferrule* [right=Cut] {
      \inferrule* [right=] {
        \inferrule* [right=,vdots=1.5em,fraction=\,] {
            \,
          }{\pi_1}          
      }{ \Gamma_{{\mathrm{1}}}  \vdash   \FILLnt{t_{{\mathrm{1}}}}  \FILLsym{:}  \FILLnt{A}  \mid  \Delta_{{\mathrm{1}}}  }
      \\
      $$\mprset{flushleft}
      \inferrule* [right=Cut] {
        \inferrule* [right=] {
        \inferrule* [right=,vdots=1.5em,fraction=\,] {
            \,
          }{\pi_2}          
      }{ \Gamma_{{\mathrm{2}}}  \vdash   \FILLnt{t_{{\mathrm{2}}}}  \FILLsym{:}  \FILLnt{B}  \mid  \Delta_{{\mathrm{2}}}  }
      \\
      \inferrule* [right=] {
          \inferrule* [right=,vdots=1.5em,fraction=\,] {
            \,
          }{\pi_3}          
        }{ \Gamma_{{\mathrm{3}}}  \FILLsym{,}  \FILLmv{x}  \FILLsym{:}  \FILLnt{A}  \FILLsym{,}  \FILLmv{y}  \FILLsym{:}  \FILLnt{B}  \FILLsym{,}  \Gamma_{{\mathrm{4}}}  \vdash  \Delta_{{\mathrm{3}}} }
      }{ \Gamma_{{\mathrm{3}}}  \FILLsym{,}  \FILLmv{x}  \FILLsym{:}  \FILLnt{A}  \FILLsym{,}  \Gamma_{{\mathrm{2}}}  \FILLsym{,}  \Gamma_{{\mathrm{4}}}  \vdash   \Delta_{{\mathrm{2}}}  \mid  \FILLsym{[}  \FILLnt{t_{{\mathrm{2}}}}  \FILLsym{/}  \FILLmv{y}  \FILLsym{]}  \Delta_{{\mathrm{3}}}  }
    }{ \Gamma_{{\mathrm{3}}}  \FILLsym{,}  \Gamma_{{\mathrm{1}}}  \FILLsym{,}  \Gamma_{{\mathrm{2}}}  \FILLsym{,}  \Gamma_{{\mathrm{4}}}  \vdash     \Delta_{{\mathrm{1}}}  \mid  \Delta_{{\mathrm{2}}}    \mid  \FILLsym{[}  \FILLnt{t_{{\mathrm{1}}}}  \FILLsym{/}  \FILLmv{x}  \FILLsym{]}  \FILLsym{[}  \FILLnt{t_{{\mathrm{2}}}}  \FILLsym{/}  \FILLmv{y}  \FILLsym{]}  \Delta_{{\mathrm{3}}}  }
  \end{math}
\end{center}
Without loss of generality suppose $\Delta_{{\mathrm{3}}}  \FILLsym{=}  \FILLnt{t_{{\mathrm{3}}}}  \FILLsym{:}  \FILLnt{C}  \FILLsym{,}  \Delta'_{{\mathrm{3}}}$.  We can see that $ \FILLsym{[}   \FILLnt{t_{{\mathrm{1}}}}  \otimes  \FILLnt{t_{{\mathrm{2}}}}   \FILLsym{/}  \FILLmv{z}  \FILLsym{]}   (  \mathsf{let}\, \FILLmv{z} \,\mathsf{be}\,  \FILLmv{x}  \otimes  \FILLmv{y}  \,\mathsf{in}\, \FILLnt{t_{{\mathrm{3}}}}  )    \FILLsym{=}   \mathsf{let}\,  \FILLnt{t_{{\mathrm{1}}}}  \otimes  \FILLnt{t_{{\mathrm{2}}}}  \,\mathsf{be}\,  \FILLmv{x}  \otimes  \FILLmv{y}  \,\mathsf{in}\, \FILLnt{t_{{\mathrm{3}}}} $ by the definition of
substitution, and by using the $\FILLdrulename{Eq\_Beta1Tensor}$ rule we obtain
$ \mathsf{let}\,  \FILLnt{t_{{\mathrm{1}}}}  \otimes  \FILLnt{t_{{\mathrm{2}}}}  \,\mathsf{be}\,  \FILLmv{x}  \otimes  \FILLmv{y}  \,\mathsf{in}\, \FILLnt{t_{{\mathrm{3}}}}   \FILLsym{=}  \FILLsym{[}  \FILLnt{t_{{\mathrm{1}}}}  \FILLsym{/}  \FILLmv{x}  \FILLsym{]}  \FILLsym{[}  \FILLnt{t_{{\mathrm{2}}}}  \FILLsym{/}  \FILLmv{y}  \FILLsym{]}  \FILLnt{t_{{\mathrm{3}}}}$.  This argument
can be repeated for any term in $\FILLsym{[}   \FILLnt{t_{{\mathrm{1}}}}  \otimes  \FILLnt{t_{{\mathrm{2}}}}   \FILLsym{/}  \FILLmv{z}  \FILLsym{]}  \FILLsym{(}    \mathsf{let}\, \FILLmv{z} \,\mathsf{be}\,  \FILLmv{x}  \otimes  \FILLmv{y}  \,\mathsf{in}\, \Delta'_{{\mathrm{3}}}    \FILLsym{)}$, and thus, $\FILLsym{[}   \FILLnt{t_{{\mathrm{1}}}}  \otimes  \FILLnt{t_{{\mathrm{2}}}}   \FILLsym{/}  \FILLmv{z}  \FILLsym{]}  \FILLsym{(}    \mathsf{let}\, \FILLmv{z} \,\mathsf{be}\,  \FILLmv{x}  \otimes  \FILLmv{y}  \,\mathsf{in}\, \Delta_{{\mathrm{3}}}    \FILLsym{)}  \FILLsym{=}  \FILLsym{[}  \FILLnt{t_{{\mathrm{1}}}}  \FILLsym{/}  \FILLmv{x}  \FILLsym{]}  \FILLsym{[}  \FILLnt{t_{{\mathrm{2}}}}  \FILLsym{/}  \FILLmv{y}  \FILLsym{]}  \Delta_{{\mathrm{3}}}$.

Note that in the second derivation of the above transformation we
first cut on $\FILLnt{B}$, and then $\FILLnt{A}$, but we could have cut on
$\FILLnt{A}$ first, and then $\FILLnt{B}$, but this would yield equivalent
derivations as above by using
Lemma~\ref{lemma:substitution_distribution}.

  
\item[Case:] principal formula vs. principal formula: par.
The proof
\begin{center}
  \scriptsize
    \begin{math}    
    $$\mprset{flushleft}
\inferrule* [right=\tiny Cut] {
  $$\mprset{flushleft}
  \inferrule* [leftskip=15px,left=\tiny Parr,rightskip=10px] {
    \inferrule* [] {
        \inferrule* [right=,vdots=1.5em,fraction=\,] {
            \,
          }{\pi_1}          
      }{ \Gamma_{{\mathrm{1}}}  \vdash     \Delta_{{\mathrm{1}}}  \mid  \FILLnt{t_{{\mathrm{1}}}}  \FILLsym{:}  \FILLnt{A}    \mid    \FILLnt{t_{{\mathrm{2}}}}  \FILLsym{:}  \FILLnt{B}  \mid  \Delta_{{\mathrm{2}}}    }
    }{ \Gamma_{{\mathrm{1}}}  \vdash     \Delta_{{\mathrm{1}}}  \mid   \FILLnt{t_{{\mathrm{1}}}}  \parr  \FILLnt{t_{{\mathrm{2}}}}   \FILLsym{:}   \FILLnt{A}  \parr  \FILLnt{B}     \mid  \Delta_{{\mathrm{2}}}  }
  \\
  $$\mprset{flushleft}
  \inferrule* [leftskip=5px,right=\tiny Parl,rightskip=30px] {
    \inferrule* [right=] {
        \inferrule* [right=,vdots=1.5em,fraction=\,] {
            \,
          }{\pi_2}          
      }{ \Gamma_{{\mathrm{2}}}  \FILLsym{,}  \FILLmv{x}  \FILLsym{:}  \FILLnt{A}  \vdash  \Delta_{{\mathrm{3}}} }
      \\
      \inferrule* [right=] {
        \inferrule* [right=,vdots=1.5em,fraction=\,] {
            \,
          }{\pi_3}          
      }{ \Gamma_{{\mathrm{3}}}  \FILLsym{,}  \FILLmv{y}  \FILLsym{:}  \FILLnt{B}  \vdash  \Delta_{{\mathrm{4}}} }
  }{ \Gamma_{{\mathrm{2}}}  \FILLsym{,}  \Gamma_{{\mathrm{3}}}  \FILLsym{,}  \FILLmv{z}  \FILLsym{:}   \FILLnt{A}  \parr  \FILLnt{B}   \vdash     \mathsf{\operatorname{let-pat} }\, \FILLmv{z} \, \FILLsym{(}   \FILLmv{x}  \parr   -    \FILLsym{)} \, \Delta_{{\mathrm{3}}}    \mid    \mathsf{\operatorname{let-pat} }\, \FILLmv{z} \, \FILLsym{(}    -   \parr  \FILLmv{y}   \FILLsym{)} \, \Delta_{{\mathrm{4}}}    }
}{ \Gamma_{{\mathrm{2}}}  \FILLsym{,}  \Gamma_{{\mathrm{3}}}  \FILLsym{,}  \Gamma_{{\mathrm{1}}}  \vdash       \Delta_{{\mathrm{1}}}  \mid  \Delta_{{\mathrm{2}}}    \mid   \FILLsym{[}   \FILLnt{t_{{\mathrm{1}}}}  \parr  \FILLnt{t_{{\mathrm{2}}}}   \FILLsym{/}  \FILLmv{z}  \FILLsym{]}  \FILLsym{(}   \mathsf{\operatorname{let-pat} }\, \FILLmv{z} \, \FILLsym{(}   \FILLmv{x}  \parr   -    \FILLsym{)} \, \Delta_{{\mathrm{3}}}   \FILLsym{)}     \mid   \FILLsym{[}   \FILLnt{t_{{\mathrm{1}}}}  \parr  \FILLnt{t_{{\mathrm{2}}}}   \FILLsym{/}  \FILLmv{z}  \FILLsym{]}  \FILLsym{(}   \mathsf{\operatorname{let-pat} }\, \FILLmv{z} \, \FILLsym{(}    -   \parr  \FILLmv{y}   \FILLsym{)} \, \Delta_{{\mathrm{4}}}   \FILLsym{)}   }
  \end{math}    
\end{center}
is transformed into the proof
\begin{center}
  \scriptsize
  \begin{math}    
    $$\mprset{flushleft}
    \inferrule* [right=  \scriptsize \tiny Series of Exchanges] {
      $$\mprset{flushleft}
    \inferrule* [right=  \scriptsize Cut] {
      $$\mprset{flushleft}
      \inferrule* [right=  \scriptsize Cut] {
        \inferrule* [right=] {
          \inferrule* [right=,vdots=1.5em,fraction=\,] {
            \,
          }{\pi_1}          
        }{ \Gamma_{{\mathrm{1}}}  \vdash     \Delta_{{\mathrm{1}}}  \mid  \FILLnt{t_{{\mathrm{1}}}}  \FILLsym{:}  \FILLnt{A}    \mid    \FILLnt{t_{{\mathrm{2}}}}  \FILLsym{:}  \FILLnt{B}  \mid  \Delta_{{\mathrm{2}}}    }
        \\
        \inferrule* [right=] {
          \inferrule* [right=,vdots=1.5em,fraction=\,] {
            \,
          }{\pi_3}          
        }{ \Gamma_{{\mathrm{3}}}  \FILLsym{,}  \FILLmv{y}  \FILLsym{:}  \FILLnt{B}  \vdash  \Delta_{{\mathrm{4}}} }
      }{ \Gamma_{{\mathrm{3}}}  \FILLsym{,}  \Gamma_{{\mathrm{1}}}  \vdash       \Delta_{{\mathrm{1}}}  \mid  \FILLnt{t_{{\mathrm{1}}}}  \FILLsym{:}  \FILLnt{A}    \mid  \Delta_{{\mathrm{2}}}    \mid  \FILLsym{[}  \FILLnt{t_{{\mathrm{2}}}}  \FILLsym{/}  \FILLmv{y}  \FILLsym{]}  \Delta_{{\mathrm{4}}}  }
      \\
      \inferrule* [right=] {
        \inferrule* [right=,vdots=1.5em,fraction=\,] {
          \,
        }{\pi_2}          
      }{ \Gamma_{{\mathrm{2}}}  \FILLsym{,}  \FILLmv{x}  \FILLsym{:}  \FILLnt{A}  \vdash  \Delta_{{\mathrm{3}}} }
    }{ \Gamma_{{\mathrm{2}}}  \FILLsym{,}  \Gamma_{{\mathrm{3}}}  \FILLsym{,}  \Gamma_{{\mathrm{1}}}  \vdash       \Delta_{{\mathrm{1}}}  \mid  \Delta_{{\mathrm{2}}}    \mid  \FILLsym{[}  \FILLnt{t_{{\mathrm{2}}}}  \FILLsym{/}  \FILLmv{y}  \FILLsym{]}  \Delta_{{\mathrm{4}}}    \mid  \FILLsym{[}  \FILLnt{t_{{\mathrm{1}}}}  \FILLsym{/}  \FILLmv{x}  \FILLsym{]}  \Delta_{{\mathrm{3}}}  }
    }{ \Gamma_{{\mathrm{2}}}  \FILLsym{,}  \Gamma_{{\mathrm{3}}}  \FILLsym{,}  \Gamma_{{\mathrm{1}}}  \vdash     \Delta_{{\mathrm{1}}}  \mid  \Delta_{{\mathrm{2}}}    \mid     \FILLsym{[}  \FILLnt{t_{{\mathrm{1}}}}  \FILLsym{/}  \FILLmv{x}  \FILLsym{]}  \Delta_{{\mathrm{3}}}   \mid  \FILLsym{[}  \FILLnt{t_{{\mathrm{2}}}}  \FILLsym{/}  \FILLmv{y}  \FILLsym{]}  \Delta_{{\mathrm{4}}}    }
  \end{math}
\end{center}
Without loss of generality consider the case when
$\Delta_{{\mathrm{3}}} =  \FILLnt{t_{{\mathrm{3}}}}  \FILLsym{:}  \FILLnt{C_{{\mathrm{1}}}}  \mid  \Delta'_{{\mathrm{3}}} $ and $\Delta_{{\mathrm{4}}} =  \FILLnt{t_{{\mathrm{4}}}}  \FILLsym{:}  \FILLnt{C_{{\mathrm{2}}}}  \mid  \Delta'_{{\mathrm{4}}} $.  First,
$\FILLsym{[}   \FILLnt{t_{{\mathrm{1}}}}  \parr  \FILLnt{t_{{\mathrm{2}}}}   \FILLsym{/}  \FILLmv{z}  \FILLsym{]}   (  \mathsf{\operatorname{let-pat} }\, \FILLmv{z} \, \FILLsym{(}   \FILLmv{x}  \parr   -    \FILLsym{)} \, \FILLnt{t_{{\mathrm{3}}}}  )   \FILLsym{=}   \mathsf{\operatorname{let-pat} }\,  (  \FILLnt{t_{{\mathrm{1}}}}  \parr  \FILLnt{t_{{\mathrm{2}}}}  )  \, \FILLsym{(}   \FILLmv{x}  \parr   -    \FILLsym{)} \, \FILLnt{t_{{\mathrm{3}}}} $,
and by $\FILLdrulename{Eq\_Beta1Par}$ we know
$ \mathsf{\operatorname{let-pat} }\,  (  \FILLnt{t_{{\mathrm{1}}}}  \parr  \FILLnt{t_{{\mathrm{2}}}}  )  \, \FILLsym{(}   \FILLmv{x}  \parr   -    \FILLsym{)} \, \FILLnt{t_{{\mathrm{3}}}}   \FILLsym{=}  \FILLsym{[}  \FILLnt{t_{{\mathrm{1}}}}  \FILLsym{/}  \FILLmv{x}  \FILLsym{]}  \FILLnt{t_{{\mathrm{3}}}}$ if $ \FILLmv{x}  \in \mathsf{FV}(  \FILLnt{t_{{\mathrm{3}}}}  ) $
or $ \mathsf{\operatorname{let-pat} }\,  (  \FILLnt{t_{{\mathrm{1}}}}  \parr  \FILLnt{t_{{\mathrm{2}}}}  )  \, \FILLsym{(}   \FILLmv{x}  \parr   -    \FILLsym{)} \, \FILLnt{t_{{\mathrm{3}}}}   \FILLsym{=}  \FILLnt{t_{{\mathrm{3}}}}$ otherwise.  In the
latter case we can see that $\FILLnt{t_{{\mathrm{3}}}}  \FILLsym{=}  \FILLsym{[}  \FILLnt{t_{{\mathrm{1}}}}  \FILLsym{/}  \FILLmv{x}  \FILLsym{]}  \FILLnt{t_{{\mathrm{3}}}}$, thus, in both cases
$ \mathsf{\operatorname{let-pat} }\,  (  \FILLnt{t_{{\mathrm{1}}}}  \parr  \FILLnt{t_{{\mathrm{2}}}}  )  \, \FILLsym{(}   \FILLmv{x}  \parr   -    \FILLsym{)} \, \FILLnt{t_{{\mathrm{3}}}}   \FILLsym{=}  \FILLsym{[}  \FILLnt{t_{{\mathrm{1}}}}  \FILLsym{/}  \FILLmv{x}  \FILLsym{]}  \FILLnt{t_{{\mathrm{3}}}}$.  This argument can
be repeated for any terms in $\Delta'_{{\mathrm{3}}}$, and hence
$\FILLsym{[}   \FILLnt{t_{{\mathrm{1}}}}  \parr  \FILLnt{t_{{\mathrm{2}}}}   \FILLsym{/}  \FILLmv{z}  \FILLsym{]}  \FILLsym{(}   \mathsf{\operatorname{let-pat} }\, \FILLmv{z} \, \FILLsym{(}   \FILLmv{x}  \parr   -    \FILLsym{)} \, \Delta_{{\mathrm{3}}}   \FILLsym{)} =  \mathsf{\operatorname{let-pat} }\,  (  \FILLnt{t_{{\mathrm{1}}}}  \parr  \FILLnt{t_{{\mathrm{2}}}}  )  \, \FILLsym{(}   \FILLmv{x}  \parr   -    \FILLsym{)} \, \Delta_{{\mathrm{3}}}   \FILLsym{=}  \FILLsym{[}  \FILLnt{t_{{\mathrm{1}}}}  \FILLsym{/}  \FILLmv{x}  \FILLsym{]}  \Delta_{{\mathrm{3}}}$.
We can apply a similar argument for\\
$\FILLsym{[}   \FILLnt{t_{{\mathrm{1}}}}  \parr  \FILLnt{t_{{\mathrm{2}}}}   \FILLsym{/}  \FILLmv{z}  \FILLsym{]}   (  \mathsf{\operatorname{let-pat} }\, \FILLmv{z} \, \FILLsym{(}    -   \parr  \FILLmv{y}   \FILLsym{)} \, \FILLnt{t_{{\mathrm{4}}}}  ) $ and
$\FILLsym{[}   \FILLnt{t_{{\mathrm{1}}}}  \parr  \FILLnt{t_{{\mathrm{2}}}}   \FILLsym{/}  \FILLmv{z}  \FILLsym{]}  \FILLsym{(}   \mathsf{\operatorname{let-pat} }\, \FILLmv{z} \, \FILLsym{(}    -   \parr  \FILLmv{y}   \FILLsym{)} \, \Delta_{{\mathrm{4}}}   \FILLsym{)}$.


  Note that we could have first cut on $\FILLnt{A}$, and then on $\FILLnt{B}$ in
  the second derivation, but we would have arrived at the same result
  just with potentially more exchanges on the right.

  \item[Case:] principal formula vs. principal formula: implication.
The proof
\begin{center}
  \scriptsize
  \begin{math}
    $$\mprset{flushleft}
    \inferrule* [right=\scriptsize Cut] {
      $$\mprset{flushleft}
      \inferrule* [right=\scriptsize Impr] {
        \inferrule* [right=] {
          \inferrule* [right=,vdots=1.5em,fraction=\,] {
            \,
          }{\pi_1}          
        }{ \Gamma  \FILLsym{,}  \FILLmv{x}  \FILLsym{:}  \FILLnt{A}  \vdash   \FILLnt{t}  \FILLsym{:}  \FILLnt{B}  \mid  \Delta  }
        \\
         \FILLmv{x}  \not\in \mathsf{FV}(  \Delta  ) 
      }{ \Gamma  \vdash    \lambda  \FILLmv{x}  .  \FILLnt{t}   \FILLsym{:}   \FILLnt{A}  \multimap   \FILLnt{B}   \mid  \Delta  }
      \\
      $$\mprset{flushleft}
      \inferrule* [right=\scriptsize Impl] {
        \inferrule* [right=] {
          \inferrule* [right=,vdots=1.5em,fraction=\,] {
            \,
          }{\pi_2}          
        }{ \Gamma_{{\mathrm{1}}}  \vdash   \FILLnt{t_{{\mathrm{1}}}}  \FILLsym{:}  \FILLnt{A}  \mid  \Delta_{{\mathrm{1}}}  }
        \\
        \inferrule* [right=] {
          \inferrule* [right=,vdots=1.5em,fraction=\,] {
            \,
          }{\pi_3}          
        }{ \Gamma_{{\mathrm{2}}}  \FILLsym{,}  \FILLmv{y}  \FILLsym{:}  \FILLnt{B}  \vdash  \Delta_{{\mathrm{2}}} }
      }{ \Gamma_{{\mathrm{1}}}  \FILLsym{,}  \FILLmv{z}  \FILLsym{:}   \FILLnt{A}  \multimap   \FILLnt{B}   \FILLsym{,}  \Gamma_{{\mathrm{2}}}  \vdash   \Delta_{{\mathrm{1}}}  \mid  \FILLsym{[}  \FILLmv{z} \, \FILLnt{t_{{\mathrm{1}}}}  \FILLsym{/}  \FILLmv{y}  \FILLsym{]}  \Delta_{{\mathrm{2}}}  }
    }{ \Gamma_{{\mathrm{1}}}  \FILLsym{,}  \Gamma  \FILLsym{,}  \Gamma_{{\mathrm{2}}}  \vdash   \Delta  \mid     \FILLsym{[}   \lambda  \FILLmv{x}  .  \FILLnt{t}   \FILLsym{/}  \FILLmv{z}  \FILLsym{]}  \Delta_{{\mathrm{1}}}   \mid   \FILLsym{[}   \lambda  \FILLmv{x}  .  \FILLnt{t}   \FILLsym{/}  \FILLmv{z}  \FILLsym{]}  \FILLsym{[}  \FILLmv{z} \, \FILLnt{t_{{\mathrm{1}}}}  \FILLsym{/}  \FILLmv{y}  \FILLsym{]}  \Delta_{{\mathrm{2}}}     }
  \end{math}
\end{center}
transforms into the proof
\begin{center}
  \scriptsize
  \begin{math}
    $$\mprset{flushleft}
    \inferrule* [right=\tiny Series of Exchanges] {
      $$\mprset{flushleft}
    \inferrule* [right=\scriptsize Cut] {
      $$\mprset{flushleft}
      \inferrule* [right=\scriptsize Cut] {
        \inferrule* [right=,rightskip=17px] {
          \inferrule* [right=,vdots=1.5em,fraction=\,] {
            \,
          }{\pi_2}          
        }{ \Gamma_{{\mathrm{1}}}  \vdash   \FILLnt{t_{{\mathrm{1}}}}  \FILLsym{:}  \FILLnt{A}  \mid  \Delta_{{\mathrm{1}}}  }
        \\
        \inferrule* [right=,rightskip=13px] {
          \inferrule* [right=,vdots=1.5em,fraction=\,] {
            \,
          }{\pi_1}          
        }{ \Gamma  \FILLsym{,}  \FILLmv{x}  \FILLsym{:}  \FILLnt{A}  \vdash   \FILLnt{t}  \FILLsym{:}  \FILLnt{B}  \mid  \Delta  }
        \\
         \FILLmv{x}  \not\in \mathsf{FV}(  \Delta  ) 
      }{ \Gamma  \FILLsym{,}  \Gamma_{{\mathrm{1}}}  \vdash   \Delta_{{\mathrm{1}}}  \mid     \FILLsym{[}  \FILLnt{t_{{\mathrm{1}}}}  \FILLsym{/}  \FILLmv{x}  \FILLsym{]}  \FILLnt{t}   \FILLsym{:}  \FILLnt{B}  \mid  \FILLsym{[}  \FILLnt{t_{{\mathrm{1}}}}  \FILLsym{/}  \FILLmv{x}  \FILLsym{]}  \Delta    }
      \\
      \inferrule* [right=] {
          \inferrule* [right=,vdots=1.5em,fraction=\,] {
            \,
          }{\pi_3}          
        }{ \Gamma_{{\mathrm{2}}}  \FILLsym{,}  \FILLmv{y}  \FILLsym{:}  \FILLnt{B}  \vdash  \Delta_{{\mathrm{2}}} }
      }{ \Gamma_{{\mathrm{2}}}  \FILLsym{,}  \Gamma  \FILLsym{,}  \Gamma_{{\mathrm{1}}}  \vdash     \Delta_{{\mathrm{1}}}  \mid  \FILLsym{[}  \FILLnt{t_{{\mathrm{1}}}}  \FILLsym{/}  \FILLmv{x}  \FILLsym{]}  \Delta    \mid  \FILLsym{[}  \FILLsym{[}  \FILLnt{t_{{\mathrm{1}}}}  \FILLsym{/}  \FILLmv{x}  \FILLsym{]}  \FILLnt{t}  \FILLsym{/}  \FILLmv{y}  \FILLsym{]}  \Delta_{{\mathrm{2}}}  }
    }{ \Gamma_{{\mathrm{1}}}  \FILLsym{,}  \Gamma  \FILLsym{,}  \Gamma_{{\mathrm{2}}}  \vdash    \FILLsym{[}  \FILLnt{t_{{\mathrm{1}}}}  \FILLsym{/}  \FILLmv{x}  \FILLsym{]}  \Delta   \mid    \Delta_{{\mathrm{1}}}  \mid   \FILLsym{[}  \FILLsym{[}  \FILLnt{t_{{\mathrm{1}}}}  \FILLsym{/}  \FILLmv{x}  \FILLsym{]}  \FILLnt{t}  \FILLsym{/}  \FILLmv{y}  \FILLsym{]}  \Delta_{{\mathrm{2}}}     }
  \end{math}
\end{center}
Without loss of generality consider the case when $\Delta_{{\mathrm{2}}} =  \FILLnt{t_{{\mathrm{2}}}}  \FILLsym{:}  \FILLnt{C}  \mid  \Delta'_{{\mathrm{2}}} $. 
First, by hypothesis we know $ \FILLmv{x}  \not\in \mathsf{FV}(  \Delta  ) $, and so we know $\Delta  \FILLsym{=}  \FILLsym{[}  \FILLnt{t_{{\mathrm{1}}}}  \FILLsym{/}  \FILLmv{x}  \FILLsym{]}  \Delta$.  We can see
that $\FILLsym{[}   \lambda  \FILLmv{x}  .  \FILLnt{t}   \FILLsym{/}  \FILLmv{z}  \FILLsym{]}  \FILLsym{[}  \FILLmv{z} \, \FILLnt{t_{{\mathrm{1}}}}  \FILLsym{/}  \FILLmv{y}  \FILLsym{]}  \FILLnt{t_{{\mathrm{2}}}}  \FILLsym{=}  \FILLsym{[}   (  \lambda  \FILLmv{x}  .  \FILLnt{t}  )  \, \FILLnt{t_{{\mathrm{1}}}}  \FILLsym{/}  \FILLmv{y}  \FILLsym{]}  \FILLnt{t_{{\mathrm{2}}}} = \FILLsym{[}  \FILLsym{[}  \FILLnt{t_{{\mathrm{1}}}}  \FILLsym{/}  \FILLmv{x}  \FILLsym{]}  \FILLnt{t}  \FILLsym{/}  \FILLmv{y}  \FILLsym{]}  \FILLnt{t_{{\mathrm{2}}}}$ by using the congruence rules
of equality and the rule $\FILLdrulename{Eq\_BetaFun}$.  This argument can be repeated for any term
in $\FILLsym{[}   \lambda  \FILLmv{x}  .  \FILLnt{t}   \FILLsym{/}  \FILLmv{z}  \FILLsym{]}  \FILLsym{[}  \FILLmv{z} \, \FILLnt{t_{{\mathrm{1}}}}  \FILLsym{/}  \FILLmv{y}  \FILLsym{]}  \Delta'_{{\mathrm{2}}}$, and so $\FILLsym{[}   \lambda  \FILLmv{x}  .  \FILLnt{t}   \FILLsym{/}  \FILLmv{z}  \FILLsym{]}  \FILLsym{[}  \FILLmv{z} \, \FILLnt{t_{{\mathrm{1}}}}  \FILLsym{/}  \FILLmv{y}  \FILLsym{]}  \Delta_{{\mathrm{2}}}  \FILLsym{=}  \FILLsym{[}  \FILLsym{[}  \FILLnt{t_{{\mathrm{1}}}}  \FILLsym{/}  \FILLmv{x}  \FILLsym{]}  \FILLnt{t}  \FILLsym{/}  \FILLmv{y}  \FILLsym{]}  \Delta_{{\mathrm{2}}}$.  Finally, by inspecting the previous
derivations we can see that $ \FILLmv{z}  \not\in \mathsf{FV}(  \Delta_{{\mathrm{1}}}  ) $, and thus, $\Delta_{{\mathrm{1}}}  \FILLsym{=}  \FILLsym{[}   \lambda  \FILLmv{x}  .  \FILLnt{t}   \FILLsym{/}  \FILLmv{z}  \FILLsym{]}  \Delta_{{\mathrm{1}}}$.  

\item[Case:] principal formula vs. principal formula: tensor unit.
    The proof
\begin{center}
  \begin{math}
    $$\mprset{flushleft}
    \inferrule* [right=Cut] {
      $$\mprset{flushleft}
      \inferrule* [right=Ir] {
        \,
      }{  \cdot   \vdash  \FILLsym{*}  \FILLsym{:}   \top  }
      \\
      $$\mprset{flushleft}
      \inferrule* [right=Il] {
        \inferrule* [right=] {
          \inferrule* [right=,vdots=1.5em,fraction=\,] {
            \,
          }{\pi}          
        }{ \Gamma  \vdash  \Delta }
      }{ \Gamma  \FILLsym{,}  \FILLmv{x}  \FILLsym{:}   \top   \vdash   \mathsf{let}\, \FILLmv{x} \,\mathsf{be}\,  *  \,\mathsf{in}\, \Delta  }
    }{ \Gamma  \vdash  \FILLsym{[}  \FILLsym{*}  \FILLsym{/}  \FILLmv{x}  \FILLsym{]}  \FILLsym{(}   \mathsf{let}\, \FILLmv{x} \,\mathsf{be}\,  *  \,\mathsf{in}\, \Delta   \FILLsym{)} }
  \end{math}
\end{center}
is transformed into the proof
\begin{center}
  \begin{math}
    \inferrule* [right=] {
      \inferrule* [right=,vdots=1.5em,fraction=\,] {
        \,
      }{\pi}          
    }{ \Gamma  \vdash  \Delta }
  \end{math}
\end{center}

Without loss of generality suppose $\Delta =  \FILLnt{t}  \FILLsym{:}  \FILLnt{A}  \mid  \Delta' $. We can see that 
$\FILLsym{[}  \FILLsym{*}  \FILLsym{/}  \FILLmv{x}  \FILLsym{]}   (  \mathsf{let}\, \FILLmv{x} \,\mathsf{be}\,  *  \,\mathsf{in}\, \FILLnt{t}  )   \FILLsym{=}   \mathsf{let}\, \FILLsym{*} \,\mathsf{be}\,  *  \,\mathsf{in}\, \FILLnt{t}  = \FILLnt{t}$ by the definition of 
substitution and the $\FILLdrulename{Eq\_EtaI}$ rule.  This argument can be repeated for any
term in $\FILLsym{[}  \FILLsym{*}  \FILLsym{/}  \FILLmv{x}  \FILLsym{]}  \FILLsym{(}   \mathsf{let}\, \FILLmv{x} \,\mathsf{be}\,  *  \,\mathsf{in}\, \Delta'   \FILLsym{)}$, and hence, $\FILLsym{[}  \FILLsym{*}  \FILLsym{/}  \FILLmv{x}  \FILLsym{]}  \FILLsym{(}   \mathsf{let}\, \FILLmv{x} \,\mathsf{be}\,  *  \,\mathsf{in}\, \Delta   \FILLsym{)}  \FILLsym{=}  \Delta$.

\item[Case:] principal formula vs. principal formula: par unit.
  The proof
\begin{center}
  \begin{math}
    $$\mprset{flushleft}
    \inferrule* [right=Cut] {
      $$\mprset{flushleft}
      \inferrule* [right=Pr] {
        \inferrule* [right=] {
          \inferrule* [right=,vdots=1.5em,fraction=\,] {
            \,
          }{\pi}          
        }{ \Gamma  \vdash  \Delta }
      }{ \Gamma  \vdash    \circ   \FILLsym{:}   \perp   \mid  \Delta  }
      \\
      $$\mprset{flushleft}
      \inferrule* [right=Pl] {
        \,
      }{ \FILLmv{x}  \FILLsym{:}   \perp   \vdash   \cdot  }
    }{ \Gamma  \vdash   \Delta  \mid  \FILLsym{[}   \circ   \FILLsym{/}  \FILLmv{x}  \FILLsym{]}   \cdot   }
  \end{math}
\end{center}
transforms into the proof
\begin{center}
  \begin{math}
    \inferrule* [right=] {
      \inferrule* [right=,vdots=1.5em,fraction=\,] {
        \,
      }{\pi}          
    }{ \Gamma  \vdash  \Delta }
  \end{math}
\end{center}
Clearly, $\FILLsym{[}   \circ   \FILLsym{/}  \FILLmv{x}  \FILLsym{]}   \cdot   \FILLsym{=}   \cdot $.

\item[Case:] secondary conclusion: left introduction of implication.
The proof 
\begin{center}
  \begin{math}
    $$\mprset{flushleft}
    \inferrule* [right=Cut] {
      $$\mprset{flushleft}
      \inferrule* [right=Impl] {
        \inferrule* [right=] {
          \inferrule* [right=,vdots=1.5em,fraction=\,] {
            \,
          }{\pi_1}          
        }{ \Gamma  \vdash   \FILLnt{t_{{\mathrm{1}}}}  \FILLsym{:}  \FILLnt{A}  \mid  \Delta  }
        \\
        \inferrule* [right=] {
          \inferrule* [right=,vdots=1.5em,fraction=\,] {
            \,
          }{\pi_2}          
        }{ \Gamma_{{\mathrm{1}}}  \FILLsym{,}  \FILLmv{x}  \FILLsym{:}  \FILLnt{B}  \FILLsym{,}  \Gamma_{{\mathrm{2}}}  \vdash   \FILLnt{t_{{\mathrm{2}}}}  \FILLsym{:}  \FILLnt{C}  \mid  \Delta_{{\mathrm{2}}}  }
      }{ \Gamma  \FILLsym{,}  \FILLmv{y}  \FILLsym{:}   \FILLnt{A}  \multimap   \FILLnt{B}   \FILLsym{,}  \Gamma_{{\mathrm{1}}}  \FILLsym{,}  \Gamma_{{\mathrm{2}}}  \vdash     \Delta  \mid   \FILLsym{[}  \FILLmv{y} \, \FILLnt{t_{{\mathrm{1}}}}  \FILLsym{/}  \FILLmv{x}  \FILLsym{]}  \FILLnt{t_{{\mathrm{2}}}}  \FILLsym{:}  \FILLnt{C}     \mid  \FILLsym{[}  \FILLmv{y} \, \FILLnt{t_{{\mathrm{1}}}}  \FILLsym{/}  \FILLmv{x}  \FILLsym{]}  \Delta_{{\mathrm{2}}}  }
      \\      
      \inferrule* [right=] {
          \inferrule* [right=,vdots=1.5em,fraction=\,] {
            \,
          }{\pi_3}          
        }{ \Gamma_{{\mathrm{3}}}  \FILLsym{,}  \FILLmv{z}  \FILLsym{:}  \FILLnt{C}  \FILLsym{,}  \Gamma_{{\mathrm{4}}}  \vdash  \Delta_{{\mathrm{3}}} }
      }{ \Gamma_{{\mathrm{3}}}  \FILLsym{,}  \Gamma  \FILLsym{,}  \FILLmv{y}  \FILLsym{:}   \FILLnt{A}  \multimap   \FILLnt{B}   \FILLsym{,}  \Gamma_{{\mathrm{1}}}  \FILLsym{,}  \Gamma_{{\mathrm{2}}}  \FILLsym{,}  \Gamma_{{\mathrm{4}}}  \vdash   \Delta  \mid     \FILLsym{[}  \FILLmv{y} \, \FILLnt{t_{{\mathrm{1}}}}  \FILLsym{/}  \FILLmv{x}  \FILLsym{]}  \Delta_{{\mathrm{2}}}   \mid  \FILLsym{[}  \FILLsym{[}  \FILLmv{y} \, \FILLnt{t_{{\mathrm{1}}}}  \FILLsym{/}  \FILLmv{x}  \FILLsym{]}  \FILLnt{t_{{\mathrm{2}}}}  \FILLsym{/}  \FILLmv{z}  \FILLsym{]}  \Delta_{{\mathrm{3}}}    }
  \end{math}
\end{center}
transforms into the proof
\begin{center}
  \begin{math}
    $$\mprset{flushleft}
    \inferrule* [right=\tiny Series of Exchanges] {
      $$\mprset{flushleft}
    \inferrule* [right=Impl] {
      \inferrule* [right=] {
          \inferrule* [right=,vdots=1.5em,fraction=\,] {
            \,
          }{\pi_1}          
        }{ \Gamma  \vdash   \FILLnt{t_{{\mathrm{1}}}}  \FILLsym{:}  \FILLnt{A}  \mid  \Delta  }
      \\
      $$\mprset{flushleft}
      \inferrule* [right=Cut] {
        \inferrule* [right=] {
          \inferrule* [right=,vdots=1.5em,fraction=\,] {
            \,
          }{\pi_2}          
        }{ \Gamma_{{\mathrm{1}}}  \FILLsym{,}  \FILLmv{x}  \FILLsym{:}  \FILLnt{B}  \FILLsym{,}  \Gamma_{{\mathrm{2}}}  \vdash   \FILLnt{t_{{\mathrm{2}}}}  \FILLsym{:}  \FILLnt{C}  \mid  \Delta_{{\mathrm{2}}}  }
        \\
        \inferrule* [right=] {
          \inferrule* [right=,vdots=1.5em,fraction=\,] {
            \,
          }{\pi_3}          
        }{ \Gamma_{{\mathrm{3}}}  \FILLsym{,}  \FILLmv{z}  \FILLsym{:}  \FILLnt{C}  \FILLsym{,}  \Gamma_{{\mathrm{4}}}  \vdash  \Delta_{{\mathrm{3}}} }
      }{ \Gamma_{{\mathrm{3}}}  \FILLsym{,}  \Gamma_{{\mathrm{1}}}  \FILLsym{,}  \FILLmv{x}  \FILLsym{:}  \FILLnt{B}  \FILLsym{,}  \Gamma_{{\mathrm{2}}}  \FILLsym{,}  \Gamma_{{\mathrm{4}}}  \vdash   \Delta_{{\mathrm{2}}}  \mid  \FILLsym{[}  \FILLnt{t_{{\mathrm{2}}}}  \FILLsym{/}  \FILLmv{z}  \FILLsym{]}  \Delta_{{\mathrm{3}}}  }
    }{ \Gamma  \FILLsym{,}  \FILLmv{y}  \FILLsym{:}   \FILLnt{A}  \multimap   \FILLnt{B}   \FILLsym{,}  \Gamma_{{\mathrm{3}}}  \FILLsym{,}  \Gamma_{{\mathrm{1}}}  \FILLsym{,}  \Gamma_{{\mathrm{2}}}  \FILLsym{,}  \Gamma_{{\mathrm{4}}}  \vdash     \Delta  \mid   \FILLsym{[}  \FILLmv{y} \, \FILLnt{t_{{\mathrm{1}}}}  \FILLsym{/}  \FILLmv{x}  \FILLsym{]}  \Delta_{{\mathrm{2}}}     \mid  \FILLsym{[}  \FILLmv{y} \, \FILLnt{t_{{\mathrm{1}}}}  \FILLsym{/}  \FILLmv{x}  \FILLsym{]}  \FILLsym{[}  \FILLnt{t_{{\mathrm{2}}}}  \FILLsym{/}  \FILLmv{z}  \FILLsym{]}  \Delta_{{\mathrm{3}}}  }
  }{ \Gamma_{{\mathrm{3}}}  \FILLsym{,}  \Gamma  \FILLsym{,}  \FILLmv{y}  \FILLsym{:}   \FILLnt{A}  \multimap   \FILLnt{B}   \FILLsym{,}  \Gamma_{{\mathrm{1}}}  \FILLsym{,}  \Gamma_{{\mathrm{2}}}  \FILLsym{,}  \Gamma_{{\mathrm{4}}}  \vdash     \Delta  \mid   \FILLsym{[}  \FILLmv{y} \, \FILLnt{t_{{\mathrm{1}}}}  \FILLsym{/}  \FILLmv{x}  \FILLsym{]}  \Delta_{{\mathrm{2}}}     \mid  \FILLsym{[}  \FILLmv{y} \, \FILLnt{t_{{\mathrm{1}}}}  \FILLsym{/}  \FILLmv{x}  \FILLsym{]}  \FILLsym{[}  \FILLnt{t_{{\mathrm{2}}}}  \FILLsym{/}  \FILLmv{z}  \FILLsym{]}  \Delta_{{\mathrm{3}}}  }
  \end{math}
\end{center}
This case holds because we can prove that $\FILLsym{[}  \FILLmv{y} \, \FILLnt{t_{{\mathrm{1}}}}  \FILLsym{/}  \FILLmv{x}  \FILLsym{]}  \FILLsym{[}  \FILLnt{t_{{\mathrm{2}}}}  \FILLsym{/}  \FILLmv{z}  \FILLsym{]}  \Delta_{{\mathrm{3}}}  \FILLsym{=}  \FILLsym{[}  \FILLsym{[}  \FILLmv{y} \, \FILLnt{t_{{\mathrm{1}}}}  \FILLsym{/}  \FILLmv{x}  \FILLsym{]}  \FILLnt{t_{{\mathrm{2}}}}  \FILLsym{/}  \FILLmv{z}  \FILLsym{]}  \Delta_{{\mathrm{3}}}$ by Lemma~\ref{lemma:substitution_distribution}
and the fact that $ \FILLmv{x}  \not\in \mathsf{FV}(  \Delta_{{\mathrm{3}}}  ) $.

\item[Case:] secondary conclusion: left introduction of exchange.
  The proof
\begin{center}
  \begin{math}
    $$\mprset{flushleft}
    \inferrule* [right=Cut] {
      $$\mprset{flushleft}
      \inferrule* [right=Exl] {
        \inferrule* [right=] {
          \inferrule* [right=,vdots=1.5em,fraction=\,] {
            \,
          }{\pi_1}          
        }{ \Gamma  \FILLsym{,}  \FILLmv{y}  \FILLsym{:}  \FILLnt{B}  \FILLsym{,}  \FILLmv{x}  \FILLsym{:}  \FILLnt{A}  \FILLsym{,}  \Gamma'  \vdash   \FILLnt{t}  \FILLsym{:}  \FILLnt{C}  \mid  \Delta  }        
      }{ \Gamma  \FILLsym{,}  \FILLmv{x}  \FILLsym{:}  \FILLnt{A}  \FILLsym{,}  \FILLmv{y}  \FILLsym{:}  \FILLnt{B}  \FILLsym{,}  \Gamma'  \vdash   \FILLnt{t}  \FILLsym{:}  \FILLnt{C}  \mid  \Delta  }
      \\
      \inferrule* [right=] {
        \inferrule* [right=,vdots=1.5em,fraction=\,] {
          \,
        }{\pi_2}          
      }{ \Gamma_{{\mathrm{1}}}  \FILLsym{,}  \FILLmv{z}  \FILLsym{:}  \FILLnt{C}  \FILLsym{,}  \Gamma_{{\mathrm{2}}}  \vdash  \Delta_{{\mathrm{2}}} }
    }{ \Gamma_{{\mathrm{1}}}  \FILLsym{,}  \Gamma  \FILLsym{,}  \FILLmv{x}  \FILLsym{:}  \FILLnt{A}  \FILLsym{,}  \FILLmv{y}  \FILLsym{:}  \FILLnt{B}  \FILLsym{,}  \Gamma'  \FILLsym{,}  \Gamma_{{\mathrm{2}}}  \vdash   \Delta  \mid  \FILLsym{[}  \FILLnt{t}  \FILLsym{/}  \FILLmv{z}  \FILLsym{]}  \Delta_{{\mathrm{2}}}  }
  \end{math}
\end{center}
transforms into the proof
\begin{center}
  \begin{math}
    $$\mprset{flushleft}
    \inferrule* [right=Exl] {
      $$\mprset{flushleft}
      \inferrule* [right=Cut] {
        \inferrule* [right=] {
          \inferrule* [right=,vdots=1.5em,fraction=\,] {
            \,
          }{\pi_1}          
        }{ \Gamma  \FILLsym{,}  \FILLmv{y}  \FILLsym{:}  \FILLnt{B}  \FILLsym{,}  \FILLmv{x}  \FILLsym{:}  \FILLnt{A}  \FILLsym{,}  \Gamma'  \vdash   \FILLnt{t}  \FILLsym{:}  \FILLnt{C}  \mid  \Delta  }        
        \\
        \inferrule* [right=] {
        \inferrule* [right=,vdots=1.5em,fraction=\,] {
          \,
        }{\pi_2}          
      }{ \Gamma_{{\mathrm{1}}}  \FILLsym{,}  \FILLmv{z}  \FILLsym{:}  \FILLnt{C}  \FILLsym{,}  \Gamma_{{\mathrm{2}}}  \vdash  \Delta_{{\mathrm{2}}} }
    }{ \Gamma_{{\mathrm{1}}}  \FILLsym{,}  \Gamma  \FILLsym{,}  \FILLmv{y}  \FILLsym{:}  \FILLnt{B}  \FILLsym{,}  \FILLmv{x}  \FILLsym{:}  \FILLnt{A}  \FILLsym{,}  \Gamma'  \FILLsym{,}  \Gamma_{{\mathrm{2}}}  \vdash   \Delta  \mid  \FILLsym{[}  \FILLnt{t}  \FILLsym{/}  \FILLmv{z}  \FILLsym{]}  \Delta_{{\mathrm{2}}}  }
  }{ \Gamma_{{\mathrm{1}}}  \FILLsym{,}  \Gamma  \FILLsym{,}  \FILLmv{x}  \FILLsym{:}  \FILLnt{A}  \FILLsym{,}  \FILLmv{y}  \FILLsym{:}  \FILLnt{B}  \FILLsym{,}  \Gamma'  \FILLsym{,}  \Gamma_{{\mathrm{2}}}  \vdash   \Delta  \mid  \FILLsym{[}  \FILLnt{t}  \FILLsym{/}  \FILLmv{z}  \FILLsym{]}  \Delta_{{\mathrm{2}}}  }
  \end{math}
\end{center}
Clearly, all terms are equivalent.

\item[Case:] secondary conclusion: left introduction of tensor.
The proof 
\begin{center}
  \begin{math}
    $$\mprset{flushleft}
    \inferrule* [right=Cut] {
      $$\mprset{flushleft}
      \inferrule* [right=Tl] {
        \inferrule* [right=] {
        \inferrule* [right=,vdots=1.5em,fraction=\,] {
          \,
        }{\pi_1}          
      }{ \Gamma  \FILLsym{,}  \FILLmv{x}  \FILLsym{:}  \FILLnt{A}  \FILLsym{,}  \FILLmv{y}  \FILLsym{:}  \FILLnt{B}  \vdash   \FILLnt{t}  \FILLsym{:}  \FILLnt{C}  \mid  \Delta  }      
      }{ \Gamma  \FILLsym{,}  \FILLmv{z}  \FILLsym{:}   \FILLnt{A}  \otimes  \FILLnt{B}   \vdash     \mathsf{let}\, \FILLmv{z} \,\mathsf{be}\,  \FILLmv{x}  \otimes  \FILLmv{y}  \,\mathsf{in}\, \FILLnt{t}    \FILLsym{:}  \FILLnt{C}  \mid   \mathsf{let}\, \FILLmv{z} \,\mathsf{be}\,  \FILLmv{x}  \otimes  \FILLmv{y}  \,\mathsf{in}\, \Delta   }
      \\
      \inferrule* [right=] {
        \inferrule* [right=,vdots=1.5em,fraction=\,] {
          \,
        }{\pi_2}          
      }{ \Gamma_{{\mathrm{1}}}  \FILLsym{,}  \FILLmv{w}  \FILLsym{:}  \FILLnt{C}  \FILLsym{,}  \Gamma_{{\mathrm{2}}}  \vdash  \Delta_{{\mathrm{2}}} }
    }{ \Gamma_{{\mathrm{1}}}  \FILLsym{,}  \Gamma  \FILLsym{,}  \FILLmv{z}  \FILLsym{:}   \FILLnt{A}  \otimes  \FILLnt{B}   \FILLsym{,}  \Gamma_{{\mathrm{2}}}  \vdash     \mathsf{let}\, \FILLmv{z} \,\mathsf{be}\,  \FILLmv{x}  \otimes  \FILLmv{y}  \,\mathsf{in}\, \Delta    \mid  \FILLsym{[}    \mathsf{let}\, \FILLmv{z} \,\mathsf{be}\,  \FILLmv{x}  \otimes  \FILLmv{y}  \,\mathsf{in}\, \FILLnt{t}    \FILLsym{/}  \FILLmv{w}  \FILLsym{]}  \Delta_{{\mathrm{2}}}  }
  \end{math}
\end{center}
transforms into the proof
\begin{center}
  \begin{math}
    $$\mprset{flushleft}
    \inferrule* [right=Tl] {
      $$\mprset{flushleft}
      \inferrule* [right=Cut] {
        \inferrule* [right=] {
        \inferrule* [right=,vdots=1.5em,fraction=\,] {
          \,
        }{\pi_1}          
      }{ \Gamma  \FILLsym{,}  \FILLmv{x}  \FILLsym{:}  \FILLnt{A}  \FILLsym{,}  \FILLmv{y}  \FILLsym{:}  \FILLnt{B}  \vdash   \FILLnt{t}  \FILLsym{:}  \FILLnt{C}  \mid  \Delta  }      
      \\
      \inferrule* [right=] {
        \inferrule* [right=,vdots=1.5em,fraction=\,] {
          \,
        }{\pi_2}          
      }{ \Gamma_{{\mathrm{1}}}  \FILLsym{,}  \FILLmv{w}  \FILLsym{:}  \FILLnt{C}  \FILLsym{,}  \Gamma_{{\mathrm{2}}}  \vdash  \Delta_{{\mathrm{2}}} }
    }{ \Gamma_{{\mathrm{1}}}  \FILLsym{,}  \Gamma  \FILLsym{,}  \FILLmv{x}  \FILLsym{:}  \FILLnt{A}  \FILLsym{,}  \FILLmv{y}  \FILLsym{:}  \FILLnt{B}  \FILLsym{,}  \Gamma_{{\mathrm{2}}}  \vdash   \Delta  \mid  \FILLsym{[}  \FILLnt{t}  \FILLsym{/}  \FILLmv{w}  \FILLsym{]}  \Delta_{{\mathrm{2}}}  }      
  }{ \Gamma_{{\mathrm{1}}}  \FILLsym{,}  \Gamma  \FILLsym{,}  \FILLmv{z}  \FILLsym{:}   \FILLnt{A}  \otimes  \FILLnt{B}   \FILLsym{,}  \Gamma_{{\mathrm{2}}}  \vdash     \mathsf{let}\, \FILLmv{z} \,\mathsf{be}\,  \FILLmv{x}  \otimes  \FILLmv{y}  \,\mathsf{in}\, \Delta    \mid   \mathsf{let}\, \FILLmv{z} \,\mathsf{be}\,  \FILLmv{x}  \otimes  \FILLmv{y}  \,\mathsf{in}\, \FILLsym{(}  \FILLsym{[}  \FILLnt{t}  \FILLsym{/}  \FILLmv{w}  \FILLsym{]}  \Delta_{{\mathrm{2}}}  \FILLsym{)}   }
  \end{math}
\end{center}
It suffices to show that $ \mathsf{let}\, \FILLmv{z} \,\mathsf{be}\,  \FILLmv{x}  \otimes  \FILLmv{y}  \,\mathsf{in}\, \FILLsym{(}  \FILLsym{[}  \FILLnt{t}  \FILLsym{/}  \FILLmv{w}  \FILLsym{]}  \Delta_{{\mathrm{2}}}  \FILLsym{)}   \FILLsym{=}  \FILLsym{[}    \mathsf{let}\, \FILLmv{z} \,\mathsf{be}\,  \FILLmv{x}  \otimes  \FILLmv{y}  \,\mathsf{in}\, \FILLnt{t}    \FILLsym{/}  \FILLmv{w}  \FILLsym{]}  \Delta_{{\mathrm{2}}}$.  This is a simple consequence of the rule
$\FILLdrulename{Eq\_NatTensor}$.

\item[Case:] secondary conclusion: left introduction of Par
The proof 
\begin{center}
  \scriptsize
  \begin{math}
    $$\mprset{flushleft}
    \inferrule* [right=\scriptsize Cut] {
      $$\mprset{flushleft}
      \inferrule* [right=\scriptsize Parl] {
        \inferrule* [right=] {
          \inferrule* [right=,vdots=1.5em,fraction=\,] {
            \,
          }{\pi_1}          
        }{ \Gamma  \FILLsym{,}  \FILLmv{x}  \FILLsym{:}  \FILLnt{A}  \vdash  \Delta }      
        \\
        \inferrule* [right=] {
          \inferrule* [right=,vdots=1.5em,fraction=\,] {
            \,
          }{\pi_2}          
        }{ \Gamma'  \FILLsym{,}  \FILLmv{y}  \FILLsym{:}  \FILLnt{B}  \vdash   \FILLnt{t'}  \FILLsym{:}  \FILLnt{C}  \mid  \Delta'  }      
      }{ \Gamma  \FILLsym{,}  \Gamma'  \FILLsym{,}  \FILLmv{z}  \FILLsym{:}   \FILLnt{A}  \parr  \FILLnt{B}   \vdash       \mathsf{\operatorname{let-pat} }\, \FILLmv{z} \, \FILLsym{(}   \FILLmv{x}  \parr   -    \FILLsym{)} \, \Delta    \mid    \mathsf{\operatorname{let-pat} }\, \FILLmv{z} \, \FILLsym{(}    -   \parr  \FILLmv{y}   \FILLsym{)} \, \FILLnt{t'}    \FILLsym{:}  \FILLnt{C}    \mid    \mathsf{\operatorname{let-pat} }\, \FILLmv{z} \, \FILLsym{(}    -   \parr  \FILLmv{y}   \FILLsym{)} \, \Delta'    }
      \\
      \inferrule* [right=] {
        \inferrule* [right=,vdots=1.5em,fraction=\,] {
          \,
        }{\pi_3}          
      }{ \Gamma_{{\mathrm{1}}}  \FILLsym{,}  \FILLmv{w}  \FILLsym{:}  \FILLnt{C}  \FILLsym{,}  \Gamma_{{\mathrm{2}}}  \vdash  \Delta_{{\mathrm{2}}} }
    }{ \Gamma_{{\mathrm{1}}}  \FILLsym{,}  \Gamma  \FILLsym{,}  \Gamma'  \FILLsym{,}  \FILLmv{z}  \FILLsym{:}   \FILLnt{A}  \parr  \FILLnt{B}   \FILLsym{,}  \Gamma_{{\mathrm{2}}}  \vdash       \mathsf{\operatorname{let-pat} }\, \FILLmv{z} \, \FILLsym{(}   \FILLmv{x}  \parr   -    \FILLsym{)} \, \Delta    \mid    \mathsf{\operatorname{let-pat} }\, \FILLmv{z} \, \FILLsym{(}    -   \parr  \FILLmv{y}   \FILLsym{)} \, \Delta'      \mid  \FILLsym{[}    \mathsf{\operatorname{let-pat} }\, \FILLmv{z} \, \FILLsym{(}    -   \parr  \FILLmv{y}   \FILLsym{)} \, \FILLnt{t'}    \FILLsym{/}  \FILLmv{w}  \FILLsym{]}  \Delta_{{\mathrm{2}}}  }
  \end{math}
\end{center}
is transformed into the proof
\begin{center}
  \begin{math}
    $$\mprset{flushleft}
    \inferrule* [right=\tiny Series of Exchanges] {
      $$\mprset{flushleft}
    \inferrule* [right=Parl] {
      \inferrule* [right=] {
        \inferrule* [right=,vdots=1.5em,fraction=\,] {
          \,
        }{\pi_1}          
      }{ \Gamma  \FILLsym{,}  \FILLmv{x}  \FILLsym{:}  \FILLnt{A}  \vdash  \Delta }      
      \\
      $$\mprset{flushleft}
      \inferrule* [right=Cut] {
        \inferrule* [right=] {
          \inferrule* [right=,vdots=1.5em,fraction=\,] {
            \,
          }{\pi_2}          
        }{ \Gamma'  \FILLsym{,}  \FILLmv{y}  \FILLsym{:}  \FILLnt{B}  \vdash   \FILLnt{t'}  \FILLsym{:}  \FILLnt{C}  \mid  \Delta'  }      
        \\
        \inferrule* [right=] {
          \inferrule* [right=,vdots=1.5em,fraction=\,] {
            \,
          }{\pi_3}          
        }{ \Gamma_{{\mathrm{1}}}  \FILLsym{,}  \FILLmv{w}  \FILLsym{:}  \FILLnt{C}  \FILLsym{,}  \Gamma_{{\mathrm{2}}}  \vdash  \Delta_{{\mathrm{2}}} }
      }{ \Gamma_{{\mathrm{1}}}  \FILLsym{,}  \Gamma'  \FILLsym{,}  \FILLmv{y}  \FILLsym{:}  \FILLnt{B}  \FILLsym{,}  \Gamma_{{\mathrm{2}}}  \vdash   \Delta'  \mid  \FILLsym{[}  \FILLnt{t'}  \FILLsym{/}  \FILLmv{w}  \FILLsym{]}  \Delta_{{\mathrm{2}}}  }
    }{ \Gamma  \FILLsym{,}  \Gamma_{{\mathrm{1}}}  \FILLsym{,}  \Gamma'  \FILLsym{,}  \Gamma_{{\mathrm{2}}}  \FILLsym{,}  \FILLmv{z}  \FILLsym{:}   \FILLnt{A}  \parr  \FILLnt{B}   \vdash       \mathsf{\operatorname{let-pat} }\, \FILLmv{z} \, \FILLsym{(}   \FILLmv{x}  \parr   -    \FILLsym{)} \, \Delta    \mid    \mathsf{\operatorname{let-pat} }\, \FILLmv{z} \, \FILLsym{(}    -   \parr  \FILLmv{y}   \FILLsym{)} \, \Delta'      \mid    \mathsf{\operatorname{let-pat} }\, \FILLmv{z} \, \FILLsym{(}    -   \parr  \FILLmv{y}   \FILLsym{)} \, \FILLsym{[}  \FILLnt{t'}  \FILLsym{/}  \FILLmv{w}  \FILLsym{]}  \Delta_{{\mathrm{2}}}    }
  }{ \Gamma_{{\mathrm{1}}}  \FILLsym{,}  \Gamma  \FILLsym{,}  \Gamma'  \FILLsym{,}  \FILLmv{z}  \FILLsym{:}   \FILLnt{A}  \parr  \FILLnt{B}   \FILLsym{,}  \Gamma_{{\mathrm{2}}}  \vdash       \mathsf{\operatorname{let-pat} }\, \FILLmv{z} \, \FILLsym{(}   \FILLmv{x}  \parr   -    \FILLsym{)} \, \Delta    \mid    \mathsf{\operatorname{let-pat} }\, \FILLmv{z} \, \FILLsym{(}    -   \parr  \FILLmv{y}   \FILLsym{)} \, \Delta'      \mid    \mathsf{\operatorname{let-pat} }\, \FILLmv{z} \, \FILLsym{(}    -   \parr  \FILLmv{y}   \FILLsym{)} \, \FILLsym{[}  \FILLnt{t'}  \FILLsym{/}  \FILLmv{w}  \FILLsym{]}  \Delta_{{\mathrm{2}}}    }
  \end{math}
\end{center}
It suffices to show that $ \mathsf{\operatorname{let-pat} }\, \FILLmv{z} \, \FILLsym{(}    -   \parr  \FILLmv{y}   \FILLsym{)} \, \FILLsym{[}  \FILLnt{t'}  \FILLsym{/}  \FILLmv{w}  \FILLsym{]}  \Delta_{{\mathrm{2}}}  = \FILLsym{[}    \mathsf{\operatorname{let-pat} }\, \FILLmv{z} \, \FILLsym{(}    -   \parr  \FILLmv{y}   \FILLsym{)} \, \FILLnt{t'}    \FILLsym{/}  \FILLmv{w}  \FILLsym{]}  \Delta_{{\mathrm{2}}}$.  This follows from the rule
$\FILLdrulename{Eq\_Nat2Par}$.

\item[Case:] secondary conclusion: left introduction of tensor unit.
The proof
\begin{center}
  \begin{math}
    $$\mprset{flushleft}
    \inferrule* [right=Cut] {
      $$\mprset{flushleft}
      \inferrule* [right=Il] {
        \inferrule* [right=] {
        \inferrule* [right=,vdots=1.5em,fraction=\,] {
          \,
        }{\pi_1}          
      }{ \Gamma  \vdash   \FILLnt{t}  \FILLsym{:}  \FILLnt{C}  \mid  \Delta  }      
      }{ \Gamma  \FILLsym{,}  \FILLmv{x}  \FILLsym{:}   \top   \vdash   \FILLnt{t}  \FILLsym{:}  \FILLnt{C}  \mid  \Delta  }
      \\
      \inferrule* [right=] {
          \inferrule* [right=,vdots=1.5em,fraction=\,] {
            \,
          }{\pi_2}          
        }{ \Gamma_{{\mathrm{1}}}  \FILLsym{,}  \FILLmv{w}  \FILLsym{:}  \FILLnt{C}  \FILLsym{,}  \Gamma_{{\mathrm{2}}}  \vdash  \Delta_{{\mathrm{1}}} }
      }{ \Gamma_{{\mathrm{1}}}  \FILLsym{,}  \Gamma  \FILLsym{,}  \FILLmv{x}  \FILLsym{:}   \top   \FILLsym{,}  \Gamma_{{\mathrm{2}}}  \vdash   \Delta  \mid  \FILLsym{[}  \FILLnt{t}  \FILLsym{/}  \FILLmv{w}  \FILLsym{]}  \Delta_{{\mathrm{1}}}  }
  \end{math}
\end{center}
is transformed into the following:
\begin{center}
  \begin{math}
    $$\mprset{flushleft}
    \inferrule* [right=\tiny Series of Exchanges] {
      $$\mprset{flushleft}
    \inferrule* [right=Il] {
      $$\mprset{flushleft}
      \inferrule* [right=Cut] {
        \inferrule* [right=] {
          \inferrule* [right=,vdots=1.5em,fraction=\,] {
            \,
          }{\pi_1}          
        }{ \Gamma  \vdash   \FILLnt{t}  \FILLsym{:}  \FILLnt{C}  \mid  \Delta  }      
        \\
        \inferrule* [right=] {
          \inferrule* [right=,vdots=1.5em,fraction=\,] {
            \,
          }{\pi_2}          
        }{ \Gamma_{{\mathrm{1}}}  \FILLsym{,}  \FILLmv{w}  \FILLsym{:}  \FILLnt{C}  \FILLsym{,}  \Gamma_{{\mathrm{2}}}  \vdash  \Delta_{{\mathrm{1}}} }  
      }{ \Gamma_{{\mathrm{1}}}  \FILLsym{,}  \Gamma  \FILLsym{,}  \Gamma_{{\mathrm{2}}}  \vdash   \Delta  \mid  \FILLsym{[}  \FILLnt{t}  \FILLsym{/}  \FILLmv{w}  \FILLsym{]}  \Delta_{{\mathrm{1}}}  }
    }{ \Gamma_{{\mathrm{1}}}  \FILLsym{,}  \Gamma  \FILLsym{,}  \Gamma_{{\mathrm{2}}}  \FILLsym{,}  \FILLmv{x}  \FILLsym{:}   \top   \vdash   \Delta  \mid  \FILLsym{[}  \FILLnt{t}  \FILLsym{/}  \FILLmv{w}  \FILLsym{]}  \Delta_{{\mathrm{1}}}  }
  }{ \Gamma_{{\mathrm{1}}}  \FILLsym{,}  \Gamma  \FILLsym{,}  \FILLmv{x}  \FILLsym{:}   \top   \FILLsym{,}  \Gamma_{{\mathrm{2}}}  \vdash   \Delta  \mid  \FILLsym{[}  \FILLnt{t}  \FILLsym{/}  \FILLmv{w}  \FILLsym{]}  \Delta_{{\mathrm{1}}}  }
  \end{math}
\end{center}
Clearly, all terms are equivalent.  Note that we do not give a case
for secondary conclusion of the left introduction of par's unit,
because it can only be introduced given an empty right context, and
thus there is no cut formula.

\item[Case:] secondary hypothesis: left introduction of tensor.
The proof
\begin{center}
  \begin{math}
    $$\mprset{flushleft}
    \inferrule* [right=Cut] {
      \inferrule* [right=] {
        \inferrule* [right=,vdots=1.5em,fraction=\,] {
          \,
        }{\pi_1}          
      }{ \Gamma  \vdash   \FILLnt{t}  \FILLsym{:}  \FILLnt{A}  \mid  \Delta  }      
      \\
      $$\mprset{flushleft}
      \inferrule* [right=Tl] {
        \inferrule* [right=] {
        \inferrule* [right=,vdots=1.5em,fraction=\,] {
          \,
        }{\pi_2}          
      }{ \Gamma_{{\mathrm{1}}}  \FILLsym{,}  \FILLmv{x}  \FILLsym{:}  \FILLnt{A}  \FILLsym{,}  \Gamma_{{\mathrm{2}}}  \FILLsym{,}  \FILLmv{y}  \FILLsym{:}  \FILLnt{B}  \FILLsym{,}  \FILLmv{z}  \FILLsym{:}  \FILLnt{C}  \FILLsym{,}  \Gamma_{{\mathrm{3}}}  \vdash   \FILLnt{t_{{\mathrm{1}}}}  \FILLsym{:}  \FILLnt{D}  \mid  \Delta_{{\mathrm{1}}}  }                  
    }{ \Gamma_{{\mathrm{1}}}  \FILLsym{,}  \FILLmv{x}  \FILLsym{:}  \FILLnt{A}  \FILLsym{,}  \Gamma_{{\mathrm{2}}}  \FILLsym{,}  \FILLmv{w}  \FILLsym{:}   \FILLnt{B}  \otimes  \FILLnt{C}   \FILLsym{,}  \Gamma_{{\mathrm{3}}}  \vdash     \mathsf{let}\, \FILLmv{w} \,\mathsf{be}\,  \FILLmv{y}  \otimes  \FILLmv{z}  \,\mathsf{in}\, \FILLnt{t_{{\mathrm{1}}}}    \FILLsym{:}  \FILLnt{D}  \mid   \mathsf{let}\, \FILLmv{w} \,\mathsf{be}\,  \FILLmv{y}  \otimes  \FILLmv{z}  \,\mathsf{in}\, \Delta_{{\mathrm{1}}}   }
  }{ \Gamma_{{\mathrm{1}}}  \FILLsym{,}  \Gamma  \FILLsym{,}  \Gamma_{{\mathrm{2}}}  \FILLsym{,}  \FILLmv{w}  \FILLsym{:}   \FILLnt{B}  \otimes  \FILLnt{C}   \FILLsym{,}  \Gamma_{{\mathrm{3}}}  \vdash   \Delta  \mid     \FILLsym{[}  \FILLnt{t}  \FILLsym{/}  \FILLmv{x}  \FILLsym{]}   (  \mathsf{let}\, \FILLmv{w} \,\mathsf{be}\,  \FILLmv{y}  \otimes  \FILLmv{z}  \,\mathsf{in}\, \FILLnt{t_{{\mathrm{1}}}}  )    \FILLsym{:}  \FILLnt{D}  \mid  \FILLsym{[}  \FILLnt{t}  \FILLsym{/}  \FILLmv{x}  \FILLsym{]}  \FILLsym{(}   \mathsf{let}\, \FILLmv{w} \,\mathsf{be}\,  \FILLmv{y}  \otimes  \FILLmv{z}  \,\mathsf{in}\, \Delta_{{\mathrm{1}}}   \FILLsym{)}    }
  \end{math}
\end{center}
transforms into the proof
\begin{center}
  \begin{math}
    $$\mprset{flushleft}
    \inferrule* [right=Tl] {
      $$\mprset{flushleft}
      \inferrule* [right=Cut] {
        \inferrule* [right=] {
        \inferrule* [right=,vdots=1.5em,fraction=\,] {
          \,
        }{\pi_1}          
      }{ \Gamma  \vdash   \FILLnt{t}  \FILLsym{:}  \FILLnt{A}  \mid  \Delta  }      
      \\
      \inferrule* [right=] {
        \inferrule* [right=,vdots=1.5em,fraction=\,] {
          \,
        }{\pi_2}          
      }{ \Gamma_{{\mathrm{1}}}  \FILLsym{,}  \FILLmv{x}  \FILLsym{:}  \FILLnt{A}  \FILLsym{,}  \Gamma_{{\mathrm{2}}}  \FILLsym{,}  \FILLmv{y}  \FILLsym{:}  \FILLnt{B}  \FILLsym{,}  \FILLmv{z}  \FILLsym{:}  \FILLnt{C}  \FILLsym{,}  \Gamma_{{\mathrm{3}}}  \vdash   \FILLnt{t_{{\mathrm{1}}}}  \FILLsym{:}  \FILLnt{D}  \mid  \Delta_{{\mathrm{1}}}  }                  
    }{ \Gamma_{{\mathrm{1}}}  \FILLsym{,}  \Gamma  \FILLsym{,}  \Gamma_{{\mathrm{2}}}  \FILLsym{,}  \FILLmv{y}  \FILLsym{:}  \FILLnt{B}  \FILLsym{,}  \FILLmv{z}  \FILLsym{:}  \FILLnt{C}  \FILLsym{,}  \Gamma_{{\mathrm{3}}}  \vdash     \Delta  \mid   \FILLsym{[}  \FILLnt{t}  \FILLsym{/}  \FILLmv{x}  \FILLsym{]}  \FILLnt{t_{{\mathrm{1}}}}   \FILLsym{:}  \FILLnt{D}    \mid  \FILLsym{[}  \FILLnt{t}  \FILLsym{/}  \FILLmv{x}  \FILLsym{]}  \Delta_{{\mathrm{1}}}  }
  }{ \Gamma_{{\mathrm{1}}}  \FILLsym{,}  \Gamma  \FILLsym{,}  \Gamma_{{\mathrm{2}}}  \FILLsym{,}  \FILLmv{w}  \FILLsym{:}   \FILLnt{B}  \otimes  \FILLnt{C}   \FILLsym{,}  \Gamma_{{\mathrm{3}}}  \vdash       \mathsf{let}\, \FILLmv{w} \,\mathsf{be}\,  \FILLmv{x}  \otimes  \FILLmv{y}  \,\mathsf{in}\, \Delta    \mid    \mathsf{let}\, \FILLmv{w} \,\mathsf{be}\,  \FILLmv{x}  \otimes  \FILLmv{y}  \,\mathsf{in}\, \FILLsym{[}  \FILLnt{t}  \FILLsym{/}  \FILLmv{x}  \FILLsym{]}  \FILLnt{t_{{\mathrm{1}}}}    \FILLsym{:}  \FILLnt{D}    \mid   \mathsf{let}\, \FILLmv{w} \,\mathsf{be}\,  \FILLmv{x}  \otimes  \FILLmv{y}  \,\mathsf{in}\, \FILLsym{[}  \FILLnt{t}  \FILLsym{/}  \FILLmv{x}  \FILLsym{]}  \Delta_{{\mathrm{1}}}   }
  \end{math}
\end{center}
First, we can see by inspection of the previous derivations that
$ \FILLmv{x} , \FILLmv{y}  \not\in \mathsf{FV}(  \Delta  ) $, thus, by using similar reasoning as above we can
use the $\FILLdrulename{EtaTensor}$ rule to obtain $ \mathsf{let}\, \FILLmv{w} \,\mathsf{be}\,  \FILLmv{x}  \otimes  \FILLmv{y}  \,\mathsf{in}\, \Delta   \FILLsym{=}  \Delta$.  It is a well-known property of substitution that 
$\FILLsym{[}  \FILLnt{t}  \FILLsym{/}  \FILLmv{x}  \FILLsym{]}   (  \mathsf{let}\, \FILLmv{w} \,\mathsf{be}\,  \FILLmv{x}  \otimes  \FILLmv{y}  \,\mathsf{in}\, \FILLnt{t_{{\mathrm{1}}}}  )  =  \mathsf{let}\, \FILLsym{[}  \FILLnt{t}  \FILLsym{/}  \FILLmv{x}  \FILLsym{]}  \FILLmv{w} \,\mathsf{be}\,  \FILLmv{x}  \otimes  \FILLmv{y}  \,\mathsf{in}\, \FILLsym{[}  \FILLnt{t}  \FILLsym{/}  \FILLmv{x}  \FILLsym{]}  \FILLnt{t_{{\mathrm{1}}}}  =  \mathsf{let}\, \FILLmv{w} \,\mathsf{be}\,  \FILLmv{x}  \otimes  \FILLmv{y}  \,\mathsf{in}\, \FILLsym{[}  \FILLnt{t}  \FILLsym{/}  \FILLmv{x}  \FILLsym{]}  \FILLnt{t_{{\mathrm{1}}}} $.

\item[Case:] secondary hypothesis: right introduction of tensor (first
  case).
The proof
\begin{center}
  \begin{math}
    $$\mprset{flushleft}
    \inferrule* [right=Cut] {
      \inferrule* [right=] {
        \inferrule* [right=,vdots=1.5em,fraction=\,] {
          \,
        }{\pi_1}          
      }{ \Gamma  \vdash   \FILLnt{t}  \FILLsym{:}  \FILLnt{A}  \mid  \Delta  }      
      \\
      $$\mprset{flushleft}
      \inferrule* [right=Tr] {
        \inferrule* [right=] {
        \inferrule* [right=,vdots=1.5em,fraction=\,] {
          \,
        }{\pi_2}          
      }{ \Gamma_{{\mathrm{1}}}  \FILLsym{,}  \FILLmv{x}  \FILLsym{:}  \FILLnt{A}  \FILLsym{,}  \Gamma_{{\mathrm{2}}}  \vdash   \FILLnt{t_{{\mathrm{1}}}}  \FILLsym{:}  \FILLnt{B}  \mid  \Delta_{{\mathrm{1}}}  }      
      \\
      \inferrule* [right=] {
        \inferrule* [right=,vdots=1.5em,fraction=\,] {
          \,
        }{\pi_3}          
      }{ \Gamma_{{\mathrm{3}}}  \vdash   \FILLnt{t_{{\mathrm{2}}}}  \FILLsym{:}  \FILLnt{C}  \mid  \Delta_{{\mathrm{2}}}  }      
      }{ \Gamma_{{\mathrm{1}}}  \FILLsym{,}  \FILLmv{x}  \FILLsym{:}  \FILLnt{A}  \FILLsym{,}  \Gamma_{{\mathrm{2}}}  \FILLsym{,}  \Gamma_{{\mathrm{3}}}  \vdash    \FILLnt{t_{{\mathrm{1}}}}  \otimes  \FILLnt{t_{{\mathrm{2}}}}   \FILLsym{:}   \FILLnt{B}  \otimes  \FILLnt{C}   \mid    \Delta_{{\mathrm{1}}}  \mid  \Delta_{{\mathrm{2}}}    }
    }{ \Gamma_{{\mathrm{1}}}  \FILLsym{,}  \Gamma  \FILLsym{,}  \Gamma_{{\mathrm{2}}}  \FILLsym{,}  \Gamma_{{\mathrm{3}}}  \vdash   \Delta  \mid     \FILLsym{[}  \FILLnt{t}  \FILLsym{/}  \FILLmv{x}  \FILLsym{]}   (  \FILLnt{t_{{\mathrm{1}}}}  \otimes  \FILLnt{t_{{\mathrm{2}}}}  )    \FILLsym{:}   \FILLnt{B}  \otimes  \FILLnt{C}   \mid     \FILLsym{[}  \FILLnt{t}  \FILLsym{/}  \FILLmv{x}  \FILLsym{]}  \Delta_{{\mathrm{1}}}   \mid  \FILLsym{[}  \FILLnt{t}  \FILLsym{/}  \FILLmv{x}  \FILLsym{]}  \Delta_{{\mathrm{2}}}      }
  \end{math}
\end{center}
transforms into the proof
\begin{center}
  \begin{math}
    $$\mprset{flushleft}
    \inferrule* [right=\tiny Series of Exchanges] {
      $$\mprset{flushleft}
    \inferrule* [right=Tr] {
      $$\mprset{flushleft}
      \inferrule* [right=Cut] {
        \inferrule* [right=] {
        \inferrule* [right=,vdots=1.5em,fraction=\,] {
          \,
        }{\pi_1}          
      }{ \Gamma  \vdash   \FILLnt{t}  \FILLsym{:}  \FILLnt{A}  \mid  \Delta  }      
      \\
      \inferrule* [right=] {
        \inferrule* [right=,vdots=1.5em,fraction=\,] {
          \,
        }{\pi_2}          
      }{ \Gamma_{{\mathrm{1}}}  \FILLsym{,}  \FILLmv{x}  \FILLsym{:}  \FILLnt{A}  \FILLsym{,}  \Gamma_{{\mathrm{2}}}  \vdash   \FILLnt{t_{{\mathrm{1}}}}  \FILLsym{:}  \FILLnt{B}  \mid  \Delta_{{\mathrm{1}}}  }      
    }{ \Gamma_{{\mathrm{1}}}  \FILLsym{,}  \Gamma  \FILLsym{,}  \Gamma_{{\mathrm{2}}}  \vdash   \Delta  \mid     \FILLsym{[}  \FILLnt{t}  \FILLsym{/}  \FILLmv{x}  \FILLsym{]}  \FILLnt{t_{{\mathrm{1}}}}   \FILLsym{:}  \FILLnt{B}  \mid  \FILLsym{[}  \FILLnt{t}  \FILLsym{/}  \FILLmv{x}  \FILLsym{]}  \Delta_{{\mathrm{1}}}    }
    \\
    \inferrule* [right=] {
        \inferrule* [right=,vdots=1.5em,fraction=\,] {
          \,
        }{\pi_3}          
      }{ \Gamma_{{\mathrm{3}}}  \vdash   \FILLnt{t_{{\mathrm{2}}}}  \FILLsym{:}  \FILLnt{C}  \mid  \Delta_{{\mathrm{2}}}  }      
    }{ \Gamma_{{\mathrm{1}}}  \FILLsym{,}  \Gamma  \FILLsym{,}  \Gamma_{{\mathrm{2}}}  \FILLsym{,}  \Gamma_{{\mathrm{3}}}  \vdash      \FILLsym{[}  \FILLnt{t}  \FILLsym{/}  \FILLmv{x}  \FILLsym{]}  \FILLnt{t_{{\mathrm{1}}}}   \otimes  \FILLnt{t_{{\mathrm{2}}}}   \FILLsym{:}   \FILLnt{B}  \otimes  \FILLnt{C}    \mid    \Delta  \mid     \FILLsym{[}  \FILLnt{t}  \FILLsym{/}  \FILLmv{x}  \FILLsym{]}  \Delta_{{\mathrm{1}}}   \mid  \Delta_{{\mathrm{2}}}      }
  }{ \Gamma_{{\mathrm{1}}}  \FILLsym{,}  \Gamma  \FILLsym{,}  \Gamma_{{\mathrm{2}}}  \FILLsym{,}  \Gamma_{{\mathrm{3}}}  \vdash   \Delta  \mid       ( \FILLsym{[}  \FILLnt{t}  \FILLsym{/}  \FILLmv{x}  \FILLsym{]}  \FILLnt{t_{{\mathrm{1}}}} )   \otimes  \FILLnt{t_{{\mathrm{2}}}}   \FILLsym{:}   \FILLnt{B}  \otimes  \FILLnt{C}    \mid     \FILLsym{[}  \FILLnt{t}  \FILLsym{/}  \FILLmv{x}  \FILLsym{]}  \Delta_{{\mathrm{1}}}   \mid  \Delta_{{\mathrm{2}}}      }
  \end{math}
\end{center}
By inspection of the previous derivations we can see that $ \FILLmv{x}  \not\in \mathsf{FV}(  \FILLnt{t_{{\mathrm{2}}}}  ) $ and $ \FILLmv{x}  \not\in \mathsf{FV}(  \Delta_{{\mathrm{2}}}  ) $.  Thus, $\FILLsym{[}  \FILLnt{t}  \FILLsym{/}  \FILLmv{x}  \FILLsym{]}  \Delta_{{\mathrm{2}}}  \FILLsym{=}  \Delta_{{\mathrm{2}}}$ and
$\FILLsym{[}  \FILLnt{t}  \FILLsym{/}  \FILLmv{x}  \FILLsym{]}   (  \FILLnt{t_{{\mathrm{1}}}}  \otimes  \FILLnt{t_{{\mathrm{2}}}}  )   \FILLsym{=}    ( \FILLsym{[}  \FILLnt{t}  \FILLsym{/}  \FILLmv{x}  \FILLsym{]}  \FILLnt{t_{{\mathrm{1}}}} )   \otimes   ( \FILLsym{[}  \FILLnt{t}  \FILLsym{/}  \FILLmv{x}  \FILLsym{]}  \FILLnt{t_{{\mathrm{2}}}} )   =   ( \FILLsym{[}  \FILLnt{t}  \FILLsym{/}  \FILLmv{x}  \FILLsym{]}  \FILLnt{t_{{\mathrm{1}}}} )   \otimes  \FILLnt{t_{{\mathrm{2}}}} $.

\item[Case:] secondary hypothesis: right introduction of tensor
  (second case).
The proof
\begin{center}
  \begin{math}
    $$\mprset{flushleft}
    \inferrule* [right=Cut] {
      \inferrule* [right=] {
        \inferrule* [right=,vdots=1.5em,fraction=\,] {
          \,
        }{\pi_1}          
      }{ \Gamma  \vdash   \FILLnt{t}  \FILLsym{:}  \FILLnt{A}  \mid  \Delta  }      
      \\
      $$\mprset{flushleft}
      \inferrule* [right=Tr] {
        \inferrule* [right=] {
        \inferrule* [right=,vdots=1.5em,fraction=\,] {
          \,
        }{\pi_2}          
      }{ \Gamma_{{\mathrm{1}}}  \vdash   \FILLnt{t_{{\mathrm{1}}}}  \FILLsym{:}  \FILLnt{B}  \mid  \Delta_{{\mathrm{1}}}  }      
      \\
      \inferrule* [right=] {
        \inferrule* [right=,vdots=1.5em,fraction=\,] {
          \,
        }{\pi_3}          
      }{ \Gamma_{{\mathrm{2}}}  \FILLsym{,}  \FILLmv{x}  \FILLsym{:}  \FILLnt{A}  \FILLsym{,}  \Gamma_{{\mathrm{3}}}  \vdash   \FILLnt{t_{{\mathrm{2}}}}  \FILLsym{:}  \FILLnt{C}  \mid  \Delta_{{\mathrm{2}}}  }      
      }{ \Gamma_{{\mathrm{1}}}  \FILLsym{,}  \Gamma_{{\mathrm{2}}}  \FILLsym{,}  \FILLmv{x}  \FILLsym{:}  \FILLnt{A}  \FILLsym{,}  \Gamma_{{\mathrm{3}}}  \vdash    \FILLnt{t_{{\mathrm{1}}}}  \otimes  \FILLnt{t_{{\mathrm{2}}}}   \FILLsym{:}   \FILLnt{B}  \otimes  \FILLnt{C}   \mid    \Delta_{{\mathrm{1}}}  \mid  \Delta_{{\mathrm{2}}}    }
    }{ \Gamma_{{\mathrm{1}}}  \FILLsym{,}  \Gamma  \FILLsym{,}  \Gamma_{{\mathrm{2}}}  \FILLsym{,}  \Gamma_{{\mathrm{3}}}  \vdash   \Delta  \mid     \FILLsym{[}  \FILLnt{t}  \FILLsym{/}  \FILLmv{x}  \FILLsym{]}   (  \FILLnt{t_{{\mathrm{1}}}}  \otimes  \FILLnt{t_{{\mathrm{2}}}}  )    \FILLsym{:}   \FILLnt{B}  \otimes  \FILLnt{C}   \mid     \FILLsym{[}  \FILLnt{t}  \FILLsym{/}  \FILLmv{x}  \FILLsym{]}  \Delta_{{\mathrm{1}}}   \mid  \FILLsym{[}  \FILLnt{t}  \FILLsym{/}  \FILLmv{x}  \FILLsym{]}  \Delta_{{\mathrm{2}}}      }
  \end{math}
\end{center}
transforms into the proof
\begin{center}
  \begin{math}
    $$\mprset{flushleft}
    \inferrule* [right=\tiny Series of Exchanges] {
      $$\mprset{flushleft}
      \inferrule* [right=Tr] {
        \inferrule* [right=] {
        \inferrule* [right=,vdots=1.5em,fraction=\,] {
          \,
        }{\pi_2}          
      }{ \Gamma_{{\mathrm{1}}}  \vdash   \FILLnt{t_{{\mathrm{1}}}}  \FILLsym{:}  \FILLnt{B}  \mid  \Delta_{{\mathrm{1}}}  }      
      \\
      $$\mprset{flushleft}
      \inferrule* [right=Cut] {
        \inferrule* [right=] {
        \inferrule* [right=,vdots=1.5em,fraction=\,] {
          \,
        }{\pi_1}          
      }{ \Gamma  \vdash   \FILLnt{t}  \FILLsym{:}  \FILLnt{A}  \mid  \Delta  }      
      \\
      \inferrule* [right=] {
        \inferrule* [right=,vdots=1.5em,fraction=\,] {
          \,
        }{\pi_3}          
      }{ \Gamma_{{\mathrm{2}}}  \FILLsym{,}  \FILLmv{x}  \FILLsym{:}  \FILLnt{A}  \FILLsym{,}  \Gamma_{{\mathrm{3}}}  \vdash   \FILLnt{t_{{\mathrm{2}}}}  \FILLsym{:}  \FILLnt{C}  \mid  \Delta_{{\mathrm{2}}}  }      
    }{ \Gamma_{{\mathrm{2}}}  \FILLsym{,}  \Gamma  \FILLsym{,}  \Gamma_{{\mathrm{3}}}  \vdash   \Delta  \mid     \FILLsym{[}  \FILLnt{t}  \FILLsym{/}  \FILLmv{x}  \FILLsym{]}  \FILLnt{t_{{\mathrm{2}}}}   \FILLsym{:}  \FILLnt{C}  \mid  \FILLsym{[}  \FILLnt{t}  \FILLsym{/}  \FILLmv{x}  \FILLsym{]}  \Delta_{{\mathrm{2}}}    }       
    }{ \Gamma_{{\mathrm{1}}}  \FILLsym{,}  \Gamma_{{\mathrm{2}}}  \FILLsym{,}  \Gamma  \FILLsym{,}  \Gamma_{{\mathrm{3}}}  \vdash     \FILLnt{t_{{\mathrm{1}}}}  \otimes   ( \FILLsym{[}  \FILLnt{t}  \FILLsym{/}  \FILLmv{x}  \FILLsym{]}  \FILLnt{t_{{\mathrm{2}}}} )    \FILLsym{:}   \FILLnt{B}  \otimes  \FILLnt{C}    \mid    \Delta_{{\mathrm{1}}}  \mid    \Delta  \mid  \FILLsym{[}  \FILLnt{t}  \FILLsym{/}  \FILLmv{x}  \FILLsym{]}  \Delta_{{\mathrm{2}}}      }
  }{ \Gamma_{{\mathrm{1}}}  \FILLsym{,}  \Gamma  \FILLsym{,}  \Gamma_{{\mathrm{2}}}  \FILLsym{,}  \Gamma_{{\mathrm{3}}}  \vdash   \Delta  \mid      \FILLnt{t_{{\mathrm{1}}}}  \otimes   ( \FILLsym{[}  \FILLnt{t}  \FILLsym{/}  \FILLmv{x}  \FILLsym{]}  \FILLnt{t_{{\mathrm{2}}}} )    \FILLsym{:}   \FILLnt{B}  \otimes  \FILLnt{C}    \mid    \Delta_{{\mathrm{1}}}  \mid   \FILLsym{[}  \FILLnt{t}  \FILLsym{/}  \FILLmv{x}  \FILLsym{]}  \Delta_{{\mathrm{2}}}       }
  \end{math}
\end{center}
This case is similar to the previous case.  

\item[Case:] secondary hypothesis: right introduction of par.
The proof
\begin{center}
  \begin{math}
    $$\mprset{flushleft}
    \inferrule* [right=Cut] {
      \inferrule* [right=] {
        \inferrule* [right=,vdots=1.5em,fraction=\,] {
          \,
        }{\pi_1}          
      }{ \Gamma  \vdash   \FILLnt{t}  \FILLsym{:}  \FILLnt{A}  \mid  \Delta  }      
      \\
      $$\mprset{flushleft}
      \inferrule* [right=Parr] {
        \inferrule* [right=] {
        \inferrule* [right=,vdots=1.5em,fraction=\,] {
          \,
        }{\pi_2}          
      }{ \Gamma_{{\mathrm{1}}}  \FILLsym{,}  \FILLmv{x}  \FILLsym{:}  \FILLnt{A}  \FILLsym{,}  \Gamma_{{\mathrm{2}}}  \vdash     \Delta_{{\mathrm{1}}}  \mid    \FILLnt{t_{{\mathrm{1}}}}  \FILLsym{:}  \FILLnt{B}  \mid  \FILLnt{t_{{\mathrm{2}}}}  \FILLsym{:}  \FILLnt{C}      \mid  \Delta_{{\mathrm{2}}}  }                  
    }{ \Gamma_{{\mathrm{1}}}  \FILLsym{,}  \FILLmv{x}  \FILLsym{:}  \FILLnt{A}  \FILLsym{,}  \Gamma_{{\mathrm{2}}}  \vdash     \Delta_{{\mathrm{1}}}  \mid    \FILLnt{t_{{\mathrm{1}}}}  \parr  \FILLnt{t_{{\mathrm{2}}}}   \FILLsym{:}   \FILLnt{B}  \parr  \FILLnt{C}      \mid  \Delta_{{\mathrm{2}}}  }
  }{ \Gamma_{{\mathrm{1}}}  \FILLsym{,}  \Gamma  \FILLsym{,}  \Gamma_{{\mathrm{2}}}  \vdash   \Delta  \mid       \FILLsym{[}  \FILLnt{t}  \FILLsym{/}  \FILLmv{x}  \FILLsym{]}  \Delta_{{\mathrm{1}}}   \mid   \FILLsym{[}  \FILLnt{t}  \FILLsym{/}  \FILLmv{x}  \FILLsym{]}   (  \FILLnt{t_{{\mathrm{1}}}}  \parr  \FILLnt{t_{{\mathrm{2}}}}  )   \FILLsym{:}   \FILLnt{B}  \parr  \FILLnt{C}      \mid  \FILLsym{[}  \FILLnt{t}  \FILLsym{/}  \FILLmv{x}  \FILLsym{]}  \Delta_{{\mathrm{2}}}    }
  \end{math}
\end{center}
transforms into the proof
\begin{center}
  \begin{math}
    $$\mprset{flushleft}
\inferrule* [right=Parr] {
  $$\mprset{flushleft}
  \inferrule* [right=Cut] {
    \inferrule* [right=] {
        \inferrule* [right=,vdots=1.5em,fraction=\,] {
          \,
        }{\pi_1}          
      }{ \Gamma  \vdash   \FILLnt{t}  \FILLsym{:}  \FILLnt{A}  \mid  \Delta  }      
      \\
      \inferrule* [right=] {
        \inferrule* [right=,vdots=1.5em,fraction=\,] {
          \,
        }{\pi_2}          
      }{ \Gamma_{{\mathrm{1}}}  \FILLsym{,}  \FILLmv{x}  \FILLsym{:}  \FILLnt{A}  \FILLsym{,}  \Gamma_{{\mathrm{2}}}  \vdash     \Delta_{{\mathrm{1}}}  \mid    \FILLnt{t_{{\mathrm{1}}}}  \FILLsym{:}  \FILLnt{B}  \mid  \FILLnt{t_{{\mathrm{2}}}}  \FILLsym{:}  \FILLnt{C}      \mid  \Delta_{{\mathrm{2}}}  }                  
    }{ \Gamma_{{\mathrm{1}}}  \FILLsym{,}  \Gamma  \FILLsym{,}  \Gamma_{{\mathrm{2}}}  \vdash   \Delta  \mid       \FILLsym{[}  \FILLnt{t}  \FILLsym{/}  \FILLmv{x}  \FILLsym{]}  \Delta_{{\mathrm{1}}}   \mid     \FILLsym{[}  \FILLnt{t}  \FILLsym{/}  \FILLmv{x}  \FILLsym{]}  \FILLnt{t_{{\mathrm{1}}}}   \FILLsym{:}  \FILLnt{B}  \mid   \FILLsym{[}  \FILLnt{t}  \FILLsym{/}  \FILLmv{x}  \FILLsym{]}  \FILLnt{t_{{\mathrm{2}}}}   \FILLsym{:}  \FILLnt{C}      \mid  \FILLsym{[}  \FILLnt{t}  \FILLsym{/}  \FILLmv{x}  \FILLsym{]}  \Delta_{{\mathrm{2}}}    }
  }{ \Gamma_{{\mathrm{1}}}  \FILLsym{,}  \Gamma  \FILLsym{,}  \Gamma_{{\mathrm{2}}}  \vdash   \Delta  \mid       \FILLsym{[}  \FILLnt{t}  \FILLsym{/}  \FILLmv{x}  \FILLsym{]}  \Delta_{{\mathrm{1}}}   \mid    \FILLsym{[}  \FILLnt{t}  \FILLsym{/}  \FILLmv{x}  \FILLsym{]}  \FILLnt{t_{{\mathrm{1}}}}   \parr   \FILLsym{[}  \FILLnt{t}  \FILLsym{/}  \FILLmv{x}  \FILLsym{]}  \FILLnt{t_{{\mathrm{2}}}}    \FILLsym{:}   \FILLnt{B}  \parr  \FILLnt{C}     \mid  \FILLsym{[}  \FILLnt{t}  \FILLsym{/}  \FILLmv{x}  \FILLsym{]}  \Delta_{{\mathrm{2}}}    }
  \end{math}
\end{center}
Clearly, $\FILLsym{[}  \FILLnt{t}  \FILLsym{/}  \FILLmv{x}  \FILLsym{]}   (  \FILLnt{t_{{\mathrm{1}}}}  \parr  \FILLnt{t_{{\mathrm{2}}}}  )   \FILLsym{=}    ( \FILLsym{[}  \FILLnt{t}  \FILLsym{/}  \FILLmv{x}  \FILLsym{]}  \FILLnt{t_{{\mathrm{1}}}} )   \parr  \FILLsym{[}  \FILLnt{t}  \FILLsym{/}  \FILLmv{x}  \FILLsym{]}  \FILLnt{t_{{\mathrm{2}}}} $.

\item[Case:] secondary hypothesis: left introduction of par (first
  case).
The proof
\begin{center}
  \begin{math}
    $$\mprset{flushleft}
    \inferrule* [leftskip=10px,right=\tiny Cut] {
      \inferrule* [right=] {
        \inferrule* [right=,vdots=1.5em,fraction=\,] {
          \,
        }{\pi_1}          
      }{ \Gamma  \vdash   \FILLnt{t}  \FILLsym{:}  \FILLnt{A}  \mid  \Delta  }      
      \\
      $$\mprset{flushleft}
      \inferrule* [right=\tiny Parl] {
        \inferrule* [right=] {
        \inferrule* [right=,vdots=1.5em,fraction=\,] {
          \,
        }{\pi_2}          
      }{ \Gamma_{{\mathrm{1}}}  \FILLsym{,}  \FILLmv{x}  \FILLsym{:}  \FILLnt{A}  \FILLsym{,}  \Gamma_{{\mathrm{2}}}  \FILLsym{,}  \FILLmv{y}  \FILLsym{:}  \FILLnt{B}  \vdash  \Delta_{{\mathrm{1}}} }      
      \\
      \inferrule* [right=] {
        \inferrule* [right=,vdots=1.5em,fraction=\,] {
          \,
        }{\pi_3}          
      }{ \Gamma_{{\mathrm{3}}}  \FILLsym{,}  \FILLmv{z}  \FILLsym{:}  \FILLnt{C}  \vdash  \Delta_{{\mathrm{2}}} }      
      }{ \Gamma_{{\mathrm{1}}}  \FILLsym{,}  \FILLmv{x}  \FILLsym{:}  \FILLnt{A}  \FILLsym{,}  \Gamma_{{\mathrm{2}}}  \FILLsym{,}  \Gamma_{{\mathrm{3}}}  \FILLsym{,}  \FILLmv{w}  \FILLsym{:}   \FILLnt{B}  \parr  \FILLnt{C}   \vdash     \mathsf{\operatorname{let-pat} }\, \FILLmv{w} \, \FILLsym{(}   \FILLmv{y}  \parr   -    \FILLsym{)} \, \Delta_{{\mathrm{1}}}    \mid    \mathsf{\operatorname{let-pat} }\, \FILLmv{w} \, \FILLsym{(}    -   \parr  \FILLmv{z}   \FILLsym{)} \, \Delta_{{\mathrm{2}}}    }
    }{ \Gamma_{{\mathrm{1}}}  \FILLsym{,}  \Gamma  \FILLsym{,}  \Gamma_{{\mathrm{2}}}  \FILLsym{,}  \Gamma_{{\mathrm{3}}}  \FILLsym{,}  \FILLmv{w}  \FILLsym{:}   \FILLnt{B}  \parr  \FILLnt{C}   \vdash   \Delta  \mid     \FILLsym{[}  \FILLnt{t}  \FILLsym{/}  \FILLmv{x}  \FILLsym{]}  \FILLsym{(}   \mathsf{\operatorname{let-pat} }\, \FILLmv{w} \, \FILLsym{(}   \FILLmv{y}  \parr   -    \FILLsym{)} \, \Delta_{{\mathrm{1}}}   \FILLsym{)}   \mid   \FILLsym{[}  \FILLnt{t}  \FILLsym{/}  \FILLmv{x}  \FILLsym{]}  \FILLsym{(}   \mathsf{\operatorname{let-pat} }\, \FILLmv{w} \, \FILLsym{(}    -   \parr  \FILLmv{z}   \FILLsym{)} \, \Delta_{{\mathrm{2}}}   \FILLsym{)}     }
  \end{math}
\end{center}
\noindent
transforms into the proof
\begin{center}
  \begin{math}
    $$\mprset{flushleft}
    \inferrule* [right=\tiny ParL] {
      $$\mprset{flushleft}
      \inferrule* [right=\tiny Cut] {
        \inferrule* [right=] {
        \inferrule* [right=,vdots=1.5em,fraction=\,] {
          \,
        }{\pi_1}          
      }{ \Gamma  \vdash   \FILLnt{t}  \FILLsym{:}  \FILLnt{A}  \mid  \Delta  }      
      \\
      \inferrule* [right=] {
        \inferrule* [right=,vdots=1.5em,fraction=\,] {
          \,
        }{\pi_2}          
      }{ \Gamma_{{\mathrm{1}}}  \FILLsym{,}  \FILLmv{x}  \FILLsym{:}  \FILLnt{A}  \FILLsym{,}  \Gamma_{{\mathrm{2}}}  \FILLsym{,}  \FILLmv{y}  \FILLsym{:}  \FILLnt{B}  \vdash  \Delta_{{\mathrm{1}}} }      
      }{ \Gamma_{{\mathrm{1}}}  \FILLsym{,}  \Gamma  \FILLsym{,}  \Gamma_{{\mathrm{2}}}  \FILLsym{,}  \FILLmv{y}  \FILLsym{:}  \FILLnt{B}  \vdash   \Delta  \mid  \FILLsym{[}  \FILLnt{t}  \FILLsym{/}  \FILLmv{x}  \FILLsym{]}  \Delta_{{\mathrm{1}}}  }
      \\
      \inferrule* [right=] {
        \inferrule* [right=,vdots=1.5em,fraction=\,] {
          \,
        }{\pi_3}          
      }{ \Gamma_{{\mathrm{3}}}  \FILLsym{,}  \FILLmv{z}  \FILLsym{:}  \FILLnt{C}  \vdash  \Delta_{{\mathrm{2}}} }      
    }{ \Gamma_{{\mathrm{1}}}  \FILLsym{,}  \Gamma  \FILLsym{,}  \Gamma_{{\mathrm{2}}}  \FILLsym{,}  \Gamma_{{\mathrm{3}}}  \FILLsym{,}  \FILLmv{w}  \FILLsym{:}   \FILLnt{B}  \parr  \FILLnt{C}   \vdash       \mathsf{\operatorname{let-pat} }\, \FILLmv{w} \, \FILLsym{(}   \FILLmv{y}  \parr   -    \FILLsym{)} \, \Delta    \mid    \mathsf{\operatorname{let-pat} }\, \FILLmv{w} \, \FILLsym{(}   \FILLmv{y}  \parr   -    \FILLsym{)} \, \FILLsym{[}  \FILLnt{t}  \FILLsym{/}  \FILLmv{x}  \FILLsym{]}  \Delta_{{\mathrm{1}}}      \mid   \mathsf{\operatorname{let-pat} }\, \FILLmv{w} \, \FILLsym{(}    -   \parr  \FILLmv{z}   \FILLsym{)} \, \Delta_{{\mathrm{2}}}   }
  \end{math}
\end{center}
First, by inspection of the previous proofs we can see that $ \FILLmv{y}  \not\in \mathsf{FV}(  \Delta  ) $ and $ \FILLmv{x}  \not\in \mathsf{FV}(  \Delta_{{\mathrm{2}}}  ) $.  Thus, $ \mathsf{\operatorname{let-pat} }\, \FILLmv{w} \, \FILLsym{(}   \FILLmv{y}  \parr   -    \FILLsym{)} \, \Delta   \FILLsym{=}  \Delta$, and $\FILLsym{[}  \FILLnt{t}  \FILLsym{/}  \FILLmv{x}  \FILLsym{]}  \FILLsym{(}   \mathsf{\operatorname{let-pat} }\, \FILLmv{w} \, \FILLsym{(}    -   \parr  \FILLmv{z}   \FILLsym{)} \, \Delta_{{\mathrm{2}}}   \FILLsym{)}  \FILLsym{=}   \mathsf{\operatorname{let-pat} }\, \FILLmv{w} \, \FILLsym{(}    -   \parr  \FILLmv{z}   \FILLsym{)} \, \Delta_{{\mathrm{2}}} $. It suffices to show that $\FILLsym{[}  \FILLnt{t}  \FILLsym{/}  \FILLmv{x}  \FILLsym{]}  \FILLsym{(}   \mathsf{\operatorname{let-pat} }\, \FILLmv{w} \, \FILLsym{(}   \FILLmv{y}  \parr   -    \FILLsym{)} \, \Delta_{{\mathrm{1}}}   \FILLsym{)}  \FILLsym{=}   \mathsf{\operatorname{let-pat} }\, \FILLmv{w} \, \FILLsym{(}   \FILLmv{y}  \parr   -    \FILLsym{)} \, \FILLsym{[}  \FILLnt{t}  \FILLsym{/}  \FILLmv{x}  \FILLsym{]}  \Delta_{{\mathrm{1}}} $ but this follows by distributing
the substitution into the let-pat, and then simplifying using the fact
that $w \neq x$.

\item[Case:] secondary hypothesis: left introduction of par (second
  case).
The proof
\begin{center}
  \begin{math}
    $$\mprset{flushleft}
    \inferrule* [right=Cut] {
      \inferrule* [right=] {
        \inferrule* [right=,vdots=1.5em,fraction=\,] {
          \,
        }{\pi_1}          
      }{ \Gamma  \vdash   \FILLnt{t}  \FILLsym{:}  \FILLnt{A}  \mid  \Delta  }      
      \\
      $$\mprset{flushleft}
      \inferrule* [right=Parl] {
        \inferrule* [right=] {
        \inferrule* [right=,vdots=1.5em,fraction=\,] {
          \,
        }{\pi_2}          
      }{ \Gamma_{{\mathrm{1}}}  \FILLsym{,}  \FILLmv{y}  \FILLsym{:}  \FILLnt{B}  \vdash  \Delta_{{\mathrm{1}}} }      
      \\
      \inferrule* [right=] {
        \inferrule* [right=,vdots=1.5em,fraction=\,] {
          \,
        }{\pi_3}          
      }{ \Gamma_{{\mathrm{2}}}  \FILLsym{,}  \FILLmv{x}  \FILLsym{:}  \FILLnt{A}  \FILLsym{,}  \Gamma_{{\mathrm{3}}}  \FILLsym{,}  \FILLmv{z}  \FILLsym{:}  \FILLnt{C}  \vdash  \Delta_{{\mathrm{2}}} }      
      }{ \Gamma_{{\mathrm{1}}}  \FILLsym{,}  \Gamma_{{\mathrm{2}}}  \FILLsym{,}  \FILLmv{x}  \FILLsym{:}  \FILLnt{A}  \FILLsym{,}  \Gamma_{{\mathrm{3}}}  \FILLsym{,}  \FILLmv{w}  \FILLsym{:}   \FILLnt{B}  \parr  \FILLnt{C}   \vdash     \mathsf{\operatorname{let-pat} }\, \FILLmv{w} \, \FILLsym{(}   \FILLmv{y}  \parr   -    \FILLsym{)} \, \Delta_{{\mathrm{1}}}    \mid    \mathsf{\operatorname{let-pat} }\, \FILLmv{w} \, \FILLsym{(}    -   \parr  \FILLmv{z}   \FILLsym{)} \, \Delta_{{\mathrm{2}}}    }
    }{ \Gamma_{{\mathrm{1}}}  \FILLsym{,}  \Gamma_{{\mathrm{2}}}  \FILLsym{,}  \Gamma  \FILLsym{,}  \Gamma_{{\mathrm{3}}}  \FILLsym{,}  \FILLmv{w}  \FILLsym{:}   \FILLnt{B}  \parr  \FILLnt{C}   \vdash   \Delta  \mid     \FILLsym{[}  \FILLnt{t}  \FILLsym{/}  \FILLmv{x}  \FILLsym{]}  \FILLsym{(}   \mathsf{\operatorname{let-pat} }\, \FILLmv{w} \, \FILLsym{(}   \FILLmv{y}  \parr   -    \FILLsym{)} \, \Delta_{{\mathrm{1}}}   \FILLsym{)}   \mid   \FILLsym{[}  \FILLnt{t}  \FILLsym{/}  \FILLmv{x}  \FILLsym{]}  \FILLsym{(}   \mathsf{\operatorname{let-pat} }\, \FILLmv{w} \, \FILLsym{(}    -   \parr  \FILLmv{z}   \FILLsym{)} \, \Delta_{{\mathrm{2}}}   \FILLsym{)}     }
  \end{math}
\end{center}
transforms into the proof
\begin{center}
  \begin{math}
    $$\mprset{flushleft}
    \inferrule* [right=Parl] {
      \inferrule* [right=] {
        \inferrule* [right=,vdots=1.5em,fraction=\,] {
          \,
        }{\pi_2}          
      }{ \Gamma_{{\mathrm{1}}}  \FILLsym{,}  \FILLmv{y}  \FILLsym{:}  \FILLnt{B}  \vdash  \Delta_{{\mathrm{1}}} }      
      \\
      $$\mprset{flushleft}
      \inferrule* [right=Cut] {
        \inferrule* [right=] {
        \inferrule* [right=,vdots=1.5em,fraction=\,] {
          \,
        }{\pi_1}          
      }{ \Gamma  \vdash   \FILLnt{t}  \FILLsym{:}  \FILLnt{A}  \mid  \Delta  }      
      \\
      \inferrule* [right=] {
        \inferrule* [right=,vdots=1.5em,fraction=\,] {
          \,
        }{\pi_3}          
      }{ \Gamma_{{\mathrm{2}}}  \FILLsym{,}  \FILLmv{x}  \FILLsym{:}  \FILLnt{A}  \FILLsym{,}  \Gamma_{{\mathrm{3}}}  \FILLsym{,}  \FILLmv{z}  \FILLsym{:}  \FILLnt{C}  \vdash  \Delta_{{\mathrm{2}}} }      
    }{ \Gamma_{{\mathrm{2}}}  \FILLsym{,}  \Gamma  \FILLsym{,}  \Gamma_{{\mathrm{3}}}  \FILLsym{,}  \FILLmv{z}  \FILLsym{:}  \FILLnt{C}  \vdash   \Delta  \mid  \FILLsym{[}  \FILLnt{t}  \FILLsym{/}  \FILLmv{x}  \FILLsym{]}  \Delta_{{\mathrm{2}}}  }
    }{ \Gamma_{{\mathrm{1}}}  \FILLsym{,}  \Gamma_{{\mathrm{2}}}  \FILLsym{,}  \Gamma  \FILLsym{,}  \Gamma_{{\mathrm{3}}}  \FILLsym{,}  \FILLmv{w}  \FILLsym{:}   \FILLnt{B}  \parr  \FILLnt{C}   \vdash     \mathsf{\operatorname{let-pat} }\, \FILLmv{w} \, \FILLsym{(}   \FILLmv{y}  \parr   -    \FILLsym{)} \, \Delta_{{\mathrm{1}}}    \mid      \mathsf{\operatorname{let-pat} }\, \FILLmv{w} \, \FILLsym{(}    -   \parr  \FILLmv{z}   \FILLsym{)} \, \Delta    \mid    \mathsf{\operatorname{let-pat} }\, \FILLmv{w} \, \FILLsym{(}    -   \parr  \FILLmv{z}   \FILLsym{)} \, \FILLsym{[}  \FILLnt{t}  \FILLsym{/}  \FILLmv{x}  \FILLsym{]}  \Delta_{{\mathrm{2}}}      }
  \end{math}
\end{center}

Similar to the previous case.

\item[Case:] secondary hypothesis: left introduction of implication
  (first case).
The proof
\begin{center}
  \begin{math}
    $$\mprset{flushleft}
    \inferrule* [right=Cut] {
      \inferrule* [right=] {
        \inferrule* [right=,vdots=1.5em,fraction=\,] {
          \,
        }{\pi_1}          
      }{ \Gamma  \vdash   \FILLnt{t}  \FILLsym{:}  \FILLnt{A}  \mid  \Delta  }      
      \\
      $$\mprset{flushleft}
      \inferrule* [right=Impl] {
        \inferrule* [right=] {
          \inferrule* [right=,vdots=1.5em,fraction=\,] {
            \,
          }{\pi_2}          
        }{ \Gamma_{{\mathrm{1}}}  \FILLsym{,}  \FILLmv{x}  \FILLsym{:}  \FILLnt{A}  \FILLsym{,}  \Gamma_{{\mathrm{2}}}  \vdash   \FILLnt{t_{{\mathrm{1}}}}  \FILLsym{:}  \FILLnt{B}  \mid  \Delta_{{\mathrm{1}}}  }      
        \\
        \inferrule* [right=] {
          \inferrule* [right=,vdots=1.5em,fraction=\,] {
            \,
          }{\pi_3}          
        }{ \Gamma_{{\mathrm{3}}}  \FILLsym{,}  \FILLmv{y}  \FILLsym{:}  \FILLnt{C}  \vdash  \Delta_{{\mathrm{2}}} }      
      }{ \Gamma_{{\mathrm{1}}}  \FILLsym{,}  \FILLmv{x}  \FILLsym{:}  \FILLnt{A}  \FILLsym{,}  \Gamma_{{\mathrm{2}}}  \FILLsym{,}  \Gamma_{{\mathrm{3}}}  \FILLsym{,}  \FILLmv{z}  \FILLsym{:}   \FILLnt{B}  \multimap   \FILLnt{C}   \vdash   \Delta_{{\mathrm{1}}}  \mid  \FILLsym{[}  \FILLmv{z} \, \FILLnt{t_{{\mathrm{1}}}}  \FILLsym{/}  \FILLmv{y}  \FILLsym{]}  \Delta_{{\mathrm{2}}}  }
    }{ \Gamma_{{\mathrm{1}}}  \FILLsym{,}  \Gamma  \FILLsym{,}  \Gamma_{{\mathrm{2}}}  \FILLsym{,}  \Gamma_{{\mathrm{3}}}  \FILLsym{,}  \FILLmv{z}  \FILLsym{:}   \FILLnt{B}  \multimap   \FILLnt{C}   \vdash   \Delta  \mid     \FILLsym{[}  \FILLnt{t}  \FILLsym{/}  \FILLmv{x}  \FILLsym{]}  \Delta_{{\mathrm{1}}}   \mid  \FILLsym{[}  \FILLnt{t}  \FILLsym{/}  \FILLmv{x}  \FILLsym{]}   \FILLsym{[}  \FILLmv{z} \, \FILLnt{t_{{\mathrm{1}}}}  \FILLsym{/}  \FILLmv{y}  \FILLsym{]}  \Delta_{{\mathrm{2}}}     }
  \end{math}
\end{center}
transforms into the proof
\begin{center}
  \begin{math}
    $$\mprset{flushleft}
    \inferrule* [right=Impl] {
      $$\mprset{flushleft}
      \inferrule* [right=Cut] {
        \inferrule* [right=] {
        \inferrule* [right=,vdots=1.5em,fraction=\,] {
          \,
        }{\pi_1}          
      }{ \Gamma  \vdash   \FILLnt{t}  \FILLsym{:}  \FILLnt{A}  \mid  \Delta  }      
      \\
      \inferrule* [right=] {
          \inferrule* [right=,vdots=1.5em,fraction=\,] {
            \,
          }{\pi_2}          
        }{ \Gamma_{{\mathrm{1}}}  \FILLsym{,}  \FILLmv{x}  \FILLsym{:}  \FILLnt{A}  \FILLsym{,}  \Gamma_{{\mathrm{2}}}  \vdash   \FILLnt{t_{{\mathrm{1}}}}  \FILLsym{:}  \FILLnt{B}  \mid  \Delta_{{\mathrm{1}}}  }      
      }{ \Gamma_{{\mathrm{1}}}  \FILLsym{,}  \Gamma  \FILLsym{,}  \Gamma_{{\mathrm{2}}}  \vdash     \Delta  \mid   \FILLsym{[}  \FILLnt{t}  \FILLsym{/}  \FILLmv{x}  \FILLsym{]}  \FILLnt{t_{{\mathrm{1}}}}   \FILLsym{:}  \FILLnt{B}    \mid  \FILLsym{[}  \FILLnt{t}  \FILLsym{/}  \FILLmv{x}  \FILLsym{]}  \Delta_{{\mathrm{1}}}  }
      \\
      \inferrule* [right=] {
          \inferrule* [right=,vdots=1.5em,fraction=\,] {
            \,
          }{\pi_3}          
        }{ \Gamma_{{\mathrm{3}}}  \FILLsym{,}  \FILLmv{y}  \FILLsym{:}  \FILLnt{C}  \vdash  \Delta_{{\mathrm{2}}} }      
    }{ \Gamma_{{\mathrm{1}}}  \FILLsym{,}  \Gamma  \FILLsym{,}  \Gamma_{{\mathrm{2}}}  \FILLsym{,}  \Gamma_{{\mathrm{3}}}  \FILLsym{,}  \FILLmv{z}  \FILLsym{:}   \FILLnt{B}  \multimap   \FILLnt{C}   \vdash     \Delta  \mid   \FILLsym{[}  \FILLnt{t}  \FILLsym{/}  \FILLmv{x}  \FILLsym{]}  \Delta_{{\mathrm{1}}}     \mid  \FILLsym{[}  \FILLmv{z} \,  ( \FILLsym{[}  \FILLnt{t}  \FILLsym{/}  \FILLmv{x}  \FILLsym{]}  \FILLnt{t_{{\mathrm{1}}}} )   \FILLsym{/}  \FILLmv{y}  \FILLsym{]}  \Delta_{{\mathrm{2}}}  }
  \end{math}
\end{center}
By inspection of the above derivations we can see that $ \FILLmv{x}  \not\in \mathsf{FV}(  \Delta_{{\mathrm{2}}}  ) $, and hence, by this fact and substitution distribution
(Lemma~\ref{lemma:substitution_distribution}) we know 
$\FILLsym{[}  \FILLnt{t}  \FILLsym{/}  \FILLmv{x}  \FILLsym{]}   \FILLsym{[}  \FILLmv{z} \, \FILLnt{t_{{\mathrm{1}}}}  \FILLsym{/}  \FILLmv{y}  \FILLsym{]}  \Delta_{{\mathrm{2}}}   \FILLsym{=}   \FILLsym{[}   ( \FILLsym{[}  \FILLnt{t}  \FILLsym{/}  \FILLmv{x}  \FILLsym{]}  \FILLmv{z} )  \,  ( \FILLsym{[}  \FILLnt{t}  \FILLsym{/}  \FILLmv{x}  \FILLsym{]}  \FILLnt{t_{{\mathrm{1}}}} )   \FILLsym{/}  \FILLmv{y}  \FILLsym{]}  \FILLsym{[}  \FILLnt{t}  \FILLsym{/}  \FILLmv{x}  \FILLsym{]}  \Delta_{{\mathrm{2}}}  =  \FILLsym{[}  \FILLmv{z} \,  ( \FILLsym{[}  \FILLnt{t}  \FILLsym{/}  \FILLmv{x}  \FILLsym{]}  \FILLnt{t_{{\mathrm{1}}}} )   \FILLsym{/}  \FILLmv{y}  \FILLsym{]}  \Delta_{{\mathrm{2}}} $.

\item[Case:] secondary hypothesis: left introduction of implication
  (second case).
The proof
\begin{center}
  \begin{math}
    $$\mprset{flushleft}
    \inferrule* [right=Cut] {
      \inferrule* [right=] {
        \inferrule* [right=,vdots=1.5em,fraction=\,] {
          \,
        }{\pi_1}          
      }{ \Gamma  \vdash   \FILLnt{t}  \FILLsym{:}  \FILLnt{A}  \mid  \Delta  }      
      \\
      $$\mprset{flushleft}
      \inferrule* [right=Impl] {
        \inferrule* [right=] {
          \inferrule* [right=,vdots=1.5em,fraction=\,] {
            \,
          }{\pi_2}          
        }{ \Gamma_{{\mathrm{1}}}  \vdash   \FILLnt{t_{{\mathrm{1}}}}  \FILLsym{:}  \FILLnt{B}  \mid  \Delta_{{\mathrm{1}}}  }      
        \\
        \inferrule* [right=] {
          \inferrule* [right=,vdots=1.5em,fraction=\,] {
            \,
          }{\pi_3}          
        }{ \Gamma_{{\mathrm{2}}}  \FILLsym{,}  \FILLmv{x}  \FILLsym{:}  \FILLnt{A}  \FILLsym{,}  \Gamma_{{\mathrm{3}}}  \FILLsym{,}  \FILLmv{y}  \FILLsym{:}  \FILLnt{C}  \vdash  \Delta_{{\mathrm{2}}} }      
      }{ \Gamma_{{\mathrm{1}}}  \FILLsym{,}  \Gamma_{{\mathrm{2}}}  \FILLsym{,}  \FILLmv{x}  \FILLsym{:}  \FILLnt{A}  \FILLsym{,}  \Gamma_{{\mathrm{3}}}  \FILLsym{,}  \FILLmv{z}  \FILLsym{:}   \FILLnt{B}  \multimap   \FILLnt{C}   \vdash   \Delta_{{\mathrm{1}}}  \mid  \FILLsym{[}  \FILLmv{z} \, \FILLnt{t_{{\mathrm{1}}}}  \FILLsym{/}  \FILLmv{y}  \FILLsym{]}  \Delta_{{\mathrm{2}}}  }
    }{ \Gamma_{{\mathrm{1}}}  \FILLsym{,}  \Gamma_{{\mathrm{2}}}  \FILLsym{,}  \Gamma  \FILLsym{,}  \Gamma_{{\mathrm{3}}}  \FILLsym{,}  \FILLmv{z}  \FILLsym{:}   \FILLnt{B}  \multimap   \FILLnt{C}   \vdash   \Delta  \mid     \FILLsym{[}  \FILLnt{t}  \FILLsym{/}  \FILLmv{x}  \FILLsym{]}  \Delta_{{\mathrm{1}}}   \mid  \FILLsym{[}  \FILLnt{t}  \FILLsym{/}  \FILLmv{x}  \FILLsym{]}   \FILLsym{[}  \FILLmv{z} \, \FILLnt{t_{{\mathrm{1}}}}  \FILLsym{/}  \FILLmv{y}  \FILLsym{]}  \Delta_{{\mathrm{2}}}     }
  \end{math}
\end{center}
transforms into the proof
\begin{center}
  \begin{math}
    $$\mprset{flushleft}
    \inferrule* [right=\tiny Series of Exchanges] {
      $$\mprset{flushleft}
    \inferrule* [right=Impl] {
      \inferrule* [right=] {
        \inferrule* [right=,vdots=1.5em,fraction=\,] {
          \,
        }{\pi_2}          
      }{ \Gamma_{{\mathrm{1}}}  \vdash   \FILLnt{t_{{\mathrm{1}}}}  \FILLsym{:}  \FILLnt{B}  \mid  \Delta_{{\mathrm{1}}}  }      
      \\
      $$\mprset{flushleft}
      \inferrule* [right=Cut] {
        \inferrule* [right=] {
        \inferrule* [right=,vdots=1.5em,fraction=\,] {
          \,
        }{\pi_1}          
      }{ \Gamma  \vdash   \FILLnt{t}  \FILLsym{:}  \FILLnt{A}  \mid  \Delta  }      
      \\
      \inferrule* [right=] {
          \inferrule* [right=,vdots=1.5em,fraction=\,] {
            \,
          }{\pi_3}          
        }{ \Gamma_{{\mathrm{2}}}  \FILLsym{,}  \FILLmv{x}  \FILLsym{:}  \FILLnt{A}  \FILLsym{,}  \Gamma_{{\mathrm{3}}}  \FILLsym{,}  \FILLmv{y}  \FILLsym{:}  \FILLnt{C}  \vdash  \Delta_{{\mathrm{2}}} }      
      }{ \Gamma_{{\mathrm{2}}}  \FILLsym{,}  \Gamma  \FILLsym{,}  \Gamma_{{\mathrm{3}}}  \FILLsym{,}  \FILLmv{y}  \FILLsym{:}  \FILLnt{C}  \vdash   \Delta  \mid  \FILLsym{[}  \FILLnt{t}  \FILLsym{/}  \FILLmv{x}  \FILLsym{]}  \Delta_{{\mathrm{2}}}  }
    }{ \Gamma_{{\mathrm{1}}}  \FILLsym{,}  \Gamma_{{\mathrm{2}}}  \FILLsym{,}  \Gamma  \FILLsym{,}  \Gamma_{{\mathrm{3}}}  \FILLsym{,}  \FILLmv{z}  \FILLsym{:}   \FILLnt{B}  \multimap   \FILLnt{C}   \vdash   \Delta_{{\mathrm{1}}}  \mid     \FILLsym{[}  \FILLmv{z} \, \FILLnt{t_{{\mathrm{1}}}}  \FILLsym{/}  \FILLmv{y}  \FILLsym{]}  \Delta   \mid   \FILLsym{[}  \FILLmv{z} \, \FILLnt{t_{{\mathrm{1}}}}  \FILLsym{/}  \FILLmv{y}  \FILLsym{]}  \FILLsym{[}  \FILLnt{t}  \FILLsym{/}  \FILLmv{x}  \FILLsym{]}  \Delta_{{\mathrm{2}}}     }
  }{ \Gamma_{{\mathrm{1}}}  \FILLsym{,}  \Gamma_{{\mathrm{2}}}  \FILLsym{,}  \Gamma  \FILLsym{,}  \Gamma_{{\mathrm{3}}}  \FILLsym{,}  \FILLmv{z}  \FILLsym{:}   \FILLnt{B}  \multimap   \FILLnt{C}   \vdash     \FILLsym{[}  \FILLmv{z} \, \FILLnt{t_{{\mathrm{1}}}}  \FILLsym{/}  \FILLmv{y}  \FILLsym{]}  \Delta   \mid    \Delta_{{\mathrm{1}}}  \mid   \FILLsym{[}  \FILLmv{z} \, \FILLnt{t_{{\mathrm{1}}}}  \FILLsym{/}  \FILLmv{y}  \FILLsym{]}  \FILLsym{[}  \FILLnt{t}  \FILLsym{/}  \FILLmv{x}  \FILLsym{]}  \Delta_{{\mathrm{2}}}      }
  \end{math}
\end{center}
By inspection of the above proofs we can see that $ \FILLmv{y}  \not\in \mathsf{FV}(  \Delta  ) $. Thus, $\FILLsym{[}  \FILLmv{z} \, \FILLnt{t_{{\mathrm{1}}}}  \FILLsym{/}  \FILLmv{y}  \FILLsym{]}  \Delta  \FILLsym{=}  \Delta$.  The same can be said for the
variable $\FILLmv{x}$ and context $\Delta_{{\mathrm{1}}}$, and hence, $\FILLsym{[}  \FILLnt{t}  \FILLsym{/}  \FILLmv{x}  \FILLsym{]}  \Delta_{{\mathrm{1}}}  \FILLsym{=}  \Delta_{{\mathrm{1}}}$.
Finally, by inspection of the above proofs $ \FILLmv{x}  \not\in \mathsf{FV}(  \FILLnt{t_{{\mathrm{1}}}}  ) $ and so
by substitution distribution
(Lemma~\ref{lemma:substitution_distribution}) we know $\FILLsym{[}  \FILLnt{t}  \FILLsym{/}  \FILLmv{x}  \FILLsym{]}   \FILLsym{[}  \FILLmv{z} \, \FILLnt{t_{{\mathrm{1}}}}  \FILLsym{/}  \FILLmv{y}  \FILLsym{]}  \Delta_{{\mathrm{2}}}   \FILLsym{=}  \FILLsym{[}  \FILLmv{z} \, \FILLnt{t_{{\mathrm{1}}}}  \FILLsym{/}  \FILLmv{y}  \FILLsym{]}  \FILLsym{[}  \FILLnt{t}  \FILLsym{/}  \FILLmv{x}  \FILLsym{]}  \Delta_{{\mathrm{2}}}$.

\item[Case:] secondary hypothesis: left introduction of implication
  (third case).
The proof
\begin{center}
  \begin{math}
    $$\mprset{flushleft}
    \inferrule* [right=Cut] {
      \inferrule* [right=] {
        \inferrule* [right=,vdots=1.5em,fraction=\,] {
          \,
        }{\pi_1}          
      }{ \Gamma  \vdash   \FILLnt{t}  \FILLsym{:}  \FILLnt{A}  \mid  \Delta  }      
      \\
      $$\mprset{flushleft}
      \inferrule* [right=Impl] {
        \inferrule* [right=] {
          \inferrule* [right=,vdots=1.5em,fraction=\,] {
            \,
          }{\pi_2}          
        }{ \Gamma_{{\mathrm{1}}}  \vdash   \FILLnt{t_{{\mathrm{1}}}}  \FILLsym{:}  \FILLnt{B}  \mid  \Delta_{{\mathrm{1}}}  }      
        \\
        \inferrule* [right=] {
          \inferrule* [right=,vdots=1.5em,fraction=\,] {
            \,
          }{\pi_3}          
        }{ \Gamma_{{\mathrm{2}}}  \FILLsym{,}  \FILLmv{y}  \FILLsym{:}  \FILLnt{C}  \FILLsym{,}  \Gamma_{{\mathrm{3}}}  \FILLsym{,}  \FILLmv{x}  \FILLsym{:}  \FILLnt{A}  \vdash  \Delta_{{\mathrm{2}}} }      
      }{ \Gamma_{{\mathrm{1}}}  \FILLsym{,}  \Gamma_{{\mathrm{2}}}  \FILLsym{,}  \FILLmv{z}  \FILLsym{:}   \FILLnt{B}  \multimap   \FILLnt{C}   \FILLsym{,}  \Gamma_{{\mathrm{3}}}  \FILLsym{,}  \FILLmv{x}  \FILLsym{:}  \FILLnt{A}  \vdash   \Delta_{{\mathrm{1}}}  \mid  \FILLsym{[}  \FILLmv{z} \, \FILLnt{t_{{\mathrm{1}}}}  \FILLsym{/}  \FILLmv{y}  \FILLsym{]}  \Delta_{{\mathrm{2}}}  }
    }{ \Gamma_{{\mathrm{1}}}  \FILLsym{,}  \Gamma_{{\mathrm{2}}}  \FILLsym{,}  \FILLmv{z}  \FILLsym{:}   \FILLnt{B}  \multimap   \FILLnt{C}   \FILLsym{,}  \Gamma_{{\mathrm{3}}}  \FILLsym{,}  \Gamma  \vdash   \Delta  \mid     \FILLsym{[}  \FILLnt{t}  \FILLsym{/}  \FILLmv{x}  \FILLsym{]}  \Delta_{{\mathrm{1}}}   \mid  \FILLsym{[}  \FILLnt{t}  \FILLsym{/}  \FILLmv{x}  \FILLsym{]}   \FILLsym{[}  \FILLmv{z} \, \FILLnt{t_{{\mathrm{1}}}}  \FILLsym{/}  \FILLmv{y}  \FILLsym{]}  \Delta_{{\mathrm{2}}}     }
  \end{math}
\end{center}
transforms into the proof
\begin{center}
  \begin{math}
    $$\mprset{flushleft}
    \inferrule* [right=\tiny Series of Exchanges] {
      $$\mprset{flushleft}
    \inferrule* [right=Impl] {
      \inferrule* [right=] {
        \inferrule* [right=,vdots=1.5em,fraction=\,] {
          \,
        }{\pi_2}          
      }{ \Gamma_{{\mathrm{1}}}  \vdash   \FILLnt{t_{{\mathrm{1}}}}  \FILLsym{:}  \FILLnt{B}  \mid  \Delta_{{\mathrm{1}}}  }      
      \\
      $$\mprset{flushleft}
      \inferrule* [right=Cut] {
        \inferrule* [right=] {
        \inferrule* [right=,vdots=1.5em,fraction=\,] {
          \,
        }{\pi_1}          
      }{ \Gamma  \vdash   \FILLnt{t}  \FILLsym{:}  \FILLnt{A}  \mid  \Delta  }      
      \\
      \inferrule* [right=] {
          \inferrule* [right=,vdots=1.5em,fraction=\,] {
            \,
          }{\pi_3}          
        }{ \Gamma_{{\mathrm{2}}}  \FILLsym{,}  \FILLmv{y}  \FILLsym{:}  \FILLnt{C}  \FILLsym{,}  \Gamma_{{\mathrm{3}}}  \FILLsym{,}  \FILLmv{x}  \FILLsym{:}  \FILLnt{A}  \vdash  \Delta_{{\mathrm{2}}} }      
      }{ \Gamma_{{\mathrm{2}}}  \FILLsym{,}  \FILLmv{y}  \FILLsym{:}  \FILLnt{C}  \FILLsym{,}  \Gamma_{{\mathrm{3}}}  \FILLsym{,}  \Gamma  \vdash   \Delta  \mid  \FILLsym{[}  \FILLnt{t}  \FILLsym{/}  \FILLmv{x}  \FILLsym{]}  \Delta_{{\mathrm{2}}}  }
    }{ \Gamma_{{\mathrm{1}}}  \FILLsym{,}  \Gamma_{{\mathrm{2}}}  \FILLsym{,}  \FILLmv{z}  \FILLsym{:}   \FILLnt{B}  \multimap   \FILLnt{C}   \FILLsym{,}  \Gamma_{{\mathrm{3}}}  \FILLsym{,}  \Gamma  \vdash   \Delta_{{\mathrm{1}}}  \mid     \FILLsym{[}  \FILLmv{z} \, \FILLnt{t_{{\mathrm{1}}}}  \FILLsym{/}  \FILLmv{y}  \FILLsym{]}  \Delta   \mid   \FILLsym{[}  \FILLmv{z} \, \FILLnt{t_{{\mathrm{1}}}}  \FILLsym{/}  \FILLmv{y}  \FILLsym{]}  \FILLsym{[}  \FILLnt{t}  \FILLsym{/}  \FILLmv{x}  \FILLsym{]}  \Delta_{{\mathrm{2}}}     }
  }{ \Gamma_{{\mathrm{1}}}  \FILLsym{,}  \Gamma_{{\mathrm{2}}}  \FILLsym{,}  \FILLmv{z}  \FILLsym{:}   \FILLnt{B}  \multimap   \FILLnt{C}   \FILLsym{,}  \Gamma_{{\mathrm{3}}}  \FILLsym{,}  \Gamma  \vdash     \FILLsym{[}  \FILLmv{z} \, \FILLnt{t_{{\mathrm{1}}}}  \FILLsym{/}  \FILLmv{y}  \FILLsym{]}  \Delta   \mid    \Delta_{{\mathrm{1}}}  \mid   \FILLsym{[}  \FILLmv{z} \, \FILLnt{t_{{\mathrm{1}}}}  \FILLsym{/}  \FILLmv{y}  \FILLsym{]}  \FILLsym{[}  \FILLnt{t}  \FILLsym{/}  \FILLmv{x}  \FILLsym{]}  \Delta_{{\mathrm{2}}}      }
  \end{math}
\end{center}
Similar to the previous case.

\item[Case:] secondary hypothesis: right introduction of implication.
The proof
\begin{center}
  \begin{math}
    $$\mprset{flushleft}
    \inferrule* [right=Cut] {
      \inferrule* [right=] {
        \inferrule* [right=,vdots=1.5em,fraction=\,] {
          \,
        }{\pi_1}          
      }{ \Gamma  \vdash   \FILLnt{t}  \FILLsym{:}  \FILLnt{A}  \mid  \Delta  }      
      \\
      $$\mprset{flushleft}
      \inferrule* [right=Impr] {
        \inferrule* [right=] {
        \inferrule* [right=,vdots=1.5em,fraction=\,] {
          \,
        }{\pi_2}          
      }{ \Gamma_{{\mathrm{1}}}  \FILLsym{,}  \FILLmv{x}  \FILLsym{:}  \FILLnt{A}  \FILLsym{,}  \Gamma_{{\mathrm{2}}}  \FILLsym{,}  \FILLmv{y}  \FILLsym{:}  \FILLnt{B}  \vdash   \FILLnt{t_{{\mathrm{1}}}}  \FILLsym{:}  \FILLnt{C}  \mid  \Delta_{{\mathrm{1}}}  }      
      \\
       \FILLmv{y}  \not\in \mathsf{FV}(  \Delta_{{\mathrm{1}}}  ) 
      }{ \Gamma_{{\mathrm{1}}}  \FILLsym{,}  \FILLmv{x}  \FILLsym{:}  \FILLnt{A}  \FILLsym{,}  \Gamma_{{\mathrm{2}}}  \vdash    \lambda  \FILLmv{y}  .  \FILLnt{t_{{\mathrm{1}}}}   \FILLsym{:}   \FILLnt{B}  \multimap   \FILLnt{C}   \mid  \Delta_{{\mathrm{1}}}  }
    }{ \Gamma_{{\mathrm{1}}}  \FILLsym{,}  \Gamma  \FILLsym{,}  \Gamma_{{\mathrm{2}}}  \vdash   \Delta  \mid     \FILLsym{[}  \FILLnt{t}  \FILLsym{/}  \FILLmv{x}  \FILLsym{]}   (  \lambda  \FILLmv{y}  .  \FILLnt{t_{{\mathrm{1}}}}  )    \FILLsym{:}   \FILLnt{B}  \multimap   \FILLnt{C}   \mid  \FILLsym{[}  \FILLnt{t}  \FILLsym{/}  \FILLmv{x}  \FILLsym{]}  \Delta_{{\mathrm{1}}}    }
  \end{math}
\end{center}
transforms into the proof
\begin{center}
  \begin{math}
    $$\mprset{flushleft}
    \inferrule* [right=Impr] {
      $$\mprset{flushleft}
      \inferrule* [right=Cut] {
        \inferrule* [right=] {
          \inferrule* [right=,vdots=1.5em,fraction=\,] {
            \,
          }{\pi_1}          
        }{ \Gamma  \vdash   \FILLnt{t}  \FILLsym{:}  \FILLnt{A}  \mid  \Delta  }      
        \\        
          \inferrule* [right=] {
            \inferrule* [right=,vdots=1.5em,fraction=\,] {
              \,
            }{\pi_2}          
          }{ \Gamma_{{\mathrm{1}}}  \FILLsym{,}  \FILLmv{x}  \FILLsym{:}  \FILLnt{A}  \FILLsym{,}  \Gamma_{{\mathrm{2}}}  \FILLsym{,}  \FILLmv{y}  \FILLsym{:}  \FILLnt{B}  \vdash   \FILLnt{t_{{\mathrm{1}}}}  \FILLsym{:}  \FILLnt{C}  \mid  \Delta_{{\mathrm{1}}}  }      
        }{ \Gamma_{{\mathrm{1}}}  \FILLsym{,}  \Gamma  \FILLsym{,}  \Gamma_{{\mathrm{2}}}  \FILLsym{,}  \FILLmv{y}  \FILLsym{:}  \FILLnt{B}  \vdash   \Delta  \mid     \FILLsym{[}  \FILLnt{t}  \FILLsym{/}  \FILLmv{x}  \FILLsym{]}  \FILLnt{t_{{\mathrm{1}}}}   \FILLsym{:}  \FILLnt{C}  \mid  \FILLsym{[}  \FILLnt{t}  \FILLsym{/}  \FILLmv{x}  \FILLsym{]}  \Delta_{{\mathrm{1}}}    }
      }{ \Gamma_{{\mathrm{1}}}  \FILLsym{,}  \Gamma  \FILLsym{,}  \Gamma_{{\mathrm{2}}}  \vdash   \Delta  \mid      \lambda  \FILLmv{y}  .  \FILLsym{[}  \FILLnt{t}  \FILLsym{/}  \FILLmv{x}  \FILLsym{]}  \FILLnt{t_{{\mathrm{1}}}}    \FILLsym{:}   \FILLnt{B}  \multimap   \FILLnt{C}   \mid  \FILLsym{[}  \FILLnt{t}  \FILLsym{/}  \FILLmv{x}  \FILLsym{]}  \Delta_{{\mathrm{1}}}    }
    \end{math}
  \end{center}
Clearly, $\FILLsym{[}  \FILLnt{t}  \FILLsym{/}  \FILLmv{x}  \FILLsym{]}   (  \lambda  \FILLmv{y}  .  \FILLnt{t_{{\mathrm{1}}}}  )   \FILLsym{=}   \lambda  \FILLmv{y}  .  \FILLsym{[}  \FILLnt{t}  \FILLsym{/}  \FILLmv{x}  \FILLsym{]}  \FILLnt{t_{{\mathrm{1}}}} $.

\item[Case:] secondary hypothesis: left introduction of tensor unit.
The proof 
\begin{center}
  \begin{math}
    $$\mprset{flushleft}
    \inferrule* [right=Cut] {
      \inferrule* [right=] {
          \inferrule* [right=,vdots=1.5em,fraction=\,] {
            \,
          }{\pi_1}          
        }{ \Gamma  \vdash   \FILLnt{t}  \FILLsym{:}  \FILLnt{A}  \mid  \Delta  }      
        \\
        $$\mprset{flushleft}
        \inferrule* [right=Il] {
          \inferrule* [right=] {
          \inferrule* [right=,vdots=1.5em,fraction=\,] {
            \,
          }{\pi_2}          
        }{ \Gamma_{{\mathrm{1}}}  \FILLsym{,}  \FILLmv{x}  \FILLsym{:}  \FILLnt{A}  \FILLsym{,}  \Gamma_{{\mathrm{2}}}  \vdash  \Delta_{{\mathrm{1}}} }      
      }{ \Gamma_{{\mathrm{1}}}  \FILLsym{,}  \FILLmv{x}  \FILLsym{:}  \FILLnt{A}  \FILLsym{,}  \Gamma_{{\mathrm{2}}}  \FILLsym{,}  \FILLmv{y}  \FILLsym{:}   \top   \vdash   \mathsf{let}\, \FILLmv{y} \,\mathsf{be}\,  *  \,\mathsf{in}\, \Delta_{{\mathrm{1}}}  }
    }{ \Gamma_{{\mathrm{1}}}  \FILLsym{,}  \Gamma  \FILLsym{,}  \Gamma_{{\mathrm{2}}}  \FILLsym{,}  \FILLmv{y}  \FILLsym{:}   \top   \vdash   \Delta  \mid  \FILLsym{[}  \FILLnt{t}  \FILLsym{/}  \FILLmv{x}  \FILLsym{]}  \FILLsym{(}   \mathsf{let}\, \FILLmv{y} \,\mathsf{be}\,  *  \,\mathsf{in}\, \Delta_{{\mathrm{1}}}   \FILLsym{)}  }
  \end{math}
\end{center}
transforms into the proof
\begin{center}
  \begin{math}
    $$\mprset{flushleft}
    \inferrule* [right=Il] {
      $$\mprset{flushleft}
      \inferrule* [right=Cut] {
        \inferrule* [right=] {
          \inferrule* [right=,vdots=1.5em,fraction=\,] {
            \,
          }{\pi_1}          
        }{ \Gamma  \vdash   \FILLnt{t}  \FILLsym{:}  \FILLnt{A}  \mid  \Delta  }      
        \\
        \inferrule* [right=] {
          \inferrule* [right=,vdots=1.5em,fraction=\,] {
            \,
          }{\pi_2}          
        }{ \Gamma_{{\mathrm{1}}}  \FILLsym{,}  \FILLmv{x}  \FILLsym{:}  \FILLnt{A}  \FILLsym{,}  \Gamma_{{\mathrm{2}}}  \vdash  \Delta_{{\mathrm{1}}} }      
      }{ \Gamma_{{\mathrm{1}}}  \FILLsym{,}  \Gamma  \FILLsym{,}  \Gamma_{{\mathrm{2}}}  \vdash   \Delta  \mid  \FILLsym{[}  \FILLnt{t}  \FILLsym{/}  \FILLmv{x}  \FILLsym{]}  \Delta_{{\mathrm{1}}}  }
    }{ \Gamma_{{\mathrm{1}}}  \FILLsym{,}  \Gamma  \FILLsym{,}  \Gamma_{{\mathrm{2}}}  \FILLsym{,}  \FILLmv{y}  \FILLsym{:}   \top   \vdash     \mathsf{let}\, \FILLmv{y} \,\mathsf{be}\,  *  \,\mathsf{in}\, \Delta    \mid   \mathsf{let}\, \FILLmv{y} \,\mathsf{be}\,  *  \,\mathsf{in}\, \FILLsym{[}  \FILLnt{t}  \FILLsym{/}  \FILLmv{x}  \FILLsym{]}  \Delta_{{\mathrm{1}}}   }
  \end{math}
\end{center}
It suffices to show that $\Delta =  \mathsf{let}\, \FILLmv{y} \,\mathsf{be}\,  *  \,\mathsf{in}\, \Delta $ and
$\FILLsym{[}  \FILLnt{t}  \FILLsym{/}  \FILLmv{x}  \FILLsym{]}  \FILLsym{(}   \mathsf{let}\, \FILLmv{y} \,\mathsf{be}\,  *  \,\mathsf{in}\, \Delta_{{\mathrm{1}}}   \FILLsym{)} =  \mathsf{let}\, \FILLmv{y} \,\mathsf{be}\,  *  \,\mathsf{in}\, \FILLsym{[}  \FILLnt{t}  \FILLsym{/}  \FILLmv{x}  \FILLsym{]}  \Delta_{{\mathrm{1}}} $.
Without loss of generality suppose $\Delta = \FILLnt{t}  \FILLsym{:}  \FILLnt{B}  \FILLsym{,}  \Delta'$.  We know
that it must be the case that $ \FILLmv{y}  \not\in \mathsf{FV}(  \FILLnt{t}  ) $, and we know that
$\FILLsym{[}  \FILLmv{y}  \FILLsym{/}  \FILLmv{z}  \FILLsym{]}  \FILLnt{t} = \FILLnt{t}$ when $ \FILLmv{z}  \not\in \mathsf{FV}(  \FILLnt{t}  ) $.  Then by
$\FILLdrulename{Eq\_Eta2I}$ we have $\FILLnt{t} =  \mathsf{let}\, \FILLmv{y} \,\mathsf{be}\,  *  \,\mathsf{in}\, \FILLnt{t} $.  This
argument can be repeated for any other term in $\Delta'$.  Thus,
$\Delta =  \mathsf{let}\, \FILLmv{y} \,\mathsf{be}\,  *  \,\mathsf{in}\, \Delta $.  It is easy to see that
$\FILLsym{[}  \FILLnt{t}  \FILLsym{/}  \FILLmv{x}  \FILLsym{]}  \FILLsym{(}   \mathsf{let}\, \FILLmv{y} \,\mathsf{be}\,  *  \,\mathsf{in}\, \Delta_{{\mathrm{1}}}   \FILLsym{)} =  \mathsf{let}\, \FILLmv{y} \,\mathsf{be}\,  *  \,\mathsf{in}\, \FILLsym{[}  \FILLnt{t}  \FILLsym{/}  \FILLmv{x}  \FILLsym{]}  \Delta_{{\mathrm{1}}} $ using
the rule $\FILLdrulename{Eq\_NatI}$.

\item[Case:] secondary hypothesis: right introduction of par unit.
The proof
\begin{center}
  \begin{math}
    $$\mprset{flushleft}
    \inferrule* [right=Cut] {
      \inferrule* [right=] {
        \inferrule* [right=,vdots=1.5em,fraction=\,] {
            \,
          }{\pi_1}          
        }{ \Gamma  \vdash   \FILLnt{t}  \FILLsym{:}  \FILLnt{A}  \mid  \Delta  }      
        \\
        $$\mprset{flushleft}
        \inferrule* [right=Pr] {
          \inferrule* [right=] {
          \inferrule* [right=,vdots=1.5em,fraction=\,] {
            \,
          }{\pi_2}          
        }{ \Gamma_{{\mathrm{1}}}  \FILLsym{,}  \FILLmv{x}  \FILLsym{:}  \FILLnt{A}  \FILLsym{,}  \Gamma_{{\mathrm{2}}}  \vdash  \Delta_{{\mathrm{1}}} }      
        }{ \Gamma_{{\mathrm{1}}}  \FILLsym{,}  \FILLmv{x}  \FILLsym{:}  \FILLnt{A}  \FILLsym{,}  \Gamma_{{\mathrm{2}}}  \vdash    \circ   \FILLsym{:}   \perp   \mid  \Delta_{{\mathrm{1}}}  }
      }{ \Gamma_{{\mathrm{1}}}  \FILLsym{,}  \Gamma  \FILLsym{,}  \Gamma_{{\mathrm{2}}}  \vdash   \Delta  \mid     \FILLsym{[}  \FILLnt{t}  \FILLsym{/}  \FILLmv{x}  \FILLsym{]}   \circ    \FILLsym{:}   \perp   \mid  \FILLsym{[}  \FILLnt{t}  \FILLsym{/}  \FILLmv{x}  \FILLsym{]}  \Delta_{{\mathrm{1}}}    }
  \end{math}
\end{center}
transforms into the proof
\begin{center}
  \begin{math}
    $$\mprset{flushleft}
    \inferrule* [right=\tiny Series of Exchanges] {
      $$\mprset{flushleft}
    \inferrule* [right=Pr] {
      $$\mprset{flushleft}
      \inferrule* [right=Cut] {
        \inferrule* [right=] {
        \inferrule* [right=,vdots=1.5em,fraction=\,] {
            \,
          }{\pi_1}          
        }{ \Gamma  \vdash   \FILLnt{t}  \FILLsym{:}  \FILLnt{A}  \mid  \Delta  }      
        \\
        \inferrule* [right=] {
          \inferrule* [right=,vdots=1.5em,fraction=\,] {
            \,
          }{\pi_2}          
        }{ \Gamma_{{\mathrm{1}}}  \FILLsym{,}  \FILLmv{x}  \FILLsym{:}  \FILLnt{A}  \FILLsym{,}  \Gamma_{{\mathrm{2}}}  \vdash  \Delta_{{\mathrm{1}}} }      
      }{ \Gamma_{{\mathrm{1}}}  \FILLsym{,}  \Gamma  \FILLsym{,}  \Gamma_{{\mathrm{2}}}  \vdash   \Delta  \mid  \FILLsym{[}  \FILLnt{t}  \FILLsym{/}  \FILLmv{x}  \FILLsym{]}  \Delta_{{\mathrm{1}}}  }
    }{ \Gamma_{{\mathrm{1}}}  \FILLsym{,}  \Gamma  \FILLsym{,}  \Gamma_{{\mathrm{2}}}  \vdash    \circ   \FILLsym{:}   \perp   \mid    \Delta  \mid  \FILLsym{[}  \FILLnt{t}  \FILLsym{/}  \FILLmv{x}  \FILLsym{]}  \Delta_{{\mathrm{1}}}    }
  }{ \Gamma_{{\mathrm{1}}}  \FILLsym{,}  \Gamma  \FILLsym{,}  \Gamma_{{\mathrm{2}}}  \vdash   \Delta  \mid     \circ   \FILLsym{:}   \perp   \mid  \FILLsym{[}  \FILLnt{t}  \FILLsym{/}  \FILLmv{x}  \FILLsym{]}  \Delta_{{\mathrm{1}}}    }
  \end{math}
\end{center}
Clearly, $\FILLsym{[}  \FILLnt{t}  \FILLsym{/}  \FILLmv{x}  \FILLsym{]}   \circ   \FILLsym{=}   \circ $.

\item[Case:] secondary hypothesis: left introduction of exchange.
The proof
\begin{center}
  \begin{math}
    $$\mprset{flushleft}
    \inferrule* [right=Cut] {
      \inferrule* [right=] {
        \inferrule* [right=,vdots=1.5em,fraction=\,] {
            \,
          }{\pi_1}          
        }{ \Gamma  \vdash   \FILLnt{t}  \FILLsym{:}  \FILLnt{A}  \mid  \Delta  }      
        \\
        $$\mprset{flushleft}
        \inferrule* [right=Exl] {
          \inferrule* [right=] {
        \inferrule* [right=,vdots=1.5em,fraction=\,] {
            \,
          }{\pi_2}          
        }{ \Gamma_{{\mathrm{1}}}  \FILLsym{,}  \FILLmv{x}  \FILLsym{:}  \FILLnt{A}  \FILLsym{,}  \Gamma_{{\mathrm{2}}}  \FILLsym{,}  \FILLmv{w}  \FILLsym{:}  \FILLnt{B}  \FILLsym{,}  \FILLmv{y}  \FILLsym{:}  \FILLnt{C}  \FILLsym{,}  \Gamma_{{\mathrm{3}}}  \vdash  \Delta_{{\mathrm{1}}} }      
      }{ \Gamma_{{\mathrm{1}}}  \FILLsym{,}  \FILLmv{x}  \FILLsym{:}  \FILLnt{A}  \FILLsym{,}  \Gamma_{{\mathrm{2}}}  \FILLsym{,}  \FILLmv{y}  \FILLsym{:}  \FILLnt{C}  \FILLsym{,}  \FILLmv{w}  \FILLsym{:}  \FILLnt{B}  \FILLsym{,}  \Gamma_{{\mathrm{3}}}  \vdash  \Delta_{{\mathrm{1}}} }
    }{ \Gamma_{{\mathrm{1}}}  \FILLsym{,}  \Gamma  \FILLsym{,}  \Gamma_{{\mathrm{2}}}  \FILLsym{,}  \FILLmv{y}  \FILLsym{:}  \FILLnt{C}  \FILLsym{,}  \FILLmv{w}  \FILLsym{:}  \FILLnt{B}  \FILLsym{,}  \Gamma_{{\mathrm{3}}}  \vdash   \Delta  \mid  \FILLsym{[}  \FILLnt{t}  \FILLsym{/}  \FILLmv{x}  \FILLsym{]}  \Delta_{{\mathrm{1}}}  }
  \end{math}
\end{center}
transforms into the proof
\begin{center}
  \begin{math}
    $$\mprset{flushleft}
    \inferrule* [right=Exl] {
      $$\mprset{flushleft}
      \inferrule* [right=Cut] {
        \inferrule* [right=] {
        \inferrule* [right=,vdots=1.5em,fraction=\,] {
            \,
          }{\pi_1}          
        }{ \Gamma  \vdash   \FILLnt{t}  \FILLsym{:}  \FILLnt{A}  \mid  \Delta  }      
        \\
        \inferrule* [right=] {
        \inferrule* [right=,vdots=1.5em,fraction=\,] {
            \,
          }{\pi_2}          
        }{ \Gamma_{{\mathrm{1}}}  \FILLsym{,}  \FILLmv{x}  \FILLsym{:}  \FILLnt{A}  \FILLsym{,}  \Gamma_{{\mathrm{2}}}  \FILLsym{,}  \FILLmv{w}  \FILLsym{:}  \FILLnt{B}  \FILLsym{,}  \FILLmv{y}  \FILLsym{:}  \FILLnt{C}  \FILLsym{,}  \Gamma_{{\mathrm{3}}}  \vdash  \Delta_{{\mathrm{1}}} }      
      }{ \Gamma_{{\mathrm{1}}}  \FILLsym{,}  \Gamma  \FILLsym{,}  \Gamma_{{\mathrm{2}}}  \FILLsym{,}  \FILLmv{w}  \FILLsym{:}  \FILLnt{B}  \FILLsym{,}  \FILLmv{y}  \FILLsym{:}  \FILLnt{C}  \FILLsym{,}  \Gamma_{{\mathrm{3}}}  \vdash   \Delta  \mid  \FILLsym{[}  \FILLnt{t}  \FILLsym{/}  \FILLmv{x}  \FILLsym{]}  \Delta_{{\mathrm{1}}}  }
    }{ \Gamma_{{\mathrm{1}}}  \FILLsym{,}  \Gamma  \FILLsym{,}  \Gamma_{{\mathrm{2}}}  \FILLsym{,}  \FILLmv{y}  \FILLsym{:}  \FILLnt{C}  \FILLsym{,}  \FILLmv{w}  \FILLsym{:}  \FILLnt{B}  \FILLsym{,}  \Gamma_{{\mathrm{3}}}  \vdash   \Delta  \mid  \FILLsym{[}  \FILLnt{t}  \FILLsym{/}  \FILLmv{x}  \FILLsym{]}  \Delta_{{\mathrm{1}}}  }
  \end{math}
\end{center}
Clearly, all terms are equivalent.

\item[Case:] secondary hypothesis: right introduction of exchange.
The proof
\begin{center}
  \begin{math}
    $$\mprset{flushleft}
    \inferrule* [right=Cut] {
      \inferrule* [right=] {
        \inferrule* [right=,vdots=1.5em,fraction=\,] {
            \,
          }{\pi_1}          
        }{ \Gamma  \vdash   \FILLnt{t}  \FILLsym{:}  \FILLnt{A}  \mid  \Delta  }      
        \\
        $$\mprset{flushleft}
        \inferrule* [right=Exr] {
          \inferrule* [right=] {
        \inferrule* [right=,vdots=1.5em,fraction=\,] {
            \,
          }{\pi_2}          
        }{ \Gamma_{{\mathrm{1}}}  \FILLsym{,}  \FILLmv{x}  \FILLsym{:}  \FILLnt{A}  \FILLsym{,}  \Gamma_{{\mathrm{2}}}  \vdash   \Delta_{{\mathrm{1}}}  \mid      \FILLnt{t_{{\mathrm{1}}}}  \FILLsym{:}  \FILLnt{B}  \mid  \FILLnt{t_{{\mathrm{2}}}}  \FILLsym{:}  \FILLnt{C}    \mid  \Delta_{{\mathrm{2}}}    }      
      }{ \Gamma_{{\mathrm{1}}}  \FILLsym{,}  \FILLmv{x}  \FILLsym{:}  \FILLnt{A}  \FILLsym{,}  \Gamma_{{\mathrm{2}}}  \vdash   \Delta_{{\mathrm{1}}}  \mid      \FILLnt{t_{{\mathrm{2}}}}  \FILLsym{:}  \FILLnt{C}  \mid  \FILLnt{t_{{\mathrm{1}}}}  \FILLsym{:}  \FILLnt{B}    \mid  \Delta_{{\mathrm{2}}}    }
    }{ \Gamma_{{\mathrm{1}}}  \FILLsym{,}  \Gamma  \FILLsym{,}  \Gamma_{{\mathrm{2}}}  \vdash   \Delta  \mid     \FILLsym{[}  \FILLnt{t}  \FILLsym{/}  \FILLmv{x}  \FILLsym{]}  \Delta_{{\mathrm{1}}}   \mid       \FILLsym{[}  \FILLnt{t}  \FILLsym{/}  \FILLmv{x}  \FILLsym{]}  \FILLnt{t_{{\mathrm{2}}}}   \FILLsym{:}  \FILLnt{C}  \mid   \FILLsym{[}  \FILLnt{t}  \FILLsym{/}  \FILLmv{x}  \FILLsym{]}  \FILLnt{t_{{\mathrm{1}}}}   \FILLsym{:}  \FILLnt{B}    \mid  \FILLsym{[}  \FILLnt{t}  \FILLsym{/}  \FILLmv{x}  \FILLsym{]}  \Delta_{{\mathrm{2}}}      }
  \end{math}
\end{center}
is transformed into 
\begin{center}
  \begin{math}
    $$\mprset{flushleft}
    \inferrule* [right=Exr] {
      $$\mprset{flushleft}
      \inferrule* [right=Cut] {
        \inferrule* [right=] {
        \inferrule* [right=,vdots=1.5em,fraction=\,] {
            \,
          }{\pi_1}          
        }{ \Gamma  \vdash   \FILLnt{t}  \FILLsym{:}  \FILLnt{A}  \mid  \Delta  }      
        \\
        \inferrule* [right=] {
        \inferrule* [right=,vdots=1.5em,fraction=\,] {
            \,
          }{\pi_2}          
        }{ \Gamma_{{\mathrm{1}}}  \FILLsym{,}  \FILLmv{x}  \FILLsym{:}  \FILLnt{A}  \FILLsym{,}  \Gamma_{{\mathrm{2}}}  \vdash   \Delta_{{\mathrm{1}}}  \mid      \FILLnt{t_{{\mathrm{1}}}}  \FILLsym{:}  \FILLnt{B}  \mid  \FILLnt{t_{{\mathrm{2}}}}  \FILLsym{:}  \FILLnt{C}    \mid  \Delta_{{\mathrm{2}}}    }      
      }{ \Gamma_{{\mathrm{1}}}  \FILLsym{,}  \Gamma  \FILLsym{,}  \Gamma_{{\mathrm{2}}}  \vdash   \Delta  \mid     \FILLsym{[}  \FILLnt{t}  \FILLsym{/}  \FILLmv{x}  \FILLsym{]}  \Delta_{{\mathrm{1}}}   \mid       \FILLsym{[}  \FILLnt{t}  \FILLsym{/}  \FILLmv{x}  \FILLsym{]}  \FILLnt{t_{{\mathrm{1}}}}   \FILLsym{:}  \FILLnt{B}  \mid   \FILLsym{[}  \FILLnt{t}  \FILLsym{/}  \FILLmv{x}  \FILLsym{]}  \FILLnt{t_{{\mathrm{2}}}}   \FILLsym{:}  \FILLnt{C}    \mid  \FILLsym{[}  \FILLnt{t}  \FILLsym{/}  \FILLmv{x}  \FILLsym{]}  \Delta_{{\mathrm{2}}}      }
    }{ \Gamma_{{\mathrm{1}}}  \FILLsym{,}  \Gamma  \FILLsym{,}  \Gamma_{{\mathrm{2}}}  \vdash   \Delta  \mid     \FILLsym{[}  \FILLnt{t}  \FILLsym{/}  \FILLmv{x}  \FILLsym{]}  \Delta_{{\mathrm{1}}}   \mid       \FILLsym{[}  \FILLnt{t}  \FILLsym{/}  \FILLmv{x}  \FILLsym{]}  \FILLnt{t_{{\mathrm{2}}}}   \FILLsym{:}  \FILLnt{C}  \mid   \FILLsym{[}  \FILLnt{t}  \FILLsym{/}  \FILLmv{x}  \FILLsym{]}  \FILLnt{t_{{\mathrm{1}}}}   \FILLsym{:}  \FILLnt{B}    \mid  \FILLsym{[}  \FILLnt{t}  \FILLsym{/}  \FILLmv{x}  \FILLsym{]}  \Delta_{{\mathrm{2}}}      }
  \end{math}
\end{center}
Clearly, all terms are equivalent.
  \end{itemize}
   \end{report}

\end{proof}

\begin{corollary}[Cut-Elimination]
  \label{corollary:cut-elimination}
  Cut-elimination holds for FILL.
\end{corollary}
% section cut-elimination (end)

\section{Full LNL Models}
\label{sec:full_lnl_models}
One of the difficult questions considering the categorical models of
linear logic was how to model Girard's exponential, $!$, which is read
``of course''.  The $!$ modality can be used to translate
intuitionistic logic into intuitionistic linear logic, and so the
correct categorical interpretation of $!$ should involve a
relationship between a cartesian closed category, and the model of
intuitionistic linear logic.

The work of de Paiva gave some of the very first categorical models for both classical
and intuitionistic linear logic in the thesis \cite{dePaiva:1988}. She
showed that a particular dialectica category called $\dial$ is a model
of FILL where $!$ is interpreted as a comonad which produces natural
comonoids, see page 76 of \cite{dePaiva:1988}.
\begin{definition}
  \label{def:dial2sets}
  The category $\dial$ consists of 
  \begin{itemize}
  \item objects that are triples, $A = (U,X,\alpha)$, where $U$ and
    $X$ are sets, and $\alpha \subseteq U \times X$ is a relation, and
  \item maps that are pairs $(f,F) : (U,X,\alpha) \to (V,Y,\beta)$
    where $f : U \to V$ and $F : Y \to X$ such that
    \begin{itemize}
    \item For any $u \in U$ and $y \in Y$, $\alpha(u,F(y))$ implies $\beta(f(u),y)$.
    \end{itemize}
  \end{itemize}
  Suppose $A = (U,X,\alpha)$, $B = (V,Y,\beta)$, and
  $C = (W,Z,\gamma)$.  Then identities are given by
  $(\id_U,\id_X) : A \to A$.  The composition of the maps
  $(f,F) : A \to B$ and $(g, G) : B \to C$ is defined as
  $(f;g,G;F) : A \to C$.
\end{definition}
In her thesis de Paiva defines a particular class of dialectica
categories called $GC$ over a base category $C$, see page 41 of
\cite{dePaiva:1988}.  The category $\dial$ defined above can be seen
as an instantiation of $GC$ by setting $C$ to be the category $\sets$
of sets and functions between them.  This model is a non-trivial model
(all four units of the multiplicative and additive conjunction and disjunction are
different objects in the category), and does not model classical
logic; see \cite{dePaiva:1988} page 58.

Seely gave a different, syntactic categorical model that confirmed
that the of-course exponential should be modeled by a comonad
\cite{Seely:1989}.  However, Seely's model turned out to be unsound,
as pointed out by Bierman \cite{Bierman:1994}.  This 
%then prompted
was noticed when
%this is back-to-front here
Benton, Bierman,  de Paiva, and Hyland,  
%to define 
worked out another categorical
model called linear categories (Definition~\ref{def:linear-cat}) that
are sound, and also model $!$ using a monoidal comonad
\cite{Bierman:1994}.  
\begin{definition}
  \label{def:linear-cat}
  A \textbf{linear category}, $\mathcal{L}$, consists of:
  \begin{center}
    \begin{itemize}
    \item A symmetric monoidal closed category $\mathcal{L}$,
    \item A symmetric monoidal comonad $(!, \epsilon, \delta, m_{A,B},
      m_I)$ such that 
      \begin{itemize}
      \item For every free $!$-coalgebra $(!A,\delta_A)$ there are two\\
        distinguished monoidal natural transformations $e_A : !A \to
        I$ and $d_A : !A \to !A \otimes !A$ which form a commutative
        comonoid and are coalgebra morphisms.
      \item If $f : (!A,\delta_A) \to (!B,\delta_B)$ is a coalgebra
        morphism between free coalgebras, then it is also a comonoid
        morphism.
      \end{itemize}
    \end{itemize}
  \end{center}
  This definition is the one given by Bierman in his thesis, see
  \cite{Bierman:1994} for full definitions.
\end{definition}
\noindent
Intuitionistic logic can be interpreted in a linear category as a
full subcategory of the category of $!$-coalgebras for the comonad,
see proposition 17 of \cite{Bierman:1994}.

Benton gave a more balanced view of linear categories called LNL
models.
\begin{definition}
  \label{def:LNL-model}
  A \textbf{linear/non-linear model (LNL model)} consists of
  \begin{itemize}
  \item a cartesian closed category $(\cat{C}, 1, \times,
    \Rightarrow)$,
  \item a SMCC $(\cat{L},I,\otimes,\limp)$, and
  \item a pair of symmetric monoidal functors $(G,n) : \cat{L} \to
    \cat{C}$ and $(F,m) : \cat{C} \to \cat{L}$ between them that form
    a symmetric monoidal adjunction with $F \dashv G$.
  \end{itemize}
  See Benton, \cite{Benton:1994}, for the definitions of symmetric
  monoidal functors and adjunctions.
\end{definition}
A non-trivial consequence of the definition of a LNL model is that the
$!$ modality can indeed be interpreted as a monoidal comonad.  Suppose
$(\cat{L}, \cat{C},F,G)$ is a LNL model. Then the comonad is given by
$(\mathop{!}, \epsilon : \mathop{!} \to \mathsf{Id}, \delta :
\mathop{!} \to \mathop{!!})$
where $! = FG$, $\epsilon$ is the counit of the adjunction and
$\delta$ is the natural transformation $\delta_A = F(\eta_{G(A)})$,
see page 15 of \cite{Benton:1994}.  We recall the following result
from Benton \cite{Benton:1994}:
\begin{theorem}[LNL Models and Linear Categories]
  \label{thm:lnl_models_are_linear_categories}
  \begin{itemize}
  \item[]
  \item[i.] (Section~2.2.1 of \cite{Benton:1994}) Every LNL model is a linear category.
  \item[ii.] (Section~2.2.2 of \cite{Benton:1994}) Every linear category is a LNL model.
  \end{itemize}
\end{theorem}
\begin{proof}
  The proof of part i. is a matter of checking that each part of the
  definition of a linear category can be constructed using the
  definition of a LNL model. See lemmata 3-7 of \cite{Benton:1994}.

  As for the proof of part ii. Given a linear category we have a SMCC and so the difficulty of
  proving this result is constructing the CCC and the adjunction
  between both parts of the model.  Suppose $\cat{L}$ is a linear
  category.  Benton constructs the CCC out of the full subcategory of
  Eilenberg-Moore category $\cat{L}^!$ whose objects are exponentiable
  coalgebras denoted $\mathsf{Exp}(\cat{L}^!)$.  He shows that
  this subcategory is cartesian closed, and contains the (co)Kleisli
  category, $\cat{L}_!$, see Lemma~11 on page 23 of \cite{Benton:1994}.
  The required adjunction
  $F : \mathsf{Exp}(\cat{L}^!) \to L : G$ can be defined using the
  adjunct functors $F(A,h_A) = A$ and $G(A) = (!A,\delta_A)$, see
  lemmata 13 - 16 of \cite{Benton:1994}.
\end{proof}
While this theorem is totally satisfactory for our goals in this paper, one should perhaps remark that it does not answer the harder question of which kind of morphisms should one consider between models of Linear Logic. A fuller discussion can be found in %\cite{Maietti:2005}.
\iffalse
@article{Maietti:2005,
  title={Relating categorical semantics for intuitionistic linear logic},
  author={Maietti, Maria Emilia and Maneggia, Paola and De Paiva, Valeria and Ritter, Eike},
  journal={Applied Categorical Structures},
  volume={13},
  number={1},
  pages={1--36},
  year={2005},
  publisher={Kluwer Academic Publishers}
}
\fi

Next we show that the category $\dial$ is a full version of a linear
category. First, we extend the definitions of linear categories and
LNL models to be equipped with the necessary categorical structure to
model par and its unit.
\begin{definition}
  \label{def:full-linear-cat}
  A \textbf{full linear category}, $\mathcal{L}$, consists of a linear
  category \\$(\mathcal{L}, \top, \otimes, \limp,!A,e_A,d_A)$, a
  symmetric monoidal structure on $L$, $(\perp, \parr)$, and
  distribution natural transformations $\mathsf{dist}_1 : A \otimes (B
  \parr C) \to (A \otimes B) \parr C$ and $\mathsf{dist}_2 : (A \parr
  B) \otimes C \to A \parr (B \otimes C)$.  The distributors must
  satisfy several coherence conditions which can  be found in
  \cite{Cockett:1997}.
\end{definition}
\begin{definition}
  \label{def:full-lnl-model}
  A \textbf{full linear/non-linear model (full LNL model)} consists of
  a LNL model $(\cat{L}, \cat{C},F,G)$, and a symmetric monoidal
  structure on $L$, $(\perp, \parr)$, as above.
\end{definition}

\noindent
First we show that $\dial$ is a full linear category, and then using
the proof by Benton that linear categories are LNL models we obtain
that \\ $\dial$ is a full LNL model. In order for this to work we
need to know that $\dial$ has a symmetric \textit{monoidal} comonad $(!,
\epsilon, \delta, m_{A,B}, m_I)$.  At the time of de Paiva's
thesis it was not known that the comonad modeling the of-course
modality needed to be \textit{monoidal}.  

The comonad $(!, \epsilon, \delta, m_{A,B}, m_I)$ can be defined by
setting $!(U , X , \alpha) = (U , U \to X^*, \alpha*)$ where $X^*$ the
commutative monoid consisting of finite sequences of elements of $X$,
and $\alpha^* \subseteq U \times X^*$ is the extension of $\alpha$ to
finite sequences, that is, $\alpha^*(u , f)$ iff $\alpha(u , x_1)
\land \cdots \land \alpha(u, x_n)$, where $f(u) = x_1,\ldots,x_n$. The
definitions of $\epsilon$ and $\delta$ can be found in the formal
development.

We can show that the monoidal maps $m_{A,B} : !A \otimes !B \to !(A
\otimes B)$ and $m_I : I \to !I$ exist in the general setting of
dialectica categories, and thus, these maps exist in $\dial$.
%need a bit more here?did you define the comonad ! yet?
Intuitively, given
two objects $A = (X,U,\alpha)$ and $B = (V,Y,\beta)$ of $\dial$ the
map $m_{A,B}$ is defined as the pair $(\id_{U \times V},F)$, where $F
= (F_1,F_2)$, $F_1 : (U \times V) \Rightarrow (V \Rightarrow X)^* \to
V \Rightarrow (U \Rightarrow X^*)$ and $F_2 : (U \times V) \Rightarrow
(U \Rightarrow Y)^* \to U \Rightarrow (V \Rightarrow Y^*)$.  The maps
$F_1$ and $F_2$ build the sequence of all the results of applying each
function in the input sequence to the input coordinate.

We can now show that the following holds.
\begin{lemma}
  \label{lemma:ddial_is_a_linear_category}
  The category $\dial$ is a full linear category.
\end{lemma}
\begin{proof}
  \begin{paper}
    We only give a sketch of the proof here, but for the full details
    see that companion report \cite{Eades:2015}\footnote{This proof
      was formalized in the Agda proof assistant see the file
      \url{https://github.com/heades/cut-fill-agda/blob/master/FullLinCat.agda}}. First,
    we must show that $\dial$ is a linear category.  The majority of
    the linear structure of $\dial$ is in de Paiva's thesis
    \cite{dePaiva:1988}.  We had to extend her definitions to show
    that the comonad $(!A,\delta,\epsilon)$ is monoidal. However, this extension
    is straightforward.

    After showing that $\dial$ is a linear category one must show that
    $\dial$ is a model of par and its unit.  This easily follows from
    de Paiva's thesis.  The bifunctor which models par is given by de
    Paiva in Definition 10 on page 47 of \cite{dePaiva:1988}.

    Finally, $\dial$ must be distributive.  The natural
    transformations $dist_1$ and $dist_2$ can be defined in terms of
    the maps $k : (A \otimes A') \otimes (B \parr C) \to (A \otimes B)
    \parr (A' \otimes C)$ and $k' : (A \parr B) \otimes (C \otimes C')
    \to (A \otimes C) \parr (B \otimes C')$ given on page 52 of
    \cite{dePaiva:1988}.  Set $A' = \top$ in $k$ and $C = \top$ in
    $k'$ to obtain $dist_1$ and $dist_2$ respectively.  They can also
    be shown to satisfy the coherence conditions given in
    \cite{Cockett:1997}.
  \end{paper}
  \begin{report}
  This proof holds by constructing each piece of a full linear
  category using the structure of $\dial$\footnote{This proof was
    formalized in the Agda proof assistant see the file
    \url{https://github.com/heades/cut-fill-agda/blob/master/FullLinCat.agda}}.
  We use some notation to make it easier to define and use functions
  over sequences.  Given a function $g : A \to X^*$ we will denote
  taking the $i$th projection of the sequence returned by $g(a)$ for
  some $a \in A$ by $g(a)_i$.  To construct set-theoretic anonymous
  functions we use $\lambda$-notation.  Lastly, we often use
  let-expressions to pattern match on sequences.  For example, $\lett
  (x_1,\ldots,x_i) = g(a) \inn (f(x_1),\cdots,f(x_i))$.
    
  First, we must construct a linear category. It is well known that
  $\mathsf{Sets}$ is a CCC, and in fact, locally cartesian closed, and
  so by using the results of de Paiva's thesis we can easily see that
  $\dial$ is symmetric monoidal closed:
  \begin{center}
    \begin{itemize}
    \item (Definition 7, page 43 of \cite{dePaiva:1988}). Suppose
      $A,B \in \obj{\dial}$.  Then there are sets $X$, $Y$, $V$, and
      $U$, and relations $\alpha \subseteq U \times X$ and
      $\beta \subseteq V \times Y$, such that, $A = (U,X,\alpha)$ and
      $B = (V,Y,\beta)$.  The tensor product of $A$ and $B$ can now be
      defined by $A \otimes B = (U \times V, (V \Rightarrow X) \times
      (U \Rightarrow Y),\alpha \otimes \beta)$, where $(- \Rightarrow
      -)$ is the internal hom of $\mathsf{Sets}$.  We define
      $((u,v),(f,g)) \in \alpha \otimes \beta$ if and only if
      $(u,f(v)) \in \alpha$ and $(v,g(u)) \in \beta$.  

      Suppose $A = (U,X,\alpha)$, $B = (V,Y,\beta)$,
      $C = (W,Z,\gamma)$, and $D = (S,T,\delta)$ objects of $\dial$,
      and $m_1 = (f, F) : A \to C$ and $m_2 = (g,G) : B \to D$ are
      maps of $\dial$.  Then the map $m_1 \otimes m_2 : A \otimes B
      \to C \otimes D$ is defined by $(f \times g, F_\otimes)$ where
      $f \times g$ is the ordinary cartesian product functor in
      $\sets$, and we define $F \otimes G$ as follows:
      \begin{center}
        \begin{math}
          \begin{array}{lll}
            F_\otimes : (S \Rightarrow Z) \times (W \Rightarrow T) \to (V \Rightarrow X) \times (U \Rightarrow Y)\\
            F_\otimes(h_1,h_2) = (\lambda v.F(h_1(g(v))),\lambda u.G(h_2(f(u))))
          \end{array}
        \end{math}
      \end{center}
      It is straightforward to confirm the relation condition on maps
      for $m_1 \otimes m_2$.

    \item (Definition 7, page 44 of \cite{dePaiva:1988}). Suppose
      $1 \in \obj{\sets}$ is the final object, and
      $\id_1 \subseteq 1 \times 1$.  Then we can define tensors
      unit by the object $\top = (1,1,\id_1)$.

    \item Suppose $A = (U,X,\alpha)$ is an object of $\dial$.  Then
      the map $\lambda_A : \top \otimes A \to A$ is defined by
      $(\hat{\lambda}_U,F_\lambda)$ where
      $\hat{\lambda}_U$ is the left unitor for the cartesian product in $\sets$,
      $F_\lambda(x) = (\diamond,\lambda y.x) : X \to (U \Rightarrow 1) \times (1 \Rightarrow X)$, 
      and $\diamond$ is the terminal arrow in $\sets$.
      It is easy to see that both
      $\hat{\lambda}_U$ and $F_\lambda$ have
      inverses, and thus, $\lambda_A$ has an inverse. It is straightforward to confirm the relation condition on maps
      for $\lambda_A$ and its inverse.

    \item Suppose $A = (U,X,\alpha)$ is an object of $\dial$.  Then
      the map $\rho_A : A \otimes \top \to A$ is defined similarly to
      $\lambda_A$ given above.

    \item Suppose $A = (U,X,\alpha) \in \obj{\dial}$ and
      $B = (V,Y,\beta) \in \obj{\dial}$.  Then we define the map
      $\beta_{A,B} : A \otimes B \to B \otimes A$ by
      $(\hat{\beta}_{U,V}, \hat{\beta}_{V \Rightarrow X,U \Rightarrow
        Y})$
      where $\hat{\beta}$ is the symmetry of the cartesian product in
      $\sets$.  Again, it is straightforward to see that $\beta$ has
      an inverse, and the relation condition on maps is satisfied.

    \item Suppose $A = (U,X,\alpha) \in \obj{\dial}$,
      $B = (V,Y,\beta) \in \obj{\dial}$, and
      $C = (W,Z,\gamma) \in \obj{\dial}$.  Then we define
      $\alpha_{A,B,C} : (A \otimes B) \otimes C \to A \otimes (B \otimes C)$
      by $(\hat{\alpha}_{U,V,W}, F_\alpha)$ where
      $\hat{\alpha}_{U,V,W}$ is the associator for the cartesian
      product in $\sets$ and $F_\alpha$ is defined as follows:
      \begin{center}
        \scriptsize
        \begin{math}
          \begin{array}{lll}
            F_\alpha : ((V \times W) \Rightarrow X) \times (U \Rightarrow ((W \Rightarrow Y) \times (V \Rightarrow Z))) \to 
            (W \Rightarrow ((V \Rightarrow X) \times (U \Rightarrow Y))) \times ((U \times V) \Rightarrow Z)\\
            F_\alpha(h_1,h_2) = (\lambda w.(\lambda v.h_1(v,w),\lambda u.h_2(u)_1(w)), \lambda (u,v).h_2(u)_2(v))\\
          \end{array}
        \end{math}
      \end{center}
      The inverse of $\alpha_{A,B,C}$ is similar, and it is
      straightforward to confirm the relation condition on maps.

    \item (Definition 9, page 44 of \cite{dePaiva:1988}). Suppose
      $A,B \in \obj{\dial}$.  Then there are sets $X$, $Y$, $V$, and
      $U$, and relations $\alpha \subseteq U \times X$ and
      $\beta \subseteq V \times Y$, such that, $A = (U,X,\alpha)$ and
      $B = (V,Y,\beta)$.  Then we define the internal hom of $\dial$
      by $A \limp B = ((U \Rightarrow V) \times (Y \Rightarrow X),U
      \times Y,\alpha \Rightarrow \beta)$.  We define $((f,g),(u,y))
      \in \alpha \Rightarrow \beta$ if and only if whenever $(u,g(y))
      \in \alpha$, then $(f(u),y) \in \beta$.  The locally cartesian 
      closed structure of $\sets$ guarantees that for any two objects $A, B \in
      \dial$ the internal hom $A \limp B \in \dial$ exists.      
    \end{itemize}
  \end{center}
  Using the constructions above $\dial$ is a SMCC by Proposition 24
  on page 46 of \cite{dePaiva:1988}.  

  Next we define the symmetric monoidal comonad
  $(!,\epsilon,\delta,m_{A,B},m_I)$ of the linear category:
  \begin{center}
    \begin{itemize}
    \item (Section 4.5, on page 76 of \cite{dePaiva:1988}). The
      endofunctor $! : \dial \to \dial$ is defined as follows:
      \begin{itemize}
      \item Objects. Suppose $A = (U,X,\alpha) \in \obj{\dial}$.  Then
        we set $!A = (U,U \Rightarrow X^*, !\alpha)$, where
        $(u,f) \in !\alpha$ if and only if
        $(u,f(u)_1) \in \alpha \text{ and } \cdots \text{ and }
        (u,f(u)_i) \in \alpha$ where $f(u)$ is a sequence of length $i$.

      \item Morphisms. Suppose $A = (U,X,\alpha) \in \obj{\dial}$,
        $B = (V,Y,\beta) \in \obj{\dial}$, and
        $(f,F) : A \to B \in \mor{\dial}$.  Then we define 
        $!(f,F) = (f,!F) : !A \to !B$, where $!F(g) = \lambda
        x.F^*(g(f(x))) : V \Rightarrow Y^* \to U \Rightarrow X^*$.
      \end{itemize}

    \item (Section 4.5, page 77 of \cite{dePaiva:1988}). The
      endofunctor $!$ defined above is the functor part of the comonad
      $(!,\epsilon, \delta)$.  Suppose
      $A = (U,X,\alpha) \in \obj{\dial}$. Then the co-unit
      $\epsilon : !A \to A$ is defined by $\epsilon =
      (\mathsf{id}_U,F_0)$ where $F_0(x) = \lambda y.(x) : X \to U
      \Rightarrow X^*$.  
      Furthermore, the co-multiplication $\delta_A
      : !A \to !!A$ is defined by $\delta_A = (\mathsf{id}_U,F_1)$
      where $F_1(g) = \lambda u.(f_1(u) \circ \cdots \circ f_i(u)) : U \Rightarrow (U
      \Rightarrow X^*)^* \to U \Rightarrow X^*$ where $g(u) = (f_1,\ldots,f_i)$.

    \item The following diagrams commute:
      \begin{center}
        \begin{math}
          \begin{array}{lll}
            \bfig
            \square[!A`!!A`!!A`!!!A;\delta_A`\delta_A`!\delta_A`\delta_{!A}]
            \efig
            &
              \,\,\,\,\,\,\,\,\,\,\,\,\,\,\,\,\,\,\,\,
            &
              \bfig
             \Atrianglepair/=`->`=`<-`->/[!A`!A`!!A`!A;`\delta_A``\epsilon_{!A}`!\epsilon_A]
           \efig
          \end{array}
        \end{math}
      \end{center}
      We show the left most diagram commutes first.  It suffices to
      show that
      $\delta_a;!\delta_A = (\id_U,!F_1;F_1) = (\id_U,F_1;F_1)$.  
      Suppose $g \in U \Rightarrow (U \Rightarrow X^*)^*$.  Then       
      \begin{center}
        \small
        \begin{math}
          \begin{array}{lll}
            F_1(F_1(g)) \\
            \,\,= F_1(\lambda u.g(u)_1(u) \circ \cdots \circ g(u)_i(u))\\
            \,\,= \lambda u.g(u)_1(u)_1(u) \circ \cdots \circ g(u)_1(u)_j(u) \circ \cdots \circ g(u)_i(u)_1(u) \circ \cdots \circ g(u)_i(u)_k(u)\\
          \end{array}
        \end{math}
      \end{center}
      Consider the other direction in the diagram.  
      \begin{center}
        \begin{math}
          \begin{array}{lll}
            F_1(!F_1(g)) 
            & = & F_1(\lambda x.F^*_1(g(x)))\\
            & = & \lambda u.F_1(g(u)_1)(u) \circ \cdots \circ F_1(g(u)_i)(u)\\            
          \end{array}
        \end{math}
      \end{center}
      Note that we have the following:
      \begin{center}
        \begin{math}
          \begin{array}{lll}
            F_1(g(u)_1)(u) & = & g(u)_1(u)_1(u) \circ \cdots \circ g(u)_1(u)_k(u)\\
            & \vdots & \\
            F_1(g(u)_i)(u) & = & g(u)_i(u)_1(u) \circ \cdots \circ g(u)_i(u)_k(u)\\
          \end{array}
        \end{math}
      \end{center}
      Clearly, the above reasoning implies that $F_1;F_1 = !F_1;F_1$.

      Now we prove that the second diagram commutes, but we break it
      into two.  
      We define $\delta_A;!\epsilon_A = (\id_U,!F_0;F_1)$ where for
      any $g \in U \Rightarrow X^*$,
      \begin{center}
        \begin{math}
          \begin{array}{lll}
            (!F_0;F_1)(g) 
            & = & F_1(!F_0(g))\\
            & = & F_1(\lambda u'.F_0^*(g(u')))\\
            & = & F_1(\lambda u'.(\lambda y.(g(u')_1),\ldots,(\lambda y.g(u')_i)))\\
            & = & \lambda u.(g(u)_1) \circ \cdots \circ (g(u)_i)\\
            & = & g\\
          \end{array}
        \end{math}
      \end{center}
      and we can define $\delta_A;\epsilon_{!A} = (\id_U,F_0;F_1)$ where for
      any $g \in U \Rightarrow X^*$,
      \begin{center}
        \begin{math}
          \begin{array}{lll}
            (F_0;F_1)(g) 
            & = & F_1(F_0(g))\\
            & = & F_1(\lambda y.(g))\\
            & = & \lambda u.g(u)\\
            & = & g\\
          \end{array}
        \end{math}
      \end{center}
      We can see by the reasoning above that
      $!F_0;F_1 = F_0;F_1 = \id_{U \Rightarrow X^*}$.

    \item The monoidal natural transformation $m_{\top} : \top \to !\top$
      is defined by $m_{\top} = (\id_1,\lambda f.\star)$ where $\star \in \top$. 
      It is easy to see that the relation condition
      on maps for $\dial$ is satisfied.  The following two diagrams
      commute:
      \begin{center}
        \begin{math}
          \begin{array}{lll}
            \bfig
            \square[\top`!\top`!\top`!!\top;m_\top`m_\top`\delta_\top`!m_\top]
            \efig
            &
              \,\,\,\,\,\,\,\,\,\,\,\,\,\,\,\,\,\,\,\,
            &
              \bfig
              \btriangle/->`=`->/[\top`!\top`\top;m_\top``\epsilon_{\top}]
           \efig
          \end{array}
        \end{math}
      \end{center}
      It is straightforward to see that the above two diagrams commute
      using the fact that the second coordinate of $m_\top$ is a
      constant function.

    \item The monoidal natural transformation
      $m_{A,B} : !A \otimes !B \to !(A \otimes B)$ is defined by
      $m_{A,B} = (\id_{U \times V},F_2)$. We need to define $F_2$,
      but two auxiliary functions are needed first:
      \begin{center}
        \begin{math}
          \begin{array}{lll}
            \begin{array}{lll}
              h_1 : (U \times V) \Rightarrow ((V \Rightarrow X) \times (U \Rightarrow Y))^* \to 
                    (V \Rightarrow (U \Rightarrow X^*))\\
              h_1(g,v,u) = (f_1(v),\ldots,f_i(v))
              \text{ where } g(u,v) = ((f_1,g_1),\ldots,(f_i,g_i))
            \end{array}
            \\
            &  \\
            \begin{array}{lll}
              h_2 : (U \times V) \Rightarrow ((V \Rightarrow X) \times (U \Rightarrow Y))^* \to 
                    (U \Rightarrow (V \Rightarrow Y^*))\\
              h_2(g,u,v) = (g_1(u),\ldots,g_i(u))
              \text{ where } g(u,v) = ((f_1,g_1),\ldots,(f_i,g_i))
            \end{array}
          \end{array}
        \end{math}
      \end{center}
      Then $F_2(g) = (h_1(g),h_2(g))$.  In order for $m_{A,B}$ to be
      considered a full fledge map in $\dial$ we have to verify that
      the relation condition on maps is satisfied.  Suppose $(u,v) \in U \times V$ and 
      $g \in (U \times V) \Rightarrow ((V \Rightarrow X) \times (U \Rightarrow Y))^*$, 
      where $g(u,v) = ((f_1,g_1), \ldots,(f_i,g_i))$. Then we know the following by definition:
      \begin{center}
        \begin{math}
          \begin{array}{lll}
            ((u,v),F(g)) \in !\alpha \otimes !\beta 
            & \text{ iff } & ((u,v),(h_1(g),h_2(g))) \in !\alpha \otimes !\beta\\
            & \text{ iff } & (u,h_1(g)(v)) \in !\alpha \text{ and } (v,h_2(g)(u)) \in !\beta\\
            & \text{ iff } & (u,f_1(v)) \in \alpha \text{ and } \cdots \text{ and } (u,f_i(v)) \text{ and }\\
            &              & (v,g_1(u)) \in \beta \text{ and } \cdots \text{ and } (v,g_i(u)) 
          \end{array}
        \end{math}
      \end{center}
      and
      \begin{center}
        \begin{math}
          \begin{array}{lll}
            ((u,v),g) \in !(\alpha \otimes \beta) 
            & \text{ iff } & ((u,v),(f_1,g_1)) \in \alpha \otimes \beta \text{ and } \cdots \text{ and } \\
            &              & ((u,v),(f_i,g_i)) \in \alpha \otimes \beta\\
            & \text{ iff } & (u,f_1(v)) \in \alpha \text{ and } (v,g_1(u)) \in \beta \text{ and } \cdots \text{ and } \\
            &              & (u,f_i(v)) \in \alpha \text{ and } (v,g_i(u)) \in \beta
          \end{array}
        \end{math}
      \end{center}
      The previous definitions imply that
      $((u,v),F(g)) \in !\alpha \otimes !\beta$ implies
      $((u,v),g) \in !(\alpha \otimes \beta)$.  Thus, $m_{A,B}$ is a map in $\dial$.
      
      At this point we show that the following diagrams commute:
      \begin{center}
          \begin{mathpar}
            \bfig
            \square/->`<-`->`->/<700,700>[!\top \otimes !A`!(\top \otimes A)`\top \otimes !A`!A;m_{\top,A}`m_\top \otimes \id_{!A}`!\lambda_A`\lambda_{!A}]
            \place(350,350)[\text{A}]
            \efig
            \and
            \bfig
            \square/->`<-`->`->/<700,700>[!A \otimes !\top`!(A \otimes \top)`!A \otimes \top`!A;m_{A,\top}`\id_{!A} \otimes m_\top`!\rho_A`\rho_{!A}]
            \place(350,350)[\text{B}]
            \efig
          \end{mathpar}
      \end{center}
      \begin{center}
        \begin{mathpar}
            \bfig
            \square/->`->`->`->/<700,700>[!A \otimes !B`!(A \otimes B)`!B \otimes !A`!(B \otimes A);m_{A,B}`\beta`!\beta`m_{B,A}]
            \place(350,350)[\text{C}]
            \efig
            \and 
            \bfig
            \hSquares/->`->`->``->`->`->/[(!A \otimes !B) \otimes !C`!(A \otimes B) \otimes !C`!((A \otimes B) \otimes C)`!A \otimes (!B \otimes !C)`!A \otimes !(B \otimes C)`!(A \otimes (B \otimes C));m_{A,B} \otimes \id_{!C}`m_{A \otimes B,C}`\alpha_{!A,!B,!C}``!\alpha_{A,B,C}`\id_{!A} \otimes m_{B,C}`m_{A,B \otimes C}]
            \place(1300,250)[\text{D}]
            \efig
          \end{mathpar}          
      \end{center}
      We first prove that diagram A commutes, and then diagrams B, and
      C will commute using similar reasoning. Following this we show
      that diagram D commutes. It suffices to show that
      \[(m_\top \otimes id_{!A});m_{\top,A};!\lambda_A = 
      (\hat{\lambda}_U,\lambda g.F_\otimes(F_2(F_\lambda(g)))) = (\hat{\lambda}_U,\lambda g.(\diamond,\lambda t.\lambda u.g(u))).\]
      Suppose
      $g \in U \Rightarrow X^*$, $(\star,u) \in 1 \times U$, and
      $g(u) = (x_1,\ldots,x_i)$. Then 
      \begin{center}
        \begin{math}
          \begin{array}{lll}
            F_\lambda(g)(\star,u) 
            & = & (\lambda x'.(\diamond,\lambda y.x'))^*(g(\hat{\lambda}_U(\star,u)))\\
            & = & (\lambda x'.(\diamond,\lambda y.x'))^*(g(u))\\
            & = & (\lambda x'.(\diamond,\lambda y.x'))^*(x_1,\ldots,x_i)\\
            & = & ((\diamond,\lambda y.x_1),\ldots,(\diamond,\lambda y.x_i))\\
          \end{array}
        \end{math}
      \end{center}
      This implies that 
      \[ F_\lambda(g) = \lambda (\star,u).\mathsf{let}\,(x_1,\ldots,x_i) = g(u)\,\mathsf{in}\,((\diamond,\lambda y.x_1),\ldots,(\diamond,\lambda y.x_i)).\]
      Using this reasoning we can see the following:
      \begin{center}
        \begin{math}
          \begin{array}{lll}
            F_2(F_\lambda(g))
            & = & (\lambda u.\lambda t.\mathsf{let}\,((\diamond,\lambda y.x_1),\ldots,(\diamond,\lambda y.x_i)) = F_\lambda(g)(t,u)\,\mathsf{in}\,\\
                  & & (\diamond(u),\ldots,\diamond(u)),\\            
            & & \lambda t.\lambda u.\mathsf{let}\,((\diamond,\lambda y.x_1),\ldots,(\diamond,\lambda y.x_i)) = F_\lambda(g)(t,u)
                \,\mathsf{in}\, \\
            & & ((\lambda y.x_1)(t),\ldots,(\lambda y.x_1)(t)))\\
            & = & (\lambda u.\lambda t.\mathsf{let}\,((\diamond,\lambda y.x_1),\ldots,(\diamond,\lambda y.x_i)) = F_\lambda(g)(t,u)\,\mathsf{in}\,\\
            & & (\star,\ldots,\star),\\            
            & & \lambda t.\lambda u.\mathsf{let}\,((\diamond,\lambda y.x_1),\ldots,(\diamond,\lambda y.x_i)) = F_\lambda(g)(t,u)
                \,\mathsf{in}\, \\
            & & (x_1,\ldots,x_1))\\
            & = & (\lambda u.\lambda t.(\star,\ldots,\star),\\            
            & & \lambda t.\lambda u.\mathsf{let}\,((\diamond,\lambda y.x_1),\ldots,(\diamond,\lambda y.x_i)) = F_\lambda(g)(t,u)
                \,\mathsf{in}\, \\
            & & (x_1,\ldots,x_1))\\
            & = & (\lambda u.\lambda t.(\star,\ldots,\star), \lambda t.\lambda u.g(u))
          \end{array}
        \end{math}
      \end{center}
      Finally, the previous allows us to infer the following:
      \begin{center}
        \begin{math}
          \begin{array}{lll}
            F_\otimes(F_2(F_\lambda(g)))
            & = & (\diamond, \lambda t.\lambda u.g(u))\\
          \end{array}
        \end{math}
      \end{center}
      Thus, we obtained our desired result.

      \noindent
      We show that diagram D commutes by observing that
      \begin{center}
        \begin{math}
          \begin{array}{lll}
            (m_{A,B} \otimes !\id_{!C});m_{A\otimes B,C};!\alpha_{A,B,C}
            & = & (\id,F_\otimes);(\id,F_2);(\hat{\alpha},!F_\alpha)\\
            & = & (\hat{\alpha},!F_\alpha;F_2;F_\otimes)\\
            & = & (\hat{\alpha},F_2;F_\otimes;F_\alpha)\\
            & = & (\hat{\alpha},F_\alpha);(\id,F_\otimes);(\id,F_2)\\
          \end{array}
        \end{math}
      \end{center}
      It suffices to show that $!F_\alpha;F_2;F_\otimes = F_2;F_\otimes;F_\alpha$:
      \begin{center}
        \begin{math}
          \begin{array}{lll}
            (!F_\alpha;F_2;F_\otimes)(g)
            & = & F_\otimes(F_2(!F_\alpha(g)))\\
            & = & F_\otimes(F_2(\lambda x.F^*_\alpha(g(x))))\\
          \end{array}
        \end{math}
      \end{center}
      Suppose
      \begin{center}
        \vspace{-20px}
        \begin{math}
          \begin{array}{lll}
            h_1 = \lambda v.\lambda u.\mathsf{let}\,((f_1,g_1),\ldots,(f_i,g_i)) = F^*_\alpha(g(u,v))\,\mathsf{in}\,
            (f_1(v),\ldots,f_i(v)),\\
            \\
            h_2 = \lambda u.\lambda v.\mathsf{let}\,((f_1,g_1),\ldots,(f_i,g_i)) = F^*_\alpha(g(u,v)) \,\mathsf{in}\,
           (g_1(u),\ldots,g_i(u)), \text{ and }\\
           \\
           F^*_\alpha(g(u,v)) = \mathsf{let}\, ((f'_1,g'_1),\ldots,(f'_j,g'_j)) = g(u,v) \,\mathsf{in}\,\\
      \,\,\,\,\,\,\,\,\,\,(\lambda w.(\lambda v'.f_1'(v',w),\lambda u.g'_1(u)_1(w)),\lambda (u,v').g'_1(u)_2(v')),\ldots,\\
      \,\,\,\,\,\,\,\,\,\,\,(\lambda w.(\lambda v'.f_j'(v',w),\lambda u.g'_j(u)_1(w)),\lambda (u,v').g'_j(u)_2(v'))).
          \end{array}
        \end{math}
      \end{center}
      Then we can simplify $h_1$ and $h_2$ as follows:
      \begin{center}
        \begin{math}
          \begin{array}{lll}
            \begin{array}{lll}
              h_1 = \lambda v.\lambda u.\mathsf{let}\,((f'_1,g'_1),\ldots,(f'_j,g'_j)) = g(u,v)\,\mathsf{in}\,\\
              \,\,\,\,\,\,\,\,\,((\lambda v'.f_1'(v',v),\lambda u'.g'_1(u')_1(v)),\ldots,(\lambda v'.f_j'(v',v),\lambda u'.g'_j(u')_1(v))) \\
            \end{array}\\
            \text{ and }\\
            \begin{array}{lll}
              h_2 = \lambda u.\lambda v.\mathsf{let}\,(u_1,u_2) = u\,\mathsf{in}\, \\
              \,\,\,\,\,\,\,\mathsf{let}\,((f'_1,g'_1),\ldots,(f'_j,g'_j)) = g((u_1,u_2),v) \,\mathsf{in}\,\\
              \,\,\,\,\,\,\,\,\,\,\,(g'_1(u_1)_2(u_2),\ldots,g'_j(u_1)_2(u_2))
            \end{array}
          \end{array}
        \end{math}
      \end{center}
      By the definition of $F_2$ the previous reasoning implies:
      \begin{center}
        \begin{math}
          \begin{array}{lll}
            F_\otimes(F_2(\lambda x.F^*_\alpha(g(x))))
            & = & F_\otimes(h_1,h_2)\\
            & = & (\lambda v.F_2(h_1(v)),h_2)\\
          \end{array}
        \end{math}
      \end{center}
      Expanding the definition of $F_2(h_1(v))$ in the above definitions yields:
      \begin{center}
        \begin{math}
          \begin{array}{lll}
            F_2(h_1(v))
            & = & (h'_1,h'_2)\\
          \end{array}
        \end{math}
      \end{center}
      where
      \begin{center}
        \begin{math}
          \begin{array}{lll}
            h'_1 = \lambda v''.\lambda u''.(f'_1(v'',v),\ldots,f'_j(v'',v))\\
            h'_2 = \lambda u''.\lambda v''.(g'_1(u')_1(v),\ldots,g'_j(u')_1(v))\\
          \end{array}
        \end{math}
      \end{center}
      At this point we can see that
      \begin{center}
        \begin{math}
          \begin{array}{lll}
            (\lambda v.F_2(h_1(v)),h_2) & = & (\lambda v.(h'_1,h'_2),h_2)\\
          \end{array}
        \end{math}
      \end{center}
      We now simplify $F_2;F_\otimes;F_\alpha$.  We know by definition:
      \begin{center}
        \begin{math}
          \begin{array}{lll}
            F_2(g) & = & (h''_1,h''_2)
          \end{array}
        \end{math}
      \end{center}
      where
      \begin{center}
        \begin{math}
          \begin{array}{lll}
            \begin{array}{lll}
              h''_1 & = & \lambda v.\lambda u.\mathsf{let}\,((f'_1,g'_1),\ldots,(f'_j,g'_j)) = g(u,v)\,\mathsf{in}\\
              & & \,\,\,\,\,\,\,\,\,\mathsf{let}\,(v',v'') = v\,\mathsf{in}\,(f'_1(v',v''),\ldots,f'_k(v',v''))
              
            \end{array}\\
            \text{ and }\\
            \begin{array}{lll}
              h''_2 & = & \lambda u.\lambda v.\mathsf{let}\,((f'_1,g'_1),\ldots,(f'_j,g'_j)) = g(u,v)\,\mathsf{in}\\
              & & \,\,\,\,\,\,\,\,\,(g'_1(u),\ldots,g'_k(u))\\
            \end{array}
          \end{array}
        \end{math}
      \end{center}
      This implies that
      \begin{center}
        \begin{math}
          \begin{array}{lll}
            F_\alpha(F_\otimes(F_2(g)))
            & = & F_\alpha(F_\otimes(h''_1,h''_2))\\
            & = & F_\alpha(h''_1,\lambda u_4.F_2(h''_2(u_4)))\\
            & = & (\lambda w.(\lambda v.h''_1(v,w),\lambda u.F_2(h''_2(u))_1(w)),\lambda (u,v).F_2(h''_2(u))_2(v))
          \end{array}
        \end{math}
      \end{center}
      Finally, by expanding the definition of $F_2$ in the last line
      of the above reasoning we can see that
      \[ (\lambda v.(h'_1,h'_2),h_2) = (\lambda w.(\lambda v.h''_1(v,w),\lambda u.F_2(h''_2(u))_1(w)),\lambda (u,v).F_2(h''_2(u))_2(v)) \]
      modulo currying of set-theoretic functions.

    \item There are two coherence diagrams that $m_{A,B}$ and $\delta$
      must ad hear to.  They are listed as follows:
      \begin{center}
        \begin{mathpar}
          \bfig
          \square/->`->`->`=/<700,700>[!A \otimes !B`!(A \otimes B)`A \otimes B`A \otimes B;m_{A,B}`\epsilon_A \otimes \epsilon_B`\epsilon_{A \otimes B}`]
            \place(350,350)[\text{E}]
            \efig            
            \and
            \bfig
            \morphism(150,500)<1615,0>[`;m_{A,B}]
            \hSquares/``->``->`->`->/[!A \otimes !B``!(A \otimes B)`!!A \otimes !!B`!(!A \otimes !B)`!!(A \otimes B);``\delta_A \otimes \delta_B``\delta_{A \otimes B}`m_{!A,!B}`!m_{A,B}]
            \place(980,250)[\text{F}]
            \efig
        \end{mathpar}
      \end{center}
      Diagram E holds by simply expanding the definitions using an
      arbitrary input of a pair of functions.  We now show diagram F
      commutes.

      It suffices to show the following:
      \begin{center}
        \begin{math}
          \begin{array}{lll}
            m_{A,B};\delta_{A \otimes B}
            & = & (\id_{U \times V},F_1;F_2)\\
            & = & (id_{u \times V},!F_2;F_2;F_\otimes)\\
            & = & (\delta_A \otimes \delta_B);m_{!A,!B};!m_{A,B}
          \end{array}
        \end{math}
      \end{center}
      Suppose $g \in (U \times V) \Rightarrow ((U \times V)
      \Rightarrow ((V \Rightarrow X) \times (U \Rightarrow Y))^*)^*$.
      Then we know by the type of $g$ and the definition of $F_1$ it
      must be the case that $F_1(g)$ first extracts all of the
      functions $(f_1,\ldots,f_i)$ returned by $g(u,v)$ for arbitrary
      $u \in U$ and $v \in V$ -- note that each $f_i$ returns a
      sequence of pairs of functions,
      $((f'_i,g'_i),\ldots,(f'_j,g'_j))$ -- then $F_1(g)$ returns the
      concatenation of all of these sequences.  Finally, $F_2(F_1(g))$
      returns two functions $h_1(v,u)$ and $h_2(u,v)$, where $h_1$
      returns the sequence $(f'_i(v),\ldots,f'_j(v))$, and $h_2$
      returns the sequence $(g'_i(u),\ldots,g'_j(u))$ from the
      sequence returned by $F_1(g)$.  Note that each $f'_i$ and $g'_i$
      returns a pair of functions.

      Now consider applying $!F_2;F_2;F_\otimes$ to $g$.  The function
      $!F_2$ will construct the function $\lambda x.F^*_2(g(x))$ by
      definition, and $F^*_2(g(x))$ will construct a sequence of pairs
      of functions $((h'_1,h''_1),\ldots,(h'_k,h''_k))$. The function
      $g$ as we saw above returns a sequence of functions,
      $(f_1,\ldots,f_i)$, where each $f_i$ returns a sequence of pairs
      of functions, $((f'_i,g'_i),\ldots,(f'_j,g'_j))$.  This tells us
      that by definition $h'_k(v,u)$ will return the sequence
      $(f'_i(v),\ldots,f'_j(v))$ and $h''_k(u,v)$ will construct the
      sequence $(g'_1(u),\ldots,g'_j(u))$.  Applying $F_2$ to $\lambda
      x.F^*_2(g(x))$ will construct two more functions $t_1(v,u)$ and
      $t_2(u,v)$ where the first returns the sequence of functions
      $(h'_1(v),\ldots,h'_k(v))$, and the second returns
      $(h''_1(u),\ldots,h''_k(u))$.  Finally, applying the function
      $F_\otimes$ to the pair $(t_1,t_2)$ will result in a pair of
      functions
      \begin{center}
        \begin{math}
          \begin{array}{lll}
            (\lambda v.F_1(t_1(v)),\lambda u.F_1(t_2(u)))
            & = & (\lambda v.\lambda u.h'_1(v)(u) \circ \cdots \circ h'_k(v)(u),\\
            &   & \,\,\,\lambda u.\lambda v.h''_1(u)(v) \circ \cdots \circ h''_k(u)(v))\\
            & = & (\lambda v.\lambda u.(f'_i(v),\ldots,f'_j(v)),\\
            &   & \,\,\,\lambda u.\lambda v.(g'_i(u), \ldots, g'_k(u)))\\
          \end{array}
        \end{math}
      \end{center} 
      We can now see that the pair
      $(\lambda v.\lambda u.(f'_i(v),\ldots,f'_j(v)),\lambda u.\lambda v.(g'_i(u), \ldots, g'_k(u)))$ is indeed
      equivalent to the pair $(h_1,h_2)$ given above, and thus, the diagram commutes.      
    \end{itemize}
  \end{center}
  Next we must show that whenever $(!A,\delta)$ is a free comonoid, we
  have the distinguished natural transformations $e_A : !A \to \top$
  and $d_A : !A \to !A \otimes !A$.  Suppose $!A = (U, U \Rightarrow
  X^*)$ and $(!A,\delta)$ is a free comonoid.  Then we have the
  following definitions:
  \begin{itemize}
  \item (Proposition 53, page 77 of \cite{dePaiva:1988}).  We define
    $e_A : !A \to \top$ as the pair $(\diamond,\lambda x.\lambda
    u.())$, where $\diamond$ is the terminal map on $U$ and $()$ is
    the empty sequence.
    
  \item (Proposition 53, page 77 of \cite{dePaiva:1988}).  We define
    $d_A : !A \to !A \otimes !A$ as the pair $(\Delta, \theta)$
    where $\Delta : U \to U \times U$ is the diagonal map in
    $\sets$, and
    \begin{center}
      \begin{math}
        \begin{array}{lll}
          \theta : ((U \times U) \Rightarrow X^*) \times ((U \times U) \Rightarrow X^*) \to U \Rightarrow X^*\\
          \theta(f,g) = \lambda u.f(u,u) \circ g(u,u).
        \end{array}
      \end{math}
    \end{center}
  \end{itemize}
  The maps $e_A$ and $d_A$ must satisfy several coherence diagrams.
  \begin{itemize}
  \item We must show that the map $e_A$ is a monoidal natural
    transformation.  This requires that the following diagrams hold
    (for any arbitrary map $f$):
    \begin{center}
      \begin{mathpar}
        \bfig
        \square/->`->`=`->/<700,700>[!A`\top`!B`\top;e_A`!f``e_B]
        \place(350,350)[\text{G}]
        \efig
        \and
        \bfig
        \btriangle/->`=`->/<700,700>[\top`!\top`\top;m_I``e_\top]
        \place(230,250)[\text{H}]
        \efig            
        \and
        \bfig
        \square/->`->`->`->/<700,700>[!A \otimes !B`\top \otimes \top`!(A \otimes B)`\top;e_A \otimes e_B`m_{A,B}`\lambda`e_{A \otimes B}]
        \place(350,350)[\text{I}]
        \efig        
      \end{mathpar}
    \end{center}
    Diagrams G and H follow easily by the definition of $e_A$ and
    $m_I$.  We now show that diagram I commutes.  It suffices to show
    the following:
    \begin{center}
      \begin{math}
        \begin{array}{lll}
          (e_A \otimes e_B);\lambda
          & = & (\diamond_U \times \diamond_V,F_\otimes);(\hat{\lambda}_\top,F_\lambda) \\
          & = & ((\diamond_U \times \diamond_V);\hat{\lambda}_\top,F_\lambda;F_\otimes)\\
          & = & (\diamond_{U \times V},F_\lambda;F_\otimes)\\
          & = & (\diamond_{U \times V},F_2(\lambda u.()))\\
          & = & (\diamond_{U \times V},(\lambda x.\lambda u.());F_2)\\          
          & = & (\id_{U \times V};\diamond_{U \times V},(\lambda x.\lambda u.());F_2)\\
          & = & (\id_{U \times V},F_2);(\diamond_{U \times V},\lambda x.\lambda u.())\\
          & = & m_{A,B};e_{A \otimes B}
        \end{array}
      \end{math}
    \end{center}
    It suffices to show $F_\lambda;F_\otimes = F_2(\lambda u.())$, but
    this easily follows by definition.

  \item The map $d_A$ must be a monoidal natural transformation.  This
    requires the following diagrams to commute:
    \begin{center}
      \begin{mathpar}
        \bfig
        \square/->`->`->`->/<700,700>[!A`!A \otimes !A`!B`!B \otimes !B;d_A`!f`!f \otimes !f`d_B]
        \place(350,350)[\text{J}]
        \efig
        \and
        \bfig
        \square/->`->`->`->/<700,700>[\top`\top \otimes \top`!\top`!\top \otimes !\top;\lambda^{-1}`m_\top`m_\top \otimes m_\top`d_\top]
        \place(350,350)[\text{K}]
        \efig
        \and
        \bfig
        \morphism(190,0)<1880,0>[`;d_{A \otimes B}]
        \hSquares/->`->`->``->``/[!A \otimes !B`(!A \otimes !A) \otimes (!B \otimes !B)`(!A \otimes !B) \otimes (!A \otimes !B)`!(A \otimes B)``!(A \otimes B) \otimes !(A \otimes B);d_A \otimes d_B`iso`m_{A,B}``m_{A,B} \otimes m_{A,B}``]
        \place(1150,250)[\text{L}]
        \efig
      \end{mathpar}
    \end{center}
    Diagrams J and K follow easily from unfolding their
    definitions. We show that diagram L next.  The morphism $iso$ in
    $\dial$ is a isomorphism that can be built out of the SMCC
    structure.  For its definition in terms of the SMCC maps see
    footnote 9 on page 141 of \cite{Bierman:1994}, but we give a
    direct definition instead.  We will need the following
    definitions:
    \begin{center}
      \begin{math}
        \begin{array}{lll}
          \begin{array}{rll}
            \hat{iso}((u,u'),(v,v')) & = & ((u,v),(u',v'))
          \end{array}\\\\        
          \begin{array}{rll}            
            F_{iso}(f,g) & = & (\lambda (v',v'').(\lambda u'.\lambda u''.f(u',v')_1(v'',u''),\lambda u'.\lambda u''.g(u',v')_1(v'',u'')),\\
            && \,\,\lambda (u',u'').(\lambda v'.\lambda v''.f(u',v')_2(u'',v''),\lambda v'.\lambda v''.g(u',v')_1(u'',v'')))\\
          \end{array}\\\\
          \begin{array}{rll}
            F_{iso}^{-1}(h_1,h_2) & = & (\lambda (u,v).(\lambda v'.\lambda u'.h_1(v,v')_1(u,u'),\lambda u'.\lambda v'.h_2(u,u')_2(v,v')),\\
            && \,\,\lambda (u,v).(\lambda v'.\lambda u'.h_1(v,v')_2(u,u'),\lambda u'.\lambda v'.h_2(u,u')_2(v,v')))\\
          \end{array}
        \end{array}
      \end{math}
    \end{center}
    We omit the proof that $F_{iso}$ is an isomorphism, but it is
    straightforward.  Now $iso = (\hat{iso},F_{iso})$.

    It suffices to show the following:
    \begin{center}
      \begin{math}
        \begin{array}{lll}
          (d_A \otimes d_B);iso;(m_{A,B} \otimes m_{A,B})
          & = & (\Delta_U \times \Delta_V,F_\otimes);(\hat{iso},F_{iso});(\id_{(U \times V) \times (U \times V)},F_\otimes)\\
          & = & ((\Delta_U \times \Delta_V);\hat{iso};\id_{(U \times V) \times (U \times V)},F_\otimes;F_{iso};F_\otimes)\\
          & = & ((\Delta_U \times \Delta_V);\hat{iso};,F_\otimes;F_{iso};F_\otimes)\\
          & = & (\Delta_{U \times V},\Theta;F_2)\\
          & = & (\id_{U \times V};\Delta_{U \times V},\Theta;F_2)\\
          & = & (\id_{U \times V},F_2);(\Delta_{U \times V},\Theta)\\
          & = & m_{A,B};d_{A,B}\\
        \end{array}
      \end{math}
    \end{center}
    At this point it suffices to show that
    $F_\otimes;F_{iso};F_\otimes = \Theta;F_2$, but this follows using
    similar reasoning as above, because $F_{iso};F_\otimes$ will
    reorganize the streams obtained by applying $g_1$ and $g_2$, and
    then the final $F_\otimes$ combines these sequences using $\Theta$.
    However, $F_2$ does the same reorganization, and then the streams
    are combined using $\Theta$.
    %% \begin{center}
    %%   \begin{math}
    %%     \begin{array}{lll}
    %%       F_\otimes(F_{iso}(F_\otimes(g_1,g_2)))
    %%       & = & F_\otimes(F_{iso}(\lambda v.F_2(g_1(v)),\lambda u.F_2(g_2(u)))\\
    %%       & = & (\lambda v.\Theta(F_2(g_1(\Delta_V(v)))),\lambda u.\Theta(F_2(g_2(\Delta_U(u)))))\\
    %%     \end{array}
    %%   \end{math}
    %% \end{center}

    %% \begin{center}
    %%   \begin{math}
    %%     \begin{array}{lll}
    %%       F_2(\Theta(g_1,g_2))
    %%       & = & F_2(\lambda u.g_1(u,u) \circ g_2(u,u))\\
    %%     \end{array}
    %%   \end{math}
    %% \end{center}

  \item Suppose $A \in \obj{\dial}$.  Then we must show that
    $(!A,d_A,e_A)$ is a commutative comonoid, but this follows
    from the following diagrams:
    \begin{center}
      \begin{mathpar}
        \bfig
        \Atrianglepair/->`->`->`<-`->/<700,500>[!A`!A \otimes \top`!A \otimes !A`\top \otimes !A;\rho^{-1}`d_A`\lambda^{-1}`\id_{!A} \otimes e_A`e_A \otimes \id_{!A}]
        \place(450,150)[\text{M}]
        \place(950,150)[\text{N}]
        \efig \and
        \bfig
        \square/->`=`->`->/[!A`!A \otimes !A`!A`!A \otimes !A;d_A``\beta_{!A,!A}`d_A]
        \place(240,250)[\text{O}]        
        \efig \and
        \bfig
        \morphism(40,500)<1855,0>[`;d_{A}]
        \hSquares/``->``->`->`->/[!A``!A \otimes !A`!A \otimes !A`(!A \otimes !A) \otimes !A`!A \otimes (!A \otimes !A);``d_A``\id_{!A} \otimes d_A`d_A \otimes \id_{!A}`\alpha]
        \place(980,250)[\text{P}]        
        \efig                
      \end{mathpar}
    \end{center}
    We prove that diagram N commutes, and then diagrams M and O will
    commute by similar reasoning.  Following this we prove that
    diagram P commutes.

    It suffices to show the following:
    \begin{center}
      \begin{math}
        \begin{array}{lll}
          d_A;(e_A \otimes \id_{!A})
          & = & (\Delta,\Theta);(\diamond \times \id_U,F_\otimes)\\
          & = & (\Delta;(\diamond \times \id_U),F_\otimes;\Theta)\\
          & = & (\lambda u.(\diamond(u),u),F_\otimes;\Theta)\\
          & = & (\hat{\lambda}^{-1}_U,F_\otimes;\Theta)\\
          & = & (\hat{\lambda}^{-1},F_{\lambda^{-1}})\\
          & = & \lambda^{-1}\\
        \end{array}
      \end{math}
    \end{center}
    We can easily see that the following holds by definition:
    \begin{center}
      \begin{math}
        \begin{array}{lll}
          (F_\otimes;\Theta)(g_1,g_2)
          & = & \Theta(F_\otimes(g_1,g_2))\\
          & = & \Theta(\lambda v.\lambda u.(),\lambda u.g_2(\diamond(u)))\\
          & = & \lambda u.() \circ g_2(\diamond(u,u))\\
          & = & \lambda u.g_2(\diamond(u,u))\\
          & = & \lambda^{-1}(g_1,g_2)\\
        \end{array}
      \end{math}
    \end{center}

    Now we show that diagram P commutes. However, it is
    straightforward to show that the following holds:
    \begin{center}
      \begin{math}
        \begin{array}{lll}
          d_A;(\id_{!A} \otimes d_A)
          & = & (\Delta,\Theta);(\id_U \times \Delta,F_\otimes)\\
          & = & (\Delta;(\id_U \times \Delta),F_\otimes;\Theta)\\
          & = & (\Delta;(\id_U \times \Delta),F_\alpha;F_\otimes;\Theta)\\
          & = & (\Delta;(\Delta \times \id_U);\hat{\alpha},F_\alpha;F_\otimes;\Theta)\\
          & = & (\Delta,\Theta);(\Delta \times \id_U,F_\otimes);(\hat{\alpha},F_\alpha)\\
          & = & d_A;(d_A \otimes \id_{!A});\alpha\\
        \end{array}
      \end{math}
    \end{center}
    We can see that $F_\otimes;\Theta = F_\alpha;F_\otimes;\Theta$,
    because the right-hand side does the same as the left-hand side,
    but first reorganizes and does it on the opposite association.

  \item The map $e_A$ must be a coalgebra morphism which amounts to
    requiring that the following diagram commute:
    \begin{center}
      \begin{math}
        \bfig
        \square/->`->`->`->/[!A`\top`!!A`!\top;e_A`\delta_A`m_\top`!e_A]
        \place(240,250)[\text{Q}]        
        \efig
      \end{math}
    \end{center}
    Diagram Q commutes by unfolding definitions, and the fact that the
    second coordinate of $e_A$ is a constant function.

  \item The map $d_A$ must also be a coalgebra morphism, and hence,
    the following diagram must commute:
    \begin{center}
      \begin{math}
        \bfig
        \morphism(40,500)<1780,0>[`;\delta_{A}]
        \hSquares/``->``->`->`->/[!A``!!A`!A \otimes !A`!!A \otimes !!A`!(!A \otimes !A);``d_A``!d_A`\delta_A \otimes \delta_A`m_{!A,!A}]
        \place(980,250)[\text{R}]        
        \efig                
      \end{math}
    \end{center}
    It suffices to show that the following holds:
    \begin{center}
      \begin{math}
        \begin{array}{lll}
          \delta_A;!d_A
          & = & (\id_U,F_1);(\Delta,!\Theta)\\
          & = & (\Delta,!\Theta;F_1)\\
          & = & (\Delta;\id_{U \times U};\id_{U \times U},!\Theta;F_1)\\
          & = & (\Delta;(\id_{U} \times \id_{U});\id_{U \times U},!\Theta;F_1)\\
          & = & (\Delta;(\id_{U} \times \id_{U});\id_{U \times U},F_2;F_\otimes;\Theta)\\
          & = & (\Delta,\Theta);(\id_{U} \times \id_{U},F_\otimes);(\id_{U \times U},F_2)\\
          & = & d_A;(\delta_A \otimes \delta_A);m_{!A,!A}\\
        \end{array}
      \end{math}
    \end{center}
    Now consider the following:
    \begin{center}
      \begin{math}
        \begin{array}{lll}
          !(\Theta;F_1)(g)
          & = & F_1(!\Theta(g))\\
          & = & F_!(\lambda x.\Theta^*(g(x)))\\
          & = & \lambda p.\lett (f_1,\cdots,f_i) = \Theta^*(g(p)) \inn f_1(p) \circ \cdots \circ f_i(p)\\          
        \end{array}
      \end{math}
    \end{center}
    Furthermore, consider the following:
    \begin{center}
      \begin{math}
        \begin{array}{lll}
          (F_2;F_\otimes;\Theta)(g)
          & = & \Theta(F_\otimes(h_1,h_2))\\
          & = & \Theta(\lambda x.F_1(h_1(x)), \lambda y.F_1(h_2(y)))\\
          & = & \lambda u.F_1(h_1(u,u)) \circ F_1(h_2(u,u))\\
        \end{array}
      \end{math}
    \end{center}
    where
    \begin{center}
      \begin{math}
        \begin{array}{lll}
          h_1(u,u') & = & \lett (f_1,\ldots,f_i) = g(u,u') \inn (f_1(u),\ldots,f_i(u))\\
          h_2(u,u') & = & \lett (f_1,\ldots,f_i) = g(u,u') \inn (f_1(u'),\ldots,f_i(u'))\\
        \end{array}        
      \end{math}
    \end{center}
    At this point we can see that $\Theta;F_1 = F_2;F_\otimes;\Theta$
    by the previous reasoning and the definition of $F_1$.

  \item Finally, we show that when given a coalgebra morphism, $f$,
    between free coalgebras $(!A,\delta_A)$ and $(!B,\delta_B)$,
    i.e. making the following diagram commute:
    \begin{center}
    \begin{math}
      \bfig
      \square/->`->`->`->/[!A`!B`!!A`!!B;f`\delta_A`\delta_B`!f]
      \place(240,250)[\text{S}]        
      \efig
    \end{math}
    \end{center}
    Then it must be the case that the following commutes:
    \begin{center}
      \begin{math}
        \bfig
        \hSquares/<-`->`=`->`->`<-`->/[\top`!A`!A \otimes !A`\top`!B`!B \otimes !B;e_A`d_A``f`f \otimes f`e_B`d_B]
        \efig                
      \end{math}
    \end{center}
    It is straightforward to show that the previous diagram commutes
    using the definitions of the respective morphisms and the 
    assumption that $f$ is a coalgebras morphism.
  \end{itemize}   
\end{report}

\end{proof}

\begin{corollary}
  \label{corollary:dial-FLNL}
  The category $\dial$ is a full LNL model.
\end{corollary}
\begin{proof}
  This follows directly from the previous lemma and
  Theorem~\ref{thm:lnl_models_are_linear_categories} which shows that
  linear categories are LNL models.
\end{proof}

%% \textbf{Remark:} It would appear, from the fact that tensorial logic
%% \cite{Mellies:2008} is a relaxing of linear logic where instead of an
%% involutive negation we have a natural transformation $\eta_A\colon A
%% \to \neg\neg A$ that $\dial$ would be a model of tensorial
%% logic. After all in $\dial$ we do have a natural transformation of the
%% shape described, taking an object $(U,X, \alpha)$ to $(X, U, \neg
%% \alpha)$ and then to $(U, X, \neg\neg \alpha)$ which is "almost" an
%% isomorphism: we use identities in U and X, but unless the predicate
%% $\alpha$ itself is double-negated, we have a morphism $\alpha \to
%% \neg\neg \alpha$, but not a converse one. But we have not had the time
%% to check whether the other structure of tensorial logic is present and
%% hence we leave this to future work.
% section full_linear_categories (end)

The point of these calculations is to show that the several different
axiomatizations available for models for linear logic are consistent
and that a model proved sound and complete according to Seely's
definition (using the Seely isomorphisms $!(A \times B)\cong !A\otimes
!B$ and $!1\cong \top$ but adding to it monoidicicty of the comonad)
is indeed sound and complete as a LNL model too.

\section{Tensorial Logic}
\label{sec:tensor_logic}

Melli\`es and Tabareau introduced tensorial logic \cite{Mellies:2008} as a means of
generalizing linear logic to a theory of tensor and a non-involutive
negation called tensorial negation.  That is, instead of an
isomorphism $A = \lnot\lnot A$ we have only a natural transformation
$A \to \lnot\lnot A$, just as one does in FILL.  Tensorial logic makes the
claim that tensor and tensorial negation are more fundamental than
tensor and negation defined via implication into false, as in FILL.  This is at odds with
FILL where implication is considered  fundamental. 

In this
section we show that multiplicative tensorial logic can be modeled by
$\dial$ (Lemma~\ref{lemma:dial-tensorial}) by showing that tensorial
negation arises as a simple property of the implication in any symmetric monoidal closed category
(Lemma~\ref{lemma:monoidal-fact1}).  While this is expected (after all
negation being defined in terms of implication into absurdity is one
of the staples of intuitionism) we think it bolsters our claim that
linear implication is a fundamental connective that should not be
redefined in terms of the multiplicative disjunction par.  In any
case, any model of FILL can be seen as a model of multiplicative
tensorial logic.  

A categorical model of tensorial logic is a symmetric monoidal
category with a tensorial negation.
\begin{definition}
  \label{def:tensorial-negation}
  A \textbf{tensorial negation} on a symmetric monoidal category\\
  $(\cat{C},\otimes,I)$ is defined as a functor $\lnot : \cat{C} \to
  \catop{C}$ together with a family of bijections $\phi_{A,B,C} :
  \homs{C}{A \otimes B}{\lnot C} \cong \homs{C}{A}{\lnot (B \otimes C)}$
  natural in $A$, $B$, and $C$.  Furthermore, the following
  diagram must commute:
  \[
  \bfig
  \morphism(0,85)|l|/<-/<0,480>[`;\phi_{A,B \otimes C,D}]
  \vSquares/->``->```->`->/<300,300>%
              [\homs{}{A \otimes (B \otimes C)}{\lnot D}`%
                \homs{}{(A \otimes B) \otimes C}{\lnot D}``%
                \homs{}{A \otimes B}{\lnot (C \otimes D)}`%
                \homs{}{A}{\lnot ((B \otimes C) \otimes D)}`%
                \homs{}{A}{\lnot (B \otimes (C \otimes D))};%
                \homs{}{\alpha_{A,B,C}}{\id_{\lnot D}}``%
                \phi_{A \otimes B,C,D}```%
                \phi_{A,B,C \otimes D}`%
                \homs{}{\id_{A}}{\lnot \alpha_{B,C,D}}]
  \efig
  \]
\end{definition}
The most basic form of tensorial logic is called multiplicative tensorial
logic and only consists of a tensor and a tensorial negation.  The model
of multiplicative tensorial logic is called a dialogue category.
\begin{definition}
  \label{def:dialogue-cat}
  A \textbf{dialogue category} is a symmetric monoidal category
  equipped with a tensorial negation.
\end{definition}

We show that tensorial negation arises as a simple
property of implication, as is traditional.
\begin{lemma}
  \label{lemma:monoidal-fact1}
  In any monoidal closed category, $\cat{C}$, there is a natural bijection
  $\phi_{A,B,C,D} : \homs{C}{A \otimes B}{C \limp D} \cong \homs{C}{A}{(B \otimes C) \limp D}$.
  Furthermore, the following diagram commutes:
  \[
  \scriptsize
  \bfig
  \morphism(0,85)|l|/<-/<0,480>[`;\phi_{A,B \otimes C,D,E}]
  \vSquares/->``->```->`->/<300,300>%
              [\homs{}{A \otimes (B \otimes C)}{D \limp E}`%
                \homs{}{(A \otimes B) \otimes C}{D \limp E}``%
                \homs{}{A \otimes B}{(C \otimes D) \limp E}`%
                \homs{}{A}{((B \otimes C) \otimes D) \limp E}`%
                \homs{}{A}{(B \otimes (C \otimes D)) \limp E};%
                \homs{}{\alpha_{A,B,C}}{\id_{D \limp E}}``%
                \phi_{A \otimes B,C,D,E}```%
                \phi_{A,B,C \otimes D,E}`%
                \homs{}{\id_{A}}{\alpha_{B,C,D} \limp E}]
  \efig
  \]
\end{lemma}
\begin{proof}
  Suppose $\cat{C}$ is a monoidal closed category.  Then we can define
  $\phi(f : A \otimes B \to C \limp D) = \cur{\alpha^{-1} ; \curi{f}}$
  and $\phi^{-1}(g : A \to (B \otimes C) \limp D) = \cur{\alpha ;
    \curi{g}}$.  Clearly, these are mutual inverses, and hence, $\phi$
  is a bijection.  Naturality of $\phi$ easily follows. \begin{paper}
    Lastly, the diagram given above also commutes.  For the complete
    proof see the companion report \cite{Eades:2015}.
  \end{paper}

  \begin{report}
    \ \\
  \noindent
  Suppose $f : A \otimes (B \otimes C) \to D \limp E$.  The required
  diagram commutes by the following equational reasoning:
  \begin{center}
    \begin{math}
      \begin{array}{lllllllll}
        \phi(\phi(\alpha;f))
        & = & \phi(\cur{\alpha^{-1};\curi{\alpha;f}}) & \quad & \text{(Definition)}\\
        & = & \cur{\alpha^{-1};\curi{\cur{\alpha^{-1};\curi{\alpha;f}}}} && \text{(Definition)}\\
        & = & \cur{\alpha^{-1};(\alpha^{-1};\curi{\alpha;f})} && \text{(Inverses)}\\
        & = & \cur{(\alpha^{-1};\alpha^{-1});\curi{\alpha;f}} && \text{(Associativity)}\\
        & = & \cur{(\alpha^{-1};\alpha^{-1});(\alpha \otimes \id);\curi{f}} && \text{(Naturality of } \mathsf{cur}\text{)}\\
        & = & \cur{(\id \otimes \alpha^{-1});\alpha^{-1};(\alpha^{-1} \otimes \id);(\alpha \otimes \id);\curi{f}} && \text{(Monoidal Pentagon)}\\
        & = & \cur{(\id \otimes \alpha^{-1});\alpha^{-1};(\alpha^{-1};\alpha \otimes \id);\curi{f}} && \text{(Functorality)}\\
        & = & \cur{(\id \otimes \alpha^{-1});\alpha^{-1};(\id \otimes \id);\curi{f}} && \text{(Inverses)}\\
        & = & \cur{(\id \otimes \alpha^{-1});\alpha^{-1};\id;\curi{f}} && \text{(Functorality)}\\
        & = & \cur{(\id \otimes \alpha^{-1});\alpha^{-1};\curi{f}} && \text{(Identity)}\\
        & = & \cur{\alpha^{-1};\curi{f}};(\alpha^{-1} \limp \id) && \text{(Naturality of } \mathsf{cur}\text{)}\\
        & = & \phi(f);(\alpha^{-1} \limp \id) && \text{(Definition)}\\
      \end{array}
    \end{math}
  \end{center}
  \end{report}
\end{proof}
Any model of FILL contains the unit of par, $\perp$, and thus, can be
used to define the negation function $\lnot A := A \limp \perp$.  Now
replacing $D$ and $E$ in the previous result with $\perp$ yields the
definition of tensorial negation.
\begin{lemma}
  \label{lemma:dial-tensorial}
  $\dial$ is a model of multiplicative tensorial logic.
\end{lemma}
\begin{proof}
  We have already shown $\dial$ to be a model of FILL, and thus, it has a
  symmetric monoidl closed structure as well as the negation functor, and thus, by
  Lemma~\ref{lemma:monoidal-fact1} it has a tensorial
  negation\footnote{We give a full proof in the formalization see the
    file
    \url{https://github.com/heades/cut-fill-agda/blob/master/Tensorial.agda}.}.
\end{proof}

Extending a model of multiplicative tensorial logic with an exponential
resource modality yields a model of full tensorial logic.
\begin{definition}
  \label{def:resource-modality}
  A \textbf{resource modality} on a symmetric monoidal category\\
  $(\cat{C}, \otimes, I)$ is an adjunction with a symmetric monoidal
  category $(\cat{M}, \otimes', I')$:
  \[
  \bfig
  \morphism(0,0)|a|/{@{>}@/^1em/}/<500,0>[\cat{M}`\cat{C};F]
  \morphism(0,0)|b|/{@{<-}@/_1em/}/<500,0>[\cat{M}`\cat{C};G]
  \efig
  \]
  A resource modality is called an \textbf{exponential resource
    modality} if $\cat{M}$ is cartesian where $\otimes'$ is the
  product and $I'$ is the terminal object.
\end{definition}

A model of full tensorial logic is defined to be a model of
multiplicative tensorial logic with an exponential resource
modality. We now know that $\dial$ is a model of multiplicative
tensorial logic.  By constructing the co-Kleisli category which
consists of the $!$-coalgebras as objects, and happens to be
cartesian, we can show that $\dial$ is a model of full tensorial
logic.  The adjunction with the co-Kleisli category naturally arises
from the proof that $\dial$ is a full LNL model
(Corollary~\ref{corollary:dial-FLNL}).
\begin{lemma}
  \label{lemma:full-tensorial-logic}
  The category $\dial$ is a model of full tensorial logic.
\end{lemma}
\begin{proof}
    It suffices to show that there is an adjunction between $\dial$ and
  a cartesian category.  Define the category $\dial_!$ as follows:
  \begin{itemize}
  \item Take as objects $(U , (U \Rightarrow X^*), \alpha_!)$
    where $U$ and $X$ are sets, and $\alpha \subseteq U \times (U \Rightarrow X^*)$.
    
  \item Take as morphisms $(f , F) : (U , (U \Rightarrow X^*),
    \alpha_!) \to (V , (V \Rightarrow Y^*), \beta_!)$ where $f : U \to
    V$ and $F : (V \Rightarrow Y^*) \to (U \Rightarrow X^*)$ subject
    to the same condition on morphisms as $\dial$.  Composition and
    identities are defined similarly to $\dial$.
  \end{itemize}
  \begin{paper}
    
    Next we must show that $\dial_!$ is cartesian.  Notice that
    $\dial_!$ is a subcategory of $\dial$, and there is a functor
    \\ $J : \dial \to \dial_!$ which is defined equivalently to the
    endofunctor $!$ from the proof of
    Lemma~\ref{lemma:ddial_is_a_linear_category}.  In fact, $\dial_!$
    is the co-Kleisli category with objects free !-coalgebras and is
    cartesian closed \cite{dePaiva:1987}.  However, we only need the
    fact that it is cartesian.
    
    To show that $\dial_!$ is cartesian it suffices to show that $J$
    preserves the cartesian structure of $\dial$ -- the proof that
    $\dial$ is cartesian can be found on page 48 of
    \cite{dePaiva:1988}.  This follows by straightforward reasoning.
    For the complete proof see the companion report \cite{Eades:2015}.
  \end{paper}
  \begin{report}
  Now we show that show that $\dial_!$ is cartesian.  Notice that
  $\dial_!$ is a subcategory of $\dial$, and there is a functor $J :
  \dial \to \dial_!$ which is defined equivalently to the endofunctor
  $!$ from the proof of Lemma~\ref{lemma:ddial_is_a_linear_category}.
  In fact, $\dial_!$ is the co-Kleisli category with objects free
  !-coalgebras and is cartesian closed \cite{dePaiva:1987}.  However,
  we only need the fact that it is cartesian.
  \begin{itemize}
  \item Suppose $A = (U , X , \alpha)$ and $B = (V , Y , \beta)$ are
    objects of $\dial$. Then we define $A \times B = (U \times V, (X +
    Y), \alpha \times \beta)$, where $((u , v), i) \in \alpha \times
    \beta$ iff when $i \in X$, then $(u , i) \in \alpha$, otherwise
    when $i \in Y$, then $(v , i) \in \beta$.  Now the cartesian
    product in $\dial_!$ is defined as $J(A \times B)$.

  \item The terminal object in $\dial$ is defined by $(\top, \perp,
    \alpha_\top)$ where $\top$ and $\perp$ are the terminal and
    initial objects in $\sets$ respectively, and $(x , y) \in
    \alpha_\top$ iff true. The proof that this is terminal, and is the
    unit to the cartesian product can be found in the formal
    development.
    
  \item Suppose $A = (U , X , \alpha)$, $B = (V , Y , \beta)$, and $C
    = (W , Z , \gamma)$ are objects of $\dial$.  Then we define
    the following morphisms -- the proofs of the morphism conditions
    have been omitted, but the details can be found in the formal
    development:
    \begin{itemize}
    \item Define the first projection by $(\pi_1, F_{\pi_1}) : J(A
      \times B) \to J(A)$, where $\pi_1$ is the first projection in
      $\sets$, and $F_{\pi_1}$ takes a function $f : U \to X^*$ and a
      pair $(u , v) \in U \times V$, and returns the sequence $(X +
      Y)^*$ by mapping the first coproduct injection over $f(u ,v)$.

    \item The second projection is defined similar to the first,
      $(\pi_2, F_{\pi_2}) : J(A \times B) \to J(B)$, but $F_{\pi_2}$ maps
      the second projection instead of the first.

    \item Suppose $j_1 = (f , F) : J(C) \to J(A)$ and $j_2 = (g , G) : J(C) \to
      J(B)$ are morphisms in $\dial_!$.  Then we define the morphism
      $(j_1, j_2) = ((f,g),\lambda j.\lambda w.F(h_1(j),w) \circ F(h_2(j),w))
      : J(C) \to J(A \times B)$, where $(f,g) = \lambda w.(f(w),g(w))$ is the unique morphism
      from the cartesian structure of $\sets$, and
      $h_1(j) = \lambda u.\iota_1(j(u,g(w)))$ and
      $h_2(j) = \lambda u.\iota_2(j(u,f(w)))$.
      Notice that $F$ is defined in terms of unique morphisms, and thus,
      $(j_1,j_2)$ is unique.
    \end{itemize}
    All of the previous morphisms satisfy the usual diagram for
    cartesian products.    
  \end{itemize}

  By Lemma~\ref{corollary:dial-FLNL} we know $\dial$ is a full LNL
  model, and thus, there is an adjunction between $\dial$ and
  $\dial_!$ where the left-adjoint is the forgetful functor $G :
  \dial_! \to \dial$, and the right adjoint is the free functor $J :
  \dial \to \dial_!$.  Therefore, $\dial$ is a model of full tensorial
  logic, as expected.
  \end{report}

\end{proof}

\subsection{Double Negation Translation}
\label{subsec:a_negative_translation_of_linear_logic_into_fill_using_tensorial_negation}

In this section we show that we can use intuitionistic negation --
which we showed was tensorial in the previous section -- to define a
negative translation of multi-conclusion linear logic
(Figure~\ref{fig:LL}) into FILL where implication plays a central
role.  Melli\`es and Tabareau give a negative translation of the
multiplicative fragment of linear logic into tensorial logic
\cite{Mellies:2010} using tensor as the main connective.  For example,
they define $( \FILLnt{A}  \otimes  \FILLnt{B} )^N =  \lnot  \FILLsym{(}     \lnot   ( \FILLnt{A} )^N     \otimes   \lnot   ( \FILLnt{B} )^N     \FILLsym{)} $, and thus,
they simulate par using tensor and negation.  This definition would
cause some syntactic issues with FILL, because the left-rule to par
requires the let-pattern term defined in
Definition~\ref{def:let-pat-term}, thus, encoding par in terms of
tensor would require the let-pattern term to be used in the left-rule
for tensor.  While simulating par, using tensor and negation, can be
seen as useful, in applications where only the tensor product can be
actually calculated, in other applications we do have an extra
bifunctor like par. This is true in the case of FILL, so we can modify
Melli\`es and Tabareau's translation into one that better fits the
source logical system.
\begin{figure}
  \begin{center}
    \begin{mathpar}
      \FILLdruleLLXXAx{} \and
      \FILLdruleLLXXCut{} \and
      \FILLdruleLLXXTl{} \and
      \FILLdruleLLXXTr{} \and
      \FILLdruleLLXXTenl{} \and
      \FILLdruleLLXXTenr{} \and
      \FILLdruleLLXXPl{} \and
      \FILLdruleLLXXPr{} \and
      \FILLdruleLLXXParl{} \and
      \FILLdruleLLXXParr{} \and
      \FILLdruleLLXXImpl{} \and
      \FILLdruleLLXXImpr{} \and
      \FILLdruleLLXXExl{} \and
      \FILLdruleLLXXExr{} \and
    \end{mathpar}
  \end{center}
  \caption{Multi-Conclusion Linear Logic}
  \label{fig:LL}
\end{figure}

The following definition provides a translation of linear logic formulas into FILL formulas.
\begin{definition}
  \label{def:ll-to-fill}
  The following is the double-negation translation of linear logic
  into FILL:
  \begin{center}
  \begin{math}
    \begin{array}{lll}
       (  \top  )^N  & = &  \top \\
       (  \perp  )^N  & = &  \perp \\
       (  \FILLnt{A} ^{\perp}  )^N  & = &  \lnot  \FILLsym{(}   ( \FILLnt{A} )^N   \FILLsym{)} \\
       (  \FILLnt{A}  \parr  \FILLnt{B}  )^N  & = &    \lnot    \lnot  \FILLsym{(}   ( \FILLnt{A} )^N   \FILLsym{)}      \parr   \lnot    \lnot  \FILLsym{(}   ( \FILLnt{B} )^N   \FILLsym{)}    \\
       (  \FILLnt{A}  \otimes  \FILLnt{B}  )^N  & = &    \lnot    \lnot  \FILLsym{(}   ( \FILLnt{A} )^N   \FILLsym{)}      \otimes   \lnot    \lnot  \FILLsym{(}   ( \FILLnt{B} )^N   \FILLsym{)}    \\
    \end{array}
  \end{math}
\end{center}
\end{definition}
\noindent
The main point of the previous definition is that because FILL has
intuitionistic versions of all of the operators of linear logic we can
give a very natural translation that preserves these operators by
double negating their arguments.  

At this point we need to extend the translation of linear logic
formulas to sequents.  However, we must be careful, because in FILL
implication has the FILL restriction, and thus, if we choose the wrong
translation then we will run into problems trying to satisfy the FILL
condition.  The method we employ here is to first translate a linear
logic sequent into a single-sided sequent, and then translate that to
FILL using the well-known translation.  That is, it is easy to see
that any linear logic sequent $ \FILLnt{A_{{\mathrm{1}}}}  \FILLsym{,} \, ... \, \FILLsym{,}  \FILLnt{A_{\FILLmv{i}}}  \vdash  \FILLnt{B_{{\mathrm{1}}}}  \FILLsym{,} \, ... \, \FILLsym{,}  \FILLnt{B_{\FILLmv{j}}} $ is
logically equivalent to the sequent $  \cdot   \vdash   \FILLnt{A_{{\mathrm{1}}}} ^{\perp}   \FILLsym{,} \, ... \, \FILLsym{,}   \FILLnt{A_{\FILLmv{i}}} ^{\perp}   \FILLsym{,}  \FILLnt{B_{{\mathrm{1}}}}  \FILLsym{,} \, ... \, \FILLsym{,}  \FILLnt{B_{\FILLmv{j}}} $.  Then we translate the latter into FILL as $ \FILLmv{x_{{\mathrm{1}}}}  \FILLsym{:}   \lnot  \FILLsym{(}   (  \FILLnt{A_{{\mathrm{1}}}} ^{\perp}  )^N   \FILLsym{)}   \FILLsym{,} \, ... \, \FILLsym{,}  \FILLmv{x_{\FILLmv{i}}}  \FILLsym{:}   \lnot  \FILLsym{(}   (  \FILLnt{A_{\FILLmv{i}}} ^{\perp}  )^N   \FILLsym{)}   \FILLsym{,}  \FILLmv{y_{{\mathrm{1}}}}  \FILLsym{:}   \lnot  \FILLsym{(}   ( \FILLnt{B_{{\mathrm{1}}}} )^N   \FILLsym{)}   \FILLsym{,} \, ... \, \FILLsym{,}  \FILLmv{y_{\FILLmv{j}}}  \FILLsym{:}   \lnot  \FILLsym{(}   ( \FILLnt{B_{\FILLmv{j}}} )^N   \FILLsym{)}   \vdash   \cdot  $ for any free variables $\FILLmv{x_{{\mathrm{1}}}},\ldots,\FILLmv{x_{\FILLmv{i}}}$ and
$\FILLmv{y_{{\mathrm{1}}}},\ldots,\FILLmv{y_{\FILLmv{j}}}$, but this is equivalent to $ \FILLmv{x_{{\mathrm{1}}}}  \FILLsym{:}   \lnot    \lnot  \FILLsym{(}   ( \FILLnt{A_{{\mathrm{1}}}} )^N   \FILLsym{)}     \FILLsym{,} \, ... \, \FILLsym{,}  \FILLmv{x_{\FILLmv{i}}}  \FILLsym{:}   \lnot    \lnot  \FILLsym{(}   ( \FILLnt{A_{\FILLmv{i}}} )^N   \FILLsym{)}     \FILLsym{,}  \FILLmv{y_{{\mathrm{1}}}}  \FILLsym{:}   \lnot  \FILLsym{(}   ( \FILLnt{B_{{\mathrm{1}}}} )^N   \FILLsym{)}   \FILLsym{,} \, ... \, \FILLsym{,}  \FILLmv{y_{\FILLmv{j}}}  \FILLsym{:}   \lnot  \FILLsym{(}   ( \FILLnt{B_{\FILLmv{j}}} )^N   \FILLsym{)}   \vdash   \cdot  $.  The reader may realize that this is indeed the
translation of single-sided classical linear logic into
single-conclusion intuitionistic linear logic.  This translation also
has the benefit that we do not have to worry about mentioning terms in
the statement of the result.

\begin{lemma}[Negative Translation]
  \label{lemma:negative_translation}
  If $ \FILLnt{A_{{\mathrm{1}}}}  \FILLsym{,} \, ... \, \FILLsym{,}  \FILLnt{A_{\FILLmv{i}}}  \vdash  \FILLnt{B_{{\mathrm{1}}}}  \FILLsym{,} \, ... \, \FILLsym{,}  \FILLnt{B_{\FILLmv{j}}} $ is derivable, then for any
  unique fresh variables
  $\FILLmv{x_{{\mathrm{1}}}},\ldots,\FILLmv{x_{\FILLmv{i}}}$, and
  $\FILLmv{y_{{\mathrm{1}}}},\ldots,\FILLmv{y_{\FILLmv{j}}}$, the sequent
  $ \FILLmv{x_{{\mathrm{1}}}}  \FILLsym{:}   \lnot    \lnot  \FILLsym{(}   ( \FILLnt{A_{{\mathrm{1}}}} )^N   \FILLsym{)}     \FILLsym{,} \, ... \, \FILLsym{,}  \FILLmv{x_{\FILLmv{i}}}  \FILLsym{:}   \lnot    \lnot  \FILLsym{(}   ( \FILLnt{A_{\FILLmv{i}}} )^N   \FILLsym{)}     \FILLsym{,}  \FILLmv{y_{{\mathrm{1}}}}  \FILLsym{:}   \lnot  \FILLsym{(}   ( \FILLnt{B_{{\mathrm{1}}}} )^N   \FILLsym{)}   \FILLsym{,} \, ... \, \FILLsym{,}  \FILLmv{y_{\FILLmv{j}}}  \FILLsym{:}   \lnot  \FILLsym{(}   ( \FILLnt{B_{\FILLmv{j}}} )^N   \FILLsym{)}   \vdash   \cdot  $
  is derivable.
\end{lemma}
\begin{proof}
  \begin{paper}
    This can be shown by induction on the assumed sequent.  For the
    complete proof see the companion report \cite{Eades:2015}.
  \end{paper}
  \begin{report}
    This is a proof by induction on the form of the assumed sequent
    $ \FILLnt{A_{{\mathrm{1}}}}  \FILLsym{,} \, ... \, \FILLsym{,}  \FILLnt{A_{\FILLmv{i}}}  \vdash  \FILLnt{B_{{\mathrm{1}}}}  \FILLsym{,} \, ... \, \FILLsym{,}  \FILLnt{B_{\FILLmv{j}}} $.  Throughout the proof if $\Gamma =
    \FILLnt{A_{{\mathrm{1}}}}  \FILLsym{,} \, ... \, \FILLsym{,}  \FILLnt{A_{\FILLmv{i}}}$, then $ \lnot (  ( \Gamma )^N  )  = \FILLmv{x_{{\mathrm{1}}}}  \FILLsym{:}   \lnot  \FILLsym{(}   ( \FILLnt{A_{{\mathrm{1}}}} )^N   \FILLsym{)}   \FILLsym{,} \, ... \, \FILLsym{,}  \FILLmv{x_{\FILLmv{i}}}  \FILLsym{:}   \lnot  \FILLsym{(}   ( \FILLnt{A_{\FILLmv{i}}} )^N   \FILLsym{)} $ for some unique fresh variables
    $\FILLmv{x_{{\mathrm{1}}}},\ldots,\FILLmv{x_{\FILLmv{i}}}$.  We will use this notation when applying
    the inductive hypothesis to make the presentation easier to read.
  \begin{itemize}
  \item[] Case.\\ 
    \begin{center}
      \begin{math}
        $$\mprset{flushleft}
        \inferrule* [right=\ifrName{Ax}] {
          \ 
        }{ \FILLnt{A}  \vdash  \FILLnt{A} }
      \end{math}
    \end{center}
    It suffices to derive the sequent
    $ \FILLmv{x}  \FILLsym{:}   \lnot    \lnot  \FILLsym{(}   ( \FILLnt{A} )^N   \FILLsym{)}     \FILLsym{,}  \FILLmv{y}  \FILLsym{:}   \lnot  \FILLsym{(}   ( \FILLnt{A} )^N   \FILLsym{)}   \vdash   \cdot  $
    which is equivalent to the sequent
    $ \FILLmv{x}  \FILLsym{:}     \lnot  \FILLsym{(}   ( \FILLnt{A} )^N   \FILLsym{)}    \multimap    \perp    \FILLsym{,}  \FILLmv{y}  \FILLsym{:}   \lnot  \FILLsym{(}   ( \FILLnt{A} )^N   \FILLsym{)}   \vdash   \cdot  $.  The latter is
    derivable as follows:
    \begin{center}
      \begin{math}
        $$\mprset{flushleft}
        \inferrule* [right=Impl] {
          $$\mprset{flushleft}
          \inferrule* [right=Ax] {
            \,
          }{ \FILLmv{y}  \FILLsym{:}   \lnot  \FILLsym{(}   ( \FILLnt{A} )^N   \FILLsym{)}   \vdash  \FILLmv{y}  \FILLsym{:}   \lnot  \FILLsym{(}   ( \FILLnt{A} )^N   \FILLsym{)}  }
          \\
          $$\mprset{flushleft}
          \inferrule* [right=Pl] {
            \,
          }{ \FILLmv{w}  \FILLsym{:}   \perp   \vdash   \cdot  }
        }{ \FILLmv{x}  \FILLsym{:}     \lnot  \FILLsym{(}   ( \FILLnt{A} )^N   \FILLsym{)}    \multimap    \perp    \FILLsym{,}  \FILLmv{y}  \FILLsym{:}   \lnot  \FILLsym{(}   ( \FILLnt{A} )^N   \FILLsym{)}   \vdash   \cdot  }
      \end{math}
    \end{center}
    

  \item[] Case.\\ 
    \begin{center}
      \begin{math}
        $$\mprset{flushleft}
        \inferrule* [right=\ifrName{Cut}] {
            \Gamma_{{\mathrm{1}}}  \vdash  \FILLnt{A}  \FILLsym{,}  \Delta_{{\mathrm{1}}}   \quad   \Gamma_{{\mathrm{2}}}  \FILLsym{,}  \FILLnt{A}  \vdash  \Delta_{{\mathrm{2}}}  
        }{ \Gamma_{{\mathrm{1}}}  \FILLsym{,}  \Gamma_{{\mathrm{2}}}  \vdash  \Delta_{{\mathrm{1}}}  \FILLsym{,}  \Delta_{{\mathrm{2}}} }
      \end{math}
    \end{center}
    By the induction hypothesis we know the following:
    \begin{center}
      \begin{math}
        \begin{array}{lll}
            \lnot (  \lnot (  ( \Gamma_{{\mathrm{1}}} )^N  )  )   \FILLsym{,}  \FILLmv{x}  \FILLsym{:}   \lnot  \FILLsym{(}   ( \FILLnt{A} )^N   \FILLsym{)}   \FILLsym{,}   \lnot (  ( \Delta_{{\mathrm{1}}} )^N  )   \vdash   \cdot  \\
            \lnot (  \lnot (  ( \Gamma_{{\mathrm{2}}} )^N  )  )   \FILLsym{,}  \FILLmv{y}  \FILLsym{:}   \lnot  \FILLsym{(}   \lnot  \FILLsym{(}   ( \FILLnt{A} )^N   \FILLsym{)}   \FILLsym{)}   \FILLsym{,}   \lnot (  ( \Delta_{{\mathrm{2}}} )^N  )   \vdash   \cdot  
        \end{array}
      \end{math}
    \end{center}
    It suffices to derive the sequent $  \lnot (  \lnot (  ( \Gamma_{{\mathrm{1}}} )^N  )  )   \FILLsym{,}   \lnot (  \lnot (  ( \Gamma_{{\mathrm{2}}} )^N  )  )   \FILLsym{,}   \lnot (  ( \Delta_{{\mathrm{1}}} )^N  )   \FILLsym{,}   \lnot (  ( \Delta_{{\mathrm{2}}} )^N  )   \vdash   \cdot  $ which we do as follows:
    \begin{center}
      \begin{math}
        $$\mprset{flushleft}
        \inferrule* [right=Cut] {
          $$\mprset{flushleft}
        \inferrule* [right=Impr] {
          $$\mprset{flushleft}
        \inferrule* [right=Pr] {
            \lnot (  \lnot (  ( \Gamma_{{\mathrm{1}}} )^N  )  )   \FILLsym{,}  \FILLmv{x}  \FILLsym{:}   \lnot  \FILLsym{(}   ( \FILLnt{A} )^N   \FILLsym{)}   \FILLsym{,}   \lnot (  ( \Delta_{{\mathrm{1}}} )^N  )   \vdash   \cdot  
        }{  \lnot (  \lnot (  ( \Gamma_{{\mathrm{1}}} )^N  )  )   \FILLsym{,}  \FILLmv{x}  \FILLsym{:}   \lnot  \FILLsym{(}   ( \FILLnt{A} )^N   \FILLsym{)}   \FILLsym{,}   \lnot (  ( \Delta_{{\mathrm{1}}} )^N  )   \vdash   \circ   \FILLsym{:}   \perp  }
        }{  \lnot (  \lnot (  ( \Gamma_{{\mathrm{1}}} )^N  )  )   \FILLsym{,}   \lnot (  ( \Delta_{{\mathrm{1}}} )^N  )   \vdash   \lambda  \FILLmv{x}  .   \circ    \FILLsym{:}   \lnot  \FILLsym{(}   \lnot  \FILLsym{(}   ( \FILLnt{A} )^N   \FILLsym{)}   \FILLsym{)}  }
        \\
          \lnot (  \lnot (  ( \Gamma_{{\mathrm{2}}} )^N  )  )   \FILLsym{,}  \FILLmv{y}  \FILLsym{:}   \lnot  \FILLsym{(}   \lnot  \FILLsym{(}   ( \FILLnt{A} )^N   \FILLsym{)}   \FILLsym{)}   \FILLsym{,}   \lnot (  ( \Delta_{{\mathrm{2}}} )^N  )   \vdash   \cdot  
        }{  \lnot (  \lnot (  ( \Gamma_{{\mathrm{1}}} )^N  )  )   \FILLsym{,}   \lnot (  \lnot (  ( \Gamma_{{\mathrm{2}}} )^N  )  )   \FILLsym{,}   \lnot (  ( \Delta_{{\mathrm{1}}} )^N  )   \FILLsym{,}   \lnot (  ( \Delta_{{\mathrm{2}}} )^N  )   \vdash   \cdot  }
      \end{math}
    \end{center}

  \item[] Case.\\ 
    \begin{center}
      \begin{math}
        $$\mprset{flushleft}
        \inferrule* [right=\ifrName{Tl}] {
           \Gamma  \vdash  \Delta 
        }{ \Gamma  \FILLsym{,}   \top   \vdash  \Delta }
      \end{math}
    \end{center}
    First, note that $ (  \top  )^N  =  \top $.  By the induction hypothesis
    we know $  \lnot (  \lnot (  ( \Gamma )^N  )  )   \FILLsym{,}   \lnot (  ( \Delta )^N  )   \vdash   \cdot  $.  We obtain our result by
    the following derivation:
    \begin{center}
      \begin{math}
        $$\mprset{flushleft}
        \inferrule* [right=Impl] {
          $$\mprset{flushleft}
          \inferrule* [right=Impr] {
            $$\mprset{flushleft}
            \inferrule* [right=Pr] {
              $$\mprset{flushleft}
              \inferrule* [right=Tl] {
                  \lnot (  \lnot (  ( \Gamma )^N  )  )   \FILLsym{,}   \lnot (  ( \Delta )^N  )   \vdash   \cdot  
              }{  \lnot (  \lnot (  ( \Gamma )^N  )  )   \FILLsym{,}   \lnot (  ( \Delta )^N  )   \FILLsym{,}  \FILLmv{z}  \FILLsym{:}   \top   \vdash   \cdot  }
            }{  \lnot (  \lnot (  ( \Gamma )^N  )  )   \FILLsym{,}   \lnot (  ( \Delta )^N  )   \FILLsym{,}  \FILLmv{z}  \FILLsym{:}   \top   \vdash  \FILLnt{t}  \FILLsym{:}   \perp  }
          }{  \lnot (  \lnot (  ( \Gamma )^N  )  )   \FILLsym{,}   \lnot (  ( \Delta )^N  )   \vdash  \FILLnt{t}  \FILLsym{:}   \lnot   \top   }
          \\
          $$\mprset{flushleft}
          \inferrule* [right=Pl] {
            \,
          }{ \FILLmv{y}  \FILLsym{:}   \perp   \vdash   \cdot  }
        }{  \lnot (  \lnot (  ( \Gamma )^N  )  )   \FILLsym{,}  \FILLmv{x}  \FILLsym{:}   \lnot  \FILLsym{(}   \lnot   \top    \FILLsym{)}   \FILLsym{,}   \lnot (  ( \Delta )^N  )   \vdash   \cdot  }
      \end{math}
    \end{center}
    Note that $ \mathsf{let}\, \FILLmv{x} \,\mathsf{be}\,  *  \,\mathsf{in}\,  \cdot   = \cdot$ for any variable $\FILLmv{x}$.

  \item[] Case.\\ 
    \begin{center}
      \begin{math}
        $$\mprset{flushleft}
        \inferrule* [right=\ifrName{Tr}] {
          \ 
        }{  \cdot   \vdash   \top  }
      \end{math}
    \end{center}
    If suffices to derive the sequent $ \FILLmv{x}  \FILLsym{:}   \lnot  \FILLsym{(}   (  \top  )^N   \FILLsym{)}   \vdash   \cdot  $, but this is equivalent to the sequent
    $ \FILLmv{x}  \FILLsym{:}   \lnot   \top    \vdash   \cdot  $ which is derivable as follows:
    \begin{center}
      \begin{math}
        $$\mprset{flushleft}
        \inferrule* [right=Impl] {
          $$\mprset{flushleft}
          \inferrule* [right=Tr] {
            \,
          }{  \cdot   \vdash  \FILLsym{*}  \FILLsym{:}   \top  }
          \\
            $$\mprset{flushleft}
          \inferrule* [right=Pl] {
            \,
          }{ \FILLmv{y}  \FILLsym{:}   \perp   \vdash   \cdot  }
        }{ \FILLmv{x}  \FILLsym{:}   \lnot   \top    \vdash   \cdot  }
      \end{math}
    \end{center}
    
  \item[] Case.\\ 
    \begin{center}
      \begin{math}
        $$\mprset{flushleft}
        \inferrule* [right=\ifrName{Tenl}] {
           \Gamma  \FILLsym{,}  \FILLnt{A}  \FILLsym{,}  \FILLnt{B}  \vdash  \Delta 
        }{ \Gamma  \FILLsym{,}   \FILLnt{A}  \otimes  \FILLnt{B}   \vdash  \Delta }
      \end{math}
    \end{center}
    By the induction hypothesis we have the following sequent:
    \begin{center}
      \begin{math}
        \begin{array}{lll}
            \lnot (  \lnot (  ( \Gamma )^N  )  )   \FILLsym{,}  \FILLmv{x}  \FILLsym{:}   \lnot    \lnot  \FILLsym{(}   ( \FILLnt{A} )^N   \FILLsym{)}     \FILLsym{,}  \FILLmv{y}  \FILLsym{:}   \lnot    \lnot  \FILLsym{(}   ( \FILLnt{B} )^N   \FILLsym{)}     \FILLsym{,}   \lnot (  ( \Delta )^N  )   \vdash   \cdot  
        \end{array}
      \end{math}
    \end{center}
    Our result follows from the following derivation:
    \begin{center}
      \begin{math}
        $$\mprset{flushleft}
        \inferrule* [right=Impl] {
         $$\mprset{flushleft}
          \inferrule* [right=Impr] {
            $$\mprset{flushleft}
            \inferrule* [right=Tenl] {
              $$\mprset{flushleft}
              \inferrule* [right=Pr] {
                  \lnot (  \lnot (  ( \Gamma )^N  )  )   \FILLsym{,}  \FILLmv{w}  \FILLsym{:}    \lnot    \lnot  \FILLsym{(}   ( \FILLnt{A} )^N   \FILLsym{)}      \FILLsym{,}  \FILLmv{v}  \FILLsym{:}   \lnot    \lnot  \FILLsym{(}   ( \FILLnt{B} )^N   \FILLsym{)}     \FILLsym{,}   \lnot (  ( \Delta )^N  )   \vdash   \cdot  
              }{  \lnot (  \lnot (  ( \Gamma )^N  )  )   \FILLsym{,}  \FILLmv{w}  \FILLsym{:}    \lnot    \lnot  \FILLsym{(}   ( \FILLnt{A} )^N   \FILLsym{)}      \FILLsym{,}  \FILLmv{v}  \FILLsym{:}   \lnot    \lnot  \FILLsym{(}   ( \FILLnt{B} )^N   \FILLsym{)}     \FILLsym{,}   \lnot (  ( \Delta )^N  )   \vdash   \circ   \FILLsym{:}   \perp  }
            }{  \lnot (  \lnot (  ( \Gamma )^N  )  )   \FILLsym{,}  \FILLmv{x}  \FILLsym{:}     \lnot    \lnot  \FILLsym{(}   ( \FILLnt{A} )^N   \FILLsym{)}      \otimes   \lnot    \lnot  \FILLsym{(}   ( \FILLnt{B} )^N   \FILLsym{)}      \FILLsym{,}   \lnot (  ( \Delta )^N  )   \vdash    \mathsf{let}\, \FILLmv{x} \,\mathsf{be}\, \FILLsym{(}   \FILLmv{w}  \otimes  \FILLmv{v}   \FILLsym{)} \,\mathsf{in}\,  \circ     \FILLsym{:}   \perp  }
          }{   \lnot (  \lnot (  ( \Gamma )^N  )  )   \FILLsym{,}   \lnot (  ( \Delta )^N  )   \vdash   \lambda  \FILLmv{x}  .    \mathsf{let}\, \FILLmv{x} \,\mathsf{be}\, \FILLsym{(}   \FILLmv{w}  \otimes  \FILLmv{v}   \FILLsym{)} \,\mathsf{in}\,  \circ      \FILLsym{:}   \lnot  \FILLsym{(}     \lnot    \lnot  \FILLsym{(}   ( \FILLnt{A} )^N   \FILLsym{)}      \otimes   \lnot    \lnot  \FILLsym{(}   ( \FILLnt{B} )^N   \FILLsym{)}      \FILLsym{)}  }
          \\
          $$\mprset{flushleft}
          \inferrule* [right=Pl] {
            \,
          }{ \FILLmv{y}  \FILLsym{:}   \perp   \vdash   \cdot  }
        }{  \lnot (  \lnot (  ( \Gamma )^N  )  )   \FILLsym{,}  \FILLmv{z}  \FILLsym{:}   \lnot    \lnot  \FILLsym{(}     \lnot    \lnot  \FILLsym{(}   ( \FILLnt{A} )^N   \FILLsym{)}      \otimes   \lnot    \lnot  \FILLsym{(}   ( \FILLnt{B} )^N   \FILLsym{)}      \FILLsym{)}     \FILLsym{,}   \lnot (  ( \Delta )^N  )   \vdash   \cdot  }
      \end{math}
    \end{center}

  \item[] Case.\\ 
    \begin{center}
      \begin{math}
        $$\mprset{flushleft}
        \inferrule* [right=\ifrName{Tenr}] {
            \Gamma  \vdash  \FILLnt{A}  \FILLsym{,}  \Delta   \quad   \Gamma'  \vdash  \FILLnt{B}  \FILLsym{,}  \Delta'  
        }{ \Gamma  \FILLsym{,}  \Gamma'  \vdash     \FILLnt{A}  \otimes  \FILLnt{B}    \FILLsym{,}  \Delta   \FILLsym{,}  \Delta' }
      \end{math}
    \end{center}
    By the induction hypothesis we know the following:
    \begin{center}
      \begin{math}
        \begin{array}{lll}
            \lnot (  \lnot (  ( \Gamma )^N  )  )   \FILLsym{,}  \FILLmv{x}  \FILLsym{:}    \lnot  \FILLsym{(}   ( \FILLnt{A} )^N   \FILLsym{)}    \FILLsym{,}   \lnot (  ( \Delta )^N  )   \vdash   \cdot  \\
            \lnot (  \lnot (  ( \Gamma' )^N  )  )   \FILLsym{,}  \FILLmv{y}  \FILLsym{:}    \lnot  \FILLsym{(}   ( \FILLnt{B} )^N   \FILLsym{)}    \FILLsym{,}   \lnot (  ( \Delta' )^N  )   \vdash   \cdot  \\
        \end{array}
      \end{math}
    \end{center}
    We obtain our result by the following derivation:
    \begin{center}
      \tiny
      \begin{math}
        $$\mprset{flushleft}
        \inferrule* [right=\ifrName{Impl}] {
          $$\mprset{flushleft}
          \inferrule* [right=\ifrName{Tenr}] {
            $$\mprset{flushleft}
            \inferrule* [right=\ifrName{Impr}] {
              $$\mprset{flushleft}
              \inferrule* [right=\ifrName{Pr}] {
                  \lnot (  \lnot (  ( \Gamma )^N  )  )   \FILLsym{,}  \FILLmv{x}  \FILLsym{:}   \lnot  \FILLsym{(}   ( \FILLnt{A} )^N   \FILLsym{)}   \FILLsym{,}   \lnot (  ( \Delta )^N  )   \vdash   \cdot  
              }{  \lnot (  \lnot (  ( \Gamma )^N  )  )   \FILLsym{,}  \FILLmv{x}  \FILLsym{:}   \lnot  \FILLsym{(}   ( \FILLnt{A} )^N   \FILLsym{)}   \FILLsym{,}   \lnot (  ( \Delta )^N  )   \vdash   \circ   \FILLsym{:}   \perp  }
            }{  \lnot (  \lnot (  ( \Gamma )^N  )  )   \FILLsym{,}   \lnot (  ( \Delta )^N  )   \vdash   \lambda  \FILLmv{x}  .   \circ    \FILLsym{:}    \lnot    \lnot  \FILLsym{(}   ( \FILLnt{A} )^N   \FILLsym{)}     }
            \\
            $$\mprset{flushleft}
            \inferrule* [right=\ifrName{Impr}] {
              $$\mprset{flushleft}
              \inferrule* [right=\ifrName{Pr}] {
                  \lnot (  \lnot (  ( \Gamma' )^N  )  )   \FILLsym{,}  \FILLmv{y}  \FILLsym{:}   \lnot  \FILLsym{(}   ( \FILLnt{B} )^N   \FILLsym{)}   \FILLsym{,}   \lnot (  ( \Delta' )^N  )   \vdash   \cdot  
              }{  \lnot (  \lnot (  ( \Gamma' )^N  )  )   \FILLsym{,}  \FILLmv{y}  \FILLsym{:}   \lnot  \FILLsym{(}   ( \FILLnt{B} )^N   \FILLsym{)}   \FILLsym{,}   \lnot (  ( \Delta' )^N  )   \vdash   \circ   \FILLsym{:}   \perp  }
            }{  \lnot (  \lnot (  ( \Gamma' )^N  )  )   \FILLsym{,}   \lnot (  ( \Delta' )^N  )   \vdash   \lambda  \FILLmv{y}  .   \circ    \FILLsym{:}   \lnot    \lnot  \FILLsym{(}   ( \FILLnt{B} )^N   \FILLsym{)}    }
          }{  \lnot (  \lnot (  ( \Gamma )^N  )  )   \FILLsym{,}   \lnot (  \lnot (  ( \Gamma' )^N  )  )   \FILLsym{,}   \lnot (  ( \Delta )^N  )   \FILLsym{,}   \lnot (  ( \Delta' )^N  )   \vdash    (  \lambda  \FILLmv{x}  .   \circ   )   \otimes   (  \lambda  \FILLmv{y}  .   \circ   )    \FILLsym{:}     \lnot    \lnot  \FILLsym{(}   ( \FILLnt{A} )^N   \FILLsym{)}      \otimes   \lnot    \lnot  \FILLsym{(}   ( \FILLnt{B} )^N   \FILLsym{)}     }
          \\
          $$\mprset{flushleft}
          \inferrule* [right=\ifrName{Pl}] {
            \,
          }{ \FILLmv{w}  \FILLsym{:}   \perp   \vdash   \cdot  }
        }{  \lnot (  \lnot (  ( \Gamma )^N  )  )   \FILLsym{,}   \lnot (  \lnot (  ( \Gamma' )^N  )  )   \FILLsym{,}  \FILLmv{z}  \FILLsym{:}    \lnot  \FILLsym{(}     \lnot    \lnot  \FILLsym{(}   ( \FILLnt{A} )^N   \FILLsym{)}      \otimes   \lnot    \lnot  \FILLsym{(}   ( \FILLnt{B} )^N   \FILLsym{)}      \FILLsym{)}    \FILLsym{,}   \lnot (  ( \Delta )^N  )   \FILLsym{,}   \lnot (  ( \Delta' )^N  )   \vdash   \cdot  }
      \end{math}
    \end{center}

  \item[] Case.\\ 
    \begin{center}
      \begin{math}
        $$\mprset{flushleft}
        \inferrule* [right=\ifrName{Pl}] {
          \ 
        }{  \perp   \vdash   \cdot  }
      \end{math}
    \end{center}
    It suffices to show that $ \FILLmv{z}  \FILLsym{:}   \lnot  \FILLsym{(}   (  \perp  )^N   \FILLsym{)}   \vdash   \cdot  $ is derivable, which is equivalent to showing that
    the sequent $ \FILLmv{z}  \FILLsym{:}   \lnot    \lnot  \FILLsym{(}   \perp   \FILLsym{)}     \vdash   \cdot  $ is derivable.  We obtain our result by the following derivation:
    \begin{center}
      \begin{math}
        $$\mprset{flushleft}
        \inferrule* [right=Impl] {
          $$\mprset{flushleft}
          \inferrule* [right=Impr] {
            $$\mprset{flushleft}
            \inferrule* [right=Pr] {
              $$\mprset{flushleft}
              \inferrule* [right=Pl] {
                \,
              }{ \FILLmv{x}  \FILLsym{:}   \perp   \vdash   \cdot  }
            }{ \FILLmv{x}  \FILLsym{:}   \perp   \vdash   \circ   \FILLsym{:}   \perp  }
          }{  \cdot   \vdash   \lambda  \FILLmv{x}  .   \circ    \FILLsym{:}   \lnot  \FILLsym{(}   \perp   \FILLsym{)}  }
          \\
            $$\mprset{flushleft}
          \inferrule* [right=Pl] {
            \,
          }{ \FILLmv{w}  \FILLsym{:}   \perp   \vdash   \cdot  }
        }{ \FILLmv{z}  \FILLsym{:}   \lnot    \lnot  \FILLsym{(}   \perp   \FILLsym{)}     \vdash   \cdot  }
      \end{math}
    \end{center}


  \item[] Case.\\ 
    \begin{center}
      \begin{math}
        $$\mprset{flushleft}
        \inferrule* [right=\ifrName{Pr}] {
           \Gamma  \vdash  \Delta 
        }{ \Gamma  \vdash   \perp   \FILLsym{,}  \Delta }
      \end{math}
    \end{center}
    By the induction hypothesis we know the following sequent is derivable:
    \[
        \lnot (  \lnot (  ( \Gamma )^N  )  )   \FILLsym{,}   \lnot (  ( \Delta )^N  )   \vdash   \cdot  
    \]
    It suffices to show that $  \lnot (  \lnot (  ( \Gamma )^N  )  )   \FILLsym{,}  \FILLmv{z}  \FILLsym{:}   \lnot  \FILLsym{(}   (  \perp  )^N   \FILLsym{)}   \FILLsym{,}   \lnot (  ( \Delta )^N  )   \vdash   \cdot  $ which is equivalent to
    the sequent $  \lnot (  \lnot (  ( \Gamma )^N  )  )   \FILLsym{,}  \FILLmv{z}  \FILLsym{:}   \lnot  \FILLsym{(}   \perp   \FILLsym{)}   \FILLsym{,}   \lnot (  ( \Delta )^N  )   \vdash   \cdot  $.  The latter is derivable as follows:
    \begin{center}
      \begin{math}
        $$\mprset{flushleft}
        \inferrule* [right=Impl] {
          $$\mprset{flushleft}
          \inferrule* [right=Pr] {
              \lnot (  \lnot (  ( \Gamma )^N  )  )   \FILLsym{,}   \lnot (  ( \Delta )^N  )   \vdash   \cdot  
          }{  \lnot (  \lnot (  ( \Gamma )^N  )  )   \FILLsym{,}   \lnot (  ( \Delta )^N  )   \vdash   \circ   \FILLsym{:}   \perp  }
          \\
            $$\mprset{flushleft}
          \inferrule* [right=Pl] {
            \,
          }{ \FILLmv{w}  \FILLsym{:}   \perp   \vdash   \cdot  }
        }{  \lnot (  \lnot (  ( \Gamma )^N  )  )   \FILLsym{,}  \FILLmv{z}  \FILLsym{:}   \lnot  \FILLsym{(}   \perp   \FILLsym{)}   \FILLsym{,}   \lnot (  ( \Delta )^N  )   \vdash   \cdot  }
      \end{math}
    \end{center}
    
  \item[] Case.\\ 
    \begin{center}
      \begin{math}
        $$\mprset{flushleft}
        \inferrule* [right=\ifrName{Parl}] {
            \Gamma  \FILLsym{,}  \FILLnt{A}  \vdash  \Delta   \quad   \Gamma'  \FILLsym{,}  \FILLnt{B}  \vdash  \Delta'  
        }{ \Gamma  \FILLsym{,}  \Gamma'  \FILLsym{,}   \FILLnt{A}  \parr  \FILLnt{B}   \vdash  \Delta  \FILLsym{,}  \Delta' }
      \end{math}
    \end{center}
    By the induction hypothesis we know the following:
    \begin{center}
      \begin{math}
        \begin{array}{lll}
            \lnot (  \lnot (  ( \Gamma )^N  )  )   \FILLsym{,}  \FILLmv{x}  \FILLsym{:}   \lnot    \lnot  \FILLsym{(}   ( \FILLnt{A} )^N   \FILLsym{)}     \FILLsym{,}   \lnot (  ( \Delta )^N  )   \vdash   \cdot  \\
            \lnot (  \lnot (  ( \Gamma' )^N  )  )   \FILLsym{,}  \FILLmv{v}  \FILLsym{:}   \lnot    \lnot  \FILLsym{(}   ( \FILLnt{B} )^N   \FILLsym{)}     \FILLsym{,}   \lnot (  ( \Delta' )^N  )   \vdash   \cdot  
        \end{array}
      \end{math}
    \end{center}
    It suffices to show that the sequent
    \[   \lnot (  \lnot (  ( \Gamma )^N  )  )   \FILLsym{,}   \lnot (  \lnot (  ( \Gamma' )^N  )  )   \FILLsym{,}  \FILLmv{z}  \FILLsym{:}   \lnot    \lnot  \FILLsym{(}   (  \FILLnt{A}  \parr  \FILLnt{B}  )^N   \FILLsym{)}     \FILLsym{,}   \lnot (  ( \Delta )^N  )   \FILLsym{,}   \lnot (  ( \Delta' )^N  )   \vdash   \cdot   \]
    is derivable which is equivalent to the sequent
    \[   \lnot (  \lnot (  ( \Gamma )^N  )  )   \FILLsym{,}   \lnot (  \lnot (  ( \Gamma' )^N  )  )   \FILLsym{,}  \FILLmv{z}  \FILLsym{:}   \lnot    \lnot  \FILLsym{(}     \lnot    \lnot  \FILLsym{(}   ( \FILLnt{A} )^N   \FILLsym{)}      \parr   \lnot    \lnot  \FILLsym{(}   ( \FILLnt{B} )^N   \FILLsym{)}      \FILLsym{)}     \FILLsym{,}   \lnot (  ( \Delta )^N  )   \FILLsym{,}   \lnot (  ( \Delta' )^N  )   \vdash   \cdot  . \]
    The latter is derivable as follows:
    \begin{center}
      \begin{math}        
        $$\mprset{flushleft}
        \inferrule* [right=\ifrName{Impl}] {
          D
          \\
            $$\mprset{flushleft}
          \inferrule* [right=\ifrName{Pl}] {
            \,
          }{ \FILLmv{w}  \FILLsym{:}   \perp   \vdash   \cdot  }
        }{  \lnot (  \lnot (  ( \Gamma )^N  )  )   \FILLsym{,}   \lnot (  \lnot (  ( \Gamma' )^N  )  )   \FILLsym{,}  \FILLmv{z}  \FILLsym{:}   \lnot    \lnot  \FILLsym{(}     \lnot    \lnot  \FILLsym{(}   ( \FILLnt{A} )^N   \FILLsym{)}      \parr   \lnot    \lnot  \FILLsym{(}   ( \FILLnt{B} )^N   \FILLsym{)}      \FILLsym{)}     \FILLsym{,}   \lnot (  ( \Delta )^N  )   \FILLsym{,}   \lnot (  ( \Delta' )^N  )   \vdash   \cdot  }
      \end{math}
    \end{center}
    where
    \begin{center}
      \begin{math}
        \begin{array}{lll}
          D:\\
          \\
          $$\mprset{flushleft}
          \inferrule* [right=\ifrName{Impr}] {
            $$\mprset{flushleft}
            \inferrule* [right=\ifrName{Pr}] {
              $$\mprset{flushleft}
              \inferrule* [right=\ifrName{Parl}] {
                  \lnot (  \lnot (  ( \Gamma )^N  )  )   \FILLsym{,}  \FILLmv{x}  \FILLsym{:}   \lnot    \lnot  \FILLsym{(}   ( \FILLnt{A} )^N   \FILLsym{)}     \FILLsym{,}   \lnot (  ( \Delta )^N  )   \vdash   \cdot  
                \\
                  \lnot (  \lnot (  ( \Gamma' )^N  )  )   \FILLsym{,}  \FILLmv{v}  \FILLsym{:}   \lnot    \lnot  \FILLsym{(}   ( \FILLnt{B} )^N   \FILLsym{)}     \FILLsym{,}   \lnot (  ( \Delta' )^N  )   \vdash   \cdot  
              }{  \lnot (  \lnot (  ( \Gamma )^N  )  )   \FILLsym{,}   \lnot (  \lnot (  ( \Gamma' )^N  )  )   \FILLsym{,}  \FILLmv{y}  \FILLsym{:}     \lnot    \lnot  \FILLsym{(}   ( \FILLnt{A} )^N   \FILLsym{)}      \parr   \lnot    \lnot  \FILLsym{(}   ( \FILLnt{B} )^N   \FILLsym{)}      \FILLsym{,}   \lnot (  ( \Delta )^N  )   \FILLsym{,}   \lnot (  ( \Delta' )^N  )   \vdash   \cdot  }
            }{  \lnot (  \lnot (  ( \Gamma )^N  )  )   \FILLsym{,}   \lnot (  \lnot (  ( \Gamma' )^N  )  )   \FILLsym{,}  \FILLmv{y}  \FILLsym{:}     \lnot    \lnot  \FILLsym{(}   ( \FILLnt{A} )^N   \FILLsym{)}      \parr   \lnot    \lnot  \FILLsym{(}   ( \FILLnt{B} )^N   \FILLsym{)}      \FILLsym{,}   \lnot (  ( \Delta )^N  )   \FILLsym{,}   \lnot (  ( \Delta' )^N  )   \vdash   \circ   \FILLsym{:}   \perp  }
          }{  \lnot (  \lnot (  ( \Gamma )^N  )  )   \FILLsym{,}   \lnot (  \lnot (  ( \Gamma' )^N  )  )   \FILLsym{,}   \lnot (  ( \Delta )^N  )   \FILLsym{,}   \lnot (  ( \Delta' )^N  )   \vdash   \lambda  \FILLmv{y}  .   \circ    \FILLsym{:}   \lnot  \FILLsym{(}     \lnot    \lnot  \FILLsym{(}   ( \FILLnt{A} )^N   \FILLsym{)}      \parr   \lnot    \lnot  \FILLsym{(}   ( \FILLnt{B} )^N   \FILLsym{)}      \FILLsym{)}  }
        \end{array}
      \end{math}
    \end{center}
  \item[] Case.\\ 
    \begin{center}
      \begin{math}
        $$\mprset{flushleft}
        \inferrule* [right=\ifrName{Parr}] {
           \Gamma  \vdash    \Delta  \FILLsym{,}  \FILLnt{A}   \FILLsym{,}  \FILLnt{B}   \FILLsym{,}  \Delta' 
        }{ \Gamma  \vdash   \Delta  \FILLsym{,}   \FILLnt{A}  \parr  \FILLnt{B}    \FILLsym{,}  \Delta' }
      \end{math}
    \end{center}
    By the induction hypothesis we know the following is derivable:
    \[   \lnot (  \lnot (  ( \Gamma )^N  )  )   \FILLsym{,}   \lnot (  ( \Delta )^N  )   \FILLsym{,}  \FILLmv{x}  \FILLsym{:}   \lnot  \FILLsym{(}   ( \FILLnt{A} )^N   \FILLsym{)}   \FILLsym{,}  \FILLmv{y}  \FILLsym{:}   \lnot  \FILLsym{(}   ( \FILLnt{B} )^N   \FILLsym{)}   \FILLsym{,}   \lnot (  ( \Delta' )^N  )   \vdash   \cdot   \]
    It suffices to show that the sequent
    $ \lnot (  \lnot (  ( \Gamma )^N  )  )   \FILLsym{,}   \lnot (  ( \Delta )^N  )   \FILLsym{,}  \FILLmv{z}  \FILLsym{:}   \lnot  \FILLsym{(}   (  \FILLnt{A}  \parr  \FILLnt{B}  )^N   \FILLsym{)}   \FILLsym{,}   \lnot (  ( \Delta' )^N  ) $
    is derivable, but this sequent is equivalent to the sequent
    $ \lnot (  \lnot (  ( \Gamma )^N  )  )   \FILLsym{,}   \lnot (  ( \Delta )^N  )   \FILLsym{,}  \FILLmv{z}  \FILLsym{:}   \lnot  \FILLsym{(}     \lnot    \lnot  \FILLsym{(}   ( \FILLnt{A} )^N   \FILLsym{)}      \parr   \lnot    \lnot  \FILLsym{(}   ( \FILLnt{B} )^N   \FILLsym{)}      \FILLsym{)}   \FILLsym{,}   \lnot (  ( \Delta' )^N  ) $.
    We derive the latter as follows:
    \begin{center}
      \scriptsize
      \begin{math}
        $$\mprset{flushleft}
        \inferrule* [right=\ifrName{Impl}] {
          $$\mprset{flushleft}
          \inferrule* [right=\ifrName{Parr}] {
            $$\mprset{flushleft}
            \inferrule* [right=\ifrName{Impr}] {
              $$\mprset{flushleft}
              \inferrule* [right=\ifrName{Pr}] {
                $$\mprset{flushleft}
                \inferrule* [right=\ifrName{Impr}] {
                  $$\mprset{flushleft}
                  \inferrule* [right=\ifrName{Pr}] {
                      \lnot (  \lnot (  ( \Gamma )^N  )  )   \FILLsym{,}   \lnot (  ( \Delta )^N  )   \FILLsym{,}  \FILLmv{x}  \FILLsym{:}   \lnot  \FILLsym{(}   ( \FILLnt{A} )^N   \FILLsym{)}   \FILLsym{,}  \FILLmv{y}  \FILLsym{:}   \lnot  \FILLsym{(}   ( \FILLnt{B} )^N   \FILLsym{)}   \FILLsym{,}   \lnot (  ( \Delta' )^N  )   \vdash   \cdot  
                  }{  \lnot (  \lnot (  ( \Gamma )^N  )  )   \FILLsym{,}   \lnot (  ( \Delta )^N  )   \FILLsym{,}  \FILLmv{x}  \FILLsym{:}   \lnot  \FILLsym{(}   ( \FILLnt{A} )^N   \FILLsym{)}   \FILLsym{,}  \FILLmv{y}  \FILLsym{:}   \lnot  \FILLsym{(}   ( \FILLnt{B} )^N   \FILLsym{)}   \FILLsym{,}   \lnot (  ( \Delta' )^N  )   \vdash   \circ   \FILLsym{:}   \perp  }
                }{  \lnot (  \lnot (  ( \Gamma )^N  )  )   \FILLsym{,}   \lnot (  ( \Delta )^N  )   \FILLsym{,}  \FILLmv{x}  \FILLsym{:}   \lnot  \FILLsym{(}   ( \FILLnt{A} )^N   \FILLsym{)}   \FILLsym{,}   \lnot (  ( \Delta' )^N  )   \vdash   (  \lambda  \FILLmv{y}  .   \circ   )   \FILLsym{:}   \lnot    \lnot  \FILLsym{(}   ( \FILLnt{B} )^N   \FILLsym{)}    }
              }{  \lnot (  \lnot (  ( \Gamma )^N  )  )   \FILLsym{,}   \lnot (  ( \Delta )^N  )   \FILLsym{,}  \FILLmv{x}  \FILLsym{:}   \lnot  \FILLsym{(}   ( \FILLnt{A} )^N   \FILLsym{)}   \FILLsym{,}   \lnot (  ( \Delta' )^N  )   \vdash   \circ   \FILLsym{:}   \perp   \FILLsym{,}   (  \lambda  \FILLmv{y}  .   \circ   )   \FILLsym{:}   \lnot    \lnot  \FILLsym{(}   ( \FILLnt{B} )^N   \FILLsym{)}    }
            }{  \lnot (  \lnot (  ( \Gamma )^N  )  )   \FILLsym{,}   \lnot (  ( \Delta )^N  )   \FILLsym{,}   \lnot (  ( \Delta' )^N  )   \vdash   (  \lambda  \FILLmv{x}  .   \circ   )   \FILLsym{:}    \lnot    \lnot  \FILLsym{(}   ( \FILLnt{A} )^N   \FILLsym{)}      \FILLsym{,}   (  \lambda  \FILLmv{y}  .   \circ   )   \FILLsym{:}   \lnot    \lnot  \FILLsym{(}   ( \FILLnt{B} )^N   \FILLsym{)}    }
          }{  \lnot (  \lnot (  ( \Gamma )^N  )  )   \FILLsym{,}   \lnot (  ( \Delta )^N  )   \FILLsym{,}   \lnot (  ( \Delta' )^N  )   \vdash    (  \lambda  \FILLmv{x}  .   \circ   )   \parr   (  \lambda  \FILLmv{y}  .   \circ   )    \FILLsym{:}     \lnot    \lnot  \FILLsym{(}   ( \FILLnt{A} )^N   \FILLsym{)}      \parr   \lnot    \lnot  \FILLsym{(}   ( \FILLnt{B} )^N   \FILLsym{)}     }
          \\
            $$\mprset{flushleft}
          \inferrule* [right=\ifrName{Pl}] {
            \,
          }{ \FILLmv{w}  \FILLsym{:}   \perp   \vdash   \cdot  }
        }{ \lnot (  \lnot (  ( \Gamma )^N  )  )   \FILLsym{,}   \lnot (  ( \Delta )^N  )   \FILLsym{,}  \FILLmv{z}  \FILLsym{:}   \lnot  \FILLsym{(}     \lnot    \lnot  \FILLsym{(}   ( \FILLnt{A} )^N   \FILLsym{)}      \parr   \lnot    \lnot  \FILLsym{(}   ( \FILLnt{B} )^N   \FILLsym{)}      \FILLsym{)}   \FILLsym{,}   \lnot (  ( \Delta' )^N  ) }
      \end{math}
    \end{center}

  \item[] Case.\\
    \begin{center}
      \begin{math}
        $$\mprset{flushleft}
        \inferrule* [right=\ifrName{Negl}] {
           \Gamma  \FILLsym{,}  \FILLnt{A}  \vdash  \Delta 
        }{ \Gamma  \vdash   \FILLnt{A} ^{\perp}   \FILLsym{,}  \Delta }
      \end{math}
    \end{center}
    By the induction hypothesis we know the following:
    \[   \lnot (  \lnot (  ( \Gamma )^N  )  )   \FILLsym{,}  \FILLmv{x}  \FILLsym{:}   \lnot    \lnot  \FILLsym{(}   ( \FILLnt{A} )^N   \FILLsym{)}     \FILLsym{,}   \lnot (  ( \Delta )^N  )   \vdash   \cdot   \]
    It suffices to show that the sequent
    $  \lnot (  \lnot (  ( \Gamma )^N  )  )   \FILLsym{,}  \FILLmv{x}  \FILLsym{:}   \lnot  \FILLsym{(}   (  \FILLnt{A} ^{\perp}  )^N   \FILLsym{)}   \FILLsym{,}   \lnot (  ( \Delta )^N  )   \vdash   \cdot  $, but this is equivalent to the
    sequent
    $  \lnot (  \lnot (  ( \Gamma )^N  )  )   \FILLsym{,}  \FILLmv{x}  \FILLsym{:}   \lnot    \lnot  \FILLsym{(}   ( \FILLnt{A} )^N   \FILLsym{)}     \FILLsym{,}   \lnot (  ( \Delta )^N  )   \vdash   \cdot  $, but this is equivalent to the induction hypothesis.

    \item[] Case.\\
    \begin{center}
      \begin{math}
        $$\mprset{flushleft}
        \inferrule* [right=\ifrName{Negl}] {
           \Gamma  \vdash  \FILLnt{A}  \FILLsym{,}  \Delta 
        }{ \Gamma  \FILLsym{,}   \FILLnt{A} ^{\perp}   \vdash  \Delta }
      \end{math}
    \end{center}
    By the induction hypothesis we know the following:
    \[   \lnot (  \lnot (  ( \Gamma )^N  )  )   \FILLsym{,}  \FILLmv{x}  \FILLsym{:}   \lnot  \FILLsym{(}   ( \FILLnt{A} )^N   \FILLsym{)}   \FILLsym{,}   \lnot (  ( \Delta )^N  )   \vdash   \cdot   \]
    It suffices to show that the sequent
    $  \lnot (  \lnot (  ( \Gamma )^N  )  )   \FILLsym{,}  \FILLmv{z}  \FILLsym{:}   \lnot    \lnot  \FILLsym{(}   (  \FILLnt{A} ^{\perp}  )^N   \FILLsym{)}     \FILLsym{,}   \lnot (  ( \Delta )^N  )   \vdash   \cdot  $, but this is equivalent to the
    sequent
    $  \lnot (  \lnot (  ( \Gamma )^N  )  )   \FILLsym{,}  \FILLmv{z}  \FILLsym{:}   \lnot    \lnot  \FILLsym{(}   \lnot  \FILLsym{(}   ( \FILLnt{A} )^N   \FILLsym{)}   \FILLsym{)}     \FILLsym{,}   \lnot (  ( \Delta )^N  )   \vdash   \cdot  $.  The latter is derived as follows:
    \begin{center}
      \begin{math}
        $$\mprset{flushleft}
        \inferrule* [right=\ifrName{Impl}] {
          $$\mprset{flushleft}
          \inferrule* [right=\ifrName{Impr}] {
            $$\mprset{flushleft}
            \inferrule* [right=\ifrName{Pr}] {
                \lnot (  \lnot (  ( \Gamma )^N  )  )   \FILLsym{,}  \FILLmv{x}  \FILLsym{:}   \lnot  \FILLsym{(}   ( \FILLnt{A} )^N   \FILLsym{)}   \FILLsym{,}   \lnot (  ( \Delta )^N  )   \vdash   \cdot  
            }{  \lnot (  \lnot (  ( \Gamma )^N  )  )   \FILLsym{,}  \FILLmv{x}  \FILLsym{:}   \lnot  \FILLsym{(}   ( \FILLnt{A} )^N   \FILLsym{)}   \FILLsym{,}   \lnot (  ( \Delta )^N  )   \vdash   \circ   \FILLsym{:}   \perp  }
          }{  \lnot (  \lnot (  ( \Gamma )^N  )  )   \FILLsym{,}   \lnot (  ( \Delta )^N  )   \vdash   \lambda  \FILLmv{x}  .   \circ    \FILLsym{:}   \lnot  \FILLsym{(}   \lnot  \FILLsym{(}   ( \FILLnt{A} )^N   \FILLsym{)}   \FILLsym{)}  }
          \\
           \FILLmv{w}  \FILLsym{:}   \perp   \vdash   \cdot  
        }{  \lnot (  \lnot (  ( \Gamma )^N  )  )   \FILLsym{,}  \FILLmv{z}  \FILLsym{:}   \lnot    \lnot  \FILLsym{(}   \lnot  \FILLsym{(}   ( \FILLnt{A} )^N   \FILLsym{)}   \FILLsym{)}     \FILLsym{,}   \lnot (  ( \Delta )^N  )   \vdash   \cdot  }
      \end{math}
    \end{center}
    
  \item[] Case.\\ 
    \begin{center}
      \begin{math}
        $$\mprset{flushleft}
        \inferrule* [right=\ifrName{Exl}] {
           \Gamma  \FILLsym{,}  \FILLnt{A}  \FILLsym{,}  \FILLnt{B}  \vdash  \Delta 
        }{ \Gamma  \FILLsym{,}  \FILLnt{B}  \FILLsym{,}  \FILLnt{A}  \vdash  \Delta }
      \end{math}
    \end{center}
    This case follows by first applying the induction hypothesis and then applying
    the rule $\ifrName{Exl}$.

  \item[] Case.\\ 
    \begin{center}
      \begin{math}
        $$\mprset{flushleft}
        \inferrule* [right=\ifrName{Exr}] {
           \Gamma  \vdash    \Delta_{{\mathrm{1}}}  \FILLsym{,}  \FILLnt{A}   \FILLsym{,}  \FILLnt{B}   \FILLsym{,}  \Delta_{{\mathrm{2}}} 
        }{ \Gamma  \vdash    \Delta_{{\mathrm{1}}}  \FILLsym{,}  \FILLnt{B}   \FILLsym{,}  \FILLnt{A}   \FILLsym{,}  \Delta_{{\mathrm{2}}} }
      \end{math}
    \end{center}
    This case follows by first applying the induction hypothesis and then applying
    the rule $\ifrName{Exl}$.  
  \end{itemize}
  \end{report}

\end{proof}
% subsection a_negative_translation_of_linear_logic_into_fill_using_tensorial_negation (end)

% section tensorial_logic (end)

\section{Agda Formalization: A Library for Dialectica Categories}
\label{sec:the_agda_formalization:_a_library_for_studying_dialectica_categories}

The majority of the results from the last few sections have been about
the category $\dial$ which, as we have said, corresponds to the
dialectica category $\mathsf{GC}$ from de Paiva's thesis
\cite{dePaiva:1988}.  All of those results have been formalized in the
proof assistant Agda\footnote{More about Agda can be found here
  \url{http://wiki.portal.chalmers.se/agda/pmwiki.php}, and the Agda
  development associated with this article can be found at
  \url{https://github.com/heades/cut-fill-agda}.}.  This formalization
is based on a general library for proving properties of dialectica
categories in sets that we believe may be of interest to the wider community.  This section introduces this library, but before
we recall a generalization of the dialectica categories from de
Paiva's thesis \cite{dePaiva:1988}.

\subsection{Lineales and Dialectica Spaces}
\label{subsec:lineales_dialectica_spaces}
As we introduced in Definition~\ref{def:dial2sets} the objects of
$\dial$ are triples $(U, X, \alpha)$ where $\alpha \subseteq U \times
X$.  Equivalently, the relation $\alpha$ can be taken as a function
$\alpha : U \times X \to \mathsf{2}$, where $\mathsf{2} =
\{0,1\}$. Hyland and de Paiva showed that these relations can be
generalized to maps of the form $\alpha : U \times X \to L$ where the
set $L$ is a lineale \cite{Hyland:1991}, but here we give a slightly
more general definition.

\subsubsection{Lineales}
\label{subsec:lineales}
A lineale is essentially a symmetric monoidal closed category in the category
of prosets -- preorder sets -- which we call a monoidal proset.
\begin{definition}
  \label{def:monoidal-proset}
  A \textbf{monoidal proset} is a proset, $(L, \leq)$, with a given
  symmetric monoidal structure $(L, \circ, e)$.  That is, a set $L$
  with a given binary relation $\leq : L \times L \to L$ satisfying
  the following:
  \begin{itemize}
  \item (reflexivity) $a \leq a$ for any $a \in L$
  \item (transitivity) If $a \leq b$ and $b \leq c$, then $a \leq c$
  \end{itemize}
  together with a monoidal structure $(\circ, e)$ consisting of a
  binary operation, called multiplication, $\circ : L \times L \to L$
  and a distinguished element $e \in L$ called the unit such that the
  following hold:
  \begin{itemize}
  \item (associativity) $(a \circ b) \circ c = a \circ (b \circ c)$
  \item (identity) $a \circ e = a = e \circ a$
  \item (symmetry) $a \circ b = b \circ a$
  \end{itemize}
  Finally, the structures must be compatible, that is, if $a \leq b$,
  then $a \circ c \leq b \circ c$ for any $c \in L$.
\end{definition}
Now a lineale can be seen as essentially a  symmetric monoidal closed
category in the category prosets.
\begin{definition}
  \label{def:lineale}
  A \textbf{lineale} is a monoidal proset, $(L, \leq, \circ, e)$, with
  a given binary operation, called implication, $\limp : L \times L
  \to L$ such that the following hold:
  \begin{itemize}
  \item (relative complement) $a \circ (a \limp b) \leq b$ 
  \item (adjunction) If $a \circ y \leq b$, then $y \leq a \limp b$
  \end{itemize}
\end{definition}
The set $\mathsf{2}$ is an example of a lineale where the order is the
usual one, the multiplication is boolean conjunction, and the
implication is boolean implication.  This example is not that
interesting, because $\mathsf{2}$ is a boolean algebra.  An example of
a proper lineale can be given using the three element set
$\{0,\frac{1}{2}, 1\}$ and the details -- along with some other
examples -- can be found in the Agda
development\footnote{\url{https://github.com/heades/dialectica-spaces/blob/master/concrete-lineales.agda\#L123}},
but one must be careful when defining lineales, because it is possible
to instead define Heyting algebras, and hence, become nonlinear.

The definition of lineales given here is a slight generalization over
the original definition given by Hyland and de Paiva -- see Definition
1 of \cite{Hyland:1991}.  They base lineales on posets instead of
prosets, but the formalization given here shows that anti-symmetry can
be safely dropped.  In fact, the formalization we present in the next
section is nearly a complete formal verification of all of their
results.
% subsubsection lineales (end)

\subsubsection{Dialectica Spaces}
\label{subsec:dialectica_spaces}
Dialectica categories come in two flavors: one called $\mathsf{GC}$
and one called $\mathsf{DC}$.  The former specialized to sets where
the third coordinate of the objects are binary relations corresponds
to $\dial$.  In this section we specialize both GC and DC to sets
using lineales whose objects we call \textbf{dialectica spaces}.

\textbf{The category $\dialSets{L}$}.  This category of dialectica
spaces is defined similarly to the definition given in
Definition~\ref{def:dial2sets}.  The following definition is due to
Hyland and de Paiva \cite{Hyland:1991}.
\begin{definition}
  \label{def:dial-space-dialSets}
  Suppose we are given a lineale $(L, \leq, \circ, e, \limp_L)$. Then
  the dialectica space $\dialSets{L}$ is a category that
  consists of
  \begin{itemize}
  \item objects, or dialectica spaces, that are triples, $A =
    (U,X,\alpha)$, where $U$ and $X$ are sets, and $\alpha : U \times
    X \to L$ is a function, and
  \item maps that are pairs $(f,F) : (U,X,\alpha) \to (V,Y,\beta)$
    where $f : U \to V$ and $F : Y \to X$ such that
    \begin{itemize}
    \item For any $u \in U$ and $y \in Y$, $\alpha(u,F(y)) \leq \beta(f(u),y)$.
    \end{itemize}
  \end{itemize}
\end{definition}
The category $\dial$ is simply an instantiation of the previous
category with the concrete lineale $\mathsf{2}$.

\textbf{The category $\dcSets{L}$}.  The dialectica
category $\mathsf{DC}$ was first proposed in de Paiva's thesis
\cite{dePaiva:1988}.  It corresponds to a model of single-conclusion
intuitionistic linear logic without par.  We can restrict its
definition to sets just as we did for the category $\mathsf{GC}$.
\begin{definition}
  \label{def:dial-space-dcSets}
  Suppose we are given a lineale $(L, \leq, \circ, e, \limp_L)$. Then
  the dialectica space $\dcSets{L}$ is a category that
  consists of
  \begin{itemize}
  \item objects, or dialectica spaces, that are triples, $A =
    (U,X,\alpha)$, where $U$ and $X$ are sets, and $\alpha : U \times
    X \to L$ is a function, and
  \item maps that are pairs $(f,F) : (U,X,\alpha) \to (V,Y,\beta)$
    where $f : U \to V$ and $F : U \times Y \to X$ such that
    \begin{itemize}
    \item For any $u \in U$ and $y \in Y$, $\alpha(u,F(u, y)) \leq \beta(f(u),y)$.
    \end{itemize}
  \end{itemize}
\end{definition}
The differences between the two categories can be seen in the
definition of morphisms between dialectica spaces. In $\dialSets{L}$
the second coordinate is the map $F : Y \to X$, but in $\dcSets{L}$
this map is $F : U \times Y \to X$.  The difference in the definition of morphims may seem small, but  while the modification prevents the definition of a second monoidal structure  modeling
the  connective par in the category $\dcSets{2}$, it is  this category that has a free comonoid structure, the only non merely syntactic example in the literature, we believe.
% subsubsection dialectica_spaces (end)

The generalizations to lineales may seem academic, but they do have
applications in concurrency.  %de Paiva showed that The dialectica
category $\dial$ shares many of the properties of Chu spaces on sets
and on $\mathsf{2}$ \cite{dePaiva:2007}.  Pratt and Gupta
\cite{Gupta:1994} showed that Chu spaces can be seen as a model of
concurrency, but in classical linear logic, while dialectica spaces
should provide a model of concurrency in full intuitionistic linear
logic.
% subsection dialectica_spaces (end)

\subsection{The Dialectica Space Agda Library}
\label{subsec:the_dialectica_space_agda_library}
Proving properties about constructions on dialectica spaces can be
quite involved due to the large number of nested definitions.  This
complexity can be seen in the proofs of
Lemma~\ref{lemma:ddial_is_a_linear_category},
Lemma~\ref{lemma:dial-tensorial}, and
Lemma~\ref{lemma:full-tensorial-logic}.  One of the many powers of
proof assistants like Agda is their ability to simplify set theoretic
based definitions automatically within goals exposing the minimal goal
within.  This greatly simplifies working with dialectica spaces.  In
fact, the developments here actually show that proving properties
about dialectica spaces is actually easier in a proof assistant than
by hand, and with a greater confidence that the proofs are correct.

We extracted a general library for working with dialectica spaces from
our formalization of the results discussed in this article. The
library itself is hosted on Github and can be found at the address
\url{https://github.com/heades/dialectica-spaces}.  It consists of the
definition of lineales, \verb!lineale.agda!, several definitions of
example concrete lineales,\\
\verb!concrete-lineales.agda!, the general
definition of $\dialSets{L}$, \\ \verb!DialSets.agda!, and the general
definition of $\dcSets{L}$, \verb!DCSets.agda!. The definitions of the
latter two are parametric in the choice of lineale.  The formalization
of the results of this paper can be seen as an example of how to use
this library to prove properties of dialectica spaces.
% subsection the_dialectica_space_agda_library (end)
% section the_agda_formalization:_a_library_for_studying_dialectica_categories (end)

\section{Conclusions
%and Future Work
}
\label{sec:conclusion_and_future_work}

We first recalled the definition of full intuitionistic linear logic using the  par rule proposed by Bellin 
%in Section~\ref{sec:full_intuitionistic_linear_logic_(fill)}, 
but using
%only proof-theoretic methods,
%proofnets are proof-theoretical
no proof nets.  We then directly proved cut-elimination for FILL in Section~\ref{sec:cut-elimination} by
adapting the well-known cut-elimination procedure for classical linear
logic to FILL.

In Section~\ref{sec:full_lnl_models} we showed that the category
$\dial$, a model of FILL, is a full LNL model by 
%showing that %it is a full linear category, and then
replaying the proof that linear
categories are LNL models by Benton. Then in
Section~\ref{sec:tensor_logic} we showed that $\dial$ is a model of full tensorial logic. The point of this exercise in categorical logic
is to show that, despite linear logicians infatuation with linear negation, there is value in keeping all your connectives independent
of each other. Only making them definable in terms of others, for
specific applications.

Games, especially programming language games are the main motivation
for Tensorial Logic and have been one of the sources of intuitions in
linear logic all along. Since we are interested in the applications of
tensorial logic to concurrency, we would like to see if our slightly
more general framework can be applied to this task, just as well as
tensorial logic.

Independently of the envisaged applications to programming, we are
also interested in developing a ``man in the street'' game-like
explanation for the finer-grained connectives of FILL, especially for
par, the multiplicative disjunction. The second author has talked
about games for FILL in the style of Lorenzen \cite{dePaiva:2011},
building up on the work of Rahman
\cite{Keiff:2011,Rahman:2005}. Rahman showed that Lorenzen games could
be defined for classical linear logic \cite{Rahman:2002} and discusses  a sound and complete semantics in Lorenzen games for
classical linear logic. Rahman suggests that one could adopt
a particular structural rule  enforcing intuitionism, but we have not
seen a fuller discussion nor a  proof of soundness and completeness for  semantics of such a system. As future work we would like to show that by adapting our work on FILL  we can actually
obtain a sound and complete semantics of Lorenzen games.

% section conclusion_and_future_work (end)

\section*{References}
\label{sec:references}
\bibliographystyle{elsarticle-num}
\bibliography{ref}
% section references (end)


\end{document}

%%% Local Variables: 
%%% mode: latex
%%% TeX-master: t
%%% End: 
